Hablar de la electrónica como en el tema 5 de la tesis larga.

Ver pagina 82 de la tesis de detectores monolíticos

Ver hoja resumen entre los distintos tipos de ASICS comerciales, página 82 de la tesis de detectores monolíticos. Plantearlo como que elegimos PETSYS porque es mejor respecto a los otros.

Leer el tema 3 (parte de electrónica) y tema 4 (parte del flexToT)

Durante los últimos años se han introducido fotodetectores de estado sólido compatibles con altos campos magnéticos: APDs y más recientemente SiPMs. Esta tendencia de uso de detectores compactos, unido al aumento del número de canales en los detectores, ha hecho imprescindible el desarrollo de electrónica integrada en PET. La arquitectura electrónica típica para PET se basa en la digitalización y lectura de la carga que proporcionan los fotosensores, para lo cual se realiza un proceso de amplificación de bajo ruido y adaptación de señal al digitalizador. Sin embargo, existen hoy en día otras formas de detección alternativas, como esquemas time over threshold (ToT) o digitalizadores directos de formas de onda.                                                                                                                                                                                                                                                                                                                                                                                                                                                                                                                                                                                                                                                                                                                                                                                                                                                                                                                                                                                                                                                