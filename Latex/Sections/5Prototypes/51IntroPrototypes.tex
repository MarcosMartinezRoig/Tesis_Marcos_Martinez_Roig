This chapter describes the different prototypes that was developed in the framework of the TRITIUM experiment, which are TRITIUM-IFIC 0, TRITIUM-IFIC 1, TRITIUM Aveiro 0 and TRITIUM-IFIC 2, listed in chronological order of their construction.

The first two prototypes built are preliminary prototypes used to learn about tritium detection and to detect and solve problems in their designs.

The other two prototypes built are prototypes with a well-defined design in which no problems were found. They were built to check more subtle effects. 

Each prototype was designed and built in the laboratories of the university (IFIC, Valencia or Aveiro, Portugal) and it was filled with tritiated water following a protocol specially developed for this task, described in appendix \ref{App:TritiumSourcePreparation}. Several water tightness and filling tests were carried out in each prototype to guarantee its radiosecurity.

Finally, the final monitor of TRITIUM detector will be explained. It is based on modular detection units for easy scalability, where each module is the chosen prototype (the one with the best results).