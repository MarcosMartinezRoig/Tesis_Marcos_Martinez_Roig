TRITIUM-Aveiro and TRITIUM-IFIC-2 have a different design than the previous prototypes that allows the reading of a large number of straight fibres with two photosensors operating in coincidence. Furthermore, the activity of the radioactive liquid source employed was much lower than for the first prototypes to investigate the MDA. The main differences between TRITIUM-Aveiro and TRITIUM-IFIC-2 are:

\begin{enumerate}

\item{} The diameter of the scintillating fibres, $2~\mm$ for TRITIUM-Aveiro and $1~\mm$ for TRITIUM-IFIC-2. The use of a larger diameter may facilitate the flow of water around the fibres, reducing issues related to surface tension and ensuring that the entire active volume of the fibres participates in tritium detection. In addition, a large radius increases the rigidity of the fibres, improving their robustness. However, the larger the radius the smaller the signal-to-background ratio. The detector active volume for $2~\mm$ fibres is smaller than for $1~\mm$ fibres for the same filling volume. Also, the internal volume of the fibres unreachable by tritium decay electrons, which contributes to background, is larger for $2~\mm$ fibres.

%On the one hand $1~\mm$ are better for the tritium detection since the mean free path of the tritium decay electrons inside the fibre is about $5~\mu\meter$. The part of the scintillating fibre deeper than that, larger for $2~\mm$ scintillating fibres will only contribute to the background of the prototype, masking the tritium signal. In addition, more $1~\mm$ fibres can be used in the same space, achieving a larger active area of the detector. Therefore, fewer photosensors are needed for the TRITIUM monitor, lowering its price. On the other hand $2~\mm$ improves the flow of the water through the scintillating fibre bunch, a crucial point since it is directly proportional to the active area of the prototype.

\item{} The whole surface-conditioning method consisting in cleaving, polishing and cleaning methods is applied to the scintillating fibres of TRITIUM-IFIC-2. However, only the cleaving method is applied to the scintillating fibres of TRITIUM-Aveiro.

\item{} TRITIUM-Aveiro uses PMTs as photosensors. Although most of the development of TRITIUM-IFIC-2 was made with PMTs, it is intended to employ SiPM arrays that provides a larger photodetection efficiency than PMTs for a similar price. In addition, no high voltage is needed for SiPMs, which reduces the price of both prototype and electronics.

\item{} TRITIUM-Aveiro uses a home-made PCB-based electronics which is cheaper than PETsys. However, the PETsys system is quite stable and meets the TRITIUM monitor scalability requirement without any additional development.

\end{enumerate}

The development and operation of these two prototypes aim at defining the best design for the final TRITIUM monitor.
