In the prototypes TRITIUM-IFIC-0 and TRITIUM-IFIC-1, the fibres were directly coupled to the photosensor, so only photons guided along the fibres were detected. However, in TRITIUM-Aveiro and TRITIUM-IFIC-2, two PMMA windows are used which allows the transmission to the photosensors of photons propagated through the water. To quantify the importance of this contribution, the TRITIUM-Aveiro prototype was simulated. The distribution of the number of photons that reach the PMMA window per tritium event is shown in Figure \ref{fig:PMMAEffect}. Fibre-guided photons are shown in red while those travelling in water are plotted in blue. It can be seen that the tritium signal obtained from the water-guided photons is as important as that obtained from the fibres. Therefore, PMMA windows improve tritium detection efficiency by around a factor of 2.

\begin{figure}[t]
\centering
\includegraphics[scale=0.32]{6Simulations/61TRITIUMDesign/615PMMA/PhotonsDetectedWaterFiber.png}
\caption{Distribution of photons reaching the PMMA windows. The red histogram corresponds to the photons guided by fibres and the blue histogram to photons guided through water \cite{SimulationPaperCarlos}.\label{fig:PMMAEffect}}
\end{figure}

