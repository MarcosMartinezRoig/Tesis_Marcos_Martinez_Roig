The water samples to be measured by the TRITIUM detector are taken directly from the Tagus river, in a site 4 km downstream from the water discharge of Almaraz NPP. These samples contain minerals, organic deposits, and living matter, which should be removed for the following reasons:

\begin{enumerate}

\item{} The mean free path of tritium electrons in water is around $5~\mu\meter$ and even less in solid materials. If the analyzed water contains particles that may deposit on the fibres, a layer of dirt could be formed, preventing tritium decay electrons from reaching the fibres and reducing drastically the tritium detection efficiency. Therefore,the detector must be kept pristine.

\item{} The tritium monitor does not have any spectrometric capability that could be used to distinguish tritium from other radioactive elements present in water.

\end{enumerate}

The water purification system was designed to remove organic matter and mineral particles with a size over $1~\mu\meter$ without modifying the tritium level in water. 

%Since tritium is the only radioactive element that can be practically equal to water (when it is in the $\ce{HTO}$ form, the majority form in wihch tritium are present in the water sample), with this process we remove all particles radioactive elements other than tritium and the amount of tritium present in the sample is not affected



%In summary, the ultrapure water system is used to keep our detector clean, ensuring the stability of its detection efficiency and to eliminate all radioactive particles other than tritium. %maintaining the activity of the tritium in the sample. Both reasons has been tested with experimental measurements, shown in secton \ref{sec:CharacterizationUltraPureWaterSystem}.