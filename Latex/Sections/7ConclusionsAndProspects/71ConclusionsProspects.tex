Que cosas se han conseguido en este experiemnto? -> DEcir que tanto l oque se ha conseguido con el detector como con las investigaciones de componentes del detector (capitulo 3)

Responder a las grandes preguntas: 
\begin{itemize}
\item{} Podemos medir tritio? 
\item{} Lo podemos hacer en tiempo quasi real? 
\item{}Lo podemos hacer a la actividad que queríamos? 
\item{} Que sensibilidad se ha llegado a conseguir?
\item{} Estabilidad temporal?
\item{} Precio?
\item{} Comparación con respecto al resto de experimentos? -> Poner la tabla 1.8 pero incluyendonos
\item{} Effecto del shield
\item{} Effecto de los vetos
\item{} Effecto de ambas cosas
\item{} Medidas a varias actividades
\end{itemize}


Tenemos datos de tritio en el agua bruta (agua del río) de esa zona desde 1998, pero tengo que solicitar permiso para poder dártelos. En cualquier caso, hay otra manera de conseguirlos, que es a partir  de los informes del CSN al Congreso de los diputados (web CSN).
Desde 2015 la concentración de tritio en el agua del río Tajo ha disminuido considerablemente porque la CNA instaló unos enfriadores por convección que emiten parte del H3 a la atmósfera. --> Tesis de Antonio Rodríguez y de Elena García.