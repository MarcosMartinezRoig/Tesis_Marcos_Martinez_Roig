The objective of the ultrapure water system is to purify the water sample before the measurement. This system is important for two reasons:

\begin{enumerate}

\item{} The mean free path of tritium electrons in water is around $5~\mu\meter$ and even less in solid materials like organic material. The electron from the tritium decay has to reach the fiber to be detected and, consequently, the detector must be kept very clean. If the analyzed water sample contains particles that may be deposited on the fibers, a layer of matter can be formed, preventing the tritium electrons from reaching the fibers and reducing drastically the tritium detection efficiency.

\item{} The tritium monitor does not have any spectrometric capabilities that can be used to distinguish other radioactive elements from tritium. That means that, any radioactive event in the analyzed water sample would be couted as a tritium event.

The ultrapure water system was designed to remove all particles up to a diameter of $1~\mu\meter$ and organic matter, which means that the only radioactive particle that passes through it is tritium. 

%Since tritium is the only radioactive element that can be practically equal to water (when it is in the $\ce{HTO}$ form, the majority form in wihch tritium are present in the water sample), with this process we remove all particles radioactive elements other than tritium and the amount of tritium present in the sample is not affected

\end{enumerate}

In summary, the ultrapure water system is used to keep our detector clean, ensuring the stability of its detection efficiency and to eliminate all radioactive particles other than tritium. %maintaining the activity of the tritium in the sample. Both reasons has been tested with experimental measurements, shown in secton \ref{sec:CharacterizationUltraPureWaterSystem}.