A third test was carried out to check the effect of the fiber diameter in the tritium detection efficiency. For this test, the same simulation explained in the section \ref{subsec:FiberLengthSimulation} was used, where a fiber length of $20~\cm$ was choseen. Two different diameters were taken into account in this study, $1~\mm$ and $2~\mm$, which is the commercial possibilities given by Saint-Gobain company.

It doesn't have sense to test it with the tritium source since, obviously, its efficiency will scale with the active surface\footnote{The active surface of the scintillating fiber is the part of the surface of the scintillating fiber that is in contact with the tritium water.} of the scintillating fiber. However, an interesting study can be performed to check how the fiber diameter affect to the cosmic detection in the fiber.

To do so, the CRY library\footnote{CRY library, Cosmic-Ray Shower library} \cite{CRYwebsite}, \cite{CRYpaper} was used in the previous simulation, which remplaced the tritium water source by a cosmic events source. 

The CRY library is able to generate cosmic-ray particle shower distributions (muons, neutrons, protons, electrons, photons and pions) at several heights (see level in our case).

The result of this simulation will be shown in the section \ref{sec:ResultsSimulations}, where it will be discussed.