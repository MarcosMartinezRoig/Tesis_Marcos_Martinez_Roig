The scintillation photons created in the core of the fibre and directed to its ends are detected by photosensors. Photosensors have a sensitive part that is optimized to detect photons in a range of energy (usually in the visible range) with a certain probability, called quantum efficiency. Photosensors produce an electronic signal that carries information about the detected photons such as their number, detection time, etc. There are many available photosensors that rely on various physical processes, such as photomultiplier tubes (PMTs), silicon photomultipliers (SiPM) or charge-coupled devices (CCD).  %Each one of these will have different properties and it has to be chosen the one which fit better for the objective of the experiment.

The optimization of the efficiency of a scintillation detector is essential. To do so, the emission spectrum of the scintillator (Figure \ref{fig:EmissionSpectrumFibers} for the fibres used) must match as much as possible the detection efficiency spectrum of the photosensor. The efficiency of a detector is proportional to the product of the emission and the detection efficiency spectra and is largest when both spectra match.

The requirements imposed on the photosensor of the TRITIUM detector are fast response, high gain and high photodetection efficiency. Two different proposals for the TRITIUM detector are investigated, SiPMs and PMTs. Both meet these requirements since they are very fast (of the order of $\nano\second$), have high gain (of the order of $10^{6}$) and have a high photodetection efficiency (around $50\%$ for SiPMs and $30\%$ for PMTs). Each proposal has its own advantages. SiPMs are more robust and need a lower supply voltage (of the order of $50~\volt$) than PMTs (of the order of $1000~\volt$). Furthermore, due to this difference in supply voltage, SiPMs have smaller cost per channel than PMTs. However, PMTs, which are the conventional choice, have lower dark count rate than SiPMs and a much lower dependence of gain with temperature.

%A certain portion (in an optimal case nearly 100\%) of the scintillation photons reach the light detector, which has to be sensitive enough to detect a small number of photons. The detector then produces a signal pulse, which has a height proportional to the number of photons hitting the detector. The signal pulse of the detector is processed by the electronics, and as a result a pulse height spectrum is produced (see Section 3.5).

%This spectrum corresponds to the energy spectrum of the detected particles.
