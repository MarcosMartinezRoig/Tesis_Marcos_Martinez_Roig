To achieve the aim of the TRITIUM project, the TRITIUM collaboration has developed a monitor consisting of several parts: 

\begin{enumerate}

\item{} The tritium detector, described in chapter \ref{chap:Prototypes}, compose of several modules read out in parallel. Each module consists of hundreds of plastic scintillating fibres in contact with the water sample measured and read out by two coincident photosensors. The photosensors considered are photomultiplier tubes (PMT) and silicon photomultipliers (SiPM) (section \ref{sec:TritiumDectectorIntro}).

\item{} The water purification system (section \ref{sec:UltraPureWaterSystem}) that prepares the water sample for the measurement. This system removes all the minerals dissolved and all the particles with a diameter greater than $1~\mu\meter$ without affecting the tritium content of the sample. This system is important for two reasons: First, because the mean free path of tritium in water is very short, $5$ to $6~\mu\meter$,  hence it is essential to avoid organic and mineral depositions onto the fibres surface since this would prevent the tritium decay electrons from reaching the fibres. Second, minerals dissolved in water may contain radioactive isotopes like $\ce{^{40}K}$, which would increase the background. As the activity limit to be measured is low (down to $100~\becquerel/\liter$), background reduction is crucial.

\item{} The background rejection system (section \ref{sec:IntroductionBackground}) which is composed of two different parts. The first one is a passive shield, consisting of a lead castle inside which the TRITIUM detector is located. This castle is employed to suppress the background from natural radioactivity and cosmic rays with energies up to $200~\MeV$. The second part is an active veto, consisting of two plastic scintillating plates located inside the passive shield, above and below the tritium detector, which are read out by photosensors. The goal of this active veto is to suppress the remaining high energy events ($>200~\mega\eV$) from cosmic rays that cross the passive shield and contribute to the background. The technique employed consists of reading the tritium detector in anti-coincidence with the active veto.
%to detect these high energy events and, for each of them, open narrow time windows in which we will not read the Tritium detector to prevent these events from affecting the tritium measurement.

\item{} A readout electronic system which allows the acquisition and processing of data to provide an alarm signal in case the tritium level exceeds the required limit of $100~\becquerel/\liter$.

\end{enumerate}

The TRITIUM system is planned to be part of the network of automatic stations, REA.