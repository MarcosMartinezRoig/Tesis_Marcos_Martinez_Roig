%Finally, an addition step was included to the fiber conditioning process, with the objective of improving the photon collection efficiency of the fibers. 

The tritium events only produce tens of photons in the scintillating fibers, so it is very important for the detector to detect as many photons as possible. As it is demostrated in the light collection characterization of scintillating fibers, subsection \ref{subsec:CharacterizationFibers}, the quality of the interface between the core of uncladded fibers and the environment (tritiated water in the case of TRITIUM detector) affects conspicuously the photon collection efficiency. To improve the quality of the interface, a fiber cleaning process was used, aiming to remove external particles deposited on the fibers, such as dust and fat that worsen the photon collection efficiency.  Through this cleaning process, the wetting property of the fibers is improved, that is to say the capacity of its surface to attract water, as illustrated in Figure \ref{fig:WettingProperty}. This implies an increase of the contact surface between the fibers and water, which prevents air molecules from attaching to them, and produce a unifrom water clad around them.

%Therefore, a mechanism, called the fiber cleaning process, was applied. As we can see in Figure \ref{fig:WettingProperty}, this cleaning process was carried out to improve the wetting properties, preventing air molecules from attaching to the fiber and achieving a uniform water clad around each fiber, avoiding variations in its refractive index which can worsen the photon collection efficiency of the fibers.

\begin{figure}[h]
\centering
\includegraphics[scale=0.5]{4ResearchAndDevelopments/41Fibers/WettingProperty.png}
\caption{Schematic representation of the wetting properties between a drop of liquid (blue) and a flat surface (gray). The wetting property is characterized by the angle formed between the surface of both objects. The smaller angle, the better wetting property of the material. \cite{WettingProperty}\label{fig:WettingProperty}}
\end{figure}

This cleaning process  was developed and carried out in the clean room of ICMOL laboratory\footnote{ICMOL, \textit{Instituto de Ciencia MOlecular}, is a research institute located in the \textit{Parc Científic} of the University of Valencia.}. It consists of filling three different glass beakers, one with alkaline soap, another with pure water\footnote{The millipore water is water in which all the ions were removed, producing a very low conductivity, on the order of $10~\mu\text{S}/\cm$} and the last one with isopropanol. First, the fibers are rubbed with gloved hands with alcalin soap during 5 minutes, then placed in the first beaker which is placed in an ultrasonic bath at $17~\kilo\hertz$ frequency during 3 minutes. Then, the fibers are cleaned with a constant flow of water during 5 minutes and they are placed in the second beaker for ultrasonic bath during 3 minutes. Third, the fibers are placed in the third beaker for for ultrasonic bath during another 3 minutes. Finally the fibers are dried with a flow of gas $\ce{N_2}$ and kept in clean conditions until their introduction into the module vessel of TRITIUM detector.

The improvement in the light collection of the scintillation fibers after this cleaning process was measured using a bundle of twenty fibers of $15~\cm$ length that have undergone this cleaning process. This bundle of fibers was arranged in the setup described in section \ref{subsec:PolishingMachine}, Figure \ref{fig:BunchWith2PMTsCoincidence}, and several energy spectra were measured, before and after cleaning the fibers, using two radioactive sources; a $\ce{^{90}Sr}$ beta source, already used in the polishing machine test, and a $\ce{^{137}Cs}$ gamma source, of $500~\becquerel$ activity. The results are plotted in Figure \ref{fig:ResultsOfCleaningProcess}. A shift of the spectrum to higher energies can be noticed for the cleaning fibers. This improvement was quantified by a parameter $F$ definded as,
\begin{equation}
F=\frac{A_{C}-A_{NC}}{A_{C}}
\label{eq:RelativeImprovement}
\end{equation}
where $A_{C}$ and $A_{NC}$ ae the integrals of the energy spectra measured after and before the cleaning process, respectively.

The value of $F$ obtained is about $21\%$ for both radioactive sources. Nevertheless, it should be taken into accout that $F$ was measured in air and the result could be different in water.

\begin{figure}
\centering
    \begin{subfigure}[b]{1\textwidth}
    \centering
    \includegraphics[width=\textwidth]{4ResearchAndDevelopments/41Fibers/Cs-137_CleaningProcess.pdf}  
    \caption{\label{subfig:EnergySpectrumCo60CleaningTest}}
    \end{subfigure}
    \hfill
    \begin{subfigure}[b]{1\textwidth}
    \centering
    \includegraphics[width=\textwidth]{4ResearchAndDevelopments/41Fibers/Sr-90_CleaningProcess.pdf}  
    \caption{\label{subfig:EnergySpectrumSr90CleaningTest}}
    \end{subfigure}
 \caption{Energy spectra obtained before and after the cleaning process using a radioactive source of a) $\ce{^{137}Cs}$ and b) $\ce{^{90}Sr}$.}
 \label{fig:ResultsOfCleaningProcess}
\end{figure}


%$(27.73 \pm 1.6)\%$ for the gamma source and $(20.72 \pm 0.9)\%$ for the beta source so, the improvement of the photon collection efficiency of the fibers was verified using the cleaning process carried out in the clean room of ICMOL laboratories. Nevertheless, it should be taken into accout that this test was carried out in air. It could be interesting to repeat it in water to obtain more realistic conclusions since the fibers of the TRITIUM detector will be immersed in water.