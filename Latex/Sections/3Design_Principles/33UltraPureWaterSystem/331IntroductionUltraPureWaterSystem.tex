The objective of the ultrapure water system is to condition the sample before the measurement. It is important for two reasons:

\begin{itemize}

\item{} On the one hand, it is important because the mean free path of tritium electrons in water, shown in section \ref{sec:TritiumProperties},  is around $5~\mu\meter$ and even less for solid materials like organic material. The electron from the tritium decay must reaches the fiber to be detected, so the detector must be kept very clean. If the analyzed water sample contains particles that can be deposited on the fibers, it can form a layer of matter, which prevents the tritium electrons from reaching the fibers, reducing drastically the tritium detection efficiency until it becomes impossible to measure tritium.

\item{} On the other hand, as it is shown in chapter \ref{chap:Prototypes}, the tritium monitor does not have any spectrometric capabilities that can be used to distinguish other radioactive elements from tritium. That means that, all the radioactive element included in the analyzed water sample will be computed as a tritium event.

The ultrapure water system is used to remove all particles up to a diameter of $1~\mu\meter$ and organic matter, which means the only radioactive particle that passes through it is tritium. 

%Since tritium is the only radioactive element that can be practically equal to water (when it is in the $\ce{HTO}$ form, the majority form in wihch tritium are present in the water sample), with this process we remove all particles radioactive elements other than tritium and the amount of tritium present in the sample is not affected

\end{itemize} 

In summary, the ultrapure water system is used to keep our detector clean, ensuring the stability of its detection efficiency and to eliminate all radioactive particles other than tritium, maintaining the activity of the tritium in the sample. Both reasons has been tested with experimental measurements, shown in secton \ref{sec:CharacterizationUltraPureWaterSystem}.