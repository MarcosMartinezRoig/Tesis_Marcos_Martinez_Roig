The Monte Carlo simulations performed in the TRITIUM project are described in this chapter. They were carried out to optimize the design of the TRITIUM detector, to understand its behaviour and to investigate its limitations. This chapter is divided into two different sections. The first section contains the results of simulations done to improve the design of the TRITIUM detector, while the second section describes the simulation results of a full TRITIUM monitor composed of several TRITIUM-IFIC-2 modules read out in anticoincidence mode with an active cosmic veto. Furthermore, several tests were carried out to check the different simulated steps such as the simulated tritium source and the energy deposition and the production of photons in the fibers. The simulation environment employed is Geant4 \cite{Geant4WebPage, Geant4P}.