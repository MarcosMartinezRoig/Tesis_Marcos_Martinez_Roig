%Finally, an addition step was included to the fiber conditioning process, with the objective of improving the photon collection efficiency of the fibers. 

As it was demonstrated in subsection \ref{subsec:CharacterizationFibers}, the quality of the interface between the core of uncladded fibers and the environment (tritiated water in the case of TRITIUM detector) affects conspicuously the photon collection efficiency. To improve the quality of the interface, a fiber cleaning process was implemented, aiming to remove external particles deposited on the fibers, such as dust and grease.  Through this cleaning process, the wetting property of the fibers is improved, as illustrated in Figure \ref{fig:WettingProperty}. The contact angle between fibers and water is decreased, preventing air molecules from attaching to the fibers and producing a uniform contact between fibers and water.
\begin{figure}[h]
\centering
\includegraphics[scale=0.5]{4ResearchAndDevelopments/41Fibers/WettingProperty.png}
\caption{Schematic representation of the wetting properties of a flat surface (gray) in contact with a drop of liquid (blue). The wetting property is characterized by the angle formed between the surface of both objects. The smaller angle, the better wetting property of the material. \cite{WettingProperty}\label{fig:WettingProperty}}
\end{figure}
This cleaning process  was developed and carried out in the clean room of ICMOL. Three different beakers were used, one filled with alkaline soap, another with pure water (conductivity of the order of $10~\mu\text{S}/\cm$) and the third with isopropanol. The fibers were first rubbed with alcalin soap during 5 minutes, afterwards placed in the first beaker which was placed in an ultrasonic bath at $17~\kilo\hertz$ frequency during 3 minutes. Then, the fibers were cleaned with a constant flow of water during 5 minutes and they were placed in the second beaker for ultrasonic bath during 3 minutes and, subsequently, placed in the third beaker for ultrasonic bath during 3 minutes. Finally, the fibers were dried with a flow of gaseous $\ce{N_2}$ and kept in clean conditions until their introduction into the vessel of the detector. The improvement in the light collection of the scintillating fibers after this cleaning process was determined by measuring the energy spectra of a bundle of twenty uncladded fibers of $15~\cm$ length before and after undergoning this cleaning process with the setup described in Figure \ref{fig:BunchWith2PMTsCoincidence}. This spectra was measured for the background, as shown in Figure \ref{fig:ResultsOfCleaningProcessBackground}, and for a $\ce{^{90}Sr}$ and $\ce{^{137}Cs}$ radioactive sources, shown in Figure \ref{fig:ResultsOfCleaningProcessSource}. A shift of the spectra to higher energies was observed in for the clean fibers with respect to the spectra without cleaning. The equation \ref{eq:RelativeImprovement} was used  quantify the improvement achieved by the cleaning process. Although no improvement in the detected events was observed for the background measurement, an improvement of about $26\%$ and $35\%$ was obtained for the $\ce{^{90}Sr}$ and $\ce{^{137}Cs}$ sources, respectively.

\begin{figure}[h]
\centering
\includegraphics[scale=0.6]{4ResearchAndDevelopments/41Fibers/Background_CleaningProcess.pdf}
\caption{Measured background energy spectra before and after cleaning.\label{fig:ResultsOfCleaningProcessBackground}}
\end{figure}

\begin{figure}
\centering
    \begin{subfigure}[b]{1\textwidth}
    \centering
    \includegraphics[width=\textwidth]{4ResearchAndDevelopments/41Fibers/Cs-137_CleaningProcess.pdf}  
    \caption{\label{subfig:EnergySpectrumCo60CleaningTest}}
    \end{subfigure}
    \hfill
    \begin{subfigure}[b]{1\textwidth}
    \centering
    \includegraphics[width=\textwidth]{4ResearchAndDevelopments/41Fibers/Sr-90_CleaningProcess.pdf}  
    \caption{\label{subfig:EnergySpectrumSr90CleaningTest}}
    \end{subfigure}
 \caption{Energy spectra obtained before and after the cleaning for a radioactive source. a) $\ce{^{137}Cs}$. b) $\ce{^{90}Sr}$.}
 \label{fig:ResultsOfCleaningProcessSource}
\end{figure}


%$(27.73 \pm 1.6)\%$ for the gamma source and $(20.72 \pm 0.9)\%$ for the beta source so, the improvement of the photon collection efficiency of the fibers was verified using the cleaning process carried out in the clean room of ICMOL laboratories. Nevertheless, it should be taken into accout that this test was carried out in air. It could be interesting to repeat it in water to obtain more realistic conclusions since the fibers of the TRITIUM detector will be immersed in water.