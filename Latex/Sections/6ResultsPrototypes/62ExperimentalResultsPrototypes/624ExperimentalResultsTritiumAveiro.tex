The first detector module was characterized, commissioned and installed in the discharge channel of Arrocampo dam to the Tagus river. Due to the high sensitivity of the detection module it requires radioactive background mitigation techniques through the use of active and passive shielding. We have obtained a MinimumDetectable Activity lower than 5 kBq/L for a single module being limited by the cosmic background. Far from the primary goal this value is already compatible with a real-time tritium environmental surveillance monitor and a power plant pre-alert system.


Medida con una fuente de 55Fe, ya que tiene gammas muy cercanas a las del tritio, 5.9keV. Esta se situo dentro del tubo de teflón ya que si no se apantallaría y no llegaría  ala fibra. Debido a esto no podíamos usar agua en su interior.


