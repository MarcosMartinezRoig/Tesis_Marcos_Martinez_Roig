Para realizar el llenado, en primer lugar necesitamos preparar la solución de agua tritiada. Para ello, hemos  empleado una fuente radiactiva de tritio, consistente en  una solución de agua tritiada ($\ce{HTO}$ en $\ce{H_2O}$) de $2.0169 \pm 0.0017\gram$ de peso y una actividad específica de $A=26.8 \pm 0.6~\mega\becquerel/\gram$, con fecha de calibración del 16 de marzo de 2017. La actividad de esta fuente, desde el punto de vista de protección radiológica,  es  exenta~\cite{IFIC}. La fuente fue proporcionada por PTB (Physikalisch-Technische Bundesanstalt, Braunscheweig and Berlin), Alemania, con número de serie 2005-1442, y número de referencía PTB-6.11-285/03.2017, y fecha de referencia de 1 de enero de 2017~\cite{IFIC}.

La solución que preparamos para el experimento, denominada solución patrón, consistió en diluir la ampolla que contenía la solución radiactiva en  $0.5~\liter$ de agua hiperpura (destilada 5 veces y sin cationes, es decir conductividad muy baja), la cual fue proporcionada por el LARUEX de Cáceres, Universidad de Extremadura. Esta disolución fue realizada por Dña. Teresa Cámara en el Laboratorio de Radiactividad Ambiental (LARAM) de la Universitat de València.



Finalmente, con la solución patrón ya preparada, se realizó el proceso de llenado del prototipo en la gammateca del IFIC, sala debidamente acondicionada para manipulación de fuentes radiactivas. Se prepararon un total de $500~\cm^3$ de solución patrón, de los cuales se trasladaron $50~\cm^3$ al IFIC.
El mecanismo seguido para el proceso de llenado se describe a continuación:

\begin{enumerate}
\item{} En primer lugar, se dispuso una bandeja de plástico recubierta con material absorbente en el interior, en la cual se realizó todo el proceso de llenado. El objetivo de esta era evitar, en la medida de lo posible, una contaminación debido a un desbordamiento imprevisto en el proceso de llenado. 

\item{} En segundo lugar, con ayuda de una pipeta y un embudo de cristal, se procedió a llenar una bureta. Para mayor seguridad, se fijó la bureta con un soporte de laboratorio. 

\item{} En tercer lugar, con la bureta ya llena del agua tritiada, se procedió a introducir ésta en el prototipo. Para ello, se introdujo la punta  de la bureta en el orificio de $8~\milli\meter$ del prototipo anteriormente descrito y, lentamente, se procedió al llenado del mismo. El proceso terminó cuando se introdujeron $39~\cm^3$ ya que, como se ha mencionado anteriormente, esta es la capacidad del prototipo.

\item{} En cuarto lugar, se procedió a cerrar y sellar con silicona los  orificios de purga y llenado del prototipo. La disolución sobrante se vertió en una botella suficientemente segura, en la cual se conservará para un futuro uso. Esta disolución, junto con el material empleado en el proceso de llenado, se introdujo en una bolsa de plástico, que a su vez  se introdujo en una caja de cartón, y todo ello fue debidamente guardado en el LARAM.

\item{} En último lugar, se trasladó el prototipo, debidamente rellenado y sellado, al Laboratorio de Reacciones Nucleares del IFIC (025)

\end{enumerate}

