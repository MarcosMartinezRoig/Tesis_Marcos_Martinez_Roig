The different prototypes developed in the framework of the TRITIUM project and their characterization are described in this chapter. They are named TRITIUM-IFIC-0, TRITIUM-IFIC-1, TRITIUM-Aveiro and TRI-TIUM-IFIC-2, listed in chronological order of their construction. The first two prototypes, TRITIUM-IFIC-0 and TRITIUM-IFIC-1, are preliminary prototypes used to learn about in-water tritium detection and to define the monitor design. The prototypes TRITIUM-Aveiro and TRITIUM-IFIC-2 have an optimized design based on the lessons learned from the former prototypes. Each prototype was filled with tritiated water following a specific method. Several water tightness and filling tests were carried out for each prototype. The measurements obtained by the different prototypes in the laboratories and at the Arrocampo dam are discussed in this chapter. At the end of the chapter, the final monitor design of the TRITIUM detector is described.