The objective of the TRITIUM project is the design, development, construction and commissioning of an automatic station for real-time monitoring of low levels of tritium in water. To achieve this aim, the TRITIUM group has developed a monitor consisting of several parts, listed below: 

\begin{enumerate}

\item{} The TRITIUM detector, described in chapter \ref{chap:Prototypes}, is based on several modules read in parallel. Each module consists of hundreds of scintillating fibers, section \ref{subsec:PlasticScintillators}, which are in conectact with the water sample measured, read by two coincident photosensors, section \ref{subsec:Photosensors}. The photosensors considered are photomultiplier tubes (PMT) (section \ref{subsubsec:PMTs}) and silicon photomultipliers (SiPM) (section \ref{subsubsec:SiPM}).

\item{} The ultrapure water system (section \ref{sec:UltraPureWaterSystem}) that prepars the water sample before measurement. This system removes all the organic particles dissolved and all the particles with a diameter greater than $1~\mu\meter$ without affecting the tritium content of the sample. This system is important for two reasons: First, because the mean free path of tritium in water is very short, $5$ or $6~\mu\meter$,  so this is essential to avoid the deposition of particles onto the fibers since this would prevent the tritium decay electrons from reaching the fibers. Second, particles disolved in water may contains raidoactive isotopes like $\ce{^{40}K}$, which whould increase the background. As the water sample has very low tritium counters, to reduce the background is a crucial matter.

\item{} The background rejection system (section \ref{sec:IntroductionBackground}), that has two different parts. The first one is a passive shield (section \ref{subsec:SetUpPassiveShield}), consisting of a lead castle inside of which the TRITIUM detector is located. This castle is employed to eliminate natural radioactive background and cosmic rays with energies up to $200~\MeV/$nucleon. The second part is an active veto (section\ref{subsec:SetUpActiveShield}), consisting of two plastic scintillation blocks located inside of a passive shielding, above and below the TRITIUM detector and read by several photosensors. The goal of this active veto is to remove the remaining high energy events ($>200~\mega\eV$), cosmic rays that can travel through the passive shielding and contribute to background. Contrary to low energy cosmics rays, high energy cosmic rays are difficult to be stopped. The technique employed to eliminate their contribution consists of reading the TRITIUM detector in anti-coincidence with the active veto.
%to detect these high energy events and, for each of them, open narrow time windows in which we will not read the Tritium detector to prevent these events from affecting the tritium measurement.

\item{} A monitoring electronic system sends an alarm if the signal limit of the tritium level, $100~\becquerel/\second$, is exceeded.

\end{enumerate}

The different parts of TRITIUM monitor were subjected to tests to verify their correct operation before installing them in the Arrocampo dam. The final goal is to include TRITIUM in the network of automatic stations, REA (section \ref{sec:Introduction}).