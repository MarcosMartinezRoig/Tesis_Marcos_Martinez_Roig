This section shows the characterization of the ultrapure water system, the objective of which is to ensure that the quality of the water sample used to be measured is good enough to overcome the requirements of the TRITIUM detector. There are three different requirements that this ultrapure water system must meet:

\begin{enumerate}
\item{} A low enough conductivity\footnote{Conductivity is the ability of the material to conduct electrical current. In liquids, conductivity is related to the presence of salts (presence of positive and negative ions)} of the water needs to be achieved, around $10~\mu\sievert/\cm$, so that external particles disolved in the water don't be deposited on the fibers, drastically reducing the detector efficiency due to such a low mean free path of the tritium electron.

\item{} It must be noted that this system does not have any spectral capabilities that can be used to distinguish between several radioactive isotopes, so to measure only tritium, this system must remove all external radioactive particles (other that tritium isotope) from the sample.

\item{} Lastly, the tritium activity should not be affected by this process. 

\end{enumerate}

To verify that these requirements have been exceeded, a characterization of the water sample was carried out before and after the ultrapure water system, called raw water and purified water respectively. This characterization consists of measuring the water sample conductivity and the activity of each radioactive element that are present in the sample. The turbidity and the chemical components of the water sample will also be measured.

It must be taken into account that, so that the sample was representative of the raw water sample, it was taken from a place located very close to the input of the ultrapure water system, which is located at 40 meters from the ultrapure water system and two meters deep in the river. It was seen that it is very important since variations of up to $25\%$ in the tritium activity was measured between both points (due to the diffusion of tritium along the river).

First, the chemical composition of the water was measured before the ultra-purification process by a physico-chemical analysis, which was carried out a few years ago. It is shown in Table \ref{tab:ChemicalComponentsRawWater}.

\begin{table}[htbp]
%%\centering
\begin{center}
\begin{tabular}{|c|c|}
\hline
Chemical components & Concentration ($\milli\gram/\liter$)\\
\hline \hline \hline
$\ce{CO_{3}H^-}$ & $154$ \\ \hline
$\ce{Mg}$ & $46$ \\ \hline
$\ce{Ca}$ & $105$ \\ \hline
$\ce{NO_{3}^-}$ & $16$ \\ \hline
$\ce{Cl^-}$ & $196$ \\ \hline
$\ce{NO_{2}^-}$ & $0.03$ \\ \hline
$\ce{K}$ & $11$ \\ \hline
$\ce{Na}$ & $173$ \\ \hline
$\ce{SO_{4}^-}$ & $217$ \\ \hline
Dry Residue & $1029$ \\ \hline
\end{tabular}
\caption{Chemical components and turbidity measured in the raw water sample.}
\label{tab:ChemicalComponentsRawWater}
\end{center}
\end{table}

This table shows that the water sample contains a number of components that must be cleaned to prevent their deposition on the scintillating fibers of the detector, reducing their sensitivity, or affecting the tritium measurement.

Its turbidity\footnote{The turbidity of water is the loss of its transparency due to dissolved particles, normally measured in NTU, Nephelometric Units of Turbidity, which measure the intensity of the scattered light at 90 degrees.} was also measured using the Hanna Hi 9829 portable multiparameter system from Hanna Instruments \cite{TurbiditySystem}, obtaining a value of $29$ NTU, much higher that the WHO recommended limit of $5$ NTU.

Second, the conductivity was measured for both, raw and purified water. To do so, the same multiparameter system was used, the Hanna Hi 9829. These measurements, together with the measurement of the conductivity of reject water, explained in section \ref{subsec:SetUpWaterSystem}, are presented in Table \ref{tab:ConductivityValues}.

\begin{table}[htbp]
%%\centering
\begin{center}
\begin{tabular}{|c|c|c|c|}
\hline
Date & Raw ($\mu\sievert/\cm$) & Pure ($\mu\sievert/\cm$) & Reject ($\mu\sievert/\cm$) \\
\hline \hline \hline
$1/8/18$ & $970$ & $11.85$ & $1442$ \\ \hline
$7/8/18$ & $958$ & $11.8$ & $1632$ \\ \hline
$14/8/18$ & $966$ & $12.04$ & $1725$ \\ \hline
$22/8/18$ & $980$ & $12.54$ & $1702$ \\ \hline
$28/8/18$ & $987$ & $9.9$ & $1692$ \\ \hline
$5/9/18$ & $1009$ & $12.02$ & $1645$ \\ \hline
\end{tabular}
\caption{Measurements of the conductivity for several samples of each water type (raw water, pure water and reject water).}
\label{tab:ConductivityValues}
\end{center}
\end{table}	

As can be seen in the first column, the raw water sample has high values of conductivity, caused because it contains many different ions, shown in Table \ref{tab:ChemicalComponentsRawWater}. It can be appreciate that, in the second column, the conductivity values of pure water was reduced by almost two orders of magnitude, reaching values close to $10~\mu\sievert/\cm$, exceeding a requirement previously mentioned.

Finally, as can be checked in the third column, the reject water has high values of conductivity. The reason of that is because it contains the ions that was removed to the pure water.

Third, the gamma radioactive elements present in the water sample of both types, raw water and purified water, was identified and their activities was measured using a high purity germanium detector, HPGe. Then, a gamma analysis was carried out to find the natural gamma emitters (those that come from the natural radioactive series, Table \ref{tab:NaturalRadioactiveSeries}) and the artificial gamma emitters with long enough lifetime to be measured (those that come from the activation of nuclear fission of neutrons). 

The radioactive elements found in the raw water sample with activities high enough to be measured by the HPGe detector was $\ce{{40}^k}$ and small quantities of $\ce{{226}^Ra}$ which were completely disappeared in the purified water sample.

Lastly, the tritium activity was measured to see how it is affected by the ultra-purification process. This measurement was carried out using the Quantulus system, which consists of a liquid scintillator mixed with tritiated water, readout by PMTs. 

Before this measurement, each water sample was filtered at 0.45 microns to remove any particles that could cause the extinction of the scintillation signal.

The Table \ref{tab:ActivityTritiumValues} show several measurements of the activity for different tritium samples of each water types (raw water, reject water and purified water).

\begin{table}[htbp]
%%\centering
\begin{center}
\begin{tabular}{|c|c|c|}
\hline
Date & Raw ($\becquerel/\liter$) & Pure ($\becquerel/\liter$) \\
\hline \hline \hline
$7/8/18$ & $24 \pm 3$ & $26 \pm 4$ \\ \hline
$11/12/19$ & $13.2 \pm 2.1$ & $13.85 \pm 2.2$ \\ \hline
$15/01/20$ & $30.6 \pm 4.2$ & $30 \pm 4$ \\ \hline
\end{tabular}
\caption{Measurements of the activity for several samples of both water types (raw water and pure water).}
\label{tab:ActivityTritiumValues}
\end{center}
\end{table}	

As can be see, tritium activity is not affected by this system, exceeding the last requirement of the ultrapure system. 