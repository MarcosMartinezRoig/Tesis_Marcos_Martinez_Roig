This appendix shows the electronic system designed to perform a complete characterization of the SiPM S13360-6075 model, which is the one proposed for the final TRITIUM monitor. This consists on three different PCBs\footnote{PCB, Printed Circuit Board}, shown in Figure \ref{fig:PCBs_LEDSpectrum}:

\begin{enumerate}
\item{} The first PCB, shown in Figure \ref{subfig:PCB1}, is used to organize the SiPMs and sensor temperature. This PCB place up to 8 different SiPMs and a temperature sensor and arrange their output signals on two HDMI connections. This PCB is placed inside a light tightness box, from Thorlabs \cite{ThorlabsCompany}. This black box has a small hole of $1~\mm$ diameter, prepared to introduce an optical fiber\footnote{The optical fiber used is BCF-98 from Saint-Gobain company \cite{OpticalFibers}} to iluminate SiPMs with a LED, model 430L from Thorlabs \cite{LEDThorlabs}. The spectrum of this LED, shown in Figure \ref{subfig:LEDSpectrum}, was measured with a spectrometer and fitted to a Gaussian function. It can be seen that the emission peak of this LED is located at $436~\nm$ with a FWHM of $19.1~\nano\meter$. With the help of this LED, the light emission of the TRITIUM scintillating fibers was simulated to calibrate the SiPMs at the working wavelength. 

\item{} The second PCB, shown in Figure \ref{subfig:PCB2}, sums the different signals of the SiPMs and amplify them by a factor $G=4187$ or $G=10761$, depending on the input resistance of the oscilloscope, $50~\varOmega$ or $1~\mega\varOmega$, respectively. This PCB uses a differential amplification that reduces the electronic noise of the system and is connected to the first PCB through two HDMI feedthroughs.

\item{} The third PCB, shown in Figure \ref{subfig:PCB3}, rearranges all the different input and output signals in an HDMI connection to avoid crosstalk between different signals. This PCB is connected to the second PCB through a HDMI feedthrough. The input signals are the supply voltage of the SiPMs and the supply voltage of the PCBs ($\pm 6~\volt$) and the output signals are the temperature sensor signal and the sum of all the SiPM signals. The output signal of the third PCB is connected to an oscilloscope, model MSO44X from Tektronix \cite{Oscilloscope}, that records the data.

\end{enumerate}

\begin{figure}
\centering
    \begin{subfigure}[b]{0.5\textwidth}
    \centering
    \includegraphics[width=\textwidth]{3DesignPrinciples/32Tritium_detector/PCB1_SiPM_Black_Box.jpg}  
    \caption{\label{subfig:PCB1}}
    \end{subfigure}
    \hfill
    \begin{subfigure}[b]{0.45\textwidth}
    \centering
    \includegraphics[width=\textwidth]{3DesignPrinciples/32Tritium_detector/PCB2_SIPMs.png}  
    \caption{\label{subfig:PCB2}}
    \end{subfigure}
    \hfill
    \begin{subfigure}[b]{0.4\textwidth}
    \centering
    \includegraphics[width=\textwidth]{3DesignPrinciples/32Tritium_detector/PCB3_SiPMs.png}  
    \caption{\label{subfig:PCB3}}
    \end{subfigure}
    \hfill
    \begin{subfigure}[b]{0.5\textwidth}
    \centering
    \includegraphics[width=\textwidth]{3DesignPrinciples/32Tritium_detector/LED_DUNE.pdf}  
    \caption{\label{subfig:LEDSpectrum}}
    \end{subfigure}
 \caption{Three PCBs used for the SiPM characterization a) The PCB 1 used to arrange 8 SiPMs and black box. b) The PCB 2 used to sum up and amplify the output signals of SiPMs. c) The PCB 3 used to rearrange the different signals of the system. d) The LED emission spectrum.}
 \label{fig:PCBs_LEDSpectrum}
\end{figure}