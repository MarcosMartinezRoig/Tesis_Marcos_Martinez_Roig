In the first prototypes, TRITIUM-IFIC 0 and TRITIUM-IFIC-1, the fibers were directly coupled to the photosensor, so the detected photons were only those guided by fibers. However, in the last prototypes, TRITIUM-Aveiro and TRITIUM-IFIC 2, two PMMA windows are used, which allows the detection of photons guided by fiber and photons that propagate through the water. To quantify the importance of the latter contribution, the TRITIUM-Aveiro prototype was simulated. The distribution of the number of photons that reach the PMMA per tritium event is shown in Figure \ref{fig:PMMAEffect}. Fiber-guided photons are shown in a red distribution, while those traveling in the water medium are plotted in the blue histogram. It can be seen that the tritium signal obtained from the water is as important as that obtained from the fibers. Therefore, the use of PMMA windows improve the tritium detection efficiency by a factor 2.

\begin{figure}[hbtp]
\centering
\includegraphics[scale=0.3]{Figures/8SimulationsResults/81TRITIUMDesign/815PMMA/PhotonsDetectedWaterFiber.png}
\caption{Distribution of photons reaching PMMA windows. The red histogram corresponds to the photons guided by fibers and the blue histogram to photons traveling in the water \cite{SimulationPaperCarlos}.\label{fig:PMMAEffect}}
\end{figure}

