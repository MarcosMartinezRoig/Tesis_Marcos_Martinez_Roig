The scintillation yield provided by the manufacturer, $8000~\text{photons}/\MeV$, is only valid for minimum ionizing particles (MIP). As tritium electron energies do not correspond to MIP particles, the output light generated by the scintillating fibers was studied. For this task, the energy deposition of tritium electrons in scintillating fibers and their subsequent emission of scintillation photons was included in the simulation.

When particles that are not MIP are detected in plastic scintillators, a quenching effect for the output light per unit of path length, $\frac{dL}{dx}$, with respect to the energy deposited per unit of path length, $\frac{dE}{dx}$, happens, that can be parametrized by the Birks coefficient\cite{BirksPaper}.
\begin{equation}
\frac{dL}{dx}= S\frac{\displaystyle{\frac{dE}{dx}}}{1+k_B\displaystyle{\frac{dE}{dx}}}
\label{eq:birkscoefficient}
\end{equation}
where S is the scintillation yield, provided by the manufacturer. The value for the Birk's coefficient of $k_B=0.126~\mm/\MeV$, typically used for scintillators based on polystyrene \cite{BirksCoefficient}, was taken. In this section, the significance of this quenching effect and how it affects the tritium detection is discussed.

A study of the energy deposition of tritium electrons on scintillating fibers was carried out. In Figure \ref{fig:InitialFinalTritiumEnergy} the initial energy of simulated tritium electrons that reach the scintillating fibers is compared to the energy deposited in the scintillating fibers. A shifth to lower energies is observed, caused by the loss of energy of tritium electrons in water. A cut about $1~\keV$ is observed in both energy distributions, produced by the default energy threshold of $990~\eV$ in the G4EmLivermorePhysics physics list.

\begin{figure}[h]
\centering
\includegraphics[scale=0.3]{6Simulations/61TRITIUMDesign/612Outputlight/InitialandFinalTritiumEnergy.png}
\caption{Distribution of the initial energy of tritium events that reach the scintillating fibers (blue histogram) and the energy deposited in these fibers (red histogram) \cite{SimulationPaperCarlos}.\label{fig:InitialFinalTritiumEnergy}}
\end{figure}

Figure \ref{fig:BirksEffectinEnergyDistribution} shows two distributions of number of photons produced in scintillating fibers by tritium events, one in which the quenching effect is not considered ($k_B=0$) and the other with the Birks coefficient set to $k_B=0.126~\mm/\MeV$.

\begin{figure}[h]
\centering
\includegraphics[scale=0.3]{6Simulations/61TRITIUMDesign/612Outputlight/BirksEnergyDistribution.png}
\caption{Energy distribution of photons produced in the scintillating fiber, without the Birks coefficient (red histogram) and with the Birks coefficient of $k_B=0.126~\mm/\MeV$ (blue histogram)\cite{SimulationPaperCarlos}.\label{fig:BirksEffectinEnergyDistribution}}
\end{figure}  

A distribution with a peak of around 40 photons per tritium event and a maximum of around 150 photons is obtained when the quenching effect is not considered. A significant reduction of the output light is observed when the Birks coefficient is taken into account, producing a distribution peaked at around $10$ photons and a maximum of $110$ photons. The quenching effect is also observed in Figure \ref{fig:2DimPlotBirks}, in which the number of produced photons as a function of the energy deposited in the fibers is displayed in a two-dimensional plot. In this figure, in addition to a reduction of the number of photons produced per unit of energy deposited, a broader distribution is obtained when the Birks coefficient is considered, indicating an increase of the fluctuations of energy deposition.

\begin{figure}
\centering
    \begin{subfigure}[b]{0.4\textwidth}
    \centering
    \includegraphics[width=\textwidth]{6Simulations/61TRITIUMDesign/612Outputlight/BidimensionalPlotBirksOFF.png}  
    \caption{\label{subfig:2DimPlotNoBirks}}
    \end{subfigure}
    \hfill
    \begin{subfigure}[b]{0.4\textwidth}
    \centering
    \includegraphics[width=\textwidth]{6Simulations/61TRITIUMDesign/612Outputlight/BidimensionalPlotBirksON.png}  
    \caption{\label{subfig:2DimPlotBirks}}
    \end{subfigure}
 \caption{Number of photons produced versus the energy deposited in the scintillating fibers when a)the Birks coefficient is not considered ($k_B=0$) b) the Birks coefficient is $k_B=0.126~\mm/\MeV$ \cite{SimulationPaperCarlos}.}
 \label{fig:2DimPlotBirks}
\end{figure}