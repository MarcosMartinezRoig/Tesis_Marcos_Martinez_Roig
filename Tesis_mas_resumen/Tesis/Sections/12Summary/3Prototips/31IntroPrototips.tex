En la col·laboració TRITIUM, un total de quatre prototips del detector de triti van ser desenvolupats. Els dos primers, anomenats TRITIUM-IFIC-0 i TRITIUM-IFIC-1, són prototips a una escala menor. Aquests van ser utilitzats com a prova de concepte per aprendre sobre la detecció de triti en aigua i estudiar la viabilitat de possibles millores per a optimitzar-la. Els dos darrers prototips, anomenats TRITIUM-Aveiro i TRITIUM-IFIC-2, són prototips funcionals amb un disseny similar, emprats per a avaluar la importància de certes diferències subtils en el disseny i elaborar un prototip final que incloga aquelles opcions que optimitzen l'eficiència de la detecció del triti. Durant l'emplenat d'aquests prototips amb aigua tritiada es van realitzar diverses proves d'estanquitat per garantir la radioseguretat de l'entorn. De forma complementària, es van realitzar diverses simulacions amb la llibreria Geant 4 \cite{Geant4WebPage}, utilitzades per obtenir més informació sobre quines possibles millores en els dissenys dels prototips incrementarien les característiques rellevants del monitor.