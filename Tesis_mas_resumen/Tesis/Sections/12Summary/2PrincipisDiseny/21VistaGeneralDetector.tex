El monitor desenvolupat per la col·laboració s'anomena TRITIUM i consisteix de tres parts principals:

\begin{enumerate}

\item{} El detector de triti, on es realitza la mesura de la mostra d'aigua. Aquest està format per diversos mòduls llegits en paral·lel. El disseny modular li confereix la propietat d'escalabilitat, permetent-li millorar certes característiques com l'eficiència o l'activitat mínima detectable de triti simplement amb la utilització de més mòduls.

Cadascun d'aquests mòduls està format per centenars de fibres centellejadores que entren en contacte directe amb la mostra d'aigua a mesurar. Aquestes fibres són capaces de detectar part de les desintegracions de triti que tenen lloc i són llegides per diversos fotosensors, el senyal dels quals és analitzat per l'electrònica corresponent.

\item{} El sistema de purificació d'aigua, el qual prepara la mostra abans d'introduir-la al detector. Aquest sistema és utilitzat per dos motius principals. En primer lloc, per a eliminar qualsevol partícula amb un diàmetre superior a $1~\mu\meter$ que es trobe dissolta a l'aigua, la qual podria ser depositada sobre les fibres centellejadores. Pel fet que el recorregut lliure mitjà del triti és de només $5-6~\mu\meter$ en la matèria, aquesta deposició provocaria una disminució de l'eficiència de detecció del triti. En segon lloc, per a eliminar qualsevol element radioactiu (diferent del triti) dissolt a l'aigua, com per exemple el $\ce{^{40}K}$ que es troba de manera natural al medi ambient, ja que aquest seria comtabilitzat pel detector com un esdeveniment de triti, falsejant les mesures.

\item{} El sistema de rebuig del fons radioactiu, utilitzat per suprimir en la mesura del que siga possible el fons radioactiu en el lloc de treball. Aquest sistema es divideix en dues parts. Un castell de plom que envolta el detector, el qual para gran part de les partícules amb energies inferiors a $200~\MeV$, procedents principalment del fons radioactiu natural del lloc de treball i dels rajos còsmics. Un veto actiu emprat per a reduir els esdeveniments detectats procedents de rajos còsmics amb energies superiors a $200~\MeV$.

\end{enumerate}

Tant el sistema de purificació de l'aigua com el sistema de rebuig del fons radioactiu són fonamentals, ja que permeten reduir gran part del fons radioactiu mesurat pel detector, una tasca essencial per a aconseguir l'objectiu del projecte (mesurar activitats de triti tan baixes com $100~\becquerel/\liter$). Finalment aquestes tres parts són controlades per diversos aparells electrònics que, en últim lloc, emetran un senyal d'alarma en cas que el límit legal fixat a Espanya per al triti en aigües potables ($100~\becquerel/\liter$) siga superat. La idea final seria poder incloure aquest monitor a la xarxa d'estacions automàtiques del CSN (REA per les sigles en castellà) \cite{REA}. Aquesta és una xarxa formada per detectors de diferents tipus que mesuren en temps quasireal les concentracions d'alguns dels diferents isòtops radioactius que es troben al medi ambient. 