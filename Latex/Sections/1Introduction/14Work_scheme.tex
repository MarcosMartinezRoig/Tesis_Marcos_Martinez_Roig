This thesis is divided into nine different chapters that structure the information as follows:

\begin{enumerate}
\item{} \textbf{Chapter 1} provides a brief introduction to tritium detection, reports some important properties of tritium, and discusses the current legislation that limits tritium levels for human consumption in many countries around the world. 

\item{} \textbf{Chapter 2} describes the State-of-the-Art of tritium detection and shortly introduces the TRITIUM project. 

\item{} \textbf{Chapter 3} outlines the different parts of the TRITIUM monitor, which are the water purification system, the background rejection system (consisting of the lead shield and the active veto)  and the tritium detector. 

\item{} \textbf{Chapter 4} reports the calibrations of the different parts of the TRITIUM monitor and describes the developments aimed at improving the efficiency of tritium detection. 

\item{} \textbf{Chapter 5} details the geometrical configuration of the different prototypes built in the TRITIUM project and the measurements taken with them. 

\item{} \textbf{Chapter 6} details the Monte Carlo simulations performed in the TRITIUM project and show the results obtained with them. 

\item{} \textbf{Chapter 7} summarizes and widely discusses the most important results achieved in this work.

\item{} \textbf{Chapter 8} briefly reviews the main results of this PhD work and discusses about the next step of the TRITIUM project. This chapter ends with the purpose of different applications for the TRITIUM monitor.

\end{enumerate}

