Photoelectron multiplier tube is one of the most used photosensors in nuclear physics during last decades. Its main objective, like all photosensors, is to detect the scintillating photons that reach its sensible part and covert it in an electronic signal large enough to be measured. The way in which PMTs achieve this objective is based in two different phases:

\begin{itemize}
\item{} First, the PMT convert photons that reach its sensible part in electrons with some probability. The sensible part is the photocathode (sensible part of the PMT) which consists in a fina capa de un material que produzca efecto photoelectrico con grán probabilidad. En nuestro caso el material utilizado es. La probabilidad de... es... el espectro de emission...  

- Hablar del photocatodo, parte sensible bla, bla, bla. Espectro de efficiencia, función de trabajo del photocatodo, etc...

\item{} After that, secondary electron multiplication

- Hablar de la etapa de ganancia, dinodos, circuito electrónico divisor de ganancia, bla, bla bla...

2 circuitos diferentes. En principio equivalentes pero el primero más seguro y el segundo mejor para mediciones precisas de tiempo y de alta tasa de cuentas.

\end{itemize}


Cuando acabe con todo esto hablar del resto de elementos, tubo de vacío, etc...


The PMT has two main functions. On the one hand it is able to convert photons, whose energy are inside of a energy range, in electrons throgh photoelectric effect. On the other hand, it capable of multiplying these electrons with high gain factors.


an electric pulse. It is based on a photocathode, which is the sensible part of the photosensor. The photocathode release an electron with some probability when a photon reach it...

IMAGEN

Nuestros objetivos para elegir el PMT y SiPM adecuado.

Leer el capitulo de PMTs del trabajo de "Centelleadores" -> Clave para explicar estas 2 etapas.

Leer el capitulo de PMTs de la tesis de fibras
Leer el capitulo de PMTs de la tesis alemanan -> Clave para los elementos externos, tubo de vacío, etc.



Linear response with the incoming photons.



Incluir lo de la intro de photosensores

%Conceptualmente, un fotomultiplicador cuenta con un fotocátodo y un multiplicador de electrones. El primero es una fina capa de un compuesto que emite electrones cuando absorbe fotones en el espectro visible o en las cercanı́as de él. Los electrones emitidos por el fotocátodo son llamados fotoelectrones. El segundo, de nombre sugestivo, es un arreglo de electrodos conectados a alta tensión que permite obtener ganancias de 10 6 . Más adelante en este capı́tulo, volveremos sobre los PMT.

