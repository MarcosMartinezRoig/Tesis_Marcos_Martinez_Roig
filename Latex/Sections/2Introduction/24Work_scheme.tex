The objetive of this three-phase project was design, develop and instalation and commisioning a automatical system capable of detection tritium which we find in the water that is used by the nuclear power plants for their cooling system. The initial idea was quantifying its activity in units of $\becquerel/\liter$ in quasi-real time before discharging it into public rivers or seas. 



Dividiremos este trabajo en seis partes:
\begin{enumerate}
\item{} En primer lugar, se realizará un estudio sobre las fibras centelleadoras para  determinar  el protocolo de manipulación para obtener un  procedimiento de preparación de  un haz de fibras centelleadoras con un buen rendimiento óptico. 

\item{} En segundo, lugar estudiaremos el procedimiento de calibración de los SiPM,  fundamental para el experimento.  No se necesita realizar una calibración de los PMT, paso igualmente importante al anterior, ya que este trabajo fue realizado recientemente por otro componente del grupo.

\item{} En tercer lugar, se describirá  el primer prototipo diseñado, formado por un haz de 35 fibras centelleadoras sin clad leídas por PMT,  incluyendo el protocolo del proceso de llenado con agua tritiada que tuvo que ser desarrollado para cumplir los requisitos de protección radiológica y evitar contaminación accidental,  y los  resultados obtenidos con el mismo.

\item{} En cuarto lugar, se presentarán las simulaciones realizadas con el programa de Geant4 en una configuración sencilla.

\item{} En quinto lugar, se expondrán aspectos a estudiar en el futuro inmediato y en etapas posteriores, durante la Tesis Doctoral. 


\item{} Se presentarán, finalmente, los resultados, logros y conclusiones alcanzadas en el desarrollo del  trabajo.

\end{enumerate}