Que cosas se han conseguido en este experiemnto? -> DEcir que tanto l oque se ha conseguido con el detector como con las investigaciones de componentes del detector (capitulo 3)

Responder a las grandes preguntas: 
\begin{itemize}
\item{} Podemos medir tritio? 
\item{} Lo podemos hacer en tiempo quasi real? 
\item{}Lo podemos hacer a la actividad que queríamos? 
\item{} Que sensibilidad se ha llegado a conseguir?
\item{} Estabilidad temporal?
\item{} Precio?
\item{} Comparación con respecto al resto de experimentos? -> Poner la tabla 1.8 pero incluyendonos
\item{} Effecto del shield
\item{} Effecto de los vetos
\item{} Effecto de ambas cosas
\item{} Medidas a varias actividades
\end{itemize}


Hemos llegado a detectar 30kBq/L con Tritium-Aveiro y se espera llegar a medir hasta menos de 5kBq/L (superando los actuales limites).

Hemos llegado a detectar 10 kBq/L con Tritium-IFIC 2 (superando los actuales limits) y se espera llegar a medir incluso menos.

Ambos valores, lejos de ser el objetivo del proyecto, sirven para una monitorización en tiempo quasí real. Además, con el monitori final que consiste en varios modiules de estos en apralelo, se pretende llegar al objetivo deseado.


Tenemos datos de tritio en el agua bruta (agua del río) de esa zona desde 1998, pero tengo que solicitar permiso para poder dártelos. En cualquier caso, hay otra manera de conseguirlos, que es a partir  de los informes del CSN al Congreso de los diputados (web CSN).
Desde 2015 la concentración de tritio en el agua del río Tajo ha disminuido considerablemente porque la CNA instaló unos enfriadores por convección que emiten parte del H3 a la atmósfera. --> Tesis de Antonio Rodríguez y de Elena García.




\begin{table}[htbp]
\begin{center}
\begin{tabular}{|c|c|c|c|c|}
\hline
Study & \parbox{5.5em}{\centering $\eta_{det}(\frac{cps \cdot{} 10^{-3}}{\kilo\becquerel/\liter})$}  & \parbox{4.5em}{\centering $F_{sci}$ ($\cm^2$)}  & \parbox{6.5em}{\centering $\varepsilon_{det}(\frac{cps \cdot{} 10^{-6}}{\kilo\becquerel/ \liter \cdot{} \cm^2})$} & LDL ($\kilo\becquerel / \liter$)\\
\hline \hline \hline
Muramatsu & $0.39$ & $123$ & $3.13$ & $370$ \\ \hline
Moghissi & $4.50$ & $>424.1$ & $<10.6$ & $37$ \\\hline
Osborne & $12$ & $3000$ & $4$ & $37$ \\ \hline
Singh & $41$ & $3000$ & $13.7$ & $<37$ \\ \hline
Hofstetter & $2.22$ & $\sim~100$ & $<22.2$ & $25$ \\ \hline
TRITIUM-IFIC 0 & $2.11$ & $219.91$ & $9.59$ & $100$ \\ \hline
TRITIUM-IFIC 1 & $38.42$ & $402.12$ & $95.55$ & $100$ \\ \hline
TRITIUM-IFIC 2 & $723.98$ & $5026.55$ & $144.03$ & $10$ \\ \hline
\end{tabular}
\caption{Results of different scintillator detector for tritiated water detection. This table shows the efficiency of the detector ($\eta_{det}$), its active surface ($F_{sci}$), its specific efficiency ($\varepsilon_{det}=\eta_{det}/F_{sci}$), defined as its efficiency normalized to its active surface, and its low detection-level (LDL) for each study listed above.}
\label{tab:ComparisonResults}
\end{center}
\end{table}