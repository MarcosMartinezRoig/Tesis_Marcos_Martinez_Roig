%\section {Bibliografía}
\begin{thebibliography}{100}
%Reference 1
\bibitem{Renovables} \textsc{F. J. Echarte},
\textit{El futuro de las energías renovables en España}, Universidad de Navarra, \textbf{TECNUN'01 IESE'13}.

%Reference 2
\bibitem{EIA},
\textit{International Energy Outlook 2013},  \textbf{U. E. Energy Information Administration}.

%Reference 3
\bibitem{HighestCO2}
\textit{https://news.un.org/en/story/2019/11/1052111}, \textbf{UN news}.

%Reference 4
\bibitem{Kyoto}
\textit{Kyoto protocol and reference manual}, 2008, \textbf{United Nations}.

%Reference 5
\bibitem{ITER}
\textit{https://www.iter.org/}, \textbf{ITER}.

%Reference 6
\bibitem{TritiumDocument} \textsc{A. Fiege}, 
\textit{Tritium}, Kernforschungszentrum Karlsruhe, \textit{1992}.

%Reference 7
\bibitem{FusionCourse} \textsc{Eduardo Oliva Gonzalo}, \textsc{Adriana Ortiz Gómez}, \textsc{Nuria Moral Fernández}, \textsc{Alejandro Carrasco Sánchez}, \textsc{José Manuel Perlado Martín}, \textsc{Raquel Suárez Hontoria}, \textsc{Manuel Cotelo Ferreiro} 
\textit{Curso Básico de Fusión Nuclear}, jóvenes nucleares, Sociedad Nuclear Española, \textit{Septiembre de 2017, Madrid, Spain}.

%Reference 8
\bibitem{ComparationEmissions} \textsc{Benjamin K. Sovacool}, 
\textit{Valuing the greenhouse gas emissions from nuclear power: A critical survey}, 
ELSEVIER, Energy Policy  \textit{Vol 36} p. 2940-2953.

%Reference 9
\bibitem{PercentageEnergySpain}
\textit{Avance del informe del sistema eléctrico español, 2019}, 
\textbf{Red eléctrica española}.

%Reference 10
\bibitem{ThreeMileIsland}
\textit{www.world-nuclear.org/information-library/safety-and-security/\\
safety-of-plants/three-mile-island-accident.aspx}, \textbf{World Nuclear Association}.

%Reference 11
\bibitem{CloseNPP}
\textit{https://cincodias.elpais.com/cincodias/2018/11/15/\\
companias/1542275699\_182457.html
}, \textbf{Cinco Días, El Pais}.

%Reference 12
\bibitem{60ReactorsChina}
\textit{https://www.europapress.es/internacional/noticia-china-\\
construira-menos-60-centrales-nucleares-proxima-\\
decada-20160916210159.html}, 
\textbf{Europa press}.

%Reference 13
\bibitem{35MillionsUSA}
\textit{https://www.energynews.es/estados-unidos-centrales-nucleares/}, 
\textbf{Energy News}.

%Reference 14
\bibitem{FiberDetector1a} \textsc{J. W. Berthold}, \textsc{L. A. Jeffers},
\textit{Phase 1 Final Report for In-Situ Tritium Beta Detector}, 
U. S. Department of Energy, McDermott Technology, Inc.,Research and Development Division, 	\textbf{DE-AC21-96MC33128}, April, 1998.

%Reference 15
\bibitem{FiberDetector1b} \textsc{McDermott Technology, Inc. (MTI)}, 
\textit{In Situ Tritium Beta Detector}, 
Technology development data sheet, \textbf{DE-AC21-96MC33128}, May, 1999.

%Reference 16
\bibitem{CommonEmissionTritium} \textsc{X- Hou},  
\textit{Tritium and \ce{^{14}C} in the environmental and nuclear facilities: Sources and analytical methods}, Journal of the Nuclear Fuel Cycle and Waste Technology (JNFCWT), 16 (2018), 11-39 \textbf{doi:10.7733/jnfcwt.2018.16.1.11}.

%Reference 17
\bibitem{FERMILAB}
\textit{https://www.fnal.gov/pub/tritium/}.

%Reference 18
\bibitem{BrookHavenNationalLaboratory}
\textit{https://www.bnl.gov/hfbr/decommission.php}.

%Reference 19
\bibitem{TrackingTritium} \textsc{Aleksandra Sawodni}, \textsc{Anna Pazdur}, \textsc{Jacek Pawlyta}, 
\textit{Measurements of Tritium Radioactivity in Surface Water on the Upper Silesia Region}, Journal on Methods and Applications of Absolute Chronology, Geochronometria, Vol. 18, pp 23-28 \textbf{2000}.

%Reference 20
\bibitem{100BqL}  
\textit{Council directive 2013/15/euratom}.

%Reference 21
\bibitem{740BqL} \textsc{Title 40},  
\textit{Protection of the Environment, US Code of Federal Regulations} Part 141, Section 66 (\textbf{June 2011}).

%Reference 22
\bibitem{CrossSeccionNeutrons}  
\textit{REFERENCIAAAAAAAA}.

%Reference 23
\bibitem{TritiumDiscovery} \textsc{M. L. Oliphant}, \textsc{P. Harteck} and \textsc{E. Rutherford},  
\textit{Transmutation Effects observed with Heavy Hydrogen}, Nature, 133, 413 (1934)\textbf{doi:10.1038/133413a0}.

%Reference 24
\bibitem{TritiumIsolate} \textsc{Luis W. Alvarez} and \textsc{R. Cornog},  
\textit{Helium and Hydrogen of Mass 3}, Physical Review Journals Archive, 53, 613 (1939)\textbf{https://doi.org/10.1103/PhysRev.56.613}.

%Reference 25
\bibitem{TritiumHandling} 
\textit{DOE Handbook: Primer on Tritium Safe Handling Practices}, U. S. Departament Of Energy Washington, D.C. 20585.

%Reference 26
\bibitem{OxigenTritium} \textsc{Robert Haight}, \textsc{Joseph Wermer} and \textsc{Michael Fikani},
\textit{Tritium Production by Fast Neutrons on Oxygen: An Integral Experiment}, Journal of Nuclear Science and Technology, 39:sup2, 1232-1235, \textbf{https://doi.org/10.1080/00223131.2002.10875326}. 

%Referencia 27
\bibitem{CrossSeccionNeutrino} \textsc{},
\textit{REFERENCIAAAA}, \textbf{}

%Referencia 28
\bibitem{TritiumDecayEnergyLevels} 
\textit{https://www-nds.iaea.org}, International Atomic Energy Agency.

%Referencia 29
\bibitem{TritiumDecayImage} 
\textit{https://conexioncausal.wordpress.com}, .

%Referencia 30
\bibitem{TesisTritio} \textsc{Zoltán Köllo},
\textit{Tesis: Studies on a plastic scintillator detector for activity measurement of tritiated water}, Facultad de Física, Instituto Tecnológico de Karlsruhe (KIT), Karlsruhe, Alemania, \textit{17/07/2015}

%Referencia 31
\bibitem{AutoRadyolisis} \textsc{Sylver Heinze}, \textsc{Thibaut Stolz}, \textsc{Didier Ducret} and \textsc{Jean-Claude Colson},
\textit{Self-Radiolysis of Tritiated Water: Experimental Study and Simulation}, Fusion Science and Technology, 48:1, 673-679, \textbf{doi:10.13182/FST05-A1014}

%Referencia 32
\bibitem{LSCLARAM} \textsc{}, \textsc{}, \textsc{} and \textsc{},
\textit{}, , , , \textbf{}

%Referencia 33
\bibitem{LSCothers} \textsc{M. N. Al-Haddad}, \textsc{A. H. Fayoumi} and \textsc{F. A. Abu-Jarad},
\textit{Calibration of a liquid scintillation counter to assess tritium levels in various samples}, Nuclear Instruments and Methods in PHysics Research A, Volume 438, Issues 2-3, December 1999, Pages 356-361, \textbf{https://doi.org/10.1016/S0168-9002(99)00272-7}

%Referencia 34
\bibitem{HofstetterSeveral} \textsc{K. J. Hofstetter} and \textsc{H. T. Wilson},
\textit{Aqueous Effluent Tritium Monitor Development}, Fusion Technology, Volume 21, 2P2, Pages 446-451, March 1992, \textbf{https://doi.org/10.13182/FST92-A29786}

%Referencia 35
\bibitem{0.6Bq_L} \textsc{M. Palomo}. \textsc{A. Peñalver}, \textsc{C. Aguilar} and \textsc{F. Borrull},
\textit{Tritium activity levels in environmental water samples from different origins}, Applied Radiation and Isotopes, Volume 65, Issue 9, September 2007, Pages 1048-1056, \textbf{https://doi.org/10.1016/j.apradiso.2007.03.013}

%Referencia 36
\bibitem{OnlineLSC} \textsc{R. A. Sigg}, \textsc{J. E. McCarty}, \textsc{R. R. Livingston} and \textsc{M. A. Sanders},
\textit{Real-time aqueous tritium monitor using liquid scintillation counting}, FNuclear Instrument and Methods in Physics Research A, Volume 353, Issues 1-3, 30 Decembre 1994, Pages 494-498 \textbf{https://doi.org/10.1016/0168-9002(94)91707-8}


%Referencia 37
\bibitem{IonizationChamber1} \textsc{N. P. Kherani},
\textit{An alternative approach to tritium-in-water monitoring}, Nuclear and Methods in PHysics Research A, Volume 484, Issues 1-3, 21 May 2002, Pages 650-659 \textbf{https://doi.org/10.1016/S0168-9002(01)02008-3}

%Referencia 38
\bibitem{IonizationChamber2} \textsc{Z. Chen}, \textsc{S. Peng}, \textsc{D. Meng} \textsc{Y. He} and \textsc{H. Wang},
\textit{Theoretical study of energy deposition in ionization chambers for tritium measurements}, Review of Scientific Instruments, 84, 103302, 2013, \textbf{https://dx.doi.org/10.1063/1.4825032}

%Referencia 39
\bibitem{Calorimeter1} \textsc{C. G. Alecu}, \textsc{U. Besserer}, \textsc{B. Bornschein}, \textsc{B. Kloppe}, \textsc{Z. Köllö} and \textsc{J. Wendel},
\textit{Reachable Accuracy and Precision for Tritium Measurements by Calorimetry at TLK}, Fusion Science and Technology, 60:3, 937-940, \textbf{https://doi.org/10.13182/FST11-A12569}

%Referencia 40
\bibitem{Calorimeter2} \textsc{A. Bükki-Deme}, \textsc{C. G. Alecu}, \textsc{B. Kloppe} and \textsc{B. Bornschein},
\textit{First results with the upgraded TLK tritium calorimeter IGC-V0.5}, Fusion Engineering and Design, Volume 88, Issue 11, November 2013, Pages 2865-2869 \textbf{https://doi.org/10.1016/j.fusengdes.2013.05.066}

%Referencia 41
\bibitem{XRays1} \textsc{M. Matsuyama}, \textsc{Y. Torikai}, \textsc{M. Hara} and \textsc{K. Watanabe},
\textit{New Technique for non-destructive measurements of tritium in future fusion reactors}, IAEA Nuclear Fusion, Volume 47, Number 7, S464, June 2007, \textbf{https://doi.org/10.1088/0029-5515/47/7/S09}

%Referencia 42
\bibitem{XRays2} \textsc{M. Matsuyama},
\textit{Development of a new detection system for monitoring high-level tritiated water}, Fusion Engineering and Design, Volume 83, Issue 10-12, December 2008, Pages 1438-1441 \textbf{https://doi.org/10.1016/j.fusengdes.2008.05.023}

%Referencia 43
\bibitem{Bremstrahlung} \textsc{S. Niemes}, \textsc{M. Sturm}, \textsc{R. Michling} and \textsc{B. Bornschein},
\textit{High Level Tritiated Water Monitoring by Bremsstrahlung Counting Using a Silicon Dift Detector}, Fusion Science and Technology, 67:3, 507-510, 2015, \textbf{https://doi.org/10.13182/FST14-T66}

%Referencia 44
\bibitem{APD} \textsc{K. S. Shah}, \textsc{P. Gothoskar}, \textsc{R. Farrell} and \textsc{J. Gordon},
\textit{High Efficiency Detection of Tritium Using Silicon Avalanche Photodiodes}, IEEE Transactions on Nuclear Science, Volume 44, Issue 3, June 1997, \textbf{10.1109/23.603750}

%Referencia 45
\bibitem{Spectrometry} \textsc{P. Jean-Baptiste}, \textsc{E. Fourré}, \textsc{A. Dapoigny}, \textsc{D. Baumier}, \textsc{N. Baglan} and \textsc{G. Alanic},
\textit{\ce{^{3}He} mass spectrometry for very low-level measurement of organic tritium in environmental samples}, Journal of Environmental Radioactivity, Volume 101, Issue 2, Febrary 2010, Pages 185-190 \textbf{https://doi.org/10.1016/j.jenvrad.2009.10.005} 

%Referencia 46
\bibitem{Ring} \textsc{C. Bray}, \textsc{A. Pailoux} and \textsc{S. Plumeri},
\textit{Tritiated water detection in the 2.17 $\mu$M spectral region by cavity ring down spectroscopy},  Nuclear Instruments and Methods in PHysics Research A, Volume 789, 21 July 2015, Pages 43-49, \textbf{https://doi.org/10.1016/j.nima.2015.03.064} 

%Referencia 47
\bibitem{Muramatsu} \textsc{M. Muramatsu}, \textsc{A. Koyano} and \textsc{N. Tokanuga},
\textit{A Scintillation Probe for Continuous Monitoring of Tritiated Water}, Nuclear Instruments and Methods, Volume 54, Issue 2, October 1967, Page 325-326, \textbf{https://doi.org/10.1016/0029-554X(67)90645-3}

%Referencia 48
\bibitem{Moghissi} \textsc{A. A. Moghissi}, \textsc{H. L. Kelley}, \textsc{C. R. Phillips} and \textsc{J. E. Regnier},
\textit{A Tritium Monitor Based on Scintillation}, Nuclear Instruments and Methods, Volume 68, Issue 1, 1 Febrary 1969, Page 159, \textbf{https://doi.org/10.1016/0029-554X(69)90705-8}

%Referencia 49
\bibitem{Osborne} \textsc{R. V. Osborne},
\textit{Detector for Tritium in Water}, Nuclear Instruments and Methods, Volume 77, Issue 1, 1 January 1970, Page 170-172, \textbf{https://doi.org/10.1016/0029-554X(70)90596-3}

%Referencia 50
\bibitem{Ratnakaran} \textsc{A. N. Singh}, \textsc{M. Ratnakaran} and \textsc{K. G. Vohra},
\textit{An Online Tritium-in-Water Monitor}, Nuclear Instruments and Methods, Volume 236, Issue 1, 1 May 1985, Page 159-164, \textbf{https://doi.org/10.1016/0168-9002(85)90141-X}

%Referencia 51
\bibitem{Ratnakaran2000} \textsc{M. Ratnakaran}, \textsc{R. M. Revetkar}, \textsc{R. K. Samant} and \textsc{M. C. Abani},
\textit{A Real-time Tritium-In-Water Monitor for Measurement Of Heavy Water Leak To The Secondary Coolant}, International congress of the INternational Radiation Protection Association, Volume 32, Issue 15, 14-19 May 2000, P-3a-197, Reference number:\textbf{32015986}

%Referencia 52
\bibitem{Hofstetter1} \textsc{K. J. Hofstetter} and \textsc{H. T. Wilson},
\textit{Aqueous Effluent Tritium Monitor Development}, Fusion Technology, Volume 21, 2P2, 1992, Pages 446-451, \textbf{https://doi.org/10.13182/FST92-A29786}

%Referencia 53
\bibitem{Hofstetter2} \textsc{K. J. Hofstetter} and \textsc{H. T. Wilson},
\textit{Continuous Tritium Effluent Water Monitor at the Savannah River Site}, International conference on advances in liquid scintillation, Vienna (Austria), 14-18 September 1992

%Referencia 54
\bibitem{TRITIUM} \textit{https://tritium-sudoe.eu/es-es/homepage}, Tritium, Interreg Sudoe Program.

%Referencia 55
\bibitem{Knoll} \textsc{Glenn F. Knoll}, 
\textit{Radiation Detection and Measurement}, Third Edition, John Wiley and Sons, Inc. 1999.

%Referencia 56
\bibitem{Leo} \textsc{William R. Leo},
\textit{Techniques for Nuclear and Particle Physics Experiments: a how-to approach}, Second Revised Edition, Springer-Verlag Berlin Heidelberg GmbH, 1994, \textbf{https://doi.org/10.1007/978-3-642-57920-2}. 

%~\cite{Ivo}
\end{thebibliography}