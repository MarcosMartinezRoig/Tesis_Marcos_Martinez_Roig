In recent decades, many nuclear-based technologies have been developed and applied in various fields such as Energy, Medicine, Industry, etc. Due to nuclear applications, anthropogenic radioactive sources have emerged, which have led to the release of radioactive elements into the environment. Tritium is one of the most abundantly emitted radioisotopes by nuclear facilities and, specifically, by nuclear power plants. Large amounts of tritium are normally produced in the water of their cooling systems, which are subsequently discharged into the environment. As large releases of tritium are hazardous for human health and the environment, several regulations are issued around the world with the aim of controlling these radioactive emissions. The main ones are the European Directive 2013/51/Euratom which establishes the tritium limit for drinking water in Europe to $100~\becquerel/\liter$ and the Environmental Protection Agency (EPA) in the United States which limits tritium in drinking water to $20~\nano\curie/\liter$.

Due to the low energy of electrons emitted in the tritium decay, very sensitive detectors are needed for measuring tritium activity. The most widely used is liquid scintillation counting which is an offline method that requires a measurement time of 2 days or more, a time too long to detect a release issue in a nuclear power plant. Detectors based on solid scintillators are promising for building a tritium detector that works in quasi-real-time (less than $1~\hour$). These detectors are being successfully developed for decades but without achieving the sensitivity required for the legal limits established in Europe.

The results of the TRITIUM project are presented in this thesis. In the framework of this project, a quasi-real-time monitor for low tritium activities in water has been developed. This monitor is composed of several detection modules read out in parallel, a water purification system and a background rejection system. Each detection module is made up of hundreds of scintillating fibres read out by photosensors (PMTs or SiPM arrays).

The final objective of this monitor is radiological protection in the water-courses located in the vicinity of nuclear power plants. This monitor will provide an alarm in case of an unexpected tritium release that exceeds the legal limits established in Europe. The TRITIUM monitor is intended to be included in the early alarm system of Extremadura consisting of a network of detectors dedicated to control the impact of the Almaraz nuclear power plant on the environment.

\vspace{1cm}

\textbf{Keywords:} Very low-energy charged particle detectors, tritiated water, tritium detection, scintillating fibres, detector modelling and simulations, real-time radiation monitoring, environmental safety, radiological protection.