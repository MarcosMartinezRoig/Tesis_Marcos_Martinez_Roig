The water samples to be measured by the TRITIUM detector are taken directly from the Tagus river, in a site 4 km downstream from the water discharge place of Almaraz NPP. This sample contains many dissolved elements such as minerals, organic deposits, and living matter dissolved in the water, reported in section \ref{sec:CharacterizationUltraPureWaterSystem}, which need to be removed for the following reasons:

\begin{enumerate}

\item{} The mean free path of tritium electrons in water is around $5~\mu\meter$ and even less in solid materials. Tritium decay electrons have to reach the fiber to be detected and, consequently, the detector must be kept prustine. If the analyzed water sample contains particles that may be deposited on the fibers, a layer of dirt could be formed, preventing tritium decay electrons from reaching the fibers and reducing drastically the tritium detection efficiency.

\item{} The tritium monitor does not have any spectrometric capability that could be used to distinguish tritium from other radioactive elements in water impurities.

\end{enumerate}

The water purification system was designed to remove organic matter and mineral particles with a size of up to $1~\mu\meter$ without modifying the tritium level in water. 

%Since tritium is the only radioactive element that can be practically equal to water (when it is in the $\ce{HTO}$ form, the majority form in wihch tritium are present in the water sample), with this process we remove all particles radioactive elements other than tritium and the amount of tritium present in the sample is not affected



%In summary, the ultrapure water system is used to keep our detector clean, ensuring the stability of its detection efficiency and to eliminate all radioactive particles other than tritium. %maintaining the activity of the tritium in the sample. Both reasons has been tested with experimental measurements, shown in secton \ref{sec:CharacterizationUltraPureWaterSystem}.