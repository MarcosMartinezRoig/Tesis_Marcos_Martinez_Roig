In this work we present simulation results for a modular tritium in-water real-time monitor. The system allows for scalability in order to achieve the required sensitivity. The modules are composed by 340 uncladed scintillating fibers immersed in water and 2 photosensors in coincidence for light readout.

- Simulación del tritio. Vemos que solo se detecta tritio hasta 5 micras. 
Debido a esto solo simularemos aros de tritio de espesor X.

Vemos la necesidad de tener que utilizar fibras sin clad. También vemos que el espectro de deposición de energía esta centrado en 10 keVs ya que las emisiones de energía pequeña pocas veces llegan a la fibra.

Mostrar aquí la simulación del espectro de tritio. Si eso mostrarla superpuesta a la referencia 10 del paper de carlos (figura 2)

- Simulación del tamaño de las fibras. Vemos que es mejor fibras cortas 

Este experimento a consistido en simular


- Simulación fibras de 1 mm y 2 mm. Cósmicos.

- Incluido el coeficiente de Birks ya que este afecta bastante. Mostrar como afecta.

 Light yield and Birks’ coefficient uncertainties for low energy beta particles is discussed.
 
  A study of the detection efficiency according to the fiber length is presented.

Simulación de cosmicos. Vemos que el resultado es mejor en fibras de 1 mm

- Expandimos esto a tritium ific 2. 
Empezar por todos los pasos que sigue la luz y demas y ver que todo se simula correctamente (similar a como se hace en el TFM). Luego mostrar los resultados.

Due to the low energetic beta emission from tritium a detection efficiency close to 3.3\% was calculated for a single 2 mm round fiber.-> Para Tritium-Aveiro 0 prototype

- Añadimos el blindaje de plomo

- Añadimos los vetos activos

- Suma de ambos efectos.

Si pongo resultados de Tritium Aveiro genial, si no tengo que cambiar el último parrafo de la seccion Geant4 Environment.