The mean free path of tritium electrons in water is only around $5~\mu\meter$, so most electrons do not reach the scintillating fibers. These electrons do not provide useful information and only consume computing resources. To optimize the simulation, the dimensions of the simulated tritium source were set to maximize the number of tritium events reaching the scintillating fibers. 

It was checked that the energy distribution of the simulated tritium events, shown in figure \ref{subfig:EnergyDistributionTritiumSource}, agrees with the input \cite{TritiumEmissionSpectrum}. In addition, the distribution of the initial energy of tritium electrons capable of penetrating a fiber and depositing energy, shown in Figure \ref{subfig:EnergySpectrumEventsDetectedandNonDetected}, is shifted to high energies an dhas a peak centred at $10~\keV$. 

\begin{figure}
\centering
    \begin{subfigure}[b]{0.45\textwidth}
    \centering
    \includegraphics[width=\textwidth]{6Simulations/61TRITIUMDesign/611TritiumSourceOptimization/TritiumSourceEnergyDistribution.png}  
    \caption{\label{subfig:EnergyDistributionTritiumSource}}
    \end{subfigure}
    \hfill
    \begin{subfigure}[b]{0.45\textwidth}
    \centering
    \includegraphics[width=\textwidth]{6Simulations/61TRITIUMDesign/611TritiumSourceOptimization/Source_Spectrum_yes_and_non_detected_events.png}  
    \caption{\label{subfig:EnergySpectrumEventsDetectedandNonDetected}}
    \end{subfigure}
 \caption{Energy distribution of a) simulated tritium decays b) Initial energy of tritium decays that reach the scintillating fibers (red histogram) compared to all simulated tritium events (blue histogram) \cite{SimulationPaperCarlos}.
 \label{fig:TritiumSourceOptimization}}
\end{figure}

A scintillating fiber $20~\cm$ long and $2~\mm$ diameter and a surrounding tritiated water source of the same length and $0.5~\mm$ thick($100$ times greater that the mean free path of tritium electrons) were simulated. The dimensions of the fiber were not important since only the energy deposition of tritium electrons in the fiber was registered, excluding optical processes. The goal of these simulations was to find the radial thickness of the simulated tritium source beyond which no significant amount of tritium decay electrons are detected. In Figure \ref{subfig:TransversalCutTritiumSource}, a transversal cut of the $2~\mm$ scintillating fiber, the $0.5~\mm$ thick tritium source surrounding the fiber, and the position where happen the tritium decays that deposit energy in the scintillating fiber are shown. Furthermore, the distribution of the radial distance between the position where tritium decays take place and the surface of the scintillating fiber is shown in figure \ref{subfig:DistanceDistributionTritiumSourceFiber}. A zoom of low energy events is shown in the inset box of Figure \ref{subfig:DistanceDistributionTritiumSourceFiber}. The chosen thickness of the simulated tritium source was $5~\mu\meter$ since $99.4\%$ of the events that deposit energy in the fibers are produced at a shorter distance.

\begin{figure}
\centering
    \begin{subfigure}[b]{0.45\textwidth}
    \centering
    \includegraphics[width=\textwidth]{6Simulations/61TRITIUMDesign/611TritiumSourceOptimization/Source_Ring.png}  
    \caption{\label{subfig:TransversalCutTritiumSource}}
    \end{subfigure}
    \hfill
    \begin{subfigure}[b]{0.45\textwidth}
    \centering
    \includegraphics[width=\textwidth]{6Simulations/61TRITIUMDesign/611TritiumSourceOptimization/SourceDistance.png}  
    \caption{\label{subfig:DistanceDistributionTritiumSourceFiber}}
    \end{subfigure}
 \caption{a)Transversal cut of simulated scintillating fiber (yellow) and tritium source (green) with various tritium decays (red dots) b) Distribution of the radial distance between the position where the tritium decay takes place and the surface of the scintillating fiber \cite{SimulationPaperCarlos}.}
 \label{fig:TritiumSourceSimulated}
\end{figure}	

