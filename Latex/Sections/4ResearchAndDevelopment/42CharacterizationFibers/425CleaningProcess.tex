Finally, an addition step was included to the fiber conditioning process, which is applied to improve the photon collection efficiency of the fibers. 

As we have seen before, the tritium events detected in the fibers produce few photons, so it is very important to conserve as many photons as we can until they are detected by the photosensors. As we have also seen in our fiber characterization study, the quality of the interface that is created between the core of no clad fibers (used in the TRITIUM detector) and the environment (tritium water in our case) greatly affects collection efficiency.

Therefore, a mechanism, called the fiber cleaning process, was applied to improve the state of this interface. This mechanism aims to remove all external particles deposited on the fibers, such as fat molecules deposited when the fibers are touched, which can affect the quality of the interface, worsening their photon collection efficiency. 

As a consequence of this cleaning process, the wetting property of the fibers, shown in Figure \ref{fig:WettingProperty}, is improved, preventing air molecules from attaching to the fiber and achieving a uniform water clad around each fiber and, therefore, improving their collection efficiency. 

%Therefore, a mechanism, called the fiber cleaning process, was applied. As we can see in Figure \ref{fig:WettingProperty}, this cleaning process was carried out to improve the wetting properties, preventing air molecules from attaching to the fiber and achieving a uniform water clad around each fiber, avoiding variations in its refractive index which can worsen the photon collection efficiency of the fibers.

\begin{figure}[h]
\centering
\includegraphics[scale=0.6]{4ResearchAndDevelopments/41Fibers/WettingProperty.png}
\caption{Wetting property produced by the cleaning process. \cite{}\label{fig:WettingProperty}}
\end{figure}


The cleaning process used was carried out within a clean room at ICMOL laboratories\footnote{ICMOL, Institute of Molecular Science, is a research institute located in the Science Park of the University of Valencia.} and it was developed by their researchers.

It consists of filling three different beakers, one with alkaline soap, another with millipore water\footnote{The millipore water is...} and the last one with isopropanol. First, the fibers must be rubbed for 5 minutes with alkaline soap and then placed in the first beaker for sonication for 3 minutes. Then, the fibers must be cleaned with a constant flow of water for 5 minutes. Second, the fibers will be placed in the second beaker for sonication for another 3 minutes, and third, they will be placed in the third beaker for sonication for another 3 minutes. Finally they will be dried with an $N_2$ air gun and introduced inside of the prototype.

The improvement in fiber response was verified using a bundle of twenty fibers with a length of $15~\cm$ that was prepared with the conditioning process previously described. This bundle of fibers was arranged in the setup described in Figure \ref{fig:BunchWith2PMTsCoincidence} and a measurement was taken. Then ,these fibers were cleaned with the fiber cleaning process and a measurement was taken again in the same conditions.

Two radioactive sources were used in this study, a beta source, $\ce{^{90}Sr}$, which is the same as that used in the polishing machine test, and a gamma source, $\ce{^{137}Cs}$, whose activity was ...

The results are shown in Figures \ref{fig:ResultsOfCleaningProcess}, where we can see a shift of the spectrum to the right. 

\begin{figure}[htbp]
 \centering
  \subfloat[Energy spectrum recorded for the Cs-137 source.]{
   \label{subfig:EnergySpectrumCo60CleaningTest}
    \includegraphics[width=0.76\textwidth]{4ResearchAndDevelopments/41Fibers/Cs-137_CleaningProcess.pdf}}
    \newline
  \subfloat[Energy spectrum recorded for the Sr-90 source.]{
   \label{subfig:EnergySpectrumSr90CleaningTest}
    \includegraphics[width=0.76\textwidth]{4ResearchAndDevelopments/41Fibers/Sr-90_CleaningProcess.pdf}}
 \caption{Energy spectrums used to test the effect of the Cleaning process}
 \label{fig:ResultsOfCleaningProcess}
\end{figure}

To quantify this improvement we can make a rude approximation and we can calculate the relative displacement of the peak of the spectrum using the equation \ref{eq:RelativeImprovement}, where $A_{C}$ is the counts per second measured after the cleaning process and $A_{NC}$ is the counts per second measured before the cleaning process.

\begin{equation}
F(a,b)=\frac{A_{C}-A_{NC}}{A_{C}}
\label{eq:RelativeImprovement}
\end{equation}

The the improvement obtained is $20.74\%$ for the gamma source and $20.99\%$ for the beta source so, we have verified that we can improve the photon collection efficiency of the fibers using the cleaning process carried out in the clean room of ICMOL laboratories. Nevertheless, we have to keep in mind that this test was carried out in air and it could be interesting to repeat it in water to obtain more realistic conclusions since the fibers of the TRITIUM detector will be immersed in water.