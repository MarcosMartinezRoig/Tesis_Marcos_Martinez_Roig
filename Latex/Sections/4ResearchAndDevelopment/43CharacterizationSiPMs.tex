This section details the characterization of some of the most relevant paramenter of the SiPM model Hamamatsu S13360-1375, which was the first choice for the TRITIUM monitor photosensor. The most relevant SiPM paramenters are its breakdown voltage, $V_{BD}$, the gain of the SiPM and its dependences with the operating voltage and temperature, $G_{SiPM}(V_{bias}, T)$, and the temperature coeficient, $e$. Additional parameters were measured and used to verify the accuracy of the characterization such as the quenching resistance, $R_q$, the pixel capacitance, $C_d$, and the terminal capacitance, $C_t$. Other relevant parameters for the TRITIUM monitor are the PDE, which can affect to the minimum detectable activity, MDA, the dark count rate and the crosstalk probability, which can generate false counts interpreted as tritium counts by the TRITIUM detector. They were not measured since it was not possible with the current setup. It is expected to be measured for the S13360-6075 model, the latest proposal for the TRITIUM detector, where all the relevant parameters will be experimentaly determined using a different experimental setup, described in appendix \ref{App:ElectronicReadoutSiPM}. The afterpulse probability was not experimentally measured since, as it is explained in seccion \ref{subsubsec:SiPM}, its probability is negligible when time coincidence windows of $10~\ns$ are used.

The SiPM characterization is carried out inside of a climatic chamber, model CCM 81 from DYCOMETAL \cite{ClimaticChamberIFIMED}. This climatic chamber allows to control the temperature and humidity with a precision of $0.1\celsius$ and $0.1\%$ respectively. In addition, this chamber is a Faraday cage. A special black blanket \cite{BlackBlancket} was used to prevent external photons from reaching the SiPM.

First, the quenching resistance and the breakdown voltage of the SiPM were obtained from the measurement of the output current generated by the SiPM as a function of its bias voltage applied in forward and reverse direction, respectively. The output current of the SiPM was directly measured using the Keithley 6487 Picoammeter/Voltage Source \cite{DataSheetKeithley6487}. The LabView sofware was used to take the data. The currents-voltage curves are shown in Figure \ref{fig:IVcurveSiPM}.

\begin{figure}
\centering
    \begin{subfigure}[b]{0.9\textwidth}
    \centering
    \includegraphics[width=\textwidth]{4ResearchAndDevelopments/42SiPM/IVCurveSiPMForward.pdf}  
    \caption{\label{subfig:IVcurveForward}}
    \end{subfigure}
    \hfill
    \begin{subfigure}[b]{0.9\textwidth}
    \centering
    \includegraphics[width=\textwidth]{4ResearchAndDevelopments/42SiPM/IVCurveSiPMReverse.pdf}  
    \caption{\label{subfig:IVcurveReverse}}
    \end{subfigure}
 \caption{I-V curves measured for the SiPM model Hamamatsu S13360-1375 with the bias voltage applied in a) forward direction b) reverse direction. The measurements were taken at $T=25\celsius$ and humidity $H=45\%$.}
 \label{fig:IVcurveSiPM}
\end{figure}

As can be seen, when the bias voltage is applied in forward direction (Figure \ref{subfig:IVcurveForward}) the output current of the SiPM does not flow until the potential difference between the n and p layers is reached, which is approximately $V_0=0.7~\volt$ for silicon photosensors, close to the value experimentally obtained, $V_0= 0.5~\volt$. When the current starts to flow, the intensity is linear with the applied voltage. The equivalent resistance, $R_{eq}$, was determined from, 
\begin{equation}
I=\frac{1}{R_{eq}}V;  \qquad \frac{1}{R_{eq}} = \sum_{i=1}^{N}\frac{1}{R_{qi}}= \frac{N}{R_{q}}
\label{QuenchingResistance}
\end{equation}
and $R_{iq}$ are the quenching resistance of each pixel of the SiPM in parallel which have the same value, $R_{q}$. A value of $R_{q}= 360.56 \pm 0.07~\kilo\ohm$ was obtained from a linear fit to the data (Figure ), which is in agreement with the typical values given by Hamamatsu.

The breakdown voltage, $V_{BD}$, was obtained from the reverse bias voltage plot (Figure \ref{subfig:IVcurveReverse}). This is the point at which the SiPM begins to operate in avalanche mode, which can be calculated from the maximum of the function 
\begin{equation}
f=\frac{1}{I}\frac{dI}{dV}
\label{BreakDownVoltageFunction}
\end{equation}

The value obtained, $V_{BD}=51.02~\volt$, is in agreement with the value provided by Hamamatsu, Table \ref{tab:PropertiesOfSiPM1375}.

To measure the SiPM gain, $G_{SiPM}$, the electronic board described in section \ref{subsubsec:SiPMsElectronicalSystem} with an amplification factor of $F_{amp}=170$ was used. An incoherent light source, LED435-03 from Roithner LaserTechnik Gmbh \cite{LEDRLT}, described in section \ref{subsec:CharacterizationFibers}, was used to illuminate the SiPM with a low enough flux of $\lambda= 435~\nm$ photons. The SiPM output signal shows various well-defined pulse heights, shown in Figure \ref{fig:OutputPulses_SPSspectrum}, corresponding to the number of pixels simultaneously fired. The single photon spectrum, SPS, is plotted in Figure \ref{fig:OutputPulses_SPSspectrum}. This spectrum was obtained by integrating and hitogramming the SiPM output pulses with time window wide enough to contain the full charge of the pulse. The time window used in these measurements was $t_w= 500~\nano\second$. The light source provides a trigger signal for the measurement, represented in green line in Figure \ref{fig:OutputPulses_SPSspectrum}.

\begin{figure}[hbtp]
\centering
\includegraphics[scale=0.3]{4ResearchAndDevelopments/42SiPM/SiPMPulses_SPS_Spectrum.png}
\caption{Above) Trigger signal (green) and SiPM output pulses (yellow). Below) SPS spectrum obtained by integrating and histograming the SiPM output pulses. This measurement was done at $25\celsius$, $V_{bias}=53.98$ and humidity of $H=60\%$. \label{fig:OutputPulses_SPSspectrum}}
\end{figure}

The well-separated peaks in the SPS spectrum correspond to the charge produced by a different number of fired pixels. The first peak in the spectrum is the pedestal, which is the charge measured when no pixel is fired. This peak is caused by the electronic noise of the system. The second peak corresponds to one fired pixel and so on. The SiPM gain, $G_{SiPM}$, can be obtained from the SPS spectrum from the equation,
\begin{equation}
G=\frac{\overline{\Delta Q}(V \cdot{} s)}{F_{amp}(V/A) \times e^-(C)}
\label{SiPMGain}
\end{equation}
where $e^-$ is the electron charge and $\overline{\Delta Q}$ is the average peak distance in the SPS spectrum, corresponding to the charge released by a fired pixel. 

To obtain the value of $\overline{\Delta Q}$ a macro was written in ROOT \cite{ROOTWebPage}. This macro finds and extract the bakcground (the output signals of the SiPM different to the pedestal when it is not illuminated by a LED), which is crucial in some cases like high temperatures or high bias voltages since it can hide its peaks. After that, this macro find all peaks in the SPS spectrum and fits each one to a Gaussian funtion, shown in Figure \ref{subfig:GaussianFitSiPMs}. The value and error of the charge produced by multiple fired pixels are obtained from the centroid and the sigma of the different fitted Gaussian functions. The obtained charges are fitted to the number of fired pixels, Figure \ref{subfig:LinearFitSiPMGain}.

\begin{figure}
\centering
    \begin{subfigure}[b]{0.9\textwidth}
    \centering
    \includegraphics[width=\textwidth]{4ResearchAndDevelopments/42SiPM/GaussianFitSPSSpectrum.pdf}  
    \caption{\label{subfig:GaussianFitSiPMs}}
    \end{subfigure}
    \hfill
    \begin{subfigure}[b]{0.9\textwidth}
    \centering
    \includegraphics[width=\textwidth]{4ResearchAndDevelopments/42SiPM/LinearFit_Gain_NPixels.pdf}  
    \caption{\label{subfig:LinearFitSiPMGain}}
    \end{subfigure}
 \caption{ROOT analysis performed to obtain the SiPM gain. a) Fit of the SPS spectrum to various Gaussian functions. b) Charge of succesive number of pixels as a function of the number of pixels fired. Error bars are within point size. This experience was carried out at $T=25\celsius$, $V_{bias}=53.98~\volt$ and humidity of $H=45\%$.}
 \label{fig:ROOTAnalysisSiPMGain}
\end{figure}

%As can bee seen in Figure \ref{subfig:GaussianFitSiPMs}, a very good fit is achieved by the ROOT script with a $\chi^2$ test of $\frac{\chi^2}{ndf}=\frac{1276}{223}$. 

Up to 10 simultaneously fired pixels were obtained with a relative uncertainty of the charge measurement of less than $2\%$. The slope of the straight line in Figure \ref{subfig:LinearFitSiPMGain} corresponds to $\overline{\Delta Q}$.

For the case studied, which corresponds to a temperature of $25~\degree$ and a bias voltage of $53.96~\volt$ (overvoltage around $3~\volt$), the value obtained for the SiPM gain is $G_{SiPM}=(4.11 \pm 0.04) \cdot{} 10^{6}$, very close to the value provided by Hamamatsu, Table \ref{tab:PropertiesOfSiPM1375}.

A method for the SiPM gain stabilization against variations due to the temperature was implemented. This is necessary for the TRITIUM project since the temperature in the final location of the tritium detector cannot be controlled with the precision required to avoid variations of the SiPM gain. This method consists in compensating for variations in the SiPM gain, caused by variations of temperature, by controlled variations of the bias voltage. For this task, first, the dependence of the SiPM gain with the temperature and bias voltage was measured. The SiPM gain was measured at several temperatures from $15\celsius$ to $41\celsius$ in steps of $2\celsius$, which is expected to be the temperature range in the final location. The bias voltage was $V_{bias} = V_{BD}+3$. The SiPM gain was measured at several overvoltages from $1~\volt$ to $5~\volt$ in steps of $0.2~\volt$. The temperature was $T=25\celsius$. Both measurements are shown in Figure \ref{fig:SiPMGainDependance}. 

\begin{figure}
\centering
    \begin{subfigure}[b]{0.9\textwidth}
    \centering
    \includegraphics[width=\textwidth]{4ResearchAndDevelopments/42SiPM/SiPMGain_vs_Temperature.pdf}  
    \caption{\label{subfig:SiPMGainvsTemperature}}
    \end{subfigure}
    \hfill
    \begin{subfigure}[b]{0.9\textwidth}
    \centering
    \includegraphics[width=\textwidth]{4ResearchAndDevelopments/42SiPM/SiPMGain_vs_Bias_Voltage.pdf}  
    \caption{\label{subfig:SiPMGainvsBiasVoltage}}
    \end{subfigure}
 \caption{Dependence of the SiPM gain with the a) Temperature b) Bias voltage.}
 \label{fig:SiPMGainDependance}
\end{figure}

As can be seen, an excellent linear trend is obtained for both cases. The  parameters of the linear fit obtained are,
\begin{equation*}
\begin{split}
G_{SiPM}=a \cdot{} T + b;& \qquad G_{SiPM}=c \cdot{} V_{bias} + d\\
a=\left( -82.53 \pm 1.59 \right) \cdot{} 10^{3};& \qquad c=\left( 137.72 \pm 1.50 \right) \cdot{} 10^{4}\\
b=\left( 617.65 \pm 4.53 \right) \cdot{} 10^{4};& \qquad d=\left( -762.16 \pm 8.13 \right) \cdot{} 10^{5} \\
\label{SiPMGainVSTempV}
\end{split}
\end{equation*} 

In addition, the breakdown voltage, $V_{BD}$, and the terminal capacitance, $C_t$, can be obtained from the linear fit of the SiPM gain as a function of the bias voltage, $V_{bias}$. Both parameters can be obtained from the definition of the SiPM gain and taking into account that the charge produced in a pixel is proportional to the capacitance of the pixel and the difference voltage in the SiPM, $V_{OV}$,
\begin{equation}
G_{SiPM}=\frac{Q_{pixel}}{e^-} = C_d \frac{V_{bias}-V_{BD}}{e^-} = c \cdot{} V_{bias}+d
\label{SiPMGain_Capacitance}
\end{equation}
where $C_d$ is the pixel capacitance.

From the linear fit obtained in Figure \ref{subfig:SiPMGainvsBiasVoltage}, a value of $V_{BD}=50.98 \pm 0.59~\volt$ and $C_d{}= 220.63 \pm 2.41~\text{f}\farad$ are obtained. The terminal capacitance of the SiPM can be calculated assuming all pixels in parallel, $C_{t}=N_{p}\times C_{d}=62.88 \pm 0.69~\pico\farad$. Both magnitudes, the breakdown voltage and the terminal capacitance, are in agreement with the values provided by Hamamatsu, Table \ref{tab:PropertiesOfSiPM1375}. 

Finally, the value of the bias voltage to be applied to compensate for the variation in the SiPM gain due to a variation of the temperature can be obtained by applying variations to linear relations:
\begin{equation*}
\begin{split}
G_{SiPM}=a \cdot{} T + b  &\longrightarrow \partial G_{SiPM}= a \partial T\\
G_{SiPM}=c \cdot{} V_{bias} + d &\longrightarrow \partial G_{SiPM}= c \partial V_{bias}
\label{Gain_compensationVariations}
\end{split}
\end{equation*} 
Therefore, the total variation of the SiPM gain, which is produced by the variation of both parameters, must be cancel:
\begin{equation*}
\begin{split}
\partial G_{SiPM, tot}= \partial G_{SiPM}(T) &+ \partial G_{SiPM}(V_{bias}) = 0\\ 
\partial G_{SiPM}(V_{bias}) = -\partial G_{SiPM}(T) &\longrightarrow c \partial V_{bias} = - a \partial T\\ 
\partial V_{bias}  = - \frac{a}{c}&\partial T = e \partial T
\label{Gain_compensation0}
\end{split}
\end{equation*} 
where the parámeter $e= 59.93 \pm 1.33~\milli\volt/\celsius $ is the ratio of $a$ and $c$ and agrees with the value of the temperature coefficient provided by Hamamatsu, Table \ref{tab:PropertiesOfSiPM1375}. Finally, integrating this expression, we obtain:
\begin{equation}
\begin{split}
\int_{V_i}^{V_f}\partial V_{bias}  = e\int_{T_i}^{T_f}\partial T \longrightarrow \Delta V_{bias} = e \Delta T
\label{Gain_compensationIntegring}
\end{split}
\end{equation} 
This equation gives the variation of the voltage $\Delta V_{bias}$ that keeps the SiPM gain when a variation in the temperature happens, $\Delta T$. More useful is to know the bias voltage $V_{bias}$ to be applied as a function of the temperature $T$. For this, it is necessary a reference case. In this case, the reference case considered is $V_i=V_{ref}= V_{BD}+3~\volt = 53.98~\volt$ and $T_i=T_{ref}=24\celsius$, at which the gain is $4.2 \cdot{} 10^{6}$ (experimentally measured). Thus, we get:
\begin{equation*}
\begin{split}
(V_{bias}-V_{ref} )= e \left( T -T_{ref} \right) 
\label{Gain_compensationEquation}
\end{split}
\end{equation*}
\begin{equation}
V_{bias}(\volt)= 59.9 \cdot{} 10^{-3} \cdot{} T(\celsius) + 52.54
\label{Gain_compensationReference}
\end{equation}  
Finally, this temperature through bias voltage compensation was tested. The temperature was varied from $21\celsius$ to $29\celsius$ and the bias voltage was modified according to the equation \ref{Gain_compensationReference}. The value of the SiPM gain obtained as a function of the temperature is shown in Figure \ref{fig:SiPMGainStabilization}.

\begin{figure}[hbtp]
\centering
\includegraphics[scale=0.75]{4ResearchAndDevelopments/42SiPM/SiPMGain_Stabilization_linear_fit.pdf}
\caption{SiPM gain measured as a function of the temperature after implementation of the gain stabilization method. \label{fig:SiPMGainStabilization}}
\end{figure}

A red dotted line is included, indicating the value of the SiPM gain to be kept. As it can be seen, the slope of the linear fit, parameter p1, is three orders less than the constant, parameter p0, so it can be depreciated and a constant dependence can be accepted. Furthermore, all experimentally measured points are in agreement with the initially value measured for the SiPM gain (red line). Therefore it can be concluded that this method works to stabilize the gain of the SiPM when variations in the temperature happens.