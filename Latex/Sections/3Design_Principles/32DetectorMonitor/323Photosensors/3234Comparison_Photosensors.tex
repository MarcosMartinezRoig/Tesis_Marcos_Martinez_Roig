The photosensors employed in TRITIUM are both, PMT and SiPM. Each kind of photosensor has its advantages and disadvantages, so both were tested to decide the most suitable. The output signal of both photosensors is proportional to the number of incident photons in our range of luminosity and they have a similar gain (of the order of $10^6$). Both properties are essential to detect tritium events and to obtain a large enough signal to be measured and recorded. Both photosensors have fast output signals, with a rise time of the order of nanoseconds, and a wide spectral sensitivity ($200-800~\nano\second$ for PMT and $300-900~\nano\second$ for SiPM). The supply voltage necessary to work with SiPM, of the order of tens of volts, is much lower than that of PMTs, which require a high voltage, of the order of a thousand volts. The electron detection efficiency at $420~\nano\meter$,  achieved with SiPM is higher, PDE around $50\%$, than with PMT, which have a QE about $30\%$. A large efficiency is essential because the number of photons produced in a tritium event is rather low. Furthermore PMTs, as they consist of a vacuum tube, are more bulky and fragile than SiPMs, which are compact and robust. This is an advantage for the SiPMs because the TRITIUM detector should work during years. Furthermore, PMTs are rather more expensive, than SiPMs. In addition, PMTs are affected by magnetic fields, contrary to SiPMs that work correctly in magnetic field up to 7 Tesla. Moreover, due to their high uniformity, SiPMs are capable of measuring the exact number of photoelectrons detected and even of resolving a single photoelectron, which is not possible with PMTs due to their gain uncertainty.

However, the dark current of PMTs is much lower (a few counts per second) than that of SiPMs, that have a dark current between 0.1 and 1 Mcps\footnote{Mega counts per second, $10^6$c/$\second$}, depending on their size, and this happens with SiPMs almost entirely at the level of a single photoelectron. This prevents to separate tritium decay signals from background in the singel photon-detection zone. Another inconvenient of SiPMs is their large crosstalk and afterpulses that need to be corrected. An additional drawback of SiPMs is that their response depends strongly on temperature. As the TRITIUM detector will be installed in an environment with significant temperature variations, this problem is solved by developing a stabilization method of the SiPM gain.