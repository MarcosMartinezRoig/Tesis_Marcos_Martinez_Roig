The TRITIUM-IFIC 1 prototype was designed to overcome the problems and limitations found in TRITIUM-IFIC 0. The main improvements were:

\begin{enumerate}

\item{} The fiber bundle was arranged straight to optimize the photon collection efficiency of the fibers.

\item{} A special fiber cleaning method, described in section \ref{subsec:SurfaceConditioningProcess}, was applied to the fibers to improve the interfaces between fiber and tritiated water. This method produces a better wetting property of the fiber, which improves the photon collection efficiency of the scintillating fibers.

\item{} A Teflon vessel was used in the Tritium prototypes to improve the photon collection in the prototype. Teflon is a convenient material for its optical properties, specifically its reflection coefficient, which is very close to $100\%$ at the working wavelength. This means that the photons that escape from fibers will hit the Teflon walls and go back to the scintillating fiber bundle.

\end{enumerate}

The TRITIUM-IFIC 1 prototype consists of 64 straight scintillating fibers of $20~\cm$ length, arranged in an $8\times 8$ Teflon squared matrix, as shown in Figure \ref{fig:TeflonStructureFibersTritiumIFIC1}.

\begin{figure}[h]
\centering
\includegraphics[scale=0.4]{5Prototypes/52PreliminarPrototypes/522TritiumIFIC1/FiberMatrixTeflonStructure.png}
\caption{Teflon structure used to arrange the fibers of TRITIUM-IFIC 1 prototype in a matrix of $8\cdot{}8$.\label{fig:TeflonStructureFibersTritiumIFIC1}}
\end{figure}
This structure is placed within a cylindrical Teflon vessel of $48~\mm$ diamenter and $200~\mm$ length, shown in Figure \ref{fig:TeflonVesselTritumIFIC1}. 

\begin{figure}
\centering
    \begin{subfigure}[b]{0.30\textwidth}
    \centering
    \includegraphics[width=\textwidth]{5Prototypes/52PreliminarPrototypes/522TritiumIFIC1/TeflonVesselTritiumIFIC1a.png}  
    \caption{\label{subfig:TeflonVesselTritumIFIC1a}}
    \end{subfigure}
    \hfill
    \begin{subfigure}[b]{0.45\textwidth}
    \centering
    \includegraphics[width=\textwidth]{5Prototypes/52PreliminarPrototypes/522TritiumIFIC1/TeflonVesselTritiumIFIC1b.png}  
    \caption{\label{subfig:TeflonVesselTritumIFIC1b}}
    \end{subfigure}
 \caption{Teflon vessel of TRITIUM-IFIC 1 prototype.}
 \label{fig:TeflonVesselTritumIFIC1}
\end{figure}

The cleaning process described in section \ref{subsec:SurfaceConditioningProcess} was applied to the fibers to achieve a better tritiated water-fiber interface. A PVC piece was used to attach the photosensor to the prototype and prevent external light from being read by photosensors. A general view of this prototype is shown in Figure \ref{fig:TritumIFIC1}.

\begin{figure}
\centering
    \begin{subfigure}[b]{0.40\textwidth}
    \centering
    \includegraphics[width=\textwidth]{5Prototypes/52PreliminarPrototypes/522TritiumIFIC1/TritiumIFIC1a.png}  
    \caption{\label{subfig:TritumIFIC1a}}
    \end{subfigure}
    \hfill
    \begin{subfigure}[b]{0.40\textwidth}
    \centering
    \includegraphics[width=\textwidth]{5Prototypes/52PreliminarPrototypes/522TritiumIFIC1/TritiumIFIC1b.png}  
    \caption{\label{subfig:TritumIFIC1b}}
    \end{subfigure}
 \caption{A general view of TRITIUM-IFIC 1 prototype.}
 \label{fig:TritumIFIC1}
\end{figure}

The prototype was read by a PMT R8520-460, from Hamamatsu Photonics company \cite{DataSheetPMTs} coupled directly to the fiber bundle using optical grease \cite{OpticalGrease}. The electronic circuit, shown in Figure \ref{fig:VoltageDividerCircuit}, was used to distribute the high voltage between the dynodes. The high voltage was $-800~\volt$, which corresponds to a quantum efficiency of $28.66\%$. The signal from this PMT was acquired by the same electronics used for the TRITIUM-IFIC 0 prototype, shown in Figure \ref{subfig:ElectronicConfiguraiton1PMT}. Unlike the first prototype, only one TRITIUM-IFIC 1 prototype was built. In a first measurement, this prototype was filled with pure water ($118~\milli\liter$, uncertainty of $0.05\%$) and several background measurements were taken over a week. Then, it was emptied and refilled with $118~\milli\liter$ (uncertainty of $0.05\%$) of a tritiated water source of the same specific activity as the one used for TRITIUM-IFIC 0 prototype, $99.696~\kilo\becquerel/\liter$. 

The measured signal and background energy spectra are shown in Figure \ref{subfig:SignalBackgroundEnergySpectraTritiumIFIC1}. The difference between both energy spectra corresponds to the tritium energy spectrum, Figura \ref{subfig:TritiumEnergySpectraTritiumIFIC1}. The detection efficiency was obtained as in the previous section. The rate measured are given in Table \ref{tab:CountsPerSecondTRITIUMIFIC1}, where the tritium counts are obtained from the substraction of signal and background spectra.

\begin{figure}
\centering
    \begin{subfigure}[b]{1\textwidth}
    \centering
    \includegraphics[width=\textwidth]{7ExperimentalResultsDetectors/71ExperimentalResultsLaboratory/712TRITIUMIFIC1/TritiumIFIC1Signals.pdf}  
    \caption{\label{subfig:SignalBackgroundEnergySpectraTritiumIFIC1}}
    \end{subfigure}
    \hfill
    \begin{subfigure}[b]{1\textwidth}
    \centering
    \includegraphics[width=\textwidth]{7ExperimentalResultsDetectors/71ExperimentalResultsLaboratory/712TRITIUMIFIC1/TritiumIFIC1Clear.pdf}  
    \caption{\label{subfig:TritiumEnergySpectraTritiumIFIC1}}
    \end{subfigure}
 \caption{Energy spectra measured with TRITIUM-IFIC 1 prototype. a) Signal and background energy spectra. b) Tritium energy spectrum.}
 \label{fig:EnergySpectraTRITIUMIFIC1}
\end{figure}

\begin{table}[htbp]
\centering{}%
\begin{tabular}{cc}
\toprule 
Spectrum & Counts/second \tabularnewline
\midrule
\midrule 
Signal prototype & $7.82 \pm 0.11$ \tabularnewline
Background prototype & $3.99 \pm 0.08$ \tabularnewline  
Tritium counts & $3.83 \pm 0.13$ \tabularnewline
\bottomrule
\end{tabular}
\caption{Counting rate obtained with the TRITIUM-IFIC 1 prototype.}
\label{tab:CountsPerSecondTRITIUMIFIC1}
\end{table}

The tritium detection efficiency obtained for TRITIUM-IFIC 1 is $(3.84 \pm 0.16)\cdot{} 10^{-2}~\liter\second^{-1}\kilo\becquerel^{-1}$. This efficiency i7s larger than that of TRITIUM-IFIC 0, as expected since this prototype has a larger active area. The specific efficiency obtained is
$$S=(9.56 \pm 0.40)\cdot{} 10^{-5}~\liter\second^{-1}\kilo\becquerel^{-1}\cm^{-2}$$
which is a factor ten better than that of TRITIUM-IFIC 0. Furthermore, compared to the scintillating detectors developed in other experiments, given in table \ref{tab:PlasticScinTritium}, the efficiency of this prototype is very close to the best result, obtained by Singh, and the specific efficiency, which is the most relevant parameter to compare, is almost 5 times larger than that obtained by Hofstetter \cite{Hofstetter1, Hofstetter2}.