This thesis is divided into eight chapters structured as follows:

\begin{enumerate}
\item{} \textbf{Chapter 1} provides a brief introduction to tritium detection, reports some important properties of tritium, and discusses the current legislation that limits tritium levels for human consumption in many countries around the world. 

\item{} \textbf{Chapter 2} describes the state-of-the-art of tritium detection and introduces the TRITIUM project. 

\item{} \textbf{Chapter 3} outlines the different parts of the TRITIUM monitor, which are the water purification system, the background rejection system (consisting of a lead shield and a active veto)  and the tritium detector. 

\item{} \textbf{Chapter 4} reports the calibrations of the different parts of the TRITIUM monitor and describes the developments aimed at improving the efficiency of tritium detection. 

\item{} \textbf{Chapter 5} details the geometrical configuration of the different prototypes built in the TRITIUM project and the measurements taken with them. 

\item{} \textbf{Chapter 6} details the Monte Carlo simulations performed in the TRITIUM project and shows the results obtained. 

\item{} \textbf{Chapter 7} summarizes and discusses the most important results achieved by the TRITIUM collaboration.

\item{} \textbf{Chapter 8} Summarize the main results of this PhD work and discusses the future of the TRITIUM project.

\end{enumerate}

