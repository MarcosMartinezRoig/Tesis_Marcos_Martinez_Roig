Tritium, which is a radiaoctive element, is one of the most abundantly produced radioisotope in nuclear facilities such as nuclear power plants or research facilities. Due to its radiotoxicity, an excessive amount of tritium released to the environment could directly or indirectly affect human health since it could produce DNA mutations, tumors, cancer, etc. Because of this risk, there exist a limitation on the amount of tritium released to the environment, which depends on how this tritium is emitted (gaseous tritium, tritiated water, organically bound tritium, etc) and the country in which it happens.

The legal limit of tritium for drinking water in Europe is $100~\becquerel/\liter$, which is one of the most restrictive limits in the world. This limit is established by the EURATOM Council Directive. Nowadays, tritium in water is mainly measured using the liquid scintillation counting technique. This technique has a very good detection sensitivity, being able to measure tritium activities as low as $1~\becquerel/\liter$. The drawback of this technique is that it takes more than 2 days to perform the measurement. It is also not reusable and contains toluene, a toxic chemical element. Promising advantages has been obtained in the line of measurements in quasi-real time (measurements taken in short times, less than $1~\hour$) using plastic scintillators but without achieving the low sensitivity required to measure low-levels of tritium in water of the order of the established limit in Europe.

The TRITIUM project was proposed to overcome the limitations found in the current techniques used to monitor tritium activities in quasi-real time. The goal of the TRITIUM project is to design, build, install and commission a tritium monitor that will measure tritium activities as low as $100~\becquerel/\liter$  in quasi-real time (1 hour or less). The TRITIUM monitor developed in the TRITIUM project consists of three different parts.

\begin{enumerate}

\item{} The TRITIUM module, where the tritium measurement takes place. Different TRITIUM modules were developed, which are based on scintillating fibers read out by photosensors (PMTs or SiPM arrays).

\item{} The background rejection system, which is used to reduce the radioactive background that affects the TRITIUM module and, therefore, its minimum detectable activity. It is based on a passive shield, used to reduce the soft component of this background (energies below than $200~\MeV/$nucleon), and the active veto, used to reduce the hard component of this background (energies above than $200~\MeV/$nucleon, mainly hard cosmic ray events).

\item{} The water purification system, which is used to prepare the water sample that will be introduced into the TRITIUM module to be measured. This preparation consits in removing all the particles dissolved in the water sample, including radiaoctive isotopes, without affecting to the tritium activity. The purification achieved with this system was measured, obtaining a conductivity of about $10~\mu\text{S}/\cm$, which is a high enough level for the requirements of the TRITIUM monitor, and it was experimentally verified that the activity of tritium is not altered by the purification process.

\end{enumerate} 
The main results obtained in this PhD work are:
\begin{enumerate}
\item{} A characterization of the scintillating fibers used in the TRITIUM modules, BCF-12 of Saint-Gobain, was carried out, in which the photon collection efficiency was measured, obtaining a factor of $76 \pm 8 \%$ for uncladded scintillating fibers of $1~\mm$ diameter. In addition a surface-conditioning method was developed and tested, which consists in cleaving, polishing and cleaning the fibers following specific rules. It was obtained an improve of the photon collection efficiency of the fiber of a factor $2$ due to polishing and $25\%$ due to cleaning.

\item{} The first SiPM model proposed for the TRITIUM monitor, model S13360-6075 from Hammamatsu, was also characterized, measuring some of the most relevant parameters of these photosensors for the tritium measurement such as their breakdown voltage and their gain. Furthermore, a stabilization method was implemented to mantain the SiPM gain when the temperature changes, which was tested in the temperature range of interest, $[20-30]\celsius$. In this test, variations of the order of $0.1\%$  were measured for the gain of the SiPM, with which a stable behaviour of the photosensors can be achieved.

\item{} An active veto was built, which is based on plastic scintillators read out by photosensors. A characterization of it was carried out, in which the optimal parameters for the detection of hard cosmic events were found. A count rate of $2.5~\text{events}/\second$ was experimentally measured, which gives an efficiency for the hard cosmic detection of $85\%$ for the active veto built.

\item{} Four different prototypes were developed for the TRITIUM module in which different improvements were applied, obtaining an increasing efficiency of the tritium detection. The last two TRITIUM prototypes, TRITIUM-Aveiro and TRITIUM-IFIC-2 have a similar design but with subtle difference such as a different diameter of the scintillating fibers.

Better specific efficiencies ($\eta_{det} = 141.45\pm5.52~\frac{cps \cdot{} 10^{-6}}{\kilo\becquerel/ \liter \cdot{} \cm^2}$) and MDA ($0.68 /0.22~\kilo\becquerel / \liter$ for an integration time of $10~\min$ and $1~\hour$ respectively) was measured with the last prototype,  TRITIUM-IFIC-2, compared to results obtained with prototypes developed in other experiments so far. It means that the state-of-the-art in tritium detection in quasi-real time has been substantially improved with it. 

One of the most relevant properties of the TRITIUM monitor is that it is scalable, which means that better results can be achieved by using a larger number of modules. The goal of the TRITIUM project (to be able to measure $100~\becquerel/\liter$ in quasi-real time) is expected to be reached using $5$ TRITIUM-IFIC-2 prototypes read out in parallel and an integration time of $1~\hour$.

\item{} Several simulations were carried out to study how the uncertainty in the tritium measurement can be reduced when a different integration time or different number of modules are used. It was found that a difference of $250~\kilo\becquerel/\liter$ in the tritium activity can be distinguised by using an integration time of $30~\min$ and three TRITIUM-IFIC-2 modules read out in parallel. Other possibility would be to use five TRITIUM-IFIC-2 and an integration time of $10~\min$ with which a difference of $200~\kilo\becquerel/\liter$ in the tritium activity would be distinguised and the MDA of $100~\becquerel/\liter$ achieved.

\item{} The background rejection system was also simulated, where it was found that a reduction of a factor $5.5$ of the cosmic ray events that affect TRITIUM-IFIC-2 prototype was obtained due to the passive shield (mainly soft cosmic ray events) and a reduction of $60\%$ of the cosmic ray events that cross the lead shield and affect the TRITIUM-IFIC-2 prototype (mainly hard cosmic ray events) was obtained due to the active veto. In summary, a reduction of $92.5\%$ of the cosmic events that affects the TRITIUM-IFIC-2 prototype was obtained through simulations.

\end{enumerate}

Currently, the water purification system and the lead shield are installed in Arrocampo dam, the final location of the TRITIUM monitor. A TRITIUM-Aveiro module is also installed, which has been monitoring the tritium activities for several months. The next step for the TRITIUM project will be to install 2 additional TRITIUM-Aveiro prototypes which will be read out in parallel with the one already there. In addition, an active veto will be installed, which will be read in anticoincidence with this three prototypes. Three TRITIUM-IFIC-2 prototypes along with an active veto are prepared to be also installed at Arrocampo as soon as possible.

Furthermore, it is necessary to quantify the improvement achieved in tritium detection when the SiPM arrays are used as photosensors. Therefore, a similar characterization to the one performed for the TRITIUM-IFIC-2 prototype must be obtained for an identical prototype in which the PMTs are remplaced by SiPM arrays.

The TRITIUM monitor has been developed with the aim of being able to measure the legal limit established by the EURATOM Concil Directive ($100~\becquerel/\liter$) but it has to be taken into account that it can be used in other different areas such us to control the correct operation of a nuclear facility (high levels of tritium in the water released by the nuclear power plant is one of the first sign of an anomalous functioning of the plant) of even other fields, different to the environmental surveillance, such as research facilities.