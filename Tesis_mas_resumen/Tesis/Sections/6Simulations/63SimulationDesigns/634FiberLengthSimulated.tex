A study to find the optimal fibre length was carried out. Two different lengths of scintillating fibres, $1~\meter$ and $20~\cm$, and two different tritium source activities, $0.5~\kilo\becquerel/\liter$ and $2.5~\kilo\becquerel/\liter$, were considered. Five detectors were simulated for the case of $20~\cm$ long fibres to have the same active area as one detector of $1~\meter$ long fibres. The advantage of using long fibres is their large active area for the same number of photosensors which would reduce the price of the TRITIUM monitor. However, short scintillating fibres reduce photon absorption, which increases the tritium detection efficiency per unit of active area. The Tritium-Aveiro prototype, consisting of $360$ scintillating fibres of $2~\mm$ diameter, was simulated. The number of photons produced in a scintillating fibre per tritium electron for all the electrons that reach the scintillating fibres and for only those that generate photons detected in coincidence by the photosensors is plotted in Figure \ref{fig:PhotonsFibersYesNoPhotosensors}. Tritium events that produce a large number of photons are almost always detected but events that produce few photons are seldom detected, resulting in a peak centred at around $25$ photons.  
\begin{figure}[h]
\centering
\includegraphics[scale=0.3]{6Simulations/61TRITIUMDesign/613Length/CollectionPhotonsInFibers.png}
\caption{Number of photons per tritium event produced in a fibre for all tritium electrons that reach the fibre (blue histogram) and for only tritium electrons producing photons detected in coincidence by photosensors (red histogram) \cite{SimulationPaperCarlos}.\label{fig:PhotonsFibersYesNoPhotosensors}}
\end{figure}
The number of counts per hour during a week as a function of time for the tritium activities and fibre lengths studied is shown in Figure \ref{fig:CountsOver60minDifferentLength}. A signal $20\%$ larger is seen for the shorter fibre length and the two activities considered, due mainly to the lower absorption and the refraction loss of photons in shorter fibres. In addition, not simulated effects like dirt and mechanical imperfections of fibre surface increase photon loss.

\begin{figure}[h]
\centering
\includegraphics[scale=0.3]{6Simulations/61TRITIUMDesign/613Length/2DifferentLength.png}
\caption{Simulated counting statistics normalized to the active area in $1~\hour$ bins during a week for $1~\meter$ (dashed lines) and $20~\cm$ (solid lines) and $0.5~\kilo\becquerel/\liter$ (blue lines) and $2.5~\kilo\becquerel/\liter$ (red lines) activities \cite{SimulationPaperCarlos}. \label{fig:CountsOver60minDifferentLength}}
\end{figure}

