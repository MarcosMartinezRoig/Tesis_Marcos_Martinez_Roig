First of all, the shape of the simulated tritated water source was optimized. The mean free path of tritium electrons in water is only around $5~\mu\meter$, so most electrons do not reach the scintillating fibers. These electrons do not provide useful information and only consume computing resources. To optimize the simulation, the dimensions of the simulated tritium source were set to minimize the number of tritium events that do not reach the scintillating fibers. 

Before that, the energy distribution of the simulated tritium events, shown in figure \ref{subfig:EnergyDistributionTritiumSource}, was compared with the input taken from ref. \cite{TritiumEmissionSpectrum}, obtaining a good agreement with it. In addition, the distribution of the initial energy of tritium electrons capable of penetrating a fiber and depositing energy was compared to the initial energy distribution of all simulated tritium events, Figure \ref{subfig:EnergySpectrumEventsDetectedandNonDetected}. A shift of the peak to high energies is observed, obtaining a peak centred at $10~\keV$. This shift occurs because the lower energy tritium decay electrons do not reach and penetrate the fibers and are not detected.

\begin{figure}
\centering
    \begin{subfigure}[b]{0.45\textwidth}
    \centering
    \includegraphics[width=\textwidth]{6Simulations/61TRITIUMDesign/611TritiumSourceOptimization/TritiumSourceEnergyDistribution.png}  
    \caption{\label{subfig:EnergyDistributionTritiumSource}}
    \end{subfigure}
    \hfill
    \begin{subfigure}[b]{0.45\textwidth}
    \centering
    \includegraphics[width=\textwidth]{6Simulations/61TRITIUMDesign/611TritiumSourceOptimization/Source_Spectrum_yes_and_non_detected_events.png}  
    \caption{\label{subfig:EnergySpectrumEventsDetectedandNonDetected}}
    \end{subfigure}
 \caption{Energy distribution of a) simulated tritium decays b) Initial energy of tritium decays that reach the scintillating fibers (red histogram) compared to all simulated tritium events (blue histogram) \cite{SimulationPaperCarlos}.
 \label{fig:TritiumSourceOptimization}}
\end{figure}

Regarding the optimization of the tritium source shape, a scintillating fiber $20~\cm$ long and $2~\mm$ in diameter and a surrounding tritiated water source of the same length and $0.5~\mm$ thick($100$ times greater that the mean free path of tritium electrons) were simulated to assess the tritium source. The dimensions of the fiber are not important in this study since only the energy deposition of tritium electrons in the fiber was simulated, excluding optical processes. The goal of this simulation was to find the radial thickness of the simulated tritium source beyond which no significant amount of tritium decay electrons are detected. In Figure \ref{subfig:TransversalCutTritiumSource}, a transversal cut of the $2~\mm$ scintillating fiber, the simulated $0.5~\mm$ thick tritium source around the fiber, and the position where happen the tritium decays that deposit energy in the scintillating fiber are shown. Furthermore, the distribution of the radial distance between the position where tritium decays take place and the surface of the scintillating fiber is shown in figure \ref{subfig:DistanceDistributionTritiumSourceFiber}. As can be seen in the Figure \ref{fig:TritiumSourceSimulated}, most of the tritium decays that are detected occur close to the scintillating fiber.  A zoom of low energy events is shown in the inset box of the Figure \ref{subfig:DistanceDistributionTritiumSourceFiber} for better viewing. The chosen thickness of the simulated tritium source was $5~\mu\meter$ since $99.4\%$ of the events that deposit energy in the fibers are produced at a shorter distance.

\begin{figure}
\centering
    \begin{subfigure}[b]{0.45\textwidth}
    \centering
    \includegraphics[width=\textwidth]{6Simulations/61TRITIUMDesign/611TritiumSourceOptimization/Source_Ring.png}  
    \caption{\label{subfig:TransversalCutTritiumSource}}
    \end{subfigure}
    \hfill
    \begin{subfigure}[b]{0.45\textwidth}
    \centering
    \includegraphics[width=\textwidth]{6Simulations/61TRITIUMDesign/611TritiumSourceOptimization/SourceDistance.png}  
    \caption{\label{subfig:DistanceDistributionTritiumSourceFiber}}
    \end{subfigure}
 \caption{a)Transversal cut of simulated scintillating fiber (yellow) and tritium source (green) with various tritium decays (red dots) b) Distribution of the radial distance between the position where the tritium decay takes place and the surface of the scintillating fiber \cite{SimulationPaperCarlos}.}
 \label{fig:TritiumSourceSimulated}
\end{figure}	

