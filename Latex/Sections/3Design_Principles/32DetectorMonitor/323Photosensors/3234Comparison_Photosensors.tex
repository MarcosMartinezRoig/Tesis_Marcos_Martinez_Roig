The photosensors employed in TRITIUM are both, PMT and SiPM. Each kind of photosensor has his advantages and disadvantages, so both were tested to make a final choice. The output signal of both photosensors is proportional to the number of incident photons and they have a similar gain (of the order of $10^6$). Both properties are essential to detect tritium events and to obtain a signal large enough to be measured and processed. Both photosensors have fast output signals, with a rise time shorter than nanoseconds, and a wide spectral sensitivity ($200-800~\nano\second$ for PMT and $300-900~\nano\second$ for SiPM). The supply voltage necessary to work with SiPM, on the order of tens of volts, is much lower than that of PMTs, which require a high voltage, of the order of a thousand volts. The PDE at $420~\nano\meter$,  achieved with SiPM is higher, around $50\%$, than with PMT, which have a PDE about $30\%$. A large PDE is essential because the number of photons produced in a tritium event is very low. Furthermore PMTs, as they consist of a vacuum tube, are more bulky and fragile than SiPMs, which are compact and robust. This is an advantage for the SiPMs because TRITIUM detector should work during years. Furthermore, PMTs are rather more expensive, than SiPMs. In addition, PMTs are affected by magnetic fields, contrary to SiPMs taht works correctly in intensities  of magnetic field up to 7 Tesla. Moreover, due to their high uniformity, SiPMs are capable of distinguishing the exact number of photoelectrons detected and even of resolving a single photoelectron, which is not possible with PMTs due to variations in their gain.

On the other hand, the dark current of PMTs is much lower (a few counts per second) than that of SiPMs, that have a dark current between 0.1 and 1 Mcps\footnote{Mega counts per second, $10^6$c/$\second$}, depending on they size, and this happens almost entirely at the level of a single photoelectron. This prevents to separate tritium decay signals from background in the signel photon-detection zone. Another inconvenient of SiPMs is large crosstalk and afterpulses that need to be corrected.

An additional drawback of SiPMs is that their output signal depends strangly with the temperature. As TRITIUM detector will be installed in an environment with significant temperature variations, this problem is solved by a suitably changing the supply voltage to compensate temperature variations.