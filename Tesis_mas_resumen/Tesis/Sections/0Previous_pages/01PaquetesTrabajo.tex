\usepackage [spanish, catalan, british]{babel}
\usepackage [utf8]{inputenc}

%\pagestyle{headings} %%Style of the page -> to set the headers and fooders
%\thispagestyle{empty} %%This applies the pagestyle only to the current page.

\usepackage{fancyhdr} %%To modify the pagestyle

\pagestyle{fancy}
\fancyhf{}
%\fancyhead[LE,RO]{\thepage}
\fancyhead[RO]{\leftmark}
\fancyhead[LE]{\rightmark}
\renewcommand{\headrulewidth}{1pt}
%\renewcommand{\footrulewidth}{1pt}
%\fancyfoot[RE,LO]{Guides and tutorials}
\fancyfoot[CE,CO]{\thepage}



\usepackage{booktabs} %%To use Tables in format style
\usepackage{eurosym} %% To use the symbol of the euro

%\usepackage[left=2.5cm,right=2.5cm,top=2.5cm,bottom=2.5cm]{geometry}

\usepackage{longtable}
 
%\setcounter{secnumdepth}{3} % le estoy indicando la profundidad hasta donde tiene que mostrar en el indice
%\setcounter{tocdepth}{4} 

\usepackage[titletoc]{appendix}

\RequirePackage[paperwidth=17cm,paperheight=24cm,inner=15mm,outer=20mm,top=25mm,bottom=25mm]{geometry} %%dimensiones externas del pdf y margenes internos.

\usepackage [T1]{fontenc}
\usepackage{graphicx}
\usepackage{graphics}
\graphicspath{ {Figures/} } %Para buscar las imágenes en otra carpeta
\usepackage{amsfonts} %Para poder escribir letras como la matriz identidad
\usepackage{mathrsfs} %Para poder escribir letras como la H del el espacio de Hilbert
\usepackage{slashed} % para la barra cruzada
\usepackage{amsmath}
\usepackage{amssymb}
\usepackage{makeidx}
\usepackage{hepunits} %para poder utilizar unidades
\usepackage[version=3]{mhchem} %para poder escribir los simpolos atómicos y químicos
%\usepackage{hep}

%\renewcommand{\figurename}{ Figura}
%\usepackage[labelsep=endash]{caption}

%\usepackage[figurename=\bold{Figura}]{caption}
%\usepackage [labelformat=empty]{caption}
%\usepackage{wrapfig} %for I can use wrapfigure
%\usepackage[font={small, up}]{caption}
%\usepackage[caption = false]{subfig} %% for putting several pictures in the same line
\usepackage[hyphens]{url} %hipervinculos del índice
\PassOptionsToPackage{hyphens}{url}
\usepackage{hyperref}
\setlength{\parindent}{4em} %para que interprete la linea en blanco como un \parragraph{}
\setlength{\parskip}{1em}
%\usepackage{subcaption}
%\usepackage[caption = false]{subfig}
\usepackage{caption}
\usepackage{subcaption}


\renewcommand{\baselinestretch}{1.15}
%\renewcommand{\thefigure}{}

%para aumentar el espacio entre elementos del índice
\usepackage{setspace}

\usepackage{eurosym} % para el euro

%\bibliographystyle{unsrthep}

%\usepackage{graphicx} % figuras
%\usepackage{subfig} % subfiguras

%\usepackage{natbib}

\setcounter{secnumdepth}{6}  %To specify the depth section to numerate
\setcounter{tocdepth}{6}  % To specify the depth section to show in the index
\usepackage{amsmath} %to use multiple line inside of a equation

%\captionscatalan{\renewcommand*{\figurename}{Figura}}
%\captionscatalan{\renewcommand*{\tablename}{Tabla}}