%Finally, an addition step was included to the fiber conditioning process, with the objective of improving the photon collection efficiency of the fibers. 

The tritium events only produce a few photons in the fibers, so it is very important to detect as many photons as possible. As it was demostrated in the fiber characterization study, the quality of the interface between the core of uncladded fibers and the environment (tritiated water in the case of TRITIUM detector) affects conspicuously the photon collection efficiency. To improve the quality of the interface, a fiber cleaning process was included, aiming to remove external particles deposited on the fibers, such as dust and fat that worsen the photon collection efficiency.  Through this cleaning process, the wetting property of the fibers, illustrated in Figure \ref{fig:WettingProperty}, is improved, preventing air molecules from attaching to the fiber and achieving a uniform water clad around the fibers, which results in an improvement of their collection efficiency. 

%Therefore, a mechanism, called the fiber cleaning process, was applied. As we can see in Figure \ref{fig:WettingProperty}, this cleaning process was carried out to improve the wetting properties, preventing air molecules from attaching to the fiber and achieving a uniform water clad around each fiber, avoiding variations in its refractive index which can worsen the photon collection efficiency of the fibers.

\begin{figure}[h]
\centering
\includegraphics[scale=0.5]{4ResearchAndDevelopments/41Fibers/WettingProperty.png}
\caption{Wetting property produced by the cleaning process. \cite{WettingProperty}\label{fig:WettingProperty}}
\end{figure}


This cleaning process  was developed and carried out in the clean room of ICMOL laboratory\footnote{ICMOL, Institute of Molecular Science, is a research institute located in the Science Park of the University of Valencia.}. It consists of filling three different glass beakers, one with alkaline soap, another with millipore water\footnote{The millipore water is water in which all the ions were removed, producing a very low conductivity of it-self, on the order of $10~\mu\sievert/\cm$} and the last one with isopropanol. First, the fibers are rubbed for 5 minutes with alkaline soap and then placed in the first beaker for sonication for 3 minutes. Then, the fibers are cleaned with a constant flow of water for 5 minutes and they are placed in the second beaker for sonication for another 3 minutes. Third, the fibers are placed in the third beaker for sonication for another 3 minutes. Finally the fibers are dried with an $\ce{N_2}$ air gun and introduced inside of the prototype.

The improvement in fiber response was verified using a bundle of twenty fibers of $15~\cm$ length  that was prepared with the conditioning process described. This bundle of fibers was arranged in the setup described in section \ref{subsec:PolishingMachine}, Figure \ref{fig:BunchWith2PMTsCoincidence}, and several energy spectra were taken using different radioactive sources. Then, these fibers were cleaned with the fiber cleaning process and spectra were measured again. Two radioactive sources were used in this study, a $\ce{^{90}Sr}$ beta source, already used in the polishing machine test, and a $\ce{^{137}Cs}$ gamma source, of $500~\becquerel$ activity. The results are plotted in Figure \ref{fig:ResultsOfCleaningProcess}. A shift of the spectrum to higher energies can be noticed for the cleaning fibers. This improvement was quantified by a parameter $F$ definded as,
\begin{equation}
F=\frac{A_{C}-A_{NC}}{A_{C}}
\label{eq:RelativeImprovement}
\end{equation}
where $A_{C}$ and $A_{NC}$ ae the integrals of the energy spectra measured after and before the cleaning process, respectively.

The value of $F$ obtained is about $21\%$ for both radioactive sources. Nevertheless, it should be taken into accout that $F$ was measured in air and the result could differ in water.

\begin{figure}
\centering
    \begin{subfigure}[b]{1\textwidth}
    \centering
    \includegraphics[width=\textwidth]{4ResearchAndDevelopments/41Fibers/Cs-137_CleaningProcess.pdf}  
    \caption{\label{subfig:EnergySpectrumCo60CleaningTest}}
    \end{subfigure}
    \hfill
    \begin{subfigure}[b]{1\textwidth}
    \centering
    \includegraphics[width=\textwidth]{4ResearchAndDevelopments/41Fibers/Sr-90_CleaningProcess.pdf}  
    \caption{\label{subfig:EnergySpectrumSr90CleaningTest}}
    \end{subfigure}
 \caption{Energy spectra obtained before and after the cleaning process using a radioactive source of a) $\ce{^{137}Cs}$ and b) $\ce{^{90}Sr}$.}
 \label{fig:ResultsOfCleaningProcess}
\end{figure}


%$(27.73 \pm 1.6)\%$ for the gamma source and $(20.72 \pm 0.9)\%$ for the beta source so, the improvement of the photon collection efficiency of the fibers was verified using the cleaning process carried out in the clean room of ICMOL laboratories. Nevertheless, it should be taken into accout that this test was carried out in air. It could be interesting to repeat it in water to obtain more realistic conclusions since the fibers of the TRITIUM detector will be immersed in water.