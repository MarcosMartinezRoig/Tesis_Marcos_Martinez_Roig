Another test was performed to find the fiber length that optimizes the tritium detection efficiency. Two different lengths of the scintillating fiber were considered in this study, $1~\meter$ and $25~\cm$. 

As the detector active area is related with its tritium detection efficiency. Therefore, the advantage to use a longer fibers is that the same active area can be achieved with a less number of cells, considerably reducing the number of used photosensors and, as a consequence, the price of the TRITIUM monitor. However, smaller length of scintillating fibers reduce de photon absortion produced in the fibers, increasing the tritium detection efficiency for the same active area.

For this task, the Tritium-Aveiro prototype was simulated. It consists of $360$ scintillating fibers readout by two photosensors. The photosensors, which consist on a windows glass and photocatode, are located at both scintillating fiber ends. These scintillating fibers are located inside of a teflon tube with two PMMA windows and a optical grease layer with a thickness of $0.5~\mm$ were simulated between the PMMA windows of the prototype and the photosensors. All optical properties mentioned in section \ref{sec:Geant4Environment} were included in this study.

The results of this study are shown in section \ref{subsec:ResultsFiberLength}, where they are discussed.