Finally, this section shows the results obtained with the simulation of a TRITIUM-IFIC 2 prototype when the background rejection system, detailed in section \ref{subsec:LeadCosmicSimulation}, is included. These simulations quantify the effect on the tritium measurement of both parts where the background rejection system are formed, lead shield and cosmic veto.

Similar to the simulations used for the study of the TRITIUM-IFIC 2 prototype, previous section, analogous variables were used as tests, which were systematically verified, to ensure that all the steps of the simulations were carried out correctly.

Three different simulations were carried out to independently quantify how the tritium detection is affected due to both, the lead shield and the cosmic veto. The first simulation consists of a TRITIUM-IFIC 2 prototype and the cosmic ray source, in the second simulation a lead shield was added and for the third simulation, the cosmic veto was also included.

The cosmic events detected by the TRITIUM-IFIC 2 prototype are reduced 5.4 times when a lead shielding with walls of $5~\cm$ is included. This reduction is mainly caused due to the stop of the weak cosmic radiation (energy lower than $200~\MeV$).%, which is in agreement with the reduction of about 4 times 

When the cosmic veto is included, the $X\%$ of the cosmic events that reach the prototype are detected in coincidence by the cosmic veto and, therefore, removed from the tritium measurement.