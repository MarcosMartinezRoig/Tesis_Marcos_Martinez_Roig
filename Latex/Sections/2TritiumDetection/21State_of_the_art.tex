Measurement of tritium activity is one of the systematic environmental controls that have been carried out for dozens of years in the vecinity of nuclear research facilities and nuclear power plants during their energy production lifetime.

Consequently, this measurement has been attempted with many different technologies so far in order to constantly improve the state of the art in tritium detection. The most used techniques are summarized in Table \ref{tab:DifferentThecnics}.

\begin{table}[htbp]
\begin{center}
\begin{tabular}{|c|c|c|c|c|}
\hline
 & LSC & IC & Calorimetry & BIXS\\
\hline \hline \hline
\parbox{5em}{\centering Measured\\ quantity} & \parbox{5em}{\centering Scintillation\\ photons} &  \parbox{5em}{\centering Ionization\\ current} & heat & X-rays\\ \hline
LDL & $\sim\becquerel$ & $10-100~\kilo\becquerel$ & $\sim~\giga\becquerel$ & $\sim~\mega\becquerel$ \\ \hline
Sample form & Liquid & Gas, vapor & All & All \\ \hline
%Disadvantages & & Gas, vapor & All & All \\ \hline
\end{tabular}
\caption{State-of-the-art in the tritium detection for different techniques. This table show the measured quantity, low detection level (LDL) and the sample form for four different technics, liquid scintillator counter (LSC), Ionization Chamber (IC), Calorimetry and the beta induced X-rays Spectrometry (BIXS)}
\label{tab:DifferentThecnics}
\end{center}
\end{table}

Nowadays, the most used technic for measuring tritium in water is the liquid scintillator counter, LSC. It consists of mixing a liquid sample (some ml for environmental measurements or less for higher activities) with liquid scintillator. In LARAM, laboratory of the University of Valencia, this mixture is usually made in a ratio of 50:50 \cite{LSCLARAM} but it depends on the detection system and the samples used \cite{LSCothers, HofstetterSeveral}. In this technique, the $\beta$ energy that is emitted from the sample excites the molecular energy levels of the liquid scintillator which promptly decays emitting several photons with a well-know energy (fluorescence), usually in the visible spectrum. Finally these photons are detected with photosensors, which convert the optical signal into measurable electrical charge.

The liquid scintillator technique has a very good detection sensitivity for low activity levels of tritiated water ($<1\becquerel/\liter$) \cite{0.6Bq_L} but it has the problems of long measurement delays (up to 2 days) and of producing chemical waste, as liquid scintillator contains toluene which is toxic. In addition, this technique requires special staff for sampling, chain-of-custody and lab analysis which consum economical and time resources. In order to avoid the last problem some unsuccessful efforts have been made in order to build a monitor of tritium with LSC \cite{OnlineLSC}. 

The ionization chamber (IC) is based on gas chamber (sample) which contains electrodes connected to different voltages. These electrodes collect the ionization current that is produced due to the $\beta$ radiation. It is a simple and fast system, but it has the problem of high Low Detection Limit ($> 10~\kilo\becquerel$) and of requiring the samples to be in a state of gas or steam \cite{IonizationChamber1, IonizationChamber2}.

The calorimetry is based on the measurement of the heat generated in the detection medium (normally platinum) \cite{Calorimeter1, Calorimeter2}. The problem with this technique is that it has a too high LDL, of the order of $\giga\becquerel$, and too long measurement time, 2 days or more.

The Beta Induced X-ray Spectrometry (BIXS) is based on the measurement of the bremsstrahlung radiation produced by the tritium decay electrons, using a \ce{NaI(Tl)} crystal couplet to a PMT  \cite{XRays1, XRays2} or Silicon Drift Detector (SDD) \cite{Bremstrahlung}. The problem with this technique is that it has too high LDL, of the order of $\mega\becquerel$.

There are many more different methods for tritium detection, although they are less used or less experimentally developed, each one with their own advantages and limitations. For example, the Avalanche PhotoDiode, APD, \cite{APD}, which cannot be used in contact with water, the mass spectrometry \cite{Spectrometry}, which needs to store the sample several months before taking the measurement or Cavity ring spectroscopy \cite{Ring}, which requires a special optical configuration that is not possible outside the laboratory.

It is necessary to keep in mind that all these techniques are offline methods that take too long to measure the activity of the samples, including the time for the sample collection and shipment to the laboratory. These techniques cannot be used for in-situ monitoring of tritium in water.

The liquid scintillation technique is, however, the only one with low enough Low-Detection-Limit to check the compliance of $100~\becquerel/\liter$ in tritium of the water samples, stablished by the EURATOM directive. 

The purpose of the TRITIUM project is to develop an alternative method, based on solid scintillators, that allows to accomplish the requirements of in-situ monitoring of levels as low as the legal limit in Europe $100~\becquerel/\liter$ in quasi-real time. There are several studies that have developed with solid scintillators so far:

\begin{itemize}

\item{} First study was done by M. Muramatsu, A. Koyano and N. Tokunaga in 1967 who used a scintillator plate read out by two PMTs in coincidence \cite{Muramatsu}.

\item{} The second study was carried out by the A. A. Moghissi, H. L. Kelley, C. R. Phillips and J. E. Regnier in 1969 that used one hundred plastic fibers coated with anthracene powder and read out by two PMTs in coincidence \cite{Moghissi}.

\item{} Third study was performed by R. V. Osborne in 1969 who used sixty stacked scintillator plates read out by two PMTs in coincidences \cite{Osborne}.

\item{} Fourth study was done by A. N. Singh, M. Ratnakaran and K. G. Vohra in 1985, who used a scintillator sponge read out by electronic coincidence \cite{Ratnakaran, Ratnakaran2000}.

\item{} Fifth study was carried out by K. J. Hofstetter and H. T. Wilson in 1991, who did different experiments for testing different shapes of scintillator plastic like several sizes of beads, fibers, etc. The better result which Hofstetter got for solid plastic scintillator was a tritium detection efficiency of the order of $10^{-3}$ \cite{Hofstetter1, Hofstetter2}.

\end{itemize}

\begin{table}[htbp]
\begin{center}
\begin{tabular}{|c|c|c|c|c|}
\hline
 & \parbox{6em}{\centering Efficiency, $\eta_{det}$\\ $(cps/(\kilo\becquerel/\liter))$}  & \parbox{5em}{\centering Surface\\ $F_{sci}$ ($\cm^2$)}  & \parbox{5em}{\centering Specific efficiency\\ $\varepsilon_{det}=\eta_{det}/F_{sci}$} & LDL ($\kilo\becquerel/\liter$)\\
\hline \hline \hline
Muramatsu & $3.85 \cdot 10^{-4}$ & $123$ & $3.13 \cdot 10^{-6}$ & $370$ \\ \hline
Moghissi & $4.5 \cdot 10^{-3}$ & $>424.1$ & $<1.06 \cdot 10^{-5}$ & $37$ \\ \hline
Osborne & $0.012$ & $3000$ & $4 \cdot 10^{-6}$ & $37$ \\ \hline
Singh & $0.041$ & $3000$ & $1.37 \cdot 10^{-5}$ & $<37$ \\ \hline
Hofstetter & $2.22 \cdot 10^{-3}$ & $\sim~100$ & $<2.22 \cdot 10^{-5}$ & $25$ \\ \hline
\end{tabular}
\caption{Results of different scintillator detector for tritiated water detection}
\label{tab:PlasticScinTritium}
\end{center}
\end{table}

%COMPROBAR QUE ESTAN BIEN TODOS LOS DATOS (sobretodo areas, lo otro esta comprobado. A lo mejor puedo calcular el area del ultimo caso)

The results of these experiments are sumarized in Table \ref{tab:PlasticScinTritium}. As can be seen, in the first column that the intrinsic detector efficiency, $\eta_{det}$, is very different in these experiments. As it is known that, in this type of detectors, one of the most important factor, which affects the efficiency, is the active surface of the plastic scintillator, $F_{sci}$, and, as can be seen in the second column, it is very different in each detector, the specific detector efficiency (third column) is used in order to compare these experiments, that's, the intrinsic detector efficiency normalized to this active surface. Now it can be checked that, effectively, these specific efficiencies are quite similar. One of the best specific efficiency was obtained by Moghissi who used scintillating fibers. This is a good point which justify our choice about using scintillating fibers as a detection medium. Finally, as can be seen in the last column, the LDL in all these experiments are of the same order of a few tens of $\kilo\becquerel/\liter$ so developing a detector which overcome these LDL values is essential for the compliance of the EURATOM directive of 100 Bq/L of tritium in water for human consumption.