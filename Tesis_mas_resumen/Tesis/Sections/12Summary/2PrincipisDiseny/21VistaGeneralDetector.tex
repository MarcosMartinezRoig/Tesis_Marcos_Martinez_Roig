El monitor desenvolupat per la col·laboració s'anomena TRITIUM i consisteix de tres parts principals:

\begin{enumerate}

\item{} El detector de triti, on es realitza la mesura de la mostra d'aigua. Aquest està format per diversos mòduls llegits en paral·lel. El disseny modular li confereix la propietat d'escalabilitat, permetent-li millorar certes característiques com la eficiencia o la activitat mínima detectable de triti simplement utilitzant més mòduls.

Cadascun d'aquests mòduls està format per centenars de fibres centellejadores que entren en contacte directe amb la mostra a mesurar i son capaçes de detectar alguns de les desintegracions de triti que tenen lloc. Aquestes fibres són llegides per varios photosensors i analitzats per la electrónica corresponent.

\item{} El sistema de purificació d'aigua el cual prepara la mostra abans d'introduir-la al detector. Aquest sistema es utilitzat per dos motius principals. En primer lloc per a eliminar cualsevol partícula amb un diàmetre superior a $1~\mu\meter$ que es trobe dissolta a l'aigua, la qual podría ser depositada sobre les fibres centellejadores. Degut a que el recorregut lliure mitjà del triti es de només $5-6~\mu\meter$ en la matèria, aquesta deposició provocaria una disminució de l'eficiencia de detecció al triti. En segon lloc per a eliminar cualsevol element radioactiu (distint al triti) disolt a l'aigua, com per exemple el $\ce{^{40}K}$ que es troba de manera natural al medi ambient, ja que aquest sería contabilitzat per el detector com un esdeveniment de triti, falsejant les mesures.

\item{} El sistema de rejecció de fons radioactiu, utilitzar per suprimir en la mesura del possible el fons radioactiu en el lloc de treball. Aquest es divideix en dos parts. Un castell de plom que envolta el detector, utilitzar per a reduir els esdeveniments de menys energia. Aquest castell de plom és capaç de parar gran part de les partícules amb energies inferiors a $200~\MeV$, que principalment procedeixen del fons radioactiu natural del lloc de treball i dels raigs còsmics. Un veto actiu, utilitzat per suprimir els esdeveniments de major energia. Aquest és capaç de suprimir gran part dels esdeveniments amb energies superior a $200~\MeV$, principalment la resta dels raigs cósmics.

\end{enumerate}

Tan el sistema de purificació de l'aigua com el sistema de rejecció de fons radioactiu son fonamentals ja que permeten reduir gran part del fons radioactiu mesurat pel detector, una tasca essencial per alcanzar l'objetiu del projecte (mesurar activitats de triti tan baixes com $100~\becquerel/\liter$).

Finalment aquestes tres parts són controlades per distints aparells electrònics que, en últim lloc, emetran un senyal d'alarma en cas que el límit legal fixat a Espanya per al triti en aigües potable, $100~\becquerel/\liter$, siga superat.

La idea final seria poder incloure aquest monitor a la xarxa d'estacions automàtiques del CSIC\cite{REA} (REA per les sigles en castellà). Aquesta es una xarxa formada per diversos detectors de tot tipus que mesuren en temps quasi real les concentracions d'alguns dels diferents isòtops radioactius que es troben de forma natural al medi ambient. 