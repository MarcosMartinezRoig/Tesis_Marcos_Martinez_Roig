First thing that was done in TRITIUM experiment was to choose the optimal fiber length at which the signal from the tritium events is optimized. To take this decision, it has to be taken into account that, on the one hand, long fibers are interesting because the efficiency of TRITIUM detector is proporcional to the active area (proporcional to the fiber length), but, on the other hand, in long fibers, scintillating photons need to be reflected in the fiber walls more times to be driven to its ends, where the photosensors are, and, because of that, some photons are lost in each reflection, deteriorating the detector signal.

Several simulations were performed using Geant4 \cite{Geant4WebPage}, a particle and nuclear physics simulation package based on C++, to quantify the importance of this effect. The result of these simulations, shown in section \ref{sec:ResultsSimulations}, Figure \ref{}, was that it is preferable to work with short fiber since a significant loss of photons was observed.

The chosen fiber length for the TRITIUM prototypes developed in Valencia, is $20~\cm$, which was also the length used for most of the experiments performed in the framework of the TRITIUM experiment. 

Due to the length of Saint-Gobain's commercial fibers, which are 1 meter long, an effective scintillator cutting technique was to be developed. It is very important to introduce strict requirements on the cutting quality of the fiber ends since it will greatly affect the transmission of photons and, thus, the efficiency of TRITIUM monitor. This cut must be perpendicular to the fiber and with very low uncertainty in the length of the fiber, both requeriments are mandatory to achieve a good coupling with the surface of the photosensor. It is also important that its final state must be as regular as possible, that means, without cracks or deformations that contribute to internal reflections, losing photons and, thus, reducing the tritium signal.

Cutting the end faces of polymer fibers is one of current challenges. There are many different techniques such as milling, laser cutting, focused-ion-beam, blade cutting, etc. The blade cleaving technique was chosen to be developed in TRITIUM experiment due to its mechanical simplicity. %I had these references in the "Smooth end face termination of microstructured, graded-index, and step-index polymer optical fibers" paper

Many commercial devices based on blade cleaving, such as the one provided by thorlabs with a diamond tipped blade \cite{DiamondThorlabs} or others similar to the guillotine designed for industrial fiber optics \cite{GuillotineIFO}, were tested in an extensive study done with unsuccessful results \cite{TFGAlberto}. As can be seen in Figures \ref{fig:BadCutsOfFibers}, it presents deformations, cracks or imperfections so the technics considered in this study don't overcome the requirements imposed.

\begin{figure}
\centering
    \begin{subfigure}[b]{0.5\textwidth}
    \centering
    \includegraphics[width=\textwidth]{4ResearchAndDevelopments/41Fibers/DeformationFiberEnds.png}  
    \caption{Fiber end deformation.\label{subfig:FiberEndDeformation}}
    \end{subfigure}
    \hfill
    \begin{subfigure}[b]{0.45\textwidth}
    \centering
    \includegraphics[width=\textwidth]{4ResearchAndDevelopments/41Fibers/CracksEndFibers.png}  
    \caption{Fiber end cracks.\label{subfig:FiberEndCracks}}
    \end{subfigure}
 \caption{Unsuccessful results of using commercial techniques to cut fibers.}
 \label{fig:BadCutsOfFibers}
\end{figure}

The microscope model PB 4161 from EUROMEX company or the Digital Microscope from Jiusion company were used to check the results in the fiber ends.

%\begin{figure}[htbp]
 %\centering
  %\subfloat[EUROMEX microscope]{
   %\label{subfig:EUROMEXMicroscope}
    %\includegraphics[width=0.25\textwidth]{4ResearchAndDevelopments/41Fibers/EuromexMicroscope.png}}
    %\newline
  %\subfloat[Jiusion microscope]{
   %\label{subfig:JiusionMicroscope}
    %\includegraphics[width=0.45\textwidth]{4ResearchAndDevelopments/41Fibers/JiusionMicroscope.png}}
 %\caption{Microscopes used to check the results.}
 %\label{fig:Microscopes}
%\end{figure}

Because commercial devices don't work for our scintillating fibers, a cutting device was designed, built and tested, shown in Figure \ref{fig:CuttingTRITIUMDevice}.

\begin{figure}
\centering
    \begin{subfigure}[b]{0.4\textwidth}
    \centering
    \includegraphics[width=\textwidth]{4ResearchAndDevelopments/41Fibers/CuttingDevice1.png}  
    \caption{TRITIUM Cutting device.\label{subfig:CuttingDevice1}}
    \end{subfigure}
    \hfill
    \begin{subfigure}[b]{0.55\textwidth}
    \centering
    \includegraphics[width=\textwidth]{4ResearchAndDevelopments/41Fibers/CuttingDevice2.png}  
    \caption{TRITIUM Cutting device.\label{subfig:CuttingDevice2}}
    \end{subfigure}
    \hfill
    \begin{subfigure}[b]{0.6\textwidth}
    \centering
    \includegraphics[width=\textwidth]{4ResearchAndDevelopments/41Fibers/AdditionalPieceCuttingDevice.png}  
    \caption{Additional piece of TRITIUM cutting device.\label{subfig:AdditionalPieceCuttingDevice}}
    \end{subfigure}
 \caption{Cutting device developed in the TRITIUM experiment and additional part to make precise measurements of fiber length.}
 \label{fig:CuttingTRITIUMDevice}
\end{figure}

It consists of fourteen rails where the fibers are fixed and a thin blade, fixed on a mobile piece, which is used to cut them. The perpendicular cut, which is one of the requirements imposed, can be ensured since the moving piece, where the blades are fixed, is placed perpendicular to the fibers.

The blade used is the typical commercial razor blade, whose thickness is $0.1~\mm$, which is the thickness with which we obtain the best results. It was positioned with a slight inclination, $5\degree$, with respect to the horizontal axis since it was seen in several studies that it helps to obtain a less aggressive and cleaner cut \cite{AngleBlade}, \cite{TemperatureBlade}.

Therefore, as can be seen in Figure \ref{subfig:CutFiberEnd}, with the developed device fiber ends without breaks or deformation was obtained, overcoming other imposed requirement.

Another important parameter that can affect the cutting quality of the fiber ends is the temperature of both, either the fiber or the blade. It was tested in a study in which both were subjected to different temperatures from room temperature (25 degrees) to 110 degrees \cite{TFGAlberto}. No significant conclusions were obtained in the temperature study, so the cutting process is carried out at room temperature to facilitate the cutting technic.

To obtain a low enough length uncertainty, which is the last requirement to overcome, an additional piece was designed and built, shown in Figure \ref{subfig:AdditionalPieceCuttingDevice}, which is used to measure the fiber. With this piece we achieve an uncertainty in the measurement of less than 1 millimeter.

With the designed cutting fiber device we have exceeded all requirements imposed, obtaining a fiber cutting device, whose effect on the light transmission is minimized.
%cutted fiber end whose quality is high enough to ensure that it will affect the transmission of light as little as possible.

In Figure \ref{subfig:CutFiberEnd}, which shows the fiber end after cutting process with TRITIUM cutting device, it can be seen a slightly darkened part at the bottom of the fiber, which is an inevitable effect of the cutting process. To reduce the effect of this imperfection, a polishing process developed by thorlabs is included \cite{DiamondThorlabs}. 

This polishing process consists of using five different polishing papers, with a decreasing grain size, whose diameters are $30~\mu\meter$, $20~\mu\meter$, $12~\mu\meter$, $5~\mu\meter$ and $0.3~\mu\meter$ respectively, in which we describe movements in the shape of 8 for two minutes (approximately 120 movements). 

The result obtained with this polishing process is shown in Figure \ref{subfig:PolishFiberEnd}. In Figure \ref{fig:ResultofPolishingProcess}, the quality of both fiber ends, before and after polishing process, can be compared, where it can be appreciated that the darkened part has completely disappeared. 

\begin{figure}
\centering
    \begin{subfigure}[b]{0.5\textwidth}
    \centering
    \includegraphics[width=\textwidth]{4ResearchAndDevelopments/41Fibers/CutEndFiberGood.png}  
    \caption{Fiber end after cutting with Tritium device.\label{subfig:CutFiberEnd}}
    \end{subfigure}
    \hfill
    \begin{subfigure}[b]{0.45\textwidth}
    \centering
    \includegraphics[width=\textwidth]{4ResearchAndDevelopments/41Fibers/CutAndPolishedFiberEnd.png}  
    \caption{Fiber end after cutting and polishing.\label{subfig:PolishFiberEnd}}
    \end{subfigure}
 \caption{Result of the polishing process. a) Fiber end after cutting with TRITIUM devices b) Fiber end after cutting with TRITIUM devices and polishing with Thorlabs technic.}
 \label{fig:ResultofPolishingProcess}
\end{figure}

The end of the cut fiber is completely clear after cutting and polishing, without any damage or imperfection, so both tasks, cutting and polishing,  make up the conditioning process developed for each fiber before any study or its introduction into the TRITIUM detector.