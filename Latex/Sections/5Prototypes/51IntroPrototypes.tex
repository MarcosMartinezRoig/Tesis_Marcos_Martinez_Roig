This chapter describes the different prototypes developed in the framework of the TRITIUM experiment, which are TRITIUM-IFIC 0, TRITIUM-IFIC 1, TRITIUM Aveiro and TRITIUM-IFIC 2, listed in chronological order of their construction. The first two prototypes built, TRITIUM-IFIC 0 and TRITIUM-IFIC 1, are preliminary prototypes used to learn about tritium detection and to detect and solve problems in their designs. The other two prototypes built, TRITIUM-Aveiro and TRITIUM-IFIC 2, are prototypes with a design in which no significant problems were found. They were built to check more subtle effects. 

Each prototype was designed and built in the laboratories of IFIC or Aveiro and it was filled with tritiated water following a protocol specially developed for this task. Several water tightness and filling tests were carried out for each prototype to guarantee its radiosecurity. At the end of the chapter, the final monitor of TRITIUM detector will be described. Its design is a modular structure for easy scalability, composed of as many units of the final prototype as needed to reach the required sensitivity.