\documentclass[12pt,a4paper]{book}

\input{./Sections/1Previous_pages/11PaquetesTrabajo}

\begin{document}
\captionsetup[figure]{labelfont={bf},labelformat={default},labelsep= endash,name={Figura}}
\input{./Sections/1Previous_pages/12Cover_page}

$\ $
\thispagestyle{empty}  % para que no se numere esta pagina
\chapter*{}

\pagenumbering{Roman} %for using romain numbers (page numering)


\begin{flushright}
\textit{Dedicated to \\
my family}
\end{flushright} 

\newpage

\chapter*{Acknowledgements} \label{chap:Acknowledgements}  %pongo el asterisco para que no se numere ni aparezca en el índice
%%\cleardoublepage
\addcontentsline{toc}{chapter}{Acknowledgements} % para que aparezca en el indice de contenidos
AGRADECER A: PEPE, NADIA, MIREIA, ANA, MARQUITOS, ANDREA, GENTE DEL LARAM (TERESA, VANESA, ROSA, CLODO), COMPAÑEROS DE DESPACHO Y DEL IFIC/UV (NOMBRAR TODOS), ANSELMO Y MIGUEL, INGENIEROS DE NEXT, GENTE DE PORTUGAL, Antonio y Jose Angel de extremadura, gente de francia...  A DAVID CALVO DEL IFIC, A DAVID CANAL DE SAMTEC, A LUIS FERR... DE PETSYS... A LIDON DEL ICMOL... Ana Ros, Jhon Barrio y Gabriela Llosa del IFIMED. Al programa interreg sudoe -> Soporte financiero!

Departamento de mecánica del IFIC (Manolo "Apellidos", Jose Luis Jordan, Jose Vicente Civera Navarrete, Tchogna Davis, Daniel), departamento de electrónica del IFIC (Jorge Nacher Arándiga, etc )


Y PENSAR GENTE QUE ME DEJO POR EL CAMINO.



\newpage

\chapter*{Abstract} \label{chap:Abstract}
%%\cleardoublepage
\addcontentsline{toc}{chapter}{Abstract} % para que aparezca en el indice de contenidos
%LEER LOS 2 ABSTRACS DE ANA MAS LOS ABSTRACS QUE TENGO YO DE LAS CONFERENCIAS Y ARTICULOS (ARTICULOS ALEATORIOS MAS EL DE CARLOS, EL MIO, EL DE NADIA, ETC)

%\begin{abstract}
%Texto           del           abstract
%\end{abstract}

\newpage

\chapter*{Nomenclature and acronyms} \label{chap:NomenclatureAcronyms}  %pongo el asterisco para que no se numere ni aparezca en el índice
%%\cleardoublepage
\addcontentsline{toc}{chapter}{Nomenclature and acronyms} % para que aparezca en el indice de contenidos
\begin{longtable}{p{25mm} c p{120mm} }
\multicolumn{3}{l}{Acronyms:}\\
\\
$OECD$ & --- & Organisation for Economic Co-operation and Development\\
$Btu$ & --- & British termal unit\\
$UN$ & --- & United Nations\\
$UNFCCC$ & --- & United Nations framework convention on Climate change\\
$ITER$ & --- & International Thermonuclear Experimental Reactor\\
$NPP$ & --- & Nuclear Power Plants\\
$DOE$ & --- & Department of Energy\\
$U.S.A.$ & --- & United States of America\\
$U.S.$ & --- & United States\\
$EPA$ & --- & Environmental Protection Agency\\
$LDL$ & --- & Lower Detection Limit\\
$PWR$ & --- & Pressurized Water Reactor\\
$BWR$ & --- & Boiled Water Reactor\\
$PHWR$ & --- & Pressurized Heavy Water Reactor\\
$quasi-real$ & --- & Less than 10 minuts\\
$LSC$ & --- & Liquid Scintillation Counting\\
%$IUPAC$ & --- & International Union of Pure and Applied Chemistry\\
$LWR$ & --- & Liquid Water Reactor\\
$STP$ & --- & Standard temperature ($0\celsius$) and pressure ($1$ atm)\\
$IC$ & --- & Ionization chamber\\
$BIXS$ & --- & Beta Induced X-ray Spectrometry\\
$PMT$ & --- & PhotoMultiplier Tube\\
$SDD$ & --- & Silicon Drift Detector\\
$EEC$ & --- & European Economical Community\\
$SiPM$ & --- & Silicon PhotoMultiplier\\
$ALARA$ & --- & As Low As Reasonably Achievable\\






$T$ & --- & Temperature (ºC).\\
$V$ & --- & Volume (m$^3$).\\


\\
\\
\multicolumn{3}{l}{Atomic and nuclear symbols}\\
\\
$\ce{CO_2}$ & --- & Carbon Dioxide\\
$\ce{CH_4}$ & --- & Methane\\
$\ce{N_2 O}$ & --- & Nitrous oxide\\
$\ce{HFC}$ & --- & Hydrofluorocarbons\\
$\ce{PFC}$ & --- & Perfluorocarbons\\
$\ce{SF_6}$ & --- & Sulfur Hexafluoride\\
$\ce{^{1}_{1}H}$ & --- & Hydrogen\\
$\ce{^{2}_{1}H}$ & --- & Deuterium (Non-radiactive hydrogen isotope, 1 neutron)\\
$\ce{^{3}_{1}H}$ & --- & Tritium (radiactive hydrogen isotope, 2 neutrons)\\
$\ce{^{4}_{2}He}$ & --- & Helium\\
$\ce{^{3}_{2}He}$ & --- & Isotope of the Helium(Non-radiactive, 1 neutrons)\\
$\ce{n}$ & --- & free neutron\\
$\becquerel$ & --- & Becquerel, Nuclear decay number per second\\
$\liter$ & --- & Liter\\
$\becquerel/\liter$ & --- & Becquerel per liter\\
$Ci$ & --- & Curios\\
$Ci/L$ & --- & Curios por litro\\
$yr$ & --- & year\\
$\giga\watt$ & --- & Giga watt\\
$T_{1/2}$ & --- & Half-life time of a radioactive element\\
$\beta$ & --- & Beta decay\\
$\ce{\overline{\nu}_e}$ & --- & Electron antineutrino\\
$\ce{e^-}$ & --- & Electron\\



$\gamma$ & --- & Gamma decay\\
$\alpha$ & --- & Alpha decay\\
\\
\\
\multicolumn{3}{l}{Añadir en un futuro:}\\
\\
D\&D & --- & Decontamination and Decommissioning.\\
DWS & --- & Drinking water standars\\
UDL & --- & Upper Detection Limit\\


NA & --- & Numerical Apertures\\
PMMA & --- & Polymethyl Methacrylates\\
UV & --- & ULtraviolet\\
WLS & --- & Wavelength shifter\\
\end{longtable}


%%añadir bibliografía al indice
\let\OLDthebibliography=\thebibliography
\def\thebibliography#1{\OLDthebibliography{#1}%
\addcontentsline{toc}{chapter}{\bibname}}

%%indice
\tableofcontents

%%lista de figuras
\listoffigures

%\cleardoublepage
\addcontentsline{toc}{chapter}{List of Figures} % para que aparezca en el indice de contenidos

%%lista de tablas
\listoftables

%\cleardoublepage
\addcontentsline{toc}{chapter}{List of Tables} % para que aparezca en el indice de contenidos

%%%%%%%%%%%%%%%%%%%%%%%%%%%%%%% MAIN BODY %%%%%%%%%%%%%%%



\chapter{Introduction}  \label{chap:GeneralIntroduction} %(%(I have to use latin numbers inside of this chapter))
\pagenumbering{arabic} %for using romain numbers (page numering)
%%\setcounter{page}{-1} %%first number of the counter
	\section{Global energy context}
	\input{./Sections/2Introduction/21Introduccion} \label{sec:Introduction}
	\newpage

	\section{Tritium properties}
	\input{./Sections/2Introduction/22Tritium_properties}\label{sec:TritiumProperties}
	\newpage
	
	\section{State-of-the-art in tritium detection}
	\input{./Sections/2Introduction/23State_of_the_art}\label{sec:StateOfTheArt}
	\newpage
	
	\section{Tritium project and Tritium monitor}
	\input{./Sections/2Introduction/24Tritium_project}\label{sec:TritiumProject}
	\newpage
	
	\section{Work scheme}
	This thesis is build up as follow:...



The objetive of this three-phase project was design, develop and instalation and commisioning a automatical system capable of detection tritium which we find in the water that is used by the nuclear power plants for their cooling system. The initial idea was quantifying its activity in quasi-real time before discharging it into public rivers or seas. 

We are speaking all the time about getting the measurement in quasi-real time, that is, in less of 10 minuts but what about the real time? It is obviously imposible because we are mesuring the activity, I mean, we are counting tritium decays in samples with low activities (supposedly less than $100~\becquerel/\liter$) so we need some time in order to get enough stadistic with which we can distinguish the tritium signal to the background in our system.

In order to get the measurement in quasi-real time we need to work \textit{in situ}, that's, we need that our detector is able to work in the same place that we take the sample. The reason of this fact is because if we need to move the sample we lose a lot of time (1 or 2 days in some cases). Furthermore if we avoid to move the samples we get a:

\begin{itemize}
\item{} faster detector because we eliminates the process of taking the sample, the chain-of-custody until this sample arrive to this laboratory and the complexity which involve these tasks. 

\item{} better detector since if we can work \textit{in site}, our measurements can be more frequent hence we will can identify cahnges in the activity earlier.

\item{} cheaper detector because we have removed all the cost associated with the sample collection, chain-of-custody of this sample, shipping of this sample to the laboratory and analysis thereof there. Not only we have eliminated the material costs attached to this tasks (material, transport, etc) but we have also eliminated the costs attached to the specialized staff who are involving in these tasks. With our detector we will only need frequent calibrations each time, which we consider suitable, in order to ensure the correct operation of this detector

\item{} safer detector since the personal exposure dose is reduced and the changes in activity are detected fastly. On top of that we remove the possibles mistakes which can be done by specialized staff who follow each protocol of this tasks.

\end{itemize} 

Dividiremos este trabajo en seis partes:
\begin{enumerate}
\item{} En primer lugar, se realizará un estudio sobre las fibras centelleadoras para  determinar  el protocolo de manipulación para obtener un  procedimiento de preparación de  un haz de fibras centelleadoras con un buen rendimiento óptico. 

\item{} En segundo, lugar estudiaremos el procedimiento de calibración de los SiPM,  fundamental para el experimento.  No se necesita realizar una calibración de los PMT, paso igualmente importante al anterior, ya que este trabajo fue realizado recientemente por otro componente del grupo.

\item{} En tercer lugar, se describirá  el primer prototipo diseñado, formado por un haz de 35 fibras centelleadoras sin clad leídas por PMT,  incluyendo el protocolo del proceso de llenado con agua tritiada que tuvo que ser desarrollado para cumplir los requisitos de protección radiológica y evitar contaminación accidental,  y los  resultados obtenidos con el mismo.

\item{} En cuarto lugar, se presentarán las simulaciones realizadas con el programa de Geant4 en una configuración sencilla.

\item{} En quinto lugar, se expondrán aspectos a estudiar en el futuro inmediato y en etapas posteriores, durante la Tesis Doctoral. 


\item{} Se presentarán, finalmente, los resultados, logros y conclusiones alcanzadas en el desarrollo del  trabajo.

\end{enumerate}

For process monitoring purposes it is needed to know how much tritium is in
the processed water, which is done mostly by offline methods, also involving handling
of tritiated water by the operating personnel. Since tritiated water is highly toxic, it
would be advantageous to have an online and inline monitor, which eliminates the need
of sample taking and also speeds up the measurement process.
In this thesis the investigation of a plastic scintillator-based tritium detector is
presented. The main goal was to test the concept for higher concentrations as before
(up to several GBq/l concentration) and investigate possible enhancements of the scin-
tillator geometry, besides understanding the detection process. For these purposes an
experimental setup consisting of standard industrial parts was designed, constructed and
calibrated for HTO. Several sample chamber geometries were tested (e.g. more scintil-
lator plates above each other, varying sample volume, etc.), and the configurations were
compared in terms of sensitivity and detected pulse height spectrum. To understand the
detection process in detail, a simulation was coded and compared with the experimen-
tally obtained results.
The thesis is built up as follows: First, a brief chapter introduces tritiated water,
concerning properties and processing in tritium handling facilities. The next chapter
presents the scintillation counting method in detail, based on the available literature.
This is followed by the discussion of the previous studies in the topic of tritiated water
activity measurement, then the goals of this study are defined. Chapter 4 deals with
the design considerations and the technical details of the experimental setup. The next
chapter presents the optimization of the setup for measurement, then the calibration of
for tritium is presented. The chapter also includes the analysis of the pulse height spec-
trum of tritiated water, and results of the measurements with varying sample amount
and scintillator surface, all followed by discussion. Chapter 6 first presents the technical
improvements done on the setup, then the results of additional measurement series and
their discussion follow. In the last chapter, the simulation of the detection process is pre-
sented, together with discussion of the results. The experimentally obtained spectrum is
compared with the simulation and the implications of the differences is discussed. This
chapter is followed by a summary of the thesis, together with the conclusions drawn
from the results.\label{sec:WorkScheme}
	\newpage
	
\chapter[Design principles]{Design principles of the Tritium monitor}\label{chap:DesignPrinciples}
	\section{Chapter scheme}
	This Chapter is divided in three different sections. First, we have a theorical explication about the interaction of fast electrons and photon with matter. Next, I show the main parts in which the TRITIUM detector consists. Finally, the foundamental explication and our own developments in scintillating fibers and photosensors are shown.
 \label{sec:IntroDesignPrinciples}
	%\newpage

	\section[Interaction of particles with matter]{Interaction of fast electrons and photons with matter}
	In this section, the explanation will only focus on the particles and energy range that are interesting for this thesis, which are electrons ($0-18~\keV$) and photons in the visible range.

On the one hand, electrons have charge so their interaction with matter are mainly produced with the orbital electrons that there are in that matter, due to the Coulomb force. The trajectory which electrons follow is much more tortuous than other heavier particles because the mass of the interacting particles is equal, electrons. Furthermore, for the same reason, these electrons lost a significant amount of energy in each collision.

In order to speak about the total energy lost of particles in matter the specific energy loss is defined as $S=-\frac{dE}{dx}$ which expresses the energy loss suffered by the particle per unit of trajectory. In the case of electrons, this total energy loss has two main contributions, the collisions (elastic and inelastic) and radiative processes (bremsstrahlung):

$$\frac{dE}{dx} \approx \left(\frac{dE}{dx}\right)_{c} + \left(\frac{dE}{dx}\right)_{br} ~\cite{Knoll} \cite{Leo} \qquad  \frac{\left(\frac{dE}{dx}\right)_{br}}{\left(\frac{dE}{dx}\right)_{c}} \approx \frac{EZ}{700} ~\cite{Knoll}$$

Where $E$ is the energy of the electron in $\MeV$ and $Z$ is the atomic number of the absorbing material. Due to this energy loss, the electrons can only penetrate a material as far as they go before losing their total energy. This distance is known as range and in the case of tritium electrons its value is seen in the table \ref{MeanFreePathTritium}.

On the other hand, photons don't have charge. Its possible interactions with the matter are photoelectric effect, Compton effect, coherent scattering and pair production and the probability of each process depends on the energy of the photon, $E_\gamma = h\nu$, and the atomic number of the material, Z, as you can see in the figure \ref{ProcessesPhotons}.

\begin{figure}[htbp]
\centering
\includegraphics[scale=0.5]{3ResearchAndDevelopment/DominantProcessesPhotons.png}
\caption{Domain regions of the three most probable types of interactions of gamma rays with matter. The lines show the values of Z and $h\nu$ where the two neighboring effects are equally likely~\cite{Knoll}.\label{ProcessesPhotons}~\cite{Knoll}}
\end{figure}

We have to take into account that the only relevant photons for this thesis are in the visible range, between $400$ and $700~\nano\meter$, that corresponds with energies of the order of the $\eV$. Therefore the last effect, pair productions, will be not explained here because it needs a photon energy equal or more than $1.022~\MeV$ for happening and it is not our case.

The photoelectric effect occurs when a photon interacts with an orbital electron in the material, losing all its energy. This energy is absorbed by the electron that is released from the atom (ionization). The energy of the resulting electron is:

$$E_e = E_\gamma - E_b ~\cite{Knoll}\cite{Leo}$$

Where $E_\gamma$ is the energy of the photon and $E_b$ is the binding energy of the electron in this material. The probability of this effect depends on the number of available electrons through the variable Z, and the energy of the electron according to the following expression:

$$\left(Pr\right)_{Ph-eff} \approx \frac{Z^n}{E_\gamma^{3.5}}~\cite{Knoll}$$

As we can see in this expression the photoelectric effect is most probably if we use elements with high atomic number. This is the reason why elements with high atomic number are the best isulators against gamma radiation and this is the reason why we use lead ($Z=82$) for building our passive shielding as we will see in the section \ref{sec:PasiveShields}. 

The Compton effect occurs when the photon interacts with an orbital electron of the material, being scattered at an angle $\theta$ with respect to the original direction and transferring part of this energy to the electron, which is released. If we neglect the binding energy, the energy transfered to this electron, $E_e$, is shown in the following equation:

$$E_e=\frac{\frac{E_\gamma^2}{m_oc^2}\left(1-cos\theta\right)}{1+ \frac{E_\gamma^2}{m_oc^2}\left(1-cos\theta\right)}~\cite{Knoll}\cite{Leo}$$

Where $m_0$ is the rest mass of the electron and $c$ is the speed of the light in the vacumm. The probability of the Compton effect is proporcional to the atomic number, Z (more available electrons) and decreases with the energy of the photon. 

As we can see in the figure \ref{ProcessesPhotons}, in the energies of the photons belonging to the visible range of the electromagnetic spectrum (of the order of eV), the Compton effect is only more likely in very light materials, (Z<4). For heavier materials the photoelectric effect is the dominant effect. This is the reason why we use elements with high number atomic in the cathode of the our PMTs.

Finally, in the coherent scattering the atom is neither excitation nor ionization and the photon conserve all their energy in this collision. This effect is more probably for photons with low energies and materials with high atomic numbers.

Because of the fact that the energy of the photon doesn't change we will not speak more about this efect but it is important because this effect change de direction of photons and it will affect to their mean free path. \label{sec:Interaction}
	\newpage	
	
	\section{The detection process in a scintillation detector}
	\input{./Sections/3Design_Principles/32Detection_Process} \label{sec:DetectionProcess}
	\newpage
	
	\section{Plastic scintillators} %\label{sec:Scintillators}
	\input{./Sections/3Design_Principles/33Plastic_Scintillators/331Plastic_Scintillators}\label{sec:PlasticScintillators}
		\subsection{Plastic scintillation fibers}
		\input{./Sections/3Design_Principles/33Plastic_Scintillators/332Fibers}\label{subsec:PlasticScintillatorFibers}
		\newpage
		
	\section{The detection process in a Photosensors} %\label{sec:Photosensors}
	\input{./Sections/3Design_Principles/34Photosensors/34IntroPhotosensors}\label{sec:Photosensors}
	
			\subsection{Photomultiplier Tubs (PMTs)}%\label{subsec:PMTs}
			Photoelectron multiplier tube is one of the most used photosensors in nuclear physics during last decades. Its main objective, like all photosensors, is to detect the scintillating photons that reach its sensible part and covert it in an electronic signal large enough to be measured. The way in which PMTs achieve this objective is based in two different phases:

\begin{itemize}
\item{} First, the PMT convert photons that reach its sensible part in electrons with some probability. The sensible part is the photocathode (sensible part of the PMT) which consists in a fina capa de un material que produzca efecto photoelectrico con grán probabilidad. En nuestro caso el material utilizado es. La probabilidad de... es... el espectro de emission...  

- Hablar del photocatodo, parte sensible bla, bla, bla. Espectro de efficiencia, función de trabajo del photocatodo, etc...

\item{} After that, secondary electron multiplication

- Hablar de la etapa de ganancia, dinodos, circuito electrónico divisor de ganancia, bla, bla bla...

2 circuitos diferentes. En principio equivalentes pero el primero más seguro y el segundo mejor para mediciones precisas de tiempo y de alta tasa de cuentas.

\end{itemize}


Cuando acabe con todo esto hablar del resto de elementos, tubo de vacío, etc...


The PMT has two main functions. On the one hand it is able to convert photons, whose energy are inside of a energy range, in electrons throgh photoelectric effect. On the other hand, it capable of multiplying these electrons with high gain factors.


an electric pulse. It is based on a photocathode, which is the sensible part of the photosensor. The photocathode release an electron with some probability when a photon reach it...

IMAGEN

Nuestros objetivos para elegir el PMT y SiPM adecuado.

Leer el capitulo de PMTs del trabajo de "Centelleadores" -> Clave para explicar estas 2 etapas.

Leer el capitulo de PMTs de la tesis de fibras
Leer el capitulo de PMTs de la tesis alemanan -> Clave para los elementos externos, tubo de vacío, etc.



Linear response with the incoming photons.



Incluir lo de la intro de photosensores

%Conceptualmente, un fotomultiplicador cuenta con un fotocátodo y un multiplicador de electrones. El primero es una fina capa de un compuesto que emite electrones cuando absorbe fotones en el espectro visible o en las cercanı́as de él. Los electrones emitidos por el fotocátodo son llamados fotoelectrones. El segundo, de nombre sugestivo, es un arreglo de electrodos conectados a alta tensión que permite obtener ganancias de 10 6 . Más adelante en este capı́tulo, volveremos sobre los PMT.

\label{subsec:PMTs}
			%\newpage
		
			\subsection{Silicon Photomultiplier array (SiPMs array)}%\label{subsec:SiPMs}
			MPPC Multi-Pixel Photon Counter

Linear response with the incoming photons.

Ver apuntes generales comentados de photosensores en la intro de photosensores

Los tubos fotodetectores (PMT) son los dispositivos adecuados para esto, aunque, sin embargo, los avances tecnologicos de las ultimas decadas en tecnologia de semiconductores ha permitido el desarrollo de los fotodiodos de avalancha operados en modo Geiger (APD, Avalanche PhotoDiode)

Actualmente, la tecnologia de semiconductores ha permitido el desarrollo de fotodiodos que, por sus prestaciones, pueden reemplazar adecuadamente a los PMT. \label{subsec:SiPM}
			\newpage
		
	\section{Detector pulse analysis}%\label{sec:PulseAnalysis}
	\input{./Sections/3Design_Principles/35Pulse_Analysis}\label{sec:PulseAnalysis}
	\newpage
	
	\section{Tritium monitor parts}%\label{sec:PulseAnalysis}
	\input{./Sections/3Design_Principles/36Tritium_Monitor_Parts}\label{sec:TritiumMonitorParts}	
	\newpage
	
\chapter[Research \& Development]{Research \& Development on detector design and components}\label{chap:ResearchandDevelopment}
	\section{Introduction (why is important)}
	%\input{./Sections/3ResearchAndDevelopment/3IntroductionChapter} \label{sec:IntroChap}
	%\newpage
	
	\section{Characterization and R\&D on scintillating fibers}
	%\input{./Sections/3ResearchAndDevelopment/33ScintillatingFibers/332RyD_SF}\label{subsec:RyDSF}
		\newpage
		
	\section{Characterization and R\&D on the SiPM arrays}
				%\input{./Sections/3ResearchAndDevelopment/34Photosensors/342SiliconPhotoMultiplier/3422RyD_SiPM}\label{subsubsec:RyDSiPM}
			\newpage

\chapter{Tritium Monitor prototypes}\label{chap:Prototypes}		
	\section[Preliminary prototypes]{Preliminary prototypes, TRITIUM-IFIC 0 and TRITIUM-IFIC 1}\label{Preliminary_prototypes}
		\subsection{Tritium-IFIC 0}
		%\input{./Sections/4Prototypes/41TritiumIFIC0}\label{subsec:TritiumIFIC0}
		\newpage
		
		\subsection{Tritium-IFIC 1}
		%\input{./Sections/4Prototypes/42TritiumIFIC1}\label{subsec:TritiumIFIC1}
		\newpage
		
		\subsection{Tritium-Aveiro}
		%\input{./Sections/4Prototypes/43TritiumAveiro}\label{subsec:TritiumAveiro}
		\newpage
		
	\section[Tritium-IFIC 2]{Advanced prototype, Tritium-IFIC 2}
		%\input{./Sections/4Prototypes/44TritiumIFIC2}\label{sec:TritiumIFIC2}
		\newpage
		
	\section[Modular TRITIUM prototype]{Modular TRITIUM prototype for in-situ tritium monitoring}
		%\input{./Sections/4Prototypes/45TritiumIFICMonitor}\label{sec:TritiumMonitor}
		\newpage
		
\chapter[Background Shields]{Tritium Monitor Background Shields}\label{chap:Shields}
	\section{Tritium Monitor Background}
	%\input{./Sections/5Shields/51IntroductionShields}\label{sec:IntroductionShields}
	\newpage
		
	\section{Passive shield (Lead)}
	%\input{./Sections/5Shields/52PasiveShields}\label{sec:PasiveShields}
	%\newpage
		
		\subsection{Introduccion}
		%\input{./Sections/5Shields/52PasiveShields}\label{sec:PasiveShields}
		%\newpage
		
		\subsection{Set up of the Passive shield}
		%\input{./Sections/5Shields/52PasiveShields}\label{sec:PasiveShields}
		%\newpage
		
		\subsection{Measurements with the Passive shield}
		%\input{./Sections/5Shields/52PasiveShields}\label{sec:PasiveShields}
		\newpage	
	
	\section{Active shield (cosmic veto)}
	%\input{./Sections/5Shields/53ActiveShields}\label{sec:ActiveShield}
	%\newpage
	
		\subsection{Introduccion}
		%\input{./Sections/5Shields/52PasiveShields}\label{sec:PasiveShields}
		%\newpage
		
		\subsection{Set up of the Active shield}
		%\input{./Sections/5Shields/52PasiveShields}\label{sec:PasiveShields}
		%\newpage
		
		\subsection{Measurements with the Active shield}
		%\input{./Sections/5Shields/52PasiveShields}\label{sec:PasiveShields}
		\newpage
	
\chapter{Ultrapure water system}\label{chap:Ultrapure}
	\section{Introduction}
	%\input{./Sections/6Shields/61IntroductionUltraPure}\label{sec:IntroductionUltrapure}
	\newpage
	
	\section{Set up of the Ultrapure water System}
	%\input{./Sections/6Shields/62SetupUltraPure}\label{sec:SetupUltrapure}
	\newpage
	
	\section{Measurements of the Ultrapure water System}
	%\input{./Sections/6Shields/63ResultsUltraPure}\label{sec:ResultsUltrapure}
	\newpage
	
\chapter[Results and Discussion]{TRITIUM Monitor results and Discussion}\label{chap:Results}
	\section{Results from Laboratory measurements}
	%\input{./Sections/7Results/71Results_prototypes}\label{sec:ResultsPrototypes}
	\newpage
		
	\section{Results from measurements at Arrocampo Dam}
	%\input{./Sections/7Results/72ResultsArrocampo}\label{sec:ResultsArrocampo}
	\newpage	

\chapter{Simulations}  \label{chap:Simulations}
%\input{./Sections/8Simulations}
\newpage	

\chapter{Conclusions and prospects}  \label{chap:Conclusions}
%\input{./Sections/9Conclusions}
\newpage



%\newpage
%\chapter{Scintillator fibers} \label{chap:GeneralFibers}
%\input{./Sections/4Fibras}
	%\section{Introduction}\label{sec:IntroductionFibers}
	%%\input{./Sections/5SiPM/52Equipo}
	
	%\section{Organic and inorganic scintillators}\label{sec:OrganicInorganicFibers}
	%\input{./Sections/5SiPM/53CalibracionTarjeta}	

	%\section{Scintillator fibers}\label{sec:ScintillatorFibers}
	%\input{./Sections/5SiPM/53CalibracionTarjeta}	
	
	%\section{Choice of the comercial scintillator fibers}\label{sec:ChoiceFiber}
	%\input{./Sections/5SiPM/53CalibracionTarjeta}
	
	%\section{Cutting device for scintillator fiber}\label{sec:CuttingFibers}
	%%\input{./Sections/5SiPM/54Analisis}
	
	%\section{Polishing task for scintillator fiber}\label{sec:PolishingTask}
	%%\input{./Sections/5SiPM/55Temperatura}

	%\section{Automatic polishing machine for scintillator fiber}\label{sec:PolishingMachine}
	%%\input{./Sections/5SiPM/56Voltaje}
	
	%\section{Splicing machine for scintillator fiber}\label{sec:SplicingMachine}
	%%\input{./Sections/5SiPM/56Voltaje}
	
	%DE aquí para abajo todos los estudios realizados con fibras...
	%\section{Estabilización de la ganancia}\label{sec:Compensacion}
	%%\input{./Sections/5SiPM/57Compensacion}

%\chapter{Photomultiplier tubes (PMTs)} \label{sec:PMT}
%\input{./Sections/4Fibras}
	%\section{Introduction}\label{sec:IntroductionPMTs}
	%%\input{./Sections/5SiPM/56Voltaje}
	
	%\section{Calibration of the PMTs}\label{sec:CalibrationPMTs}
	%%\input{./Sections/5SiPM/56Voltaje}
	
	%\subsection{Gain calibration of the PMTs}\label{sec:GainCalibrationPMTs}
	%%\input{./Sections/5SiPM/56Voltaje}
	
	%\subsection{Aquí para abajo las demas calibraciones que haré con los PMTs}\label{sec:demáscalibraciones}
	%%\input{./Sections/5SiPM/56Voltaje}

%\newpage
%\chapter{Calibracion de los fotomultiplicadores de silicio (SiPM)} \label{chap:SiPM}
%%\input{./Sections/5SiPM/51IntroSiPM}
	%\section{Equipo y montaje experimental}\label{sec:Equipo}
	%%\input{./Sections/5SiPM/52Equipo}
	
	%\section{Características de la tarjeta}\label{sec:Tarjeta}
	%\input{./Sections/5SiPM/53CalibracionTarjeta}
	
	%\section{Análisis de datos}\label{sec:Analisis}
	%%\input{./Sections/5SiPM/54Analisis}
	
	%\section{Calibración en temperatura}\label{sec:Temperatura}
	%%\input{./Sections/5SiPM/55Temperatura}

	%\section{Calibración en voltaje de operación}\label{sec:Voltaje}
	%%\input{./Sections/5SiPM/56Voltaje}
	
	%\section{Estabilización de la ganancia}\label{sec:Compensacion}
	%%\input{./Sections/5SiPM/57Compensacion}

%\newpage
%\chapter{Prototipo} \label{chap:Prototipo}  
%%\input{./Sections/6Prototipos/61IntroPrototipos}
	%\section{Configuración del prototipo}\label{sec:Configuracion}
	%%\input{./Sections/6Prototipos/62Configuracionprototipo}
	
	%\section{Procedimiento de llenado}\label{sec:Llenado}
	%%\input{./Sections/6Prototipos/63Llenado}
	
	%\section{Configuración de la electrónica}\label{sec:Electronica}
	%%\input{./Sections/6Prototipos/64Configuracionelectronica}
	
	%\section{Resultados}\label{sec:Resultados}
	%%\input{./Sections/6Prototipos/65Resultados}

%\newpage
%\chapter{Simulaciones} \label{chap:Simulaciones}
%%\input{./Sections/7Simulaciones}

%\newpage
%\chapter{Previsiones de futuro} \label{chap:Futuro}
%%\input{./Sections/8Previsiones}

%\newpage
%\chapter{Resultados y conclusiones} \label{chap:Conclusiones}
%%\input{./Sections/9Conclusiones2}

\appendix
\appendixpage
\noappendicestocpagenum
\addappheadtotoc

\chapter{Más cosas}\label{App:A}
Aún faltan cosas por decir.

\chapter{Y más cosas aún}\label{App:B}
Y más cosas aún.

%\chapter{Bibliografía} \label{chap:bibliographia}
%\section {Bibliografía}
\begin{thebibliography}{100}
%Reference 1
\bibitem{IAEA} \textsc{IAEA}, 
\textit{The International Atomic Energy Agency} \href{https://www.iaea.org/}{\textbf{Webpage}}. 

%Reference 2
\bibitem{UNSCEAR} \textsc{UNSCEAR}, 
\textit{The United Nations Scientific Committee on the Effects of Atomic Radiation} \href{https://www.unscear.org/}{\textbf{Webpage}}. 

%Reference 3
\bibitem{CSN} \textsc{CSN}, 
\textit{Consejo de Seguridad Nuclear, Spain} \href{https://www.csn.es/home}{\textbf{Webpage}}.

%Reference 4
\bibitem{ICRU} \textsc{ICRU}, 
\textit{Internation Commission of Radiological Units and Measurements} \href{https://www.icru.org/}{\textbf{Webpage}}.

%Reference 5
\bibitem{ICRP} \textsc{ICRP}, 
\textit{International Commission on Radiololgical Proteccion} \href{https://www.icrp.org/}{\textbf{Webpage}}.

%Reference 6
\bibitem{ISR} \textsc{ISR}, 
\textit{International Society of Radiology} \href{https://www.isradiology.org/}{\textbf{Webpage}}.

%Reference 7
\bibitem{UN} \textsc{UN}, 
\textit{United Nations} \href{https://www.un.org/en/}{\textbf{Webpage}}. 

%Reference 8
\bibitem{REA} \textsc{CSN}, 
\textit{Red de Estaciones Automáticas, REA} \href{https://www.csn.es/mapa-de-valores-ambientales}{\textbf{Webpage}}. 

%Reference 9
\bibitem{REM} \textsc{CSN}, 
\textit{Red de Estaciones de Muestreo, REM} \href{https://www.csn.es/kprgisweb2/index.html?lang=es}{\textbf{Webpage}}. 

%Reference 10
\bibitem{100BqL}  
\href{https://eur-lex.europa.eu/eli/dir/2013/59/oj}{\textit{Council directive 2013/15/euratom}}.

%Reference 11
\bibitem{FiberDetector1a} \textsc{J. W. Berthold}, \textsc{L. A. Jeffers},
\href{https://www.osti.gov/biblio/2225-phase-final-report-situ-tritium-beta-detector}{\textit{Phase 1 Final Report for In-Situ Tritium Beta Detector}}, 
U. S. Department of Energy, McDermott Technology, Inc.,Research and Development Division, 	\textbf{DE-AC21-96MC33128}, April, 1998.

%Reference 12
\bibitem{FiberDetector1b} \textsc{J. W. Berthold}, \textsc{L. A. Jeffers}, 
\href{https://www.osti.gov/biblio/836625-MxOOUa/native/}{\textit{In Situ Tritium Beta Detector}}, U. S. Department of Energy, McDermott Technology, Inc. (MTI), Technology development data sheet, \textbf{DE-AC21-96MC33128}, May, 1999.

%Reference 13
\bibitem{CommonEmissionTritium} \textsc{X- Hou},  
\textit{Tritium and \ce{^{14}C} in the environmental and nuclear facilities: Sources and analytical methods}, Journal of the Nuclear Fuel Cycle and Waste Technology (JNFCWT), 16 (2018), 11-39 \href{https://doi.org/10.7733/jnfcwt.2018.16.1.11}{\textbf{DOI: 10.7733/jnfcwt.2018.16.1.11}}.

%Reference 14
\bibitem{CrossSeccionNeutrons}  
\textit{REFERENCIAAAAAAAA}.

%Reference 15
\bibitem{PercentageEnergySpain}
\href{https://www.ree.es/es/datos/publicaciones/informe-anual-sistema/informe-del-sistema-electrico-espanol-2019}{\textit{Avance del informe del sistema eléctrico español, 2019}}, 
\textbf{Red eléctrica española}.

%Reference 16
\bibitem{60ReactorsChina}
\href{https://www.europapress.es/internacional/noticia-china-construira-menos-60-centrales-nucleares-proxima-decada-20160916210159.html}{\textit{China construirá 60 centrales nucleares en la próxima década}}, 
\textbf{Europa press}.

%Reference 17
\bibitem{35MillionsUSA}
\href{https://www.energynews.es/estados-unidos-centrales-nucleares/}{\textit{Inversión de EE. UU. de 35 millones para centrales nucelares}}, \textbf{Energy News}

%Reference 18
\bibitem{ThreeMileIsland}
\href{www.world-nuclear.org/information-library/safety-and-security/safety-of-plants/three-mile-island-accident.aspx}{\textit{Three mile island accident}}, \textbf{World Nuclear Association}.

%Reference 19
\bibitem{EIAOutlook}
\textit{International Energy Outlook 2013}. \href{https://www.eia.gov/outlooks/ieo/}{\textbf{U. E. Energy Information Administration}}.

%Reference 20
\bibitem{FERMILAB}
\href{https://www.fnal.gov/pub/tritium/}{Tritium at Fermilab}.

%Reference 21
\bibitem{BrookHavenNationalLaboratory}
\href{https://www.bnl.gov/hfbr/decommission.php}{\textbf{Brookhaven National Laboratory (BNL)}}.

%Reference 22
\bibitem{TrackingTritium} \textsc{Aleksandra Sawodni}, \textsc{Anna Pazdur}, \textsc{Jacek Pawlyta}, 
\href{http://yadda.icm.edu.pl/baztech/element/bwmeta1.element.baztech-article-BAT3-0035-0005}{\textit{Measurements of Tritium Radioactivity in Surface Water on the Upper Silesia Region}}, Journal on Methods and Applications of Absolute Chronology, Geochronometria, Vol. 18, pp 23-28 \textbf{2000}.

%Reference 23
\bibitem{TritiumDiscovery} \textsc{M. L. Oliphant}, \textsc{P. Harteck} and \textsc{E. Rutherford},  
\href{https://royalsocietypublishing.org/doi/10.1098/rspa.1934.0077}{\textit{Transmutation Effects observed with Heavy Hydrogen}}, Nature, 133, 413 (1934)\href{https://doi.org/10.1038/133413a0}{\textbf{DOI: 10.1038/133413a0}}.

%Reference 24
\bibitem{TritiumIsolate} \textsc{Luis W. Alvarez} and \textsc{R. Cornog},  
\textit{Helium and Hydrogen of Mass 3}, Physical Review Journals Archive, 56, 613 (1939)\href{https://doi.org/10.1103/PhysRev.56.613}{\textbf{DOI: 10.1103/PhysRev.56.613}}.

%Reference 25
\bibitem{TritiumHandling} 
\href{https://www.twirpx.com/file/1977676/}{\textit{DOE Handbook: Primer on Tritium Safe Handling Practices}}, U. S. Departament Of Energy Washington, D.C. 20585.

%Reference 26
\bibitem{OxigenTritium} \textsc{Robert Haight}, \textsc{Joseph Wermer} and \textsc{Michael Fikani},
\textit{Tritium Production by Fast Neutrons on Oxygen: An Integral Experiment}, Journal of Nuclear Science and Technology, 39:sup2, 1232-1235, \href{https://doi.org/10.1080/00223131.2002.10875326}{\textbf{DOI: 10.1080/00223131.2002.10875326}}. 

%Reference 27
\bibitem{FranceTritiumEnvironment} \textsc{Institut de Radioprotection et de Sureté Nucléaire}
\textit{Tritium and the environment}, \href{https://www.irsn.fr/EN/Research/publications-documentation/radionuclides-sheets/environment/Pages/Tritium-environment.aspx}{\textit{Tritium and the environment}}, IRSN, Enhancing nuclear safety. 

%Referencia 28
\bibitem{CrossSeccionNeutrino} \textsc{},
\textit{REFERENCIAAAA}, \textbf{}

%Referencia 29
\bibitem{TritiumDecayEnergyLevels} 
\href{https://www-nds.iaea.org}{\textit{International Atomic Energy Agency}}.

%Referencia 30
\bibitem{TritiumDecayImage} 
\href{https://conexioncausal.wordpress.com}{\textit{Tritium decay image}}.

%Referencia 31
\bibitem{TritiumEspectrum} \textsc{Zhang Lin},
\href{https://www.mdpi.com/2073-4352/10/2/105/htm}{\textit{Simulation and Optimization Design of SiC-Basaed PN Betavoltaic Microbattery Using Tritium Source}}, MDPI Open Access Journal \textit{12/02/2020}, \textbf{DOI:10.3390/cryst10020105}

%Reference 32
\bibitem{MeanFreePathDocument} \textsc{Blauvelt, R.K.}, \textsc{Deaton, M.R.} and \textsc{Gill, J.T.},
\textit{Health Physics Manual of Good Practices for Tritium Facilities}, EG and G Mound Applied Technologies, Miamisburg, OH (United States), Technical Report,  01 December 1991, \href{https://doi.org/10.2172/266889}{\textbf{DOI: 10.2172/266889}}. 

%Reference 33
\bibitem{EstimationTritiumDosi} \textsc{Tsuyoshi Masuda} and \textsc{Toshitada Yoshioka},
\textit{Estimation of radiation dose from ingested tritium in humans by administration of deuterium-labelled compounds and food}, Scientific reports, 02 Febrary 2021, \href{https://doi.org/10.1038/s41598-021-82460-5}{\textbf{DOI: 10.1038/s41598-021-82460-5}}. 

%Reference 34
\bibitem{EstimationTritiumDosiRats} \textsc{Z. Pietrzak-Flis}, \textsc{I. Radwan}, \textsc{Z. Major} and \textsc{M. Kowalska},
\textit{Tritium Incorporation in Rats Chronically Exposed to Tritiated Food or Tritiated Water for Three Successive Generations}, Journal of Radiation Research, Vol 22, Issue 4, December 1981, page 434-442 \href{https://doi.org/10.1269/jrr.22.434}{\textbf{DOI: 10.1269/jrr.22.434}}. 

%Reference 35
\bibitem{EstimationTritiumDosiKangarooRats} \textsc{J.R. Martin} and \textsc{J.J. Koranda},
\textit{Biological Half-Life Studies of Tritium in Chronically Exposed Knagaroo Rats}, Journal of Radiation Research, Vol 50, Issue 2, May 1972, page 426-440 \href{https://www.jstor.org/stable/3573500?seq=1#metadata_info_tab_contents}{\textbf{PMID: 5025235}}. 

%Reference 36
\bibitem{StraumeTritiumHazard} \textsc{T Straume} and \textsc{A. L. Carsten},
\textit{Tritium radiobiology and relative biological effectiveness}, Health Physics, Vol. 65, Number 6, December 1993, \href{https://pubmed.ncbi.nlm.nih.gov/8244712/}{\textbf{DOI: 10.1097/00004032-199312000-00005 }}. 

%Reference 37
\bibitem{RytoemaaTritiumHazard} \textsc{Rytoemaa, T.}, \textsc{Saltevo, J.} and \textsc{Toivonen, H.},
\href{http://inis.iaea.org/search/search.aspx?orig_q=RN:11535484}{\textit{Radiotoxicity of Tritium-Labelled Molecules}}, International Atomic Energy Agency symposium, IAEA, Vienna: Biological Implications of Radionuclides Released from Nuclear Industries, INIS Vol. 11, INIS Issue. 13, Reference Number, 11535484, 1979. 

%Reference 38
\bibitem{ICRP_GL} \textsc{International Commission on Radiological Protection, ICRP},
\href{https://www.icrp.org/publication.asp?id=icrp\%20publication\%2060}{\textit{Recommendations of the ICRP. Annals of the ICRP, 21(1.3), 1991a. 1990. Oxford, Pergamon Press (Publication 60).}}. 

%Reference 39
\bibitem{WHO_GL} \textsc{World Health Organization, WHO}, 
\href{http://www.who.int/water_sanitation_health/dwq/
GDWQ2004web.pdf}{\textit{Guidelines for Drinking-Water Quality. Vol 1. Third Edition. Geneve, Switzerland, 2004}}. 

%Reference 40
\bibitem{ICRP_factor} \textsc{International Commission on Radiological Protection, ICRP}, 
\href{https://www.icrp.org/publication.asp?id=ICRP\%20Publication\%2072}{\textit{Age-dependent doses to members of the public from intake of radionuclides: Part 5. Compilation of ingestion and inhalation dose coefficients. Oxford, Pergamon Press (International Commission on Radiological Protection Publication 72), 1996}}. 

%Reference 41
\bibitem{Switzerland_GL} \textsc{Département fédéral de l'intérieur, DFI (Federal Department of the Interior)},
\href{www.admin.ch/ch/f/rs/8/817.021.23.fr.pdf}{\textit{Ordonnance du DFI sur les substances etrangères et les composants dans les denrées alimentaires (817.021.23)}}, 2006, Switzerland (in French).

%Reference 42
\bibitem{Ontario_GL} \textsc{Ontario Ministry of the Environment},
\href{https://atrium.lib.uoguelph.ca/xmlui/handle/10214/15832}{\textit{Ontario Drinking Water Objectives. Toronto, Ontario, 1994}}. 

%Reference 43
\bibitem{Quebec_GL} \textsc{Québec},
\href{https://numerique.banq.qc.ca/patrimoine/details/52327/3582272?docref=fxoJ-qgA5cus5Upw-L_NHg}{\textit{Résultats du programme de surveillance de l’environnement du site de Gentilly. Rapport annuel 2006. Québec, Canada.}}. 

%Reference 44
\bibitem{Russia_GL} \textsc{Russia},
\href{http://www.wdcb.ru/mining/zakon/NRB99.htm}{\textit{NRB-99 Radiation Safety Norms}}, 2007. 

%Reference 45
\bibitem{Australia_GL} \textsc{Australian Government}, \textsc{National Health and Medical Reserch Council} and \textsc{Natural Resource Management Ministerial Council},
\href{https://www.nhmrc.gov.au/about-us/publications/australian-drinking-water-guidelines}{\textit{AustralianDrinking Water Guideilnes 6}}, National Water Quality Managment Strategy,Version 3.6, Updated March 2011. 

%Reference 46
\bibitem{Finland_GL} \textsc{Nuclear Energy Agency, NEA},
\href{https://www.oecd-nea.org/jcms/pl_23551/finland}{\textit{Radiation and Nuclear Safety Authority}}, 1993. Radioactivity of Household Water. ST 12.3. Erweko Paintuote, Helsinki, Finland, 1994. 

%Reference 47
\bibitem{California_GL} \textsc{Office of Environmental HEalth Hazard Assessment, OEHHA},
\href{https://oehha.ca.gov/water/public-health-goal/public-health-goals-six-chemicals-drinking-water}{\textit{Public Health Goals for Chemicals in Drinking Water-Tritium. OEHHA, California ENfironmental Protection Agency, California USA, September, 2007}}. 

%Reference 48
\bibitem{USEPA_GL} \textsc{United States Environmental Protection Agency, US EPA},
\href{https://www.epa.gov/dwreginfo/radionuclides-rule}{\textit{Drinking Water Requirements for States and Public Water Systems}}, Radionuclides Rule, 1976. 

%Reference 49
\bibitem{France_GL} \textsc{Institut de radioprotection et de sûreté nucléaire, IRSN (Radioprotection and Nucelar Safety Institute)},
\href{https://www.google.com/url?sa=t&rct=j&q=&esrc=s&source=web&cd=&ved=2ahUKEwiskum8mYLwAhXLB2MBHWLgAkoQFjAAegQIBBAD&url=https\%3A\%2F\%2Fwww.actu-environnement.com\%2Fmedia\%2Fpdf\%2Fnews-32705-bilan.pdf&usg=AOvVaw0oCSJP78IgV1Tek0T4_6z1}{\textit{Bilan de l’état radiologique  de l’environnement français  de 2015 à 2017}}. France. 

%Reference 50
\bibitem{Germany_GL} \textsc{Bundesamt für Strahlenschutz, BMU (Federal Office for Radiation Protection)},
\href{http://doris.bfs.de/jspui/handle/urn:nbn:de:0221-20100331990}{\textit{ Environmental Radioactivity and Radiation Exposure}}, Annual Report, 2005, (Jahresbericht 2005). BMU, Bonn, Germany (in German). 

%Reference 51
\bibitem{Spain_GL} \textsc{Consejo de Seguridad Nuclear, CSN, Nuclear Safety Council},
\href{https://www.csn.es/en/normativa-del-csn/normativa-espanola}{\textit{National Regulation of Radionuclides}}. 

%Reference 52
\bibitem{EURATOM_GL} \textsc{European Atomic Energy Community, EURATOM},
\href{https://eur-lex.europa.eu/eli/dir/2013/59/oj}{\textit{Council directive 2013/15/euratom}}, October, 2013. Laying down requirements for the protection of the health of the general public with regard to radioactive substances in water intended for human consumption. 

%Referencia 53
\bibitem{LSCothers} \textsc{M. N. Al-Haddad}, \textsc{A. H. Fayoumi} and \textsc{F. A. Abu-Jarad},
\textit{Calibration of a liquid scintillation counter to assess tritium levels in various samples}, Nuclear Instruments and Methods in PHysics Research A, Volume 438, Issues 2-3, December 1999, Pages 356-361, \href{https://doi.org/10.1016/S0168-9002(99)00272-7}{\textbf{DOI: 10.1016/S0168-9002(99)00272-7}}.

%Referencia 54
\bibitem{HofstetterSeveral} \textsc{K. J. Hofstetter} and \textsc{H. T. Wilson},
\textit{Aqueous Effluent Tritium Monitor Development}, Fusion Technology, Volume 21, 2P2, Pages 446-451, March 1992, \href{https://doi.org/10.13182/FST92-A29786}{\textbf{DOI: 10.13182/FST92-A29786}}.

%Referencia 55
\bibitem{0.6Bq_L} \textsc{M. Palomo}. \textsc{A. Peñalver}, \textsc{C. Aguilar} and \textsc{F. Borrull},
\textit{Tritium activity levels in environmental water samples from different origins}, Applied Radiation and Isotopes, Volume 65, Issue 9, September 2007, Pages 1048-1056, \href{https://doi.org/10.1016/j.apradiso.2007.03.013}{\textbf{DOI: 10.1016/j.apradiso.2007.03.013}}.

%Referencia 56
\bibitem{OnlineLSC} \textsc{R. A. Sigg}, \textsc{J. E. McCarty}, \textsc{R. R. Livingston} and \textsc{M. A. Sanders},
\textit{Real-time aqueous tritium monitor using liquid scintillation counting}, FNuclear Instrument and Methods in Physics Research A, Volume 353, Issues 1-3, 30 Decembre 1994, Pages 494-498 \href{https://doi.org/10.1016/0168-9002(94)91707-8}{\textbf{DOI: 10.1016/0168-9002(94)91707-8}}.

%Referencia 57
\bibitem{IonizationChamber1} \textsc{N. P. Kherani},
\textit{An alternative approach to tritium-in-water monitoring}, Nuclear and Methods in PHysics Research A, Volume 484, Issues 1-3, 21 May 2002, Pages 650-659 \href{https://doi.org/10.1016/S0168-9002(01)02008-3}{\textbf{DOI: 10.1016/S0168-9002(01)02008-3}}

%Referencia 58
\bibitem{IonizationChamber2} \textsc{Z. Chen}, \textsc{S. Peng}, \textsc{D. Meng} \textsc{Y. He} and \textsc{H. Wang},
\textit{Theoretical study of energy deposition in ionization chambers for tritium measurements}, Review of Scientific Instruments, 84, 103302, 2013, \href{https://dx.doi.org/10.1063/1.4825032}{\textbf{DOI: 10.1063/1.4825032}}.

%Referencia 59
\bibitem{Calorimeter1} \textsc{C. G. Alecu}, \textsc{U. Besserer}, \textsc{B. Bornschein}, \textsc{B. Kloppe}, \textsc{Z. Köllö} and \textsc{J. Wendel},
\textit{Reachable Accuracy and Precision for Tritium Measurements by Calorimetry at TLK}, Fusion Science and Technology, 60:3, 937-940, \href{https://doi.org/10.13182/FST11-A12569}{\textbf{DOI: 10.13182/FST11-A12569}}.

%Referencia 60
\bibitem{Calorimeter2} \textsc{A. Bükki-Deme}, \textsc{C. G. Alecu}, \textsc{B. Kloppe} and \textsc{B. Bornschein},
\textit{First results with the upgraded TLK tritium calorimeter IGC-V0.5}, Fusion Engineering and Design, Volume 88, Issue 11, November 2013, Pages 2865-2869 \href{https://doi.org/10.1016/j.fusengdes.2013.05.066}{\textbf{DOI: 10.1016/j.fusengdes.2013.05.066}}.

%Referencia 61
\bibitem{XRays1} \textsc{M. Matsuyama}, \textsc{Y. Torikai}, \textsc{M. Hara} and \textsc{K. Watanabe},
\textit{New Technique for non-destructive measurements of tritium in future fusion reactors}, IAEA Nuclear Fusion, Volume 47, Number 7, S464, June 2007, \href{https://doi.org/10.1088/0029-5515/47/7/S09}{\textbf{DOI: 10.1088/0029-5515/47/7/S09}}.

%Referencia 62
\bibitem{XRays2} \textsc{M. Matsuyama},
\textit{Development of a new detection system for monitoring high-level tritiated water}, Fusion Engineering and Design, Volume 83, Issue 10-12, December 2008, Pages 1438-1441 \href{https://doi.org/10.1016/j.fusengdes.2008.05.023}{\textbf{DOI: 10.1016/j.fusengdes.2008.05.023}}.

%Referencia 63
\bibitem{Bremstrahlung} \textsc{S. Niemes}, \textsc{M. Sturm}, \textsc{R. Michling} and \textsc{B. Bornschein},
\textit{High Level Tritiated Water Monitoring by Bremsstrahlung Counting Using a Silicon Dift Detector}, Fusion Science and Technology, 67:3, 507-510, 2015, \href{https://doi.org/10.13182/FST14-T66}{\textbf{DOI: 10.13182/FST14-T66}}.

%Referencia 64
\bibitem{APD} \textsc{K. S. Shah}, \textsc{P. Gothoskar}, \textsc{R. Farrell} and \textsc{J. Gordon},
\textit{High Efficiency Detection of Tritium Using Silicon Avalanche Photodiodes}, IEEE Transactions on Nuclear Science, Volume 44, Issue 3, June 1997, \href{https://doi.org/10.1109/23.603750}{\textbf{DOI: 10.1109/23.603750}}

%Referencia 65
\bibitem{Spectrometry} \textsc{P. Jean-Baptiste}, \textsc{E. Fourré}, \textsc{A. Dapoigny}, \textsc{D. Baumier}, \textsc{N. Baglan} and \textsc{G. Alanic},
\textit{\ce{^{3}He} mass spectrometry for very low-level measurement of organic tritium in environmental samples}, Journal of Environmental Radioactivity, Volume 101, Issue 2, Febrary 2010, Pages 185-190, \href{https://doi.org/10.1016/j.jenvrad.2009.10.005}{\textbf{DOI: https://doi.org/10.1016/j.jenvrad.2009.10.005}}. 

%Referencia 66
\bibitem{Ring} \textsc{C. Bray}, \textsc{A. Pailoux} and \textsc{S. Plumeri},
\textit{Tritiated water detection in the 2.17 $\mu$M spectral region by cavity ring down spectroscopy},  Nuclear Instruments and Methods in Physics Research A, Volume 789, 21 July 2015, Pages 43-49, \href{https://doi.org/10.1016/j.nima.2015.03.064}{\textbf{DOI: 10.1016/j.nima.2015.03.064}}. 

%Referencia 67
\bibitem{Muramatsu} \textsc{M. Muramatsu}, \textsc{A. Koyano} and \textsc{N. Tokanuga},
\textit{A Scintillation Probe for Continuous Monitoring of Tritiated Water}, Nuclear Instruments and Methods, Volume 54, Issue 2, October 1967, Page 325-326, \href{https://doi.org/10.1016/0029-554X(67)90645-3}{\textbf{DOI: 10.1016/0029-554X(67)90645-3}}.

%Referencia 68
\bibitem{Moghissi} \textsc{A. A. Moghissi}, \textsc{H. L. Kelley}, \textsc{C. R. Phillips} and \textsc{J. E. Regnier},
\textit{A Tritium Monitor Based on Scintillation}, Nuclear Instruments and Methods, Volume 68, Issue 1, 1 Febrary 1969, Page 159, \href{https://doi.org/10.1016/0029-554X(69)90705-8}{\textbf{DOI: 10.1016/0029-554X(69)90705-8}}.

%Referencia 69
\bibitem{Osborne} \textsc{R. V. Osborne},
\textit{Detector for Tritium in Water}, Nuclear Instruments and Methods, Volume 77, Issue 1, 1 January 1970, Page 170-172, \href{https://doi.org/10.1016/0029-554X(70)90596-3}{\textbf{DOI: 10.1016/0029-554X(70)90596-3}}.

%Referencia 70
\bibitem{Ratnakaran} \textsc{A. N. Singh}, \textsc{M. Ratnakaran} and \textsc{K. G. Vohra},
\textit{An Online Tritium-in-Water Monitor}, Nuclear Instruments and Methods, Volume 236, Issue 1, 1 May 1985, Page 159-164, \href{https://doi.org/10.1016/0168-9002(85)90141-X}{\textbf{DOI: 10.1016/0168-9002(85)90141-X}}.

%Referencia 71
\bibitem{Ratnakaran2000} \textsc{M. Ratnakaran}, \textsc{R. M. Revetkar}, \textsc{R. K. Samant} and \textsc{M. C. Abani},
\href{https://inis.iaea.org/search/search.aspx?orig_q=RN:32015986}{\textit{A Real-time Tritium-In-Water Monitor for Measurement Of Heavy Water Leak To The Secondary Coolant}}, International congress of the International Radiation Protection Association, Volume 32, Issue 15, 14-19 May 2000, P-3a-197, Reference number: \textbf{32015986}

%Referencia 72
\bibitem{Hofstetter1} \textsc{K. J. Hofstetter} and \textsc{H. T. Wilson},
\textit{Aqueous Effluent Tritium Monitor Development}, Fusion Technology, Volume 21, 2P2, 1992, Pages 446-451, \href{https://doi.org/10.13182/FST92-A29786}{\textbf{DOI: 10.13182/FST92-A29786}}.

%Referencia 73
\bibitem{Hofstetter2} \textsc{K. J. Hofstetter} and \textsc{H. T. Wilson},
\href{https://www.osti.gov/biblio/6865647-continuous-tritium-effluent-water-monitor-savannah-river-site}{\textit{Continuous Tritium Effluent Water Monitor at the Savannah River Site}}, International conference on advances in liquid scintillation, Vienna (Austria), 14-18 September 1992.

%Referencia 74
\bibitem{TRITIUM} \textit{Tritium, Interreg Sudoe Program}. 
\href{https://tritium-sudoe.eu/es-es/homepage}{\textbf{Tritium website}}.

%Referencia 75
\bibitem{Geant4WebPage} \textsc{CERN Collaboration},
\textit{Geant4: A toolkit for the simulation of the passage of particles through matter.}. \href{https://geant4.web.cern.ch/node/1}{\textbf{Website}}.

%Referencia 76
\bibitem{Knoll} \textsc{Glenn F. Knoll}, 
\textit{Radiation Detection and Measurement}, Third Edition, John Wiley and Sons, Inc. 1999.

%Referencia 77
\bibitem{Leo} \textsc{William R. Leo},
\textit{Techniques for Nuclear and Particle Physics Experiments: a how-to approach}, Second Revised Edition, Springer-Verlag Berlin Heidelberg GmbH, 1994. \href{https://doi.org/10.1007/978-3-642-57920-2}{\textbf{DOI: 10.1007/978-3-642-57920-2}}. 

%Referencia 78
\bibitem{DataSheetBCF12Fiber} \textsc{Saint-Gobain Ceramics and Plastics, Inc.},
\textit{Scintillating Optical Fibers}, It's What's Inside that Counts, 2005-14. \href{https://www.crystals.saint-gobain.com/products/scintillating-fiber}{\textbf{Data sheet}}. 

%Referencia 79
\bibitem{TFGAlberto} \textsc{},
\textit{}, . \href{}{\textbf{}}. 

%Referencia 80
%\bibitem{DataSheetKuraray}
%\textit{Plastic Scintillating Fibers}, Scintillating Fibers, Wavelength Shifting Fibers and Clear Fibers. \href{https://www.kuraray.com/products/psf}{\textbf{Data sheet}}. 

%Referencia 80
\bibitem{Snell} \textsc{},
\textit{}, \href{}{\textbf{}}. 

%Referencia 81
\bibitem{DataSheetPMTs} \textsc{HAMAMATSU PHOTONICS K.K.},
\textit{Photonmultiplier tube R8520-406/R8520-506}. \href{https://www.hamamatsu.com/eu/en/product/type/R8520-406/index.html}{\textbf{Data sheet}}.

%Referencia 82
\bibitem{CalibrationPMTsNEXT} \textsc{Javier Pérez Pérez},
 \href{https://next.ific.uv.es/cgi-bin/DocDB/public/ShowDocument?docid=48}{\textit{Caracterización de los Fotomultiplicadores R8520-06SEL para NEXT}, 25-06-2010}.

%Referencia 83
\bibitem{TesisNEXTSiPMs} \textsc{David Lorca Galindo},
\href{https://dialnet.unirioja.es/servlet/tesis?codigo=101465}{\textit{Tesis: SiPM based tracking for detector calibration in NEXT}}, Departamento de física atómica, molecular y nuclear, Universidad de Valencia (UV), Valencia, Spain, \textit{03/2015}.

%Referencia 84
\bibitem{OSI} \textsc{OSI Optoelectronics}, 
\href{https://osioptoelectronics.com/standard-products/default.aspx?gclid=EAIaIQobChMIkYrLif_37QIVDNTtCh3NuwpkEAAYASAAEgKMJ_D_BwE}{\textit{Characteristics and Applications}}.

%Referencia 85
\bibitem{DataSheetHammamatsu_1_SiPM_50} \textsc{HAMAMATSU PHOTONICS K.K. Solid State Division},
\textit{MPPC Multi-Pixel Photon Counter S13360-6050}. \href{https://www.hamamatsu.com/eu/en/product/type/S13360-6050CS/index.html}{\textbf{Data sheet}}.

%Referencia 86
\bibitem{DataSheetHammamatsu_1_SiPM_75} \textsc{HAMAMATSU PHOTONICS K.K. Solid State Division},
\textit{MPPC Multi-Pixel Photon Counter S13360-6075}. \href{https://www.hamamatsu.com/eu/en/product/type/S13360-6075CS/index.html}{\textbf{Data sheet}}.

%Referencia 87
\bibitem{DataSheetHammamatsu_array_SiPM_6050} \textsc{HAMAMATSU PHOTONICS K.K. Solid State Division},
\textit{MPPC Multi-Pixel Photon Counter S13361-6050}. \href{https://www.hamamatsu.com/us/en/product/type/S13361-6050AE-04/index.html}{\textbf{Data sheet}}.

%Referencia 88
\bibitem{DataSheetHammamatsu_array_SiPM_3050} \textsc{HAMAMATSU PHOTONICS K.K. Solid State Division},
\textit{MPPC Multi-Pixel Photon Counter S13361-3050}. \href{https://www.hamamatsu.com/jp/en/product/type/S13361-3050AE-08/index.html}{\textbf{Data sheet}}.

%Referencia 89
\bibitem{DataSheetSensL} \textsc{SensL sense light},
\textit{Introduction to the SPM TECHNICAL NOTE}. February 2017 \href{https://sensl.com/}{\textbf{Document}}.

%Referencia 90
\bibitem{DataSheetKeithley6487} \textsc{KEITHLEY, a greater measure of confidence},
\textit{Model 6487 Picoammeter/voltage source, Manual reference}. \href{https://pdf.directindustry.com/pdf/keithley-instruments/6487-picoammeter-voltage-source/1438-619876.html}{\textbf{Data sheet}}.

%Referencia 91
\bibitem{DataSheetHVSupplyTennelec} \textsc{Tennelec},
\textit{Model TC 952 High Voltage Supply, Manual reference}. \href{https://groups.nscl.msu.edu/nscl_library/manuals/tennelec/tennelec.htm}{\textbf{Data sheet}}.

%Referencia 92
\bibitem{DataSheetHVSupplyWenzel} \textsc{Wenzel Electronik},
\textit{Model N 1330-4 High Voltage Power Supply}. \href{https://wenzel-elektronik.de}{\textbf{Website}}.

%Referencia 93
\bibitem{DataSheetFANINOUT} \textsc{Philips Scientific},
\textit{Model 740 Quad Linear Fan-In/Out, Manual reference}. \href{https://prep.fnal.gov/catalog/hardware_info/phillips_scientific/740.html}{\textbf{Data sheet}}.

%Referencia 94
\bibitem{DataSheetPreAmp} \textsc{ORTEC},
\textit{Model 9326 FastPreamplifier, Manual reference}. \href{https://www.ortec-online.com/products/electronics/preamplifiers/9326}{\textbf{Data sheet}}.

%Referencia 95
\bibitem{DataSheet575Amp} \textsc{ORTEC},
\textit{Model 575A Amplifier, Manual reference}. \href{https://www.ortec-online.com/products/electronics/amplifiers/575a}{\textbf{Data sheet}}.

%Referencia 96
\bibitem{DataSheet671Amp} \textsc{ORTEC},
\textit{Model 671 Spectroscopy Amplifier, Manual reference}. \href{https://www.ortec-online.com/products/electronics/amplifiers/671}{\textbf{Data sheet}}.

%Referencia 97
\bibitem{DataSheetDiscriminator} \textsc{ORTEC},
\textit{Model CF8000 Octal Constant-Fraction Discriminator, Manual reference}. \href{https://www.ortec-online.com/products/electronics/fast-timing-discriminators/cf8000}{\textbf{Data sheet}}.

%Referencia 98
\bibitem{DataSheetDiscriminatorCAEN} \textsc{CAEN},
\textit{Model 84, 4 channels discriminator}. \href{https://www.caen.it/}{\textbf{Website}}.

%Referencia 99
\bibitem{DataSheetCoincidenceLeCroy} \textsc{LeCroy},
\textit{Model 465 Coincidence Unit, Manual reference}. \href{https://prep.fnal.gov/catalog/hardware_info/lecroy/nim/465.html}{\textbf{Data sheet}}.

%Referencia 100
\bibitem{DataSheetCoincidenceCERN} \textsc{CERN},
\textit{Coincidence Unit Type N6234, Manual reference}. \href{}{\textbf{Data sheet}}.

%Referencia 101
\bibitem{DataSheetGateAndDelay} \textsc{ORTEC},
\textit{Model 416A Gate and Delay Generator, Manual reference}. \href{https://www.ortec-online.com/products/electronics/delays-gates-and-logic-modules/416a}{\textbf{Data sheet}}.

%Referencia 102
\bibitem{DataSheetMCA} \textsc{AmpTek},
\textit{MCA8000D, Pocket MCA, Digital Multichannel Analyzer, Manual reference}. \href{https://www.amptek.com/products/multichannel-analyzers/mca-8000d-digital-multichannel-analyzer}{\textbf{Data sheet}}.

%Referencia 103
\bibitem{PETSYS} \textit{PETsys Electronics}. \href{https://www.petsyselectronics.com/web/private/login}{\textbf{Website}}.

%Referencia 104
\bibitem{OpticalFibers} \textsc{Saint-Gobain Ceramics and Plastics, Inc.},
\textit{Optical fiber BCF-98, Manual reference}. \href{https://www.crystals.saint-gobain.com/products/scintillating-fiber}{\textbf{Manual reference}}.

%Referencia 105
\bibitem{LEDThorlabs} \textsc{Thorlabs},
\textit{LED430L - 430 nm LED with a Glass Lens, 8 mW, TO-18}. \href{https://www.thorlabs.com/thorproduct.cfm?partnumber=LED430L}{\textbf{Datasheet}}.

%Referencia 106
\bibitem{NaturalRadioactiveSeries1} \textsc{Pall Theodórsson},
\textit{Measurement of weak radioactivity}, World Scientific, 1996.

%Referencia 107
\bibitem{NaturalRadioactiveSeries2} \textsc{R D Evans},
\textit{The Atomic Nucleus}, McGraw-Hill, Inc., 1996.

%Referencia 108
\bibitem{PDG} \textsc{P.A. Zyla et al.},
\textit{(Particle Data Grup), PDG, Prog. Theor. Exp. Phys. \textbf{2020} no. 8, 083C01 (2020)}. \href{https://pdg.lbl.gov/}{\textbf{Website}}~\href{https://academic.oup.com/ptep/article/2020/8/083C01/5891211}{\textbf{DOI: 10.1093/ptep/ptaa104}}.

%Referencia 109
\bibitem{HardCosmicMuonRate} \textsc{Hiroyuki SAGAWA \& Itsumasa URABE (2001)},
\textit{Estimation of Absorbed Dose Rates in Air Based on Flux Densities of Cosmic Ray Muons and Electrons on the Ground Level in Japan}, Journal of Nuclear Science and Technology, 38:12, 1103-1108, \href{https://doi.org/10.1080/18811248.2001.9715142}{\textbf{DOI: 10.1080/18811248.2001.9715142}}.

%Referencia 110
\bibitem{HardCosmicMuonRatePlot} \textsc{T. Szücs, D. Bemmerer, T. P. Reinhardt, K. Schmidt, M. P Takács, A. Wagner, L. Wagner, D. Weinberger and K. Zuber},
\textit{Cosmic-ray induced background intercomparison with actively shielded HPGe detectors at underground locations}. \href{https://arxiv.org/abs/1503.00457v2}{\textbf{DOI:10.1140/epja/i2015-15033-0}}.

%Referencia 111
\bibitem{ScintillatorVeto} \textsc{Epic Crystal},
\textit{Plastic scintillator of Epic Crystal, Manual reference}. \href{http://www.epic-crystal.com/others/plastic-scintillator.html}{\textbf{Data sheet}}.

%Referencia 112
\bibitem{DiamondThorlabs} \textsc{Thorlabs},
\textit{Guide to connectorization and polishing optical fibers}, 2006. \href{https://www.thorlabs.de/thorproduct.cfm?partnumber=FN96A}{\textbf{Manual Reference}}.

%Referencia 113
\bibitem{GuillotineIFO} \textsc{Indistroañ fiber optical},
\textit{POF Cutter block}. \href{https://i-fiberoptics.com/tool-detail.php?id=105&cat=cutters}{\textbf{Website}}.

%Referencia 114
\bibitem{AngleBlade} \textsc{David Sáez-Rodríguez, Kristian Nielsen, Ole Bang and David John Webb},
\textit{Simple Room Temperature Method for Polymer Optical Fibre CLeaving}, Journal of lightwave technology, vol 33, No. 23, December 1, 2015. \href{https://ieeexplore.ieee.org/document/7274313}{\textbf{DOI:10.1109/JLT.2015.2479365}}.

%Referencia 115
\bibitem{TemperatureBlade} \textsc{S.H. Law, J.D. Harvey, R.J. Kruhlak, M. Song, E. Wu, G.W. Barton, M.A. van Eijkelenborg and M.C.J. Large},
\textit{Cleaving of microstructured polymer optical fibres}. \href{https://www.researchgate.net/publication/228880071_Cleaving_of_microstructured_polymer_optical_fibres}{\textbf{DOI:10.1016/j.optcom.2005.08.011}}.

%Referencia 116
\bibitem{StepperMotors} \textsc{Nanotec},
\textit{ST4209S1404-A - STEPPER MOTOR NEMA 17}. \href{https://en.nanotec.com/products/463-st4209s1404-a}{\textbf{Data sheet}}.

%Referencia 117
\bibitem{ArduinoUNO}
\textit{ARDUINO}, \href{https://www.arduino.cc/}{\textbf{Website}}.

%Referencia 118
\bibitem{CNCShield}
\textit{CNC shield V3.0}, \href{https://osoyoo.com/2017/04/07/arduino-uno-cnc-shield-v3-0-a4988/}{\textbf{Reference manual}}.

%Referencia 119
\bibitem{A4988Driver} \textsc{Allegro}
\textit{Driver Pololu A4988, DMOS Microstepping Driver with Translator And Overcurrent Protection}, \href{https://www.alldatasheet.es/datasheet-pdf/pdf/455036/ALLEGRO/A4988.html}{\textbf{Data sheet}}.

%Referencia 120
\bibitem{DRV8825Driver} \textsc{Texas Instruments}
\textit{Driver DRV8825 Stepper Motor Controller IC}, \href{https://www.ti.com/product/DRV8825?utm_source=google&utm_medium=cpc&utm_campaign=asc-null-null-GPN_EN-cpc-pf-google-wwe&utm_content=DRV8825&ds_k=DRV8825+Datasheet&DCM=yes&gclid=EAIaIQobChMIworWtYba7gIVqoFQBh10_QfhEAAYASAAEgLPn_D_BwE&gclsrc=aw.ds}{\textbf{Data sheet}}.

%Referencia 121
\bibitem{TMC2208Driver}
\textit{Driver TMC2208, Step/Dir Drivers for Two-Phase Bipolar Stepper Motors up to 2A peak- StealthChop for Quiet Movement- UART Interface Option}, \href{https://datasheetspdf.com/pdf/1142008/TRINAMIC/TMC2225/1}{\textbf{Data sheet}}.

%Referencia 122
\bibitem{LEDRLT} \textsc{Roithner LaserTechnik Gmbh}
\textit{LED435-03, 20 mW, 20 mA}, \href{http://www.roithner-laser.com/led_diverse.html}{\textbf{Reference}}.

%Referencia 123
\bibitem{FiberConnector} \textsc{},
\textit{}, \href{}{\textbf{Reference}}.

%Referencia 124
\bibitem{OpticalGrease} \textsc{Saint-Gobain Ceramics and Plastics, Inc.},
\textit{BC-630, Silicone Optical Grease}, \href{https://www.crystals.saint-gobain.com/}{\textbf{Website}}.

%Referencia 125
\bibitem{BlackBlancket} \textsc{Thorlabs},
\textit{BK5 - Black Nylon, Polyurethane-Coated Fabric, 5'x9' (1.5m x 2.7m) x 0.005" (0.12 mm) Thick}, \href{https://www.thorlabs.com/thorproduct.cfm?partnumber=BK5}{\textbf{Datasheet}}.

%Referencia 126
\bibitem{WettingProperty} \textsc{San Nopco company},
\textit{Wetting property}, \href{https://www.sannopco.co.jp/eng/products/function/function4.php}{\textbf{Website}}.

%Referencia 127
\bibitem{TurbiditySystem} \textsc{Hanna Instruments},
\textit{Multiparamétrico con opciones GPS, sonda autoregistradora, turbidez e ISE}, \href{https://www.hannainst.es/parametros/4654-multiparametrico-portatil-con-portasondas-multisensor-ph-orp-ce-od-temperatura.html#/507-cable_m-4_m/512-portasondas-si/513-portasondas_registrador-no/514-gps-no/515-turbidez-no}{\textbf{Website}}.

%Referencia 107
%\bibitem{ElectronicMicroscopeSCSIE} \textsc{SCSIE}
%\textit{Microscopio electrónico de barrido por emisión de campo, Marca: Hitachi S4800}, \href{https://www.uv.es/uvweb/servicio-central-soporte-investigacion-experimental/es/equipamientos-instalaciones/todos-equipamientos-instalaciones/microscopio-electronico-barrido-emision-campo-1285887466621/OCTRecurs.html?id=1286169920510}{\textbf{Website}}.

%Referencia 128
\bibitem{ScalerDataSheet} \textsc{CAEN company},
\textit{Quad Scaler And Preset Counter-Timer, N1145}, \href{https://www.caen.it/products/n1145/}{\textbf{Datasheet}}.

%Referencia 129
\bibitem{TritiumSourceTechnicalFile} \textsc{Physikalisch-Technische Bundesanstalt, PTB, Braunschweig and Berlin, Germany}
\textit{Calibration Certificate of tritium source, PTB-6.11-2005-1442}.

%Referencia 130
\bibitem{DataSheetBCF10Fiber} \textsc{Saint-Gobain Ceramics and Plastics, Inc.},
\textit{Scintillating Optical Fibers}, It's What's Inside that Counts, 2005-14. \href{https://www.crystals.saint-gobain.com/products/scintillating-fiber}{\textbf{Data sheet}}. 

%Referencia 131
\bibitem{DataSheetPMTsAveiro} \textsc{HAMAMATSU PHOTONICS K.K.},
\textit{Photonmultiplier tube R2154-02 2"}. \href{https://www.hamamatsu.com/eu/en/product/type/R2154-02/index.html}{\textbf{Data sheet}}.

%Referencia 132
\bibitem{PowerSupplyAveiroDataSheet} \textsc{HAMAMATSU PHOTONICS K.K.},
\textit{High Voltage Power Supply C11152-01}. \href{https://www.hamamatsu.com/jp/en/product/type/C11152-01/index.html}{\textbf{Data sheet}}.

%Referencia 133
\bibitem{MAX5500DataSheet} \textsc{Maxim Integrated},
\textit{Low-Power, Quad, 12-Bit, Votlage-Output DACs with Serial Interface}. \href{https://www.maximintegrated.com/en/products/analog/data-converters/digital-to-analog-converters/MAX5500.html}{\textbf{Data sheet}}.

%Referencia 134
\bibitem{VetoAveiro} \textsc{Saint-Gobain},
\textit{Scintillating plastic grown with polymeric method}. \href{https://www.epic-crystal.com/others/plastic-scintillator.html}{\textbf{Data sheet}}.

%Referencia 135
\bibitem{CREMATPreAmplifierDataSheet} \textsc{CREMAT Inc.},
\textit{CR 111-R2.1 Charge sensitive preamplifier}. \href{https://www.cremat.com/home/charge-sensitive-preamplifiers/}{\textbf{Data sheet}}.

%Referencia 136
\bibitem{OPA656} \textsc{Texas Instruments},
\textit{OPA656 Wideband, Unity-Gain Stable, FET-Input Operational Amplifier}. \href{https://www.ti.com/product/OPA656}{\textbf{Data sheet}}.

%Referencia 137
\bibitem{LT111} \textsc{Linear Technology},
\textit{LT111A}. \href{https://datasheetspdf.com/pdf/57354/LinearTechnology/LT111/1}{\textbf{Data sheet}}.

%Referencia 138
\bibitem{Stretcher} \textsc{Texas Instruments},
\textit{SN74AHC1G32 Single 2-Input Positive-OR Gate}. \href{https://www.ti.com/product/SN74AHC1G32}{\textbf{Data sheet}}.

%Referencia 139
\bibitem{ANDGate} \textsc{Texas Instruments},
\textit{SN74LVC1G11DBVR Single 3-Input Positive-AND Gate}. \href{https://www.ti.com/store/ti/en/p/product/?p=SN74LVC1G11DBVR}{\textbf{Data sheet}}.

%Referencia 140
%\bibitem{Geant4WP} \textsc{CERN collaboration}, 
%\textit{Geant4, A simulation toolkit}. \href{https://geant4.web.cern.ch/node/1}{\textbf{Website}}.

%Referencia 141
\bibitem{Geant4P} \textsc{J. Allison}, 
\textit{Geant4 - A simulation toolkit}. \href{https://doi.org/10.1016/S0168-9002(03)01368-8}{\textbf{DOI:10.1016/S0168-9002(03)01368-8}}

%Referencia 142
\bibitem{CRYwebsite} \textsc{Plot Nuclear Data (NADS)}, 
\textit{Physics simulation packages, CRY (cosmic-ray particle showers)}. \href{https://nuclear.llnl.gov/simulation/}{\textbf{Website}}

%Referencia 143
\bibitem{CRYpaper} \textsc{Chris Hagmann}, \textsc{David Lange} and \textsc{Douglas Wright}, 
\textit{Cosmic-Ray particle Showers Generator (CRY) for Monte Carlo Transport Codes}, IEEE Nuclear Scinece Symposium conference record. Nuclear Scinece Symposium 2:1143-1146, January 2007. \href{https://www.researchgate.net/publication/4313740_Cosmic-ray_shower_generator_CRY_for_Monte_Carlo_transport_codes}{\textbf{DOI:10.1109/NSSMIC.2007.437209}}

%Referencia 144
\bibitem{WaterPropertiesSimulation} \textsc{H. Buiteveld}, \textsc{J.H.M. Hakvoort}, \textsc{M. Donze}, 
\textit{Optical properties of pure water, in:Proc. 2258 Ocean Optics XII, Bergen, Norway, 1994}. \href{https://www.spiedigitallibrary.org/conference-proceedings-of-spie/2258/1/Optical-properties-of-pure-water/10.1117/12.190060.short?SSO=1}{\textbf{DOI:10.1117/12.190060}}

%Referencia 145
\bibitem{TritiumEmissionSpectrum} \textsc{S. Maertens}, \textsc{et al.},  \textit{Sensitivity of next-generation tritium beta-decay experiments for keV-scale sterile neutrinos, J. Cosmol. Astropart. Phys. 2015 (2015) 020}. \href{https://iopscience.iop.org/article/10.1088/1475-7516/2015/02/020}\textbf{DOI:10.1088/1475-7516/2015/02/020}

%Referencia 146
\bibitem{NEMODataSimulation} \textsc{J. Argyriades}, \textsc{et al.},
\textit{Spectral modeling of scintillator for the NEMO-3 and SuperNEMO detectors, Nucl.Instrum. Methods A 625 (2011) 20-28}. \href{}{\textbf{}}

%Referencia 147
\bibitem{CAENV1724} \textsc{CAEN, Tools for Discovery},
\textit{CAEN V1724, 8 Channels, 14 bit, 100MS/s Digitalizer}. \href{https://www.caen.it/products/v1724/}{\textbf{Data sheet}}

%Referencia 147
\bibitem{CurieLimit} \textsc{Lloyd, A. Currie},
\textit{Limits for Qualitative Detection and QUantitative Determination. Application to Radiochemistry, Anal. Chem. 1968, 40, 3, 586-593, March 1, 1968}. \href{https://doi.org/10.1021/ac60259a007}{\textbf{DOI: 10.1021/ac60259a007}}

%~\cite{Ivo}
\end{thebibliography}

\end{document}