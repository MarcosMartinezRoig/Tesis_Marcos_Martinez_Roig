This section details the characterization of the SiPM S13360-1375 model, which was the first choose for the TRITIUM monitor photosensors. It has to be taken into account that this characterization is incomplete since some important SiPM parameters for the TRITIUM monitor, which are its PDE, its dark count rate and its crosstalk probability, was not experimentally measured. 

A complete characterization is already underway for the S13360-6075 model, where all interesting parameters, explained in section \ref{subsubsec:SiPM}, will be experimentaly determined using a different experimental setup, shown in appendix \ref{}, and its results will be finalized as soon as possible.

The setup used for this characterization is shown in section \ref{subsubsec:SiPMsElectronicalSystem}. Furthermore, these measurements were carried out inside of a climatic chamber, model CCM 81 from DYCOMETAL \cite{ClimaticChamberIFIMED}, whose temperature and humidity were controled with a precision of $0.1~\degree$ and $0.1\%$ respectively. 

First of all the breakdown voltage and the quenching resistance of the SiPM were experimentally obtained. Both parameters can be calculated from the measurement of the current-voltage curves of the SiPM, bias voltage applied in reverse and forward direction respectively. This measurement should be done without the amplification of the electronic board to achieve a better precision. Therefore, the output current of the SiPM was directly measured ussing, which are plotted in Figure \ref{}. The output current was measured with the Keithley 6487 Picoammeter/Voltage Source \cite{DataSheetKeithley6487} and the LabView program was used to automate the taking of measurements.

CURVAS IV, DIRECTA E INVERSA.

As can be seen when the bias voltage is applied in forward direction, Figure \ref{}, the output current of the SiPM doesn't flow until the potencial difference existing between the n and p layers are reached, which is approximately $V_0=0.7~\volt$ for silicon, quite similar to the value experimentally obtained $V_0= ~\volt$. When the current start to flow, the intensity is linear with the forward voltage:

\begin{equation}
I=\frac{1}{R_{eq}}V;  \qquad \frac{1}{R_{eq}} = \sum_{i=1}^{N}\frac{1}{R_{qi}}= \frac{N}{R_{q}}
\label{QuenchingResistance}
\end{equation}
Where $R_{eq}$ is the equivalent resistance of all quenching resistance of the SiPM, which are in parallel. Therefore, a value of $R_{q}= ~\ohm$ is obtained from the slope of the linear fit, which is in agreement with the value provided by Hamamatsu, Table \ref{tab:PropertiesOfSiPM1375}.

Regardness to the experience where a reverse bias voltage was applied, the output current of SiPM start to flow when the breakdown voltage is reached, which can be calculated from the maximum of the function 
\begin{equation}
f=\frac{1}{I}\frac{dI}{dV}
\label{BreakDownVoltageFunction}
\end{equation}

The value obtained is $V_{BR}=~\volt$, quite in agreement with the value providad by Hamamatsu, Table \ref{tab:PropertiesOfSiPM1375}.






Tiempo de coincidencia de los SiPMs -> 10 nanosegundos. Por este motivo no es importante estudiar los afterpulses.


Furthermore, a temperature compensation method has been developed and experimentally tested to compensate for temperature variations of the SiPM since it has been seen to greatly affect to its correct operation and, therefore, the measurement of tritium.


Pequeñas zonas de deplexión crean grandes capacidades que producen alto ruido de los SiPMs

la banda prohibida es pequeña por lo que algunos electrones pueden excitarse termicamente y pasar a la banda de conducción -> RUIDO
Cuando hable del ruido, afterpulses, crosstalk... intro en MPPC hammatsu data sheet

resumen de como varía cada magnitud con la temperatura y el voltaje. Tesis SiPMs.

Aunque el aumentar el voltaje inverso mejora la eficiencia de detección de fotones, también aumenta la corriente oscura. Reducir la temperatura sin embargo disminuye la corriente oscura.

En este trabajo se estudió tanto la variación en la resolución en energía de un SiPM como la variación del centroide de un pico (explicado a detalle en los próximos capítulos) con la temperatura y el voltaje. Para ello se estableció una electrónica de adquisición adecuada para la mayor eliminación de ruido posible y óptima resolución (explicada en el Capítulo 3).


Superponer 2 plots con el LED a distintos voltajes (2 o mas... probar varios voltajes a Vov recomendado y quedarnos con los mejores).

Figura 11 de la tesis de cristales monoliticos
