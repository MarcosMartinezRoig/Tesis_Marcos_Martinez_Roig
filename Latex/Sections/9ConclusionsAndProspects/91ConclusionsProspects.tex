Se han caracterizado las distintas partes del detector de tritio

Se han desarrollado y caracterizado vetos activos, vetos pasivos, sistema de agua ultrapura.

Se ha desarrollado un método de compensación d ela temperatura



Punto 4.3 del paper de simulacion de carlos.

Conclusiones de los papers de carlos.
Conclusiones de nuestros papers.
Conclusiones presentaciones.

Que cosas se han conseguido en este experiemnto? -> DEcir que tanto l oque se ha conseguido con el detector como con las investigaciones de componentes del detector (capitulo 3)

Responder a las grandes preguntas: 
\begin{itemize}
\item{} Podemos medir tritio? 
\item{} Lo podemos hacer en tiempo quasi real? 
\item{}Lo podemos hacer a la actividad que queríamos? 
\item{} Que sensibilidad se ha llegado a conseguir?
\item{} Estabilidad temporal?
\item{} Precio?
\item{} Comparación con respecto al resto de experimentos? -> Poner la tabla 1.8 pero incluyendonos
\item{} Effecto del shield
\item{} Effecto de los vetos
\item{} Effecto de ambas cosas
\item{} Medidas a varias actividades
\end{itemize}


Hemos llegado a detectar 30kBq/L con Tritium-Aveiro y se espera llegar a medir hasta menos de 5kBq/L (superando los actuales limites).

Hemos llegado a detectar 10 kBq/L con Tritium-IFIC 2 (superando los actuales limits) y se espera llegar a medir incluso menos.

Ambos valores, lejos de ser el objetivo del proyecto, sirven para una monitorización en tiempo quasí real. Además, con el monitori final que consiste en varios modiules de estos en apralelo, se pretende llegar al objetivo deseado.


Tenemos datos de tritio en el agua bruta (agua del río) de esa zona desde 1998, pero tengo que solicitar permiso para poder dártelos. En cualquier caso, hay otra manera de conseguirlos, que es a partir  de los informes del CSN al Congreso de los diputados (web CSN).
Desde 2015 la concentración de tritio en el agua del río Tajo ha disminuido considerablemente porque la CNA instaló unos enfriadores por convección que emiten parte del H3 a la atmósfera. --> Tesis de Antonio Rodríguez y de Elena García.

El prototipo más avanzado hasta la fecha ha sido el de Aveiro. Prometedores resultados se estan obteniendo con el último prototipo, IFIC 2.

Como se ha dicho, se leerán varios prototipos en coincidencia, lo cual aumenta el area activa del detector. Esto nos permite obtener un mayor número de cuentas para una misma actividad de la fuente ya que estas depende linealmente del area activa, mejorando la efficiencia del detector y permitiendonos llegar a LDLs más bajos. Sin embargo, no se espera que la eficiencia especifica del detector se mejor ya que esta no depende del area activa.

Además el uso de los vetos activos nos permitirá reducir el nivel de background de la muestra. Dado que se espera que la incertidumbre relativa del background se mantenga, obtendremos incertidumbres más pequeñas, dando lugar a menores MDAs.

\begin{table}[htbp]
\begin{center}
\begin{tabular}{|c|c|c|c|c|}
\hline
Study & \parbox{5.5em}{\centering $\varepsilon_{det}(\frac{cps \cdot{} 10^{-3}}{\kilo\becquerel/\liter})$}  & \parbox{4.5em}{\centering $F_{sci}$ ($\cm^2$)}  & \parbox{6.5em}{\centering $\eta_{det}(\frac{cps \cdot{} 10^{-6}}{\kilo\becquerel/ \liter \cdot{} \cm^2})$} & LDL ($\kilo\becquerel / \liter$)\\
\hline \hline \hline
Muramatsu & $0.39$ & $123$ & $3.13$ & $370$ \\ \hline
Moghissi & $4.50$ & $>424.1$ & $<10.6$ & $37$ \\\hline
Osborne & $12$ & $3000$ & $4$ & $37$ \\ \hline
Singh & $41$ & $3000$ & $13.7$ & $<37$ \\ \hline
Hofstetter & $2.22$ & $\sim~100$ & $<22.2$ & $25$ \\ \hline
T-IFIC 0 & $2.11 \pm 0.85$ & $219.91$ & $9.59 \pm 3.87$ & $100$* \\ \hline
T-IFIC 1 & $38.42 \pm 1.61$ & $402.12$ & $95.55 \pm 4.01$ & $100$* \\ \hline
T-Aveiro 0 & $64.87 \pm 19.41$ & $4071.50$ & $15.93 \pm 4.77$ & $29.8$ \\ \hline
T-IFIC 2 & $711.03 \pm 27.77$ & $5026.55$ & $141.45 \pm 5.52$ & $10$* \\ \hline
\end{tabular}
\caption{Results of different scintillator detector for tritiated water detection. This table shows the efficiency of the detector ($\varepsilon_{det}$), its active surface ($F_{sci}$), its specific efficiency ($\eta_{det}=\varepsilon_{det}/F_{sci}$), defined as its efficiency normalized to its active surface, and its low detection-level (LDL) for each study listed above. The "*" symbol indicates that this is the specified activity that the detector can distinguish from the background, but it is not its LDL.}
\label{tab:ComparisonResults}
\end{center}
\end{table}