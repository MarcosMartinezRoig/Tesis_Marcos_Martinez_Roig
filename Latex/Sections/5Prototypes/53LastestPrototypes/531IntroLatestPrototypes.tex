This section reports on prototypes TRITIUM-Aveiro and TRITIUM-IFIC 2. A different design was employed in both prototypes so that they can safely allow the reading of a large number of fibers arranged in a straight position with two photosensors in time coincidence. 

Particular attention to tritium detection efficiency was paid in these prototypes, obtained by the use of more fibers than the preliminary prototypes and time coincidences of the photosensors. Furthermore, the activity of the radioactive liquid source of tritium used to fill these prototypes is much lower than that of the first prototypes to be able to measure their low detection level.

The design of both prototypes is quite similar and their objective is to test the importance of to decide the final design of the TRITIUM monitor.
