Another test was performed to find the fiber length that optimizes the tritium detection efficiency. Two different lengths of the scintillating fiber were considered in this study, $1~\meter$ and $25~\cm$. 

As the active area of the detector is related with its tritium detection efficiency, the advantage to use a longer fibers is that the same active area can be achieved with a less number of cells, considerably reducing the number of used photosensors and, as a consequence, the price of the TRITIUM monitor. However, smaller length of scintillating fibers reduce de photon absortion produced in the fibers, increasing the tritium detection efficiency for the same active area.

For this task, the Tritium-Aveiro prototype was simulated, consisting of a similar design of TRITIUM-IFIC 2 prototype but using $360$ scintillating fibers of $2~\mm$ diameter and readout by two different photosensors, model R2154-02 2" from Hamamatsu company \cite{DataSheetPMTsAveiro}. All optical properties mentioned in section \ref{sec:Geant4Environment} were included in this study.

The results of this study are shown in section \ref{subsec:ResultsFiberLength}, where they are discussed.