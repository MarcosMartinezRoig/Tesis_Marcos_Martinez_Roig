This section will explain the interaction of particles with matter. It will focus on the interesting particles and energy range for this thesis, which are electrons ($0-18~\keV$) and photons in the visible range (approx. $380-750~\nm$).

On the one hand, electrons have charge so their interaction with matter is mainly produced with the orbital electrons of the matter through the Coulomb force. The electron trajectory is much more tortuous than other heavier particles because the mass of both interacting particles is equal, electrons. Furthermore, for the same reason, these electrons lost a significant amount of energy in each collision.

In order to speak about the total energy lost of particles in matter the specific energy loss is defined as $S=-\frac{dE}{dx}$ which expresses the energy loss suffered by the particle per unit of trajectory. In the case of electrons, this total energy loss has two main contributions, the collisions (elastic and inelastic) and radiative processes (bremsstrahlung):

\begin{equation}
\frac{dE}{dx} \approx \left(\frac{dE}{dx}\right)_{c} + \left(\frac{dE}{dx}\right)_{br} ~\cite{Knoll, Leo} \qquad  \frac{\left(\frac{dE}{dx}\right)_{br}}{\left(\frac{dE}{dx}\right)_{c}} \approx \frac{EZ}{700} ~\cite{Knoll}
\label{eq:ElectronInteraction}
\end{equation}

Where $E$ is the energy of the electron in $\MeV$ and $Z$ is the atomic number of the absorbing material. Due to this energy loss, the electrons can only penetrate a material as far as they go before losing their total kinetic energy. This distance is known as range and, in the case of tritium electrons, its value is seen in Table \ref{tab:MeanFreePathTritium}.

On the other hand, photons don't have charge. Its possible interactions with the matter are photoelectric effect, Compton effect, coherent scattering and pair production and the probability of each process depends on the energy of the photon, $E_\gamma = h\nu$, and the atomic number of the material, Z, as can be seen in Figure \ref{fig:ProcessesPhotons}.

\begin{figure}[htbp]
\centering
\includegraphics[scale=0.75]{3DesignPrinciples/32Tritium_detector/DominantProcessesPhotons.png}
\caption{Domain regions of the three most probable types of interactions of gamma rays with matter. The lines show the values of Z and $h\nu$ where the two neighboring effects are equally likely.\label{fig:ProcessesPhotons}~\cite{Knoll, Leo}}
\end{figure}

It has to be taken into account that the only relevant photons for this thesis are in the visible range, between $400$ and $700~\nano\meter$, that corresponds with energies of the order of the $\eV$. Therefore the last effect, pair productions, will be not explained here because it requires a photon energy equal or more than $1.022~\MeV$.

The photoelectric effect occurs when a photon interacts with an orbital electron in the material, losing all its energy. This energy is absorbed by the electron that is released from the atom (ionization). The energy of the resulting electron, $E_e$, is:


\begin{equation}
E_e = E_\gamma - E_b ~\cite{Knoll, Leo}
\label{eq:PhotoelectricEffect}
\end{equation}

Where $E_b$ is the binding energy of the electron in this material. The probability of this effect depends on the number of available electrons in the matter through the variable Z, and the energy of the electron according to the following expression:

\begin{equation}
\left(Pr\right)_{Ph-eff} \approx \frac{Z^n}{E_\gamma^{3.5}}~\cite{Knoll}
\label{eq:PhotoelectricProb}
\end{equation}

As it is shown in this expression and in Figure \ref{fig:ProcessesPhotons}, the photoelectric effect is most probably if elements with high atomic number are used. This is the reason why elements with high atomic number are the best isulators against gamma radiation and this is the reasons why the passive shielding of TRITIUM monitor consists of lead bricks ($Z=82$), shown in section \ref{subsec:SetUpPassiveShield}. This is also the reason why elements with high atomic number like $\ce{Sb}$ ($Z=51$), $\ce{Rb}$ ($Z=37$) or $\ce{Cs}$ ($Z=55$), are used in the cathodes of PMTs. 

The Compton effect occurs when the photon interacts with an orbital electron of the material, transferring part of this energy to the electron, which is released, and this photon is scattered at an angle $\theta$ with respect to the original direction. If the binding energy are neglected, the energy transfered to this electron, $E_e$, is shown in the following equation:

\begin{equation}
E_e=\frac{\frac{E_\gamma^2}{m_oc^2}\left(1-cos\theta\right)}{1+ \frac{E_\gamma^2}{m_oc^2}\left(1-cos\theta\right)}~\cite{Knoll, Leo}
\label{eq:ComptonEffect}
\end{equation}

Where $m_0$ is the rest mass of the electron and $c$ is the speed of the light in the vacumm. The probability of the Compton effect is proporcional to the atomic number(available electrons in the matter), Z,  and decreases with the energy of the photon. 

As can be seen in Figure \ref{fig:ProcessesPhotons}, in the energies of the photons belonging to the visible range of the electromagnetic spectrum (of the order of eV), the Compton effect is only more likely in very light materials, (Z<4). For heavier materials the photoelectric effect is the dominant effect.

Finally, in the coherent scattering, the atom is neither excitation nor ionization and the photon conserve all their energy in this collision. It is more probably for photons with low energies and materials with high atomic numbers and, as it will be shown in section \ref{subsec:PlasticScintillators}, it explains why the produced photons are guided into the scintillating fibers. 


%Because of the fact that the energy of the photon doesn't change we will not speak more about this effect but it is important since this effect change de direction of photons and it will affect to their mean free path.