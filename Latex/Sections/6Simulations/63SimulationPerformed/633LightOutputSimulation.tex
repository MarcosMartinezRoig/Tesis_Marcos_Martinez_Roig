The scintillation yield provided by the manufacturer, $8000~\text{phot.}/\MeV$, only works for Minimum Ionizing Particles (MIP). As tritium electron energies are far from being MIP particles, the output light generated by the scintillating fibers was studied. For this task, the energy deposition of tritium electrons in scintillating fibers and their subsequent emission of scintillation photons was added to the simulation.

When particles that are not MIP are detected in plastic scintillators a light quenching effect affect to the proportionality between the output light per unit of path length, $\frac{dL}{dx}$, and the energy deposited per unit of path length, $\frac{dE}{dx}$, through the so-called Birks coefficient, following equation \cite{BirksPaper}.
\begin{equation}
\frac{dL}{dx}= S\frac{\displaystyle{\frac{dE}{dx}}}{1+k_B\displaystyle{\frac{dE}{dx}}}
\label{eq:birkscoefficient}
\end{equation}
where S is the scintillation yield, provided by the manufacturer. A value of $k_B=0.126~\mm/\MeV$ was considered for the Birk's coefficient which is the one used for scintillators based on polystyrene \cite{BirksCoefficient}. The effect of this correction is shown and discussed in section \ref{subsec:ResultsOutputLight}.