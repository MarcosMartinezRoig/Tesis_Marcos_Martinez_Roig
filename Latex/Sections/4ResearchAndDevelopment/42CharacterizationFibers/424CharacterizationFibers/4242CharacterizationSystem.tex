%Before characterizing a fiber several tasks had to be performed to check that the system is working properly. The quality of the tightness to the light of the black box used and the correct operation of the PMT for this study, which involves checking the correct operation of the PCB designed and checking the linearity of the PMT output signal in the study range, must be verified.

Before characterizing a fiber several tasks had to be performed to check that the black box is light-tight enough and that the PMT response is linear.

%First, the quality of the light tightness of the black box used was verified. It is important because we are detecting small signals, a few hundred photons per nanosecond, so it must be verified that the background of the system are below that.

A light leak in the black box would produce a background larger than the signal. To check the light-tightness of the black box a uncladded fiber of $20~\cm$ length was arranged in the setup. The LED was fed with four different intensities ($0.05~\milli\ampere$, $0.1~\milli\ampere$, $0.15~\milli\ampere$ and $0.2~\milli\ampere$) and the PMT response was measured with and without a special black blanket from Thorlabs \cite{BlackBlancket}, that prevents external photons to reach the system. This test was repeated for three different fibers and the mean and standard deviation of the light output were calculated.

%\begin{equation}
%\bar{x}=\frac{\sum_{i=0}^{N}x_i}{N}; \qquad \sigma = \frac{\sqrt{\sum_{i=0}^{N}(x_i-\bar{x})^2}}{N-1};
%\label{eq:MeanAndStandardDesviation}
%\end{equation}

The difference of the PMT responses in both cases is plotted in Figure \ref{fig:LightTightnessTest} as a function of the LED intensity. As it can be seen in this figure, there are no statistically relevant differences between covered and uncovered fibers. Therefore, the light tightness of the black box is sufficient for this study.


%\begin{figure}[]
 %\centering
  %\subfloat[The measurement obtained by covering the setup with a special black blanket and not covering.]{
   %\label{subfig:LightTightnessTestData}
    %\includegraphics[width=0.9\textwidth]{4ResearchAndDevelopments/41Fibers/Light_tightness_Measurements.pdf}}
    %\newline
  %\subfloat[.]{
   %\label{subfig:LightTightnessTestDifference}
    %\includegraphics[width=0.9\textwidth]{4ResearchAndDevelopments/41Fibers/Light_tightness_difference.pdf}}
 %\caption{Energy spectrums used to test the effect of the Polishing machine}
 %\label{fig:LightTightnessTest}
%\end{figure}

\begin{figure}[h]
\centering
\includegraphics[scale=0.6]{4ResearchAndDevelopments/41Fibers/Light_tightness_difference.pdf}
\caption{Difference between the results obtained in both tests carried out to check the light-tight quality of the system.\label{fig:LightTightnessTest}}
\end{figure}

The optimal voltage of the PCB without the PMT internal gain was obtained by finding the voltage plateau at which the electron collection efficiency in the first dynode was practically $100\%$.- With no fibers in the setup, the LED was fed at $1~\milli\ampere$ intensity and the PMT output current was measured for different PMT supply voltages, between $0$ and $500~\volt$. The number of photons detected by the PMT is plotted in Figure \ref{fig:PlateauNoGainPMT}. As it can be seen, the plateau starts at voltages higher than $150~\volt$. The chosen voltage for the characterization was $250~\volt$.

\begin{figure}[h]
\centering
\includegraphics[scale=0.7]{4ResearchAndDevelopments/41Fibers/PCBNoGainPlateau_Calibrated.pdf}
\caption{Response of the PMT as a function of its high voltage using the designed PCB with which no internal gain of the PMT is obtained. Error bars are included but they are too small to be visible.\label{fig:PlateauNoGainPMT}}
\end{figure}

Finally, the linearity of the PMT was verified. The LED was powered with intensities ranging from 0 to $10~\milli\ampere$ (LED linearity range) to check that the LED emission light does not saturate. The linearity was tested in the of the number of photons range expected for a tritium event (a few tens of photons per tritium event, which gives tens of photons per nanosecond) and in the range around two thousand five hundred photons per nanosecond. To test the linearity of the PMT in the range of tritium events, the setup described above was used without any fiber but with one of the connectors and the collimators kept to make sure that the active area of the PMT is the same as in the characterization study. To test the linearity of the PMT in the range of more than a thousand photons per nanosecond, the remaining connector was removed in order to increase the photons that reach the photosensor but the collimator was also kept. The results for both intensity ranges are shown in Figures \ref{fig:LinearityRangesOfPMT}. As it can be seen, the PMT output current is linear in both intensity ranges.

\begin{figure}
\centering
    \begin{subfigure}[b]{1\textwidth}
    \centering
    \includegraphics[width=\textwidth]{4ResearchAndDevelopments/41Fibers/Linearity_test_0_30_range.pdf}  
    \caption{\label{subfig:LinearityTritiumRange}}
    \end{subfigure}
    \hfill
    \begin{subfigure}[b]{1\textwidth}
    \centering
    \includegraphics[width=\textwidth]{4ResearchAndDevelopments/41Fibers/Linearity_test_0_2500_range.pdf}  
    \caption{\label{subfig:LinearityStudyRange}}
    \end{subfigure}
 \caption{Linearity tests of the PMT response. (Above) Response of the PMT in the intensity range of tritium events. (Below) Response of the PMT in the range $0-2500~\text{photons}/\nano\second$. Error bars are included but they are too small to be visible.}
 \label{fig:LinearityRangesOfPMT}
\end{figure}