A test was performed to find the fiber length that optimizes the tritium detection efficiency. Two different lengths of scintillating fibers were considered in this study, $1~\meter$ and $25~\cm$. As the active area of the detector is related to its tritium detection efficiency, the advantage to use long fibers is their large active areas with a small number of cells, reducing the number of photosensors and, consequently, the price of the TRITIUM monitor. However, a smaller length of scintillating fibers reduce de photon absortion produced in the fibers, increasing the tritium detection efficiency per active area.

To find the optical fiber length, the Tritium-Aveiro prototype, consisting of a similar design as the TRITIUM-IFIC 2 prototype but with $360$ scintillating fibers of $2~\mm$ diameter, was simulated. All optical properties were included in this study.

The results of this study are reported in section \ref{subsec:ResultsFiberLength}.