A test was carried out to study the influence of the fiber diameter in the tritium measurement. To do so, a simulation, for a single fiber length of $20~\cm$ and two different diameters, $1~\mm$ and $2~\mm$, the commercial options given by Saint-Gobain company, were compared.

%It doesn't have sense to test it with the tritium source since its efficiency will scale with the active surface of the scintillating fiber.
An important point is how the fiber diameter affect to cosmic ray detection in the fiber, which is an important component of the background. The tritiated water source was replaced by a cosmic ray source, generated by the CRY library\footnote{CRY library, Cosmic-Ray Shower library} \cite{CRYwebsite}, \cite{CRYpaper}. The CRY library is a package based on objected-oriented technology and implemented in the C++ programming language. This library is used to generate cosmic-ray shower distributions for different particles (muons, neutrons, protons, electrons, photons and pions). The cosmic source shape used in this simulation is a horizontal square of $1 \times 1~\meter ^2$ located at a height of $35~\cm$ (above the detector) with the typical distribution of cosmic particles at see level. The result of this simulation are presented in section \ref{subsec:ResultsFiberDiameter}.