This section show and discuss the results obtianed with the simulation described in section \ref{subsec:SourceShapeSimulation}, the objective of which is to decide which is the radial thick of the simulated tritium source that optimize the simulations. 

First, the energy of the simulated tritium events are verified. For this task, the energy distribution of the simulated tritium electrons is obtained and shown in figure \ref{subfig:EnergyDistributionTritiumSource}, where this is compared with that obtained in the reference \cite{TritiumEmissionSpectrum}. As can be checked, there is a good agreement between both.

In addition, an energy spectrum of the initial energy of the detected tritium electrons (electrons that are able to reach the fiber and deposit energy in it) are shown in Figure \ref{subfig:EnergySpectrumEventsDetectedandNonDetected}, red histogram, which is compared to the energy distribution of all simulated tritium events, blue histogram. As can be seen, the red histogram are centred at $10~\keV$ since the lower energy tritium events do not reach the scintillating fibers. This occurs mainly because they are absorbed at a very short distance in the water volume or they don't have enough energy to overcome the water-fiber interface.

\begin{figure}[h]
 \centering
  \subfloat[Energy distribution of tritium decays simulated]{
   \label{subfig:EnergyDistributionTritiumSource}
    \includegraphics[width=0.5\textwidth]{8SimulationsResults/81TRITIUMDesign/811TritiumSourceOptimization/TritiumSourceEnergyDistribution.png}}
   %\newline
  \subfloat[]{
   \label{subfig:EnergySpectrumEventsDetectedandNonDetected}
    \includegraphics[width=0.5\textwidth]{8SimulationsResults/81TRITIUMDesign/811TritiumSourceOptimization/Source_Spectrum_yes_and_non_detected_events.png}}
 \caption{ Energy distribution of a) simulated tritium decays b) Initial energy of tritium decays that reach the scintillating fibers (red histogram) compared the all simulated tritium events (blue histogram).}
 \label{fig:TritiumSourceOptimization}
\end{figure}

Figure \ref{subfig:TransversalCutTritiumSource} shows a transversal cut of the $2~\mm$ scintillating fiber, yellow, the simulated tritium source $0.5~\mm$ thick around the fiber, green, and the tritium decays, red dots, the electrons of which has deposited their energy in the scintillating fiber. Furthermore, the distribution of the radial distance between the position where tritium decays take place and the surface of the scintillating fiber are shown in figure \ref{subfig:DistanceDistributionTritiumSourceFiber}.

\begin{figure}[h]
 \centering
  \subfloat[]{
   \label{subfig:TransversalCutTritiumSource}
    \includegraphics[width=0.5\textwidth]{8SimulationsResults/81TRITIUMDesign/811TritiumSourceOptimization/Source_Ring.png}}
   %\newline
  \subfloat[]{
   \label{subfig:DistanceDistributionTritiumSourceFiber}
    \includegraphics[width=0.5\textwidth]{8SimulationsResults/81TRITIUMDesign/811TritiumSourceOptimization/SourceDistance.png}}
 \caption{a)Transversal cut of simulated scintillating fiber (yellow) and tritium source (green) with various tritium decays (red dots) b) Distribution of the radial distance between the position where the tritium decay takes place and the surface of the scintillating fiber.}
 \label{fig:TritiumSourceSimulated}
\end{figure}	

As can be seen in both figures, most of the tritium decays that are detected occur in close proximity to the scintillating fiber.  A zoom is applied in the inset box of the Figure \ref{subfig:DistanceDistributionTritiumSourceFiber}. It was obtained that the $99.4\%$ of the detected events are produced at least of $5~\mu\meter$ (the mean free path of tritium electrons). Therefore, the thick of the simulated tritium source chosen to optimize the simulation are $5~\mu\meter$. 

