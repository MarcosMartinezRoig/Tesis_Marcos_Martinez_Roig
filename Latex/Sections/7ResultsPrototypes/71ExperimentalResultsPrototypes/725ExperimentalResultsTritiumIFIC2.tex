This section shows the results of the lastest prototype developed in the TRITIUM experiment, TRITIUM-IFIC 2, during its installation in the Nuclear Radiation Laboratory at IFIC laboratory. The design of this prototype is shown in section \ref{subsec:TritiumIFIC2}.

The energy spectra of the signal and background prototypes, were measured, which are shown in Figure \ref{}. As it was mentioned in section \ref{subsection:TritiumIFIC2}, the signal prototype was filled with a tritiated water solution with an activity of $10~\kilo\becquerel/\liter$ and the background prototype was filled with ultrapure water.

FIGURES

A difference between both signals is clearly visible, which corresponds to the energy spectrum of tritium, Figure \ref{}. The number of counts per second obtained for both, the signal and background, are $19.05 \pm 0.18$ and $11.54 \pm 0.14$ respectively. Therefore, $7.11 \pm 0.23$ counts per second was obtained for the tritiated water source used.

The tritium detection efficiency obtained for this prototype is $(7.11 \pm 0.28)\cdot{} 10^{-1}~ \frac{\text{c}/\second}{\kilo\becquerel/\liter}$, calculated from the quotient of both, the counts per second measured and the specific activity o the tritium liquid source used. This efficiency is larger than all scintillating detectors developed so far, including the prototypes developed in TRITIUM experiment, Table \ref{} and sections \ref{}, \ref{} and \ref{}.

The specific efficiency obtained for this prototype is $(1.59 \pm 0.48)\cdot{} 10^{-5}~ \frac{\text{c}/\second}{\kilo\becquerel/\liter}\frac{1}{\cm^{2}}$, which is, again, the larger specific efficiency obtained so far with a scintillating detector used for tritium detection.

Therefore, as it has demostrated, the intrinsec and specific efficiency obtained so far for scintillating detectors used for tritium detection has been exceeded with the last TRITIUM prototype, TRITIUM-IFIC 2.


la monitorización de la señal y el fondo.

Rellenar con uan actividad grande y volver a medir. 

Vaciar el prototipo y rellenar con actividades más pequeñas. 1000 y 100 Bq/L.

Comparar fondos en el laboratorio, caceres, almaraz, etc.

Medir en el prototipo con SIPM.

Medir la distribución energética del photoelectron y expresar en unidades de photones detectados. Basarme en el punto 2.1 del artículo experimental de carlos.

Se midieron los dos PMTs en coincidencia bajo la manta negra y sin fibras ni nada y no se observó nada. Indica que no hay entrada de luz en estos.

Menos fondo que con el prototipo de AVEIRO en laboratorio Aveiro pero más que en laboratorio extremadura!!!!!!!!!!!!!!!!!!!!