As hard radiation cannot be stopped by a moderate lead thickness cosmic vetos are employed, which consist of at least two complementary detectors in coincidence that reject events simoultaneously detected in both of them. As shown in Figure \ref{fig:VetoAndPrototype}, the two complementary detectors are placed one above and the other below the TIRTIUM detector. The distance between both detectors, $34.2~\cm$ for our latest prototype developed, is set by the TRITIUM prototype to be placed inside.

\begin{figure}[h]
\centering
\includegraphics[scale=0.45]{3DesignPrinciples/34BackgroundRejectionSystem/Vetos_y_prototipo.png}
\caption{Cosmic veto and Tritium-IFIC 2 prototype in an aluminum mechanical structure developed by IFIC's mechanical engineering department.\label{fig:VetoAndPrototype}}
\end{figure}

A hard cosmic event crossing simultaneously both cosmic detectors is schematically sketched in figure \ref{subfig:RealHardCosmicEvent}. Each cosmic detector has two photosensors, so the electronic configuration given in Figure \ref{subfig:ElectronicConfiguraiton4PMT} is used to make time coincidences. The TRITIUM detector is read out in anti-coincidence with the cosmic veto to reject the hard cosmic events from the tritium measurement. Random coincidences from two different hard cosmic events, one in each detector, as drawn in Figure \ref{subfig:FakeHardCosmicEvent}, are negligible. The expected hard cosmic rate at sea level for muons is $7\times 10^{-3}~\cm^{-2}\second^{-1}\steradian^{-1}$ \cite{PDG, HardCosmicMuonRate}, as shown in the plot of Figure \ref{fig:HardCoscmicRate}. As time coincidences are triggered by logical gates of about $10~\nano\second$, the probability of recording two different hard cosmic events in temporal coincidence is less than $10^{-9}$ which is negligible.

\begin{figure}[h]
\centering
    \begin{subfigure}[b]{0.45\textwidth}
    \centering
    \includegraphics[width=\textwidth]{3DesignPrinciples/34BackgroundRejectionSystem/Real_Event.png}  
    \caption{\label{subfig:RealHardCosmicEvent}}
    \end{subfigure}
    \hfill
    \begin{subfigure}[b]{0.45\textwidth}
    \centering
    \includegraphics[width=\textwidth]{3DesignPrinciples/34BackgroundRejectionSystem/Fake_Event.png}  
    \caption{\label{subfig:FakeHardCosmicEvent}}
    \end{subfigure}
   \caption{Hard cosmic events detected with the cosmic veto of TRITIUM: a) Real coincidence, b) Fake coincidence.}
 \label{fig:HardCosmicEventsSimulation}
\end{figure}

\begin{figure}[h]
\centering
\includegraphics[scale=0.6]{3DesignPrinciples/34BackgroundRejectionSystem/HardCosmicRate.png}
\caption{Hard cosmic muon rate \cite{HardCosmicMuonRatePlot}.\label{fig:HardCoscmicRate}}
\end{figure}

The vetos are made of a plastic scintillator block from Epic-Crystal \cite{ScintillatorVeto}. Its properties are given in Table \ref{tab:ParametersScintillatorVeto} and its energy emission spectrum is displayed in Figure \ref{fig:EmissionEnergySpectrumVeto}.

\begin{table}[htbp]
\centering{}%
\begin{tabular}{lc}
\toprule 
Property & Value \tabularnewline
\midrule
\midrule 
Base material & Polystyrene \tabularnewline
Growth method & Polymeric \tabularnewline
Density ($\gram/\cm^3$)& 1.05 \tabularnewline
Refractive index & 1.58 \tabularnewline
Soften temperature ($\degree$) & 75-80 \tabularnewline
Light output (Anthracene) & 50-60\% \tabularnewline
H/C raito & 1.1 \tabularnewline
Emission peak (nm) & 415 (Blue) \tabularnewline
Decay Time, (ns) & 2.4 \tabularnewline
Hygroscopic & No \tabularnewline
\bottomrule
\end{tabular}
\caption{Properties of plastic scintillators from Epic-Crystals~\cite{ScintillatorVeto}.}
\label{tab:ParametersScintillatorVeto}
\end{table}

\begin{figure}[]
\centering
\includegraphics[scale=0.35]{3DesignPrinciples/34BackgroundRejectionSystem/EmissionEnergySpectrumVetos.png}
\caption{Emission energy spectrum of the plastic scintillation used for the cosmic vetos\label{fig:EmissionEnergySpectrumVeto}~\cite{ScintillatorVeto}.}
\end{figure}

The energy spectrum has a peak very close to that of the scintillating fibers used, so the same photosensors are used to read them out. The dimensions of the scintillator block are $45 \times 17 \dot{} 1~\cm^2$. They are wrapped by three different layers, Teflon, aluminum and black tape, exhibited in Figure \ref{fig:LayersVeto}. These layers prevent external photons from reaching the scintillator plastic and avoid photons generated by the scintillator plastic from escaping before reaching the photosensor. Two $2.5\times 2.5 ~\cm^2$ windows are made on the wrapping to allow reading by the photosensors.


\begin{figure}[h]
\centering
    \begin{subfigure}[b]{0.23\textwidth}
    \centering
    \includegraphics[width=\textwidth]{3DesignPrinciples/34BackgroundRejectionSystem/NoCoating.jpeg}  
    \caption{\label{subfig:PlasticScintillatorNoCoating}}
    \end{subfigure}
    \hfill
    \begin{subfigure}[b]{0.23\textwidth}
    \centering
    \includegraphics[width=\textwidth]{3DesignPrinciples/34BackgroundRejectionSystem/TeflonCoating.jpeg}  
    \caption{\label{subfig:PlasticScintillatorTeflon}}
    \end{subfigure}
    \hfill
    \begin{subfigure}[b]{0.23\textwidth}
    \centering
    \includegraphics[width=\textwidth]{3DesignPrinciples/34BackgroundRejectionSystem/AluminiumCoating.jpeg}  
    \caption{\label{subfig:PlasticScintillatorAluminium}}
    \end{subfigure}
    \hfill
    \begin{subfigure}[b]{0.23\textwidth}
    \centering
    \includegraphics[width=\textwidth]{3DesignPrinciples/34BackgroundRejectionSystem/BlackTapeCoating.jpeg}  
    \caption{\label{subfig:PlasticScintillatorBlackTape}}
    \end{subfigure}
 \caption{Different layers used to wrap the cosmic veto detectors. a) Scintillator without coating. b) Teflon coating. c) Aluminium coating. c) d) Black tape coating.}
 \label{fig:LayersVeto}
\end{figure}

As mentioned above, the expected hard cosmic rate at sea level is $7 \times 10^{-3}~\cm^{-2}\second^{-1}\steradian^{-1}$. Taking into account that the solid angle of our detectors is $\omega=0.5434$, calculated by integrating the solid angle of one scintillator on the other, and that the area of the veto is $765~\cm^2$, the expected hard cosmic rate on our cosmic vetos should be $2,909~$event$/\second$. This is an important result which is used in section \ref{sec:TritiumActiveVeto} to determine the efficiency of the cosmic veto.