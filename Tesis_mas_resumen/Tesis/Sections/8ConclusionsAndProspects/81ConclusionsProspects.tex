In this chapter, the main results and conclusions obtained in this PhD thesis are the following:

\begin{enumerate}
\item{} A scalable modular monitor for measuring low activities of tritium in water is proposed. The goal of this monitor is to reach a sensitivity of $100~\becquerel/\liter$ of tritium activity, which is the maximum level allowed for drinking water by the EU directive. The modularity allows reaching the required sensitivity by selecting a given integration time and including the corresponding number of modules.

\item{} The IFIC modules of the proposed TRITIUM monitor consist of $1~\mm$ diameter scintillating fibres read out by photosensors.

\item{} Three different prototypes of the TRITIUM-IFIC module in which different improvements were successively incorporated were developed.

\item{} A surface-conditioning method for the scintillating fibres, consisting of specific rules for cleaving, polishing and cleaning the fibres was developed. This conditioning method was applied to the selected $1~\mm$ diameter uncladded scintillating fibres from Saint-Gobain. An improvement of the fibre photon collection efficiency of a factor $2$ due to polishing and an additional $25\%$ due to cleaning was obtained.

\item{} Although most of the laboratory tests were carried out with PMTs to read out the scintillating fibres, SiPM arrays are proposed by the TRITIUM Valencia team for the final monitor.

\item{} A S13360-1375 SiPM from Hamamatsu, selected for the monitor, was characterized. The most relevant parameters for tritium detection of this photosensor such as breakdown voltage and gain were measured.

\item{} As SiPM gain varies strongly with temperature, a stabilization method to maintain the gain constant with temperature was implemented. This method, which consists in compensating the gain variations with the bias voltage, was tested in the $[20-30]\celsius$ temperature range of interest. Relative variations of the gain of the order of $0.1\%$ were found which guarantees a stable operation of the photosensors.

\item{} In order to obtain the required tritium activity sensitivity, a background rejection system consisting of a lead shield castle and an active plastic scintillator veto was proposed and implemented.

\item{} The passive background rejection system was simulated. A reduction factor of the cosmic ray events detected by TRITIUM-IFIC-2 of $5.5$ was obtained.

\item{} An active veto based on plastic scintillators read out by photosensors was designed and built.

\item{} The characterization of the active veto was done and the optimal parameters for the detection of hard cosmic rays were determined. A count rate of $2.5~\text{events}/\second$ was measured, which gives an efficiency for cosmic ray detection of $85\%$ by comparing this counting rate to the cosmic ray rate at sea level quoted in the literature.

\item{} The active veto was simulated, obtaining a suppression of $60\%$ of the cosmic rays that cross the lead shield and are detected by TRITIUM-IFIC-2.

\item{} The total background suppression by the rejection system obtained from simulations for TRITIUM-IFIC-2 is $92.5\%$.

\item{} A specific efficiency $\eta = (141 \pm 6) \times 10^{-6}~ \second^{-1}  \liter ~ \kilo\becquerel^{-1} \cm^{-2}$ for the TRITIUM-IFIC-2  prototype was measured, which is about an order of magnitude higher than the specific efficiencies reported in the literature. The state-of-the-art of tritium detection with plastic scintillators is substantially improved.

\item{} The MDA obtained with TRITIUM-IFIC-2 is $677~\becquerel/\liter$  for $10~\min$ integration time and $218~\becquerel/\liter$ for $1~\hour$ integration time.

\item{} The goal of the TRITIUM project of measuring an MDA of $100~\becquerel/\liter$ in quasi-real time is expected to be reached with 5 TRITIUM-IFIC-2 prototypes read out in parallel and $1~\hour$ integration time.

\item{} Simulations of the detection resolution of a TRITIUM monitor based on TRITIUM-IFIC-2 were done. A resolution of $100~\becquerel/\liter$ for an integration time of $1~\hour$ and five modules was obtained.

\item{} Three TRITIUM-IFIC-2 prototypes with the corresponding active veto are ready to be installed at the Arrocampo site as soon as possible.

%\item{} Furthermore, it is necessary to quantify the improvement achieved in tritium detection when the SiPM arrays are used as photosensors. Therefore, a similar characterization to the one performed for the TRITIUM-IFIC-2 prototype must be obtained for an identical prototype in which the PMTs are remplaced by SiPM arrays.

\end{enumerate}

%The TRITIUM monitor has been developed with the aim of being able to measure the legal limit established by the EURATOM Concil Directive ($100~\becquerel/\liter$) but it has to be taken into account that it can be used in other different areas such us to control the correct operation of a nuclear facility (high levels of tritium in the water released by the nuclear power plant is one of the first sign of an anomalous functioning of the plant) of even other fields, different to the environmental surveillance, such as research facilities.