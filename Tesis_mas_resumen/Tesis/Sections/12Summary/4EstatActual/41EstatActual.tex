El triti és un dels elements radioactius produïts més abundantment en instal·lacions nuclears. A causa de la seua radiotoxicitat, l'exposició a un excés de triti podria afectar la salut humana i el medi ambient, motiu pel qual molts països han implementat un límit legal per al triti en aigües potables. Aquest límit és establert a Espanya a $100~\becquerel/\liter$ pel CSN seguint la directiva del consell EURATOM \cite{100BqL}.

El projecte TRITIUM es va fundar amb l'objectiu de dissenyar i construir un monitor capaç de mesurar en temps quasireal ($1~\hour$ o menys) activitats de triti en l'aigua inferior al límit legal. El monitor construit fins ara, anomenat TRITIUM consta de tres parts, el detector de triti, on es fa la mesura de l'activitat del triti en aigua, el sistema de rebuig del fons radioactiu, destinat a minimitzar el fons radioactiu mesurat pel detector de triti, i el sistema de purificació de l'aigua, utilitzat per a eliminar partícules i minerals a la mostra d'aigua que pugen reduir l'eficiència del detector. Els avanços realitzats fins ara per la col·laboració TRITIUM es llisten a continuació:

\begin{enumerate}

\item{} Es va construir un monitor per a mesurar baixes activitats de triti en aigua. Aquest presenta un disseny modular el qual li confereix la propietat d'escalabilitat, és a dir, algunes de les característiques rellevants del detector, com ara l'MDA o la resolució, poden ser millorades mitjançant la utilització de més mòduls llegits en paral·lel.

\item{} Es van dissenyar, construir i caracteritzar un total de quatre prototips del detector de triti basats en fibres centellejadores llegides per fotosensors. Amb aquests prototips fou possible identificar i solucionar problemes greus en el disseny, a més d'implementar possibles modificacions que optimitzen la detecció del triti.

\item{} Es van realitzar diverses simulacions amb la llibreria Geant4 per comparar diversos dissenys del detector. L'objectiu d'aquestes simulacions era decidir la configuració que optimitza l'eficiència de detecció del triti en aigua.

\item{} Es va desenvolupar un protocol de tractament de la superfície de les fibres centellejadores que va ser avaluat, obtenint una millora de la col·lecció de fotons de més d'un factor $2$. També es va realitzar una caracterització de les fibres centellejadores, obtenint una eficiència de col·lecció de $(76 \pm 8)\%$ i una incertesa en el nombre de fotons llegits pel fotosensor d'al voltant d'un $3\%$.

\item{} Així mateix, es va caracteritzar un SiPM, model S13360-1375 de Hamamatsu on molts dels paràmetres rellevants per al projecte TRTIUM, com per exemple el guany, van ser mesurats amb una excel·lent precisió (resolució de l'espectre del fotoelectró al voltant de l'$1\%$). A més, a causa de la forta dependència del resposta del SiPM amb la temperatura, es va aplicar amb èxit un mètode d'estabilització del guany, obtenint una variació d'al voltant del $0,1\%$ en el rang de temperatures de treball ($[20-30]\celsius$).

\item{} Es va dissenyar, construir i caracteritzar un sistema de rebuig del fons radioactiu basat en dues parts, un castell de plom amb parets de $5~\cm$ de gruix i un veto actiu basat en plàstics centellejadors. A partir de simulacions es va obtenir una reducció del nombre de rajos còsmics detectats pel monitor TRITIUM d'un factor $5,5$ a causa del castell de plom i del $60\%$ a causa del veto actiu, donant una reducció total del $92,7\%$. Aquesta reducció causada pel veto actiu s'inferior a l'obtinguda experimentalment, probablement causat pel fet de no incloure el recobriment del veto en les simulacions. Per tant, s'espera un factor de reducció total major a l'obtingut per simulacions.

\item{} També es va dissenyar, construir i caracteritzar un sistema de purificació de l'aigua, necessari per a eliminar les partícules i els minerals dissolts. Amb aquest sistema va ser possible assolir nivells de conductivitat de l'aigua d'al voltant de $10~\mu\text{S}/\cm$ sense afectar l'activitat del triti.

\item{} Amb el prototip TRITIUM-IFIC-2, l'últim prototip desenvolupat en la col·laboració TRITIUM, es va obtenir una eficiència específica en la detecció del triti de
$$S = (14,1 \pm 0,6)\cdot{} 10^{-5}~\liter\:\kilo\becquerel^{-1}\second^{-1}\cm^{-2}$$
i un MDA de $220~\becquerel/\liter$ per a mesures d'$1$ hora, millorant l'actual estat de l'art de la detecció del triti.

\item{} A partir dels resultats experimentals obtinguts amb TRITIUM-IFIC-2 i tenint en compte la propietat d'escalabilitat del monitor, s'espera aconseguir l'activitat de $100~\becquerel/\liter$, objectiu del projecte TRITIUM, amb $5$ mòduls llegits en paral·lel.

\item{} L'estabilitat del prototip TRITIUM-IFIC-2 va ser provada al llarg de diversos mesos, tant pel senyal com pel fons, obtenint una desviació estàndard relativa d'al voltant de $2,5\%$.

\end{enumerate}

Actualment el prototip TRITIUM-Aveiro està instal·lat a Arrocampo juntament amb el sistema de purificació de l'aigua i el castell de plom. Aquest prototip ha estat realitzant mesures del fons radioactiu d'Arrocampo durant més de quatre mesos, mostrant una gran estabilitat. En un futur pròxim s'espera poder instal·lar dos prototips TRITIUM-Aveiro addicionals i un veto actiu. A més, tres mòduls TRITIUM-IFIC-2 i un veto actiu estan llests per ser instal·lats tan prompte com siga possible.