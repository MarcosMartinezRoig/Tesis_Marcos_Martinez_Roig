\selectlanguage{spanish}
Muchas personas han colaborado a lo largo de estos cuatro años para hacer posible este trabajo. En primer lugar, me gustaría agradecer enormemente a mis tutores, Jose Díaz Medina y Nadia Yahlali Haddou, por intentar transmitirme en todo momento tantos conocimientos como les ha sido posible. También agradecer su tiempo ya que, a pesar de sus muchas obligaciones, siempre han encontrado un hueco en sus agendas para ayudarme rápidamente con cualquier problema. Estoy muy agradecido por su cercanía y sus consejos, tanto en el mundo laboral como en la vida personal, los cuales me han ayudado a llegar a donde estoy ahora. Gracias de todo corazón por todo, nada de esto hubiera sido posible sin vosotros.

En segundo lugar, me gustaría agradecer a mis compañeros de laboratorio Andrea Esparcia Córcoles, Ana Bueno Fernández, Marcos Llanos Expósito y sobre todo a Mireia Simeó Vinaixa, con quienes los largos e interminables trabajos de laboratorio han sido mucho más fáciles, ayudándome en todo momento a comprender y mejorar los resultados experimentales obtenidos.  También me gustaría agradecer a los técnicos e investigadores del LARAM, Clodoaldo Roldán García, Teresa Cámara García, Vanesa Delgado Belmar y Rosa María Rodríguez Galán, por su ayuda, su tiempo y sus esfuerzos en hacer que todo esto haya sido posible.

En tercer lugar, me gustaría agradecer a todos los investigadores del proyecto TRITIUM, gente que me ha acogido con los brazos abiertos, que me ha ayudado en todo lo que les ha sido posible y con la que he tenido el placer de trabajar y aprender a su lado. Me gustaría agradecer en especial a Carlos Azevedo, junto al cual tuve el placer de trabajar. Gracias por todo el tiempo que invertiste en formarme en temas tan interesantes y dispares como programación con Arduino y Rasberry Pi (llegando a realizar proyectos en común) o simulaciones con el paquete de programación Geant4 (llegando a realizar mis propias simulaciones).

En cuarto lugar, me gustaría agradecer a David Calvo Díaz-Aldagalán, ingeniero del IFIC, por los diseños de las distintas PCBs utilizadas en el proyecto TRITIUM.

Finalmente, me gustaría dar las gracias a los investigadores de otros proyectos los cuales, pese a no ayudarme directamente en mi labor de investigación, han aportado a mi trabajo debido a las sinergias existentes entre los desarrollos experimentales de los proyectos:

\begin{itemize}

\item{} A los investigadores del proyecto DUNE, Anselmo Cervera, Miguel Angel García Peris y Justo Martín-Albo, los cuales me han ayudado con su tiempo, sus conocimientos y su instrumentación experimental a completar la caracterización de los SiPMs de TRITIUM. 

\item{} A los ingenieros del proyecto NEXT, Vicente Álvarez, Marc Querol, Javier Rodríguez y Sara Cárcel, con quienes he compartido laboratorio y, aun sin trabajar directamente con ellos, siempre han estado dispuestos a echarme una mano. 

\item{} A los investigadores del IFIMED, Gabriela Llosá, Ana Ros, Marina Borja, John Barrio, Rita Viegas y Jorge Roser e investigadores del proyecto ATLAS (Urmila Soldevila y Carlos Mariña) por prestarnos sus instalaciones y, en concreto, sus sistemas de control de temperatura los cual fueron necesario para la caracterización de los SiPMs. 

\item{} A los investigadores del laboratorio HYMNS, Ion Ladarescu y César Domingo, quienes me ayudaron en mis primeros pasos con la electrónica PETsys además de asesorarme con las questiones que me han surgido a lo largo del camino.

\item{} A los investigadores del ICMOL, Henk Bolink, Alejandra Soriano, Lidón Gil y Jorge Ferrando por ayudarnos con el protocolo de limpieza de las fibras centelleadoras además de permitirnos usar sus instalaciones.

\item{} A toda la gente de los distintos departamentos del IFIC que me han ayudado con su tiempo y sus consejos. Me gustaría agradecir especialmente la ayuda de Jose Vicente Civera, Manolo Montserrate, José Luis Jordan y Daniel Tchogna del departamento de mecánica, Ximo Navajas, Ximo Nadal, Carlos Martínez y Francisco Javier Sánchez del departamento de informática y Manuel López, Jorge Nacher y Jose Mazorra del departamento de electrónica.

\item{} A mis amigos del IFIC, Pepe, Kevin, José Antonio, Pablo, Stefan y Victor, los cuales, aun sin haber aportado directamente a mi trabajo, compartir mis días con ellos ha sido uno de los mayores apoyos de esta aventura. No solo me llevo amigos de esta etapa, me llevo una familia.

\item{} A mi amigo, Galo, una amistad que empezó en el master, donde tuve el placer de compartir muchas de las duras horas de estudio, y continuó en el doctorado, donde hemos compartido muchos buenos momentos. Siento que esta amistad nos hemos ayudado a crecer mutuamente, tanto profesional como personalmente. Una amistad que, sinceramente, espero que dure para toda la vida.

\item{} A todos los miembros de mi familia, quienes sin comprender exactamente en qué consiste mi trabajo, me han apoyado y animado ciegamente desde el principio. Fueron, son y serán una pieza imprescindible en todas las áreas de mi vida.

\end{itemize} 

Finalmente me gustaría agradecer al programa INTERREG-SUDOE por financiar el proyecto TRITIUM, gracias al cual todo esto se ha podido materializar.