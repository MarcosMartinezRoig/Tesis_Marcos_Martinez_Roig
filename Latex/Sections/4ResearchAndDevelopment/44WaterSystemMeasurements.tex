This section describes the characterization of the ultrapure water system, employed to ensure that the quality of the water sample used to be measured fulfills the requirements of the TRITIUM detector. There are three different requirements that this ultrapure water system must satisfy:

\begin{enumerate}
\item{} A quite low conductivity\footnote{Conductivity is the ability of a material to conduct electrical current. In liquids, conductivity is related to the presence of salts (presence of positive and negative ions)} of the water, around $10~\mu\sievert/\cm$, to avoid that external particles disolved in the water be deposited on the fibers, drastically reducing the detector efficiency.

\item{} The radioactive particles (other that tritium isotope) from the water sample should be removed because tritium cannot be separated from other radioactive isotops.

\item{} Lastly, the tritium activity should not be affected by the water purification process. 

\end{enumerate}

To verify that these requirements are complied with, a characterization of the water sample for both, raw water and purified water, was done. This characterization consisted of measuring the water sample conductivity and the activity of the different radioactive element present in the sample. The turbidity and the chemical components of the water sample were also measured. The sample of the raw water was taken at 40 meters from the ultrapure water system and two meters deep in the river since it is the place where the samples used in TRITIUM monitor will be taken. Variations of up to $25\%$ in the tritium activity was measured between both points (due to the diffusion of tritium along the river).

The chemical composition of the water was measured before the ultra-purification process by a physico-chemical analysis, shown in Table \ref{tab:ChemicalComponentsRawWater}.

\begin{table}[htbp]
%%\centering
\begin{center}
\begin{tabular}{|c|c|}
\hline
Chemical components & Concentration ($\milli\gram/\liter$)\\
\hline \hline \hline
$\ce{CO_{3}H^-}$ & $154$ \\ \hline
$\ce{Mg}$ & $46$ \\ \hline
$\ce{Ca}$ & $105$ \\ \hline
$\ce{NO_{3}^-}$ & $16$ \\ \hline
$\ce{Cl^-}$ & $196$ \\ \hline
$\ce{NO_{2}^-}$ & $0.03$ \\ \hline
$\ce{K}$ & $11$ \\ \hline
$\ce{Na}$ & $173$ \\ \hline
$\ce{SO_{4}^-}$ & $217$ \\ \hline
Dry Residue & $1029$ \\ \hline
\end{tabular}
\caption{Chemical components and turbidity measured in the raw water sample.}
\label{tab:ChemicalComponentsRawWater}
\end{center}
\end{table}

The water sample contains a number of components given in the Table, that must be removed to prevent their deposition on the scintillating fibers of the detector, that would reduce their sensitivity to tritium activity.

The water turbidity\footnote{The turbidity of water is the loss of transparency due to dissolved particles, normally measured in Nephelometric Units of Turbidity, NTU, as the intensity of scattered light at 90 degrees.} was measured using the Hanna Hi 9829 portable multiparameter system from Hanna Instruments \cite{TurbiditySystem}, obtaining a value of $29$ NTU, much higher that the WHO recommended limit of $5$ NTU for drinking water.

The water conductivity was measured for both, raw and purified water, using the same system, the Hanna Hi 9829. The results of these measurements, together with the measurement of the conductivity of rejected water, described in section \ref{subsec:SetUpWaterSystem}, are presented in Table \ref{tab:ConductivityValues}.

\begin{table}[htbp]
%%\centering
\begin{center}
\begin{tabular}{|c|c|c|c|}
\hline
Date & Raw ($\mu\sievert/\cm$) & Pure ($\mu\sievert/\cm$) & Reject ($\mu\sievert/\cm$) \\
\hline \hline \hline
$1/8/18$ & $970$ & $11.85$ & $1442$ \\ \hline
$7/8/18$ & $958$ & $11.8$ & $1632$ \\ \hline
$14/8/18$ & $966$ & $12.04$ & $1725$ \\ \hline
$22/8/18$ & $980$ & $12.54$ & $1702$ \\ \hline
$28/8/18$ & $987$ & $9.9$ & $1692$ \\ \hline
$5/9/18$ & $1009$ & $12.02$ & $1645$ \\ \hline
\end{tabular}
\caption{Measurements of the conductivity for several samples of water.}
\label{tab:ConductivityValues}
\end{center}
\end{table}	

As it can be seen in the first column, the raw water sample has high values of conductivity, due to its content of ions, shown in Table \ref{tab:ChemicalComponentsRawWater}. It can be noticed in the second column of the table that the conductivity of pure water was reduced by almost two orders of magnitude, to values close to $10~\mu\sievert/\cm$, fulfilling the requirement. In the third column, it can be remarked that the rejected water conductivity is higher than that of raw water, because this water contains the removed ions from pure water.

The gamma radioactive elements present in both, raw water and purified water, were identified and their activities measured by a HPGe, high purity germanium detector. A gamma analysis was carried out to determine the emitters with long enough lifetime to be measured. The radioactive isotopes found in the raw water sample with measurable activities were $\ce{^{40}K}$ and $\ce{^{226}Ra}$ which were completely removed in the purified water.

%A gamma analysis was carried out to find the natural gamma emitters (those that come from the natural radioactive series, Table \ref{tab:NaturalRadioactiveSeries}) and the artificial gamma emitters with long enough lifetime to be measured (those that come from the activation of nuclear fission of neutrons).  

The tritium activity was measured by liquid scintillation counting (LSC) to find its modification by the ultra-purification process. The raw water was filtered at 0.45 microns to remove any particles that could cause the extinction of the scintillation signal. Table \ref{tab:ActivityTritiumValues} show several measurements of the tritium activity for different water samples before and after purification. As it is see, tritium activity is not affected by the purification process.

\begin{table}[htbp]
%%\centering
\begin{center}
\begin{tabular}{|c|c|c|}
\hline
Date & Raw ($\becquerel/\liter$) & Pure ($\becquerel/\liter$) \\
\hline \hline \hline
$7/8/18$ & $24 \pm 3$ & $26 \pm 4$ \\ \hline
$11/12/19$ & $13.2 \pm 2.1$ & $13.85 \pm 2.2$ \\ \hline
$15/01/20$ & $30.6 \pm 4.2$ & $30 \pm 4$ \\ \hline
\end{tabular}
\caption{Measurements of the tritium activity for several samples of both, raw and pure water.}
\label{tab:ActivityTritiumValues}
\end{center}
\end{table}	