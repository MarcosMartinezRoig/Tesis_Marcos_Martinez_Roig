Finally the lead shielding and active vetos, explained in section \ref{subsec:TritiumIFIC2Simulation}, were included in the simulation of the Tritium-IFIC 2 prototype. The objective of these simulations was to demonstrate its need, quantifying its effect in reducing the cosmic events detected by the prototype.

For this task, similar to that done in section \ref{subsec:FiberDiameterSimulation}, the tritium source was not simulated. Insted, a cosmic events source was used, which was simulated through the CRY library previously explained.

As can be seen in Figures \ref{subfig:RealHardCosmicEvent} and \ref{subfig:FakeHardCosmicEvent}, two plastic scintillators were simulated with the dimensions mentioned in section \ref{subsec:SetUpActiveShield} and located above and below of the Tritium-IFIC 2 prototype simulated. 

The optical properties given to this plastic scintillators were the same as the one used for the fibers, which are the refractive index, the light attenuation spectrum and energy emission spectrum, whose values was obtained from their data sheet provided by the manufacturer \cite{ScintillatorVeto}.

As shown in this figure, two PMTs, model R8520-460 from Hamamatsu company, were simulated to read each plastic scintillator, similar to that presented in section \ref{subsec:SetUpActiveShield}.

Finally, a lead shielding was simulated, whose properties were taken from the Geant4 NIST database. The dimensions of the simulated lead shielding were $60~\cm$ long, $60~\cm$ wide, $70~\cm$ high, which is the minimum needed to accomodate the active vetos and Tritium detector module inside.  The length of this is smaller than the built lead shielding, $148~\cm$, shown in section \ref{subsec:SetUpActiveShield}. The reason for this is that only one tritium detector module was simulated, so the dimension of the lead shielding can be reduced to optimize simulation time and computing resources.

The results of these simulation are shown in section \ref{sec:ResultsSimulatedBackgroundRejectionSystem}, where they are discussed.