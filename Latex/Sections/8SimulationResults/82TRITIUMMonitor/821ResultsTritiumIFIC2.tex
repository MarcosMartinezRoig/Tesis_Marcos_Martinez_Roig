This section shows the results obtained with the simulation of several prototypes of TRITIUM-IFIC 2, which was used for two different objectives. On the one hand, these simulations were carried out to find the Low Detection Limit, LDL, of the prototype for tritiated water activity, which is an important parameter to know the limitation of the prototype. On the other hand, these simulations serve to study how tritium detection is affected due to the integration count time and the number of prototypes read in parallel.

First, tritium detection was studied using only one TRITIUM-IFIC 2 prototype, throguh the simulation of various activities of tritiated water. The integration count time used was $10~\min$ and continuous use of the detector during 3 months was simulated for each activity studied.

Several variables were used as tests in each measurement to verify the correct simulation of the different steps such as the simulated tritium source, the simulated energy deposition in scintillating fibers and their subsequent emission of photons, etc. Some of these variables are detailed in the appendix \ref{App:TestVariablesSimulations}.



Se obtienen un máximo de 15 fotones por evento de tritio, similar a lo que se obtiene experimentalmente.

Imagen del detector con una fibra tocada.

Varios tests para ver que la simulación es correcta: (Quizá en un appendice?)

- Simulación de la fuente de tritio correcta tanto en energía (ya visto en una sección anterior) como en distribución espacial (Proyección en ejes X, Y y Z).

- Deposición de energía del tritio en la fibra correcta (ya visto en uan sección anterior). También distribución espacial y temporal de esta correcta. Distancia con la fibra esperada (tritio generado y tritio detectado).

- Deposición de tirtio tanto en el agua como en la fibra y sus recorridos libres medios son los esperados.

- Número máximo de fotones detectados son los esperados.

Datos obtenidos para 1, 5 y 10 detectores.

Ajuste lineal cuentas por segundo (con su error) frente a actividad de la muestra.

Estudio del counting time: Simulaciones para 10 min, 30 min y 60 min.

Ver como se mejora la resolución en \% para tiempos mayores. Extraer algún tipo de relación de estas?

PEORES RESULTADOS SIMULADOS QUE CON AVEIRO PERO MEJORES EXPERIMENTALMENTE. La diferencia debe de estar en que uno usa fibras pulida y limpiadas y el otro no.