The lead shield and the active vetos, described above, were included in the simulation of the Tritium-IFIC-2 prototype. The purpose of these simulations was to quantify the reduction of cosmic background detected by the prototype. For this task, the tritium source was replaced by a cosmic event source, simulated using the CRY library. The optical properties included for the plastic scintillators of the active veto are the refractive index, the light attenuation spectrum and energy emission spectrum, the values of which were obtained from their data sheet provided by the manufacturer \cite{ScintillatorVeto}. Two PMTs, model R8520-460 from Hamamatsu, were simulated to read each plastic scintillator, as described in section \ref{subsec:SetUpActiveShield}. The lead shield was simulated with properties taken from the Geant4 NIST database. The dimensions of the simulated lead shield were $60 \times 60 \times 70~\cm^3$, which is the minimum needed to accomodate an active veto and a TRITIUM-IFIC-2 prototype inside. The length of the simulated lead castle, $60~\cm$, is smaller than the real dimension, $148~\cm$, of the lead shield at Arrocampo. The reason for this is that only one tritium detector module was simulated, so the dimension of the lead shield was reduced to optimize simulation time and computing resources. As for the simulations of the TRITIUM-IFIC-2 prototype, the characteristics of the events generated (energy distribution, position and momentum distribution, etc) were checked to verify the simulation.

Three different simulations were carried out to quantify the tritium detection improvement due to the lead shield and the cosmic veto. The first simulation consists of a TRITIUM-IFIC-2 prototype and the cosmic ray source. In the second simulation, a lead shield was added and for the third simulation, the cosmic veto was also included. It is found that the cosmic rays detected by the TRITIUM-IFIC-2 prototype are reduced by around a factor 5.5 when a lead shield with walls of $5~\cm$ is included, the width of the shield currently installed in Arrocampo. This reduction is most probably caused by the suppression of the soft cosmic radiation (energy lower than $200~\MeV$). It has to be taken into account that the natural background of the installation site was not included in this simulation. This radioactive background would also be mitigated by the lead shield, so the expected reduction of the radioactive background due to the passive veto would be even better. Around $60\%$ of the cosmic events that penetrate the lead shield and reach the TRITIUM IFIC 2 prototype, which are the hard cosmic rays, are detected by the cosmic veto and, therefore, would be mitigated from the background in the detector.