As hard radiation cannot be stopped by a moderate lead thickness, cosmic vetos are employed, which consist of at least two complementary detectors in coincidence that reject events simoultaneously detected in both of them. As shown in Figure \ref{fig:VetoAndPrototype}, the two complementary detectors are placed one above and the other below the TRITIUM detector. The distance between both detectors, $34.2~\cm$ for the latest prototype developed at IFIC, is set by the dimension of the detector to be placed inside.

\begin{figure}[h]
\centering
\includegraphics[scale=0.45]{3DesignPrinciples/34BackgroundRejectionSystem/Vetos_y_prototipo.png}
\caption{Cosmic veto and Tritium-IFIC 2 prototype in an aluminum mechanical structure developed at IFIC.\label{fig:VetoAndPrototype}}
\end{figure}

A hard cosmic event crossing simultaneously both cosmic detectors is schematically sketched in figure \ref{subfig:RealHardCosmicEvent}. Each cosmic detector has two photosensors, so the electronic configuration given in Figure \ref{subfig:ElectronicConfiguraiton4PMT} is used to make time coincidences. The TRITIUM detector is read out in anti-coincidence with the cosmic veto to reject the hard cosmic events from the tritium measurement. The expected hard cosmic rate at sea level for muons is $7\times 10^{-3}~\cm^{-2}\second^{-1}\steradian^{-1}$ \cite{PDG, HardCosmicMuonRate}, as shown in the plot of Figure \ref{fig:HardCoscmicRate}. As time coincidences are triggered by logical gates of about $10~\nano\second$, the probability of recording two different hard cosmic events in temporal coincidence, one in each detector, as drawn in Figure \ref{subfig:FakeHardCosmicEvent}, are negligible.

\begin{figure}[h]
\centering
    \begin{subfigure}[b]{0.45\textwidth}
    \centering
    \includegraphics[width=\textwidth]{3DesignPrinciples/34BackgroundRejectionSystem/Real_Event.png}  
    \caption{\label{subfig:RealHardCosmicEvent}}
    \end{subfigure}
    \hfill
    \begin{subfigure}[b]{0.45\textwidth}
    \centering
    \includegraphics[width=\textwidth]{3DesignPrinciples/34BackgroundRejectionSystem/Fake_Event.png}  
    \caption{\label{subfig:FakeHardCosmicEvent}}
    \end{subfigure}
   \caption{Hard cosmic events detected with the cosmic veto of TRITIUM: a) Real coincidence event, b) randomly coincidence envent that may mimic a hard cosmic event.}
 \label{fig:HardCosmicEventsSimulation}
\end{figure}

\begin{figure}[h]
\centering
\includegraphics[scale=0.6]{3DesignPrinciples/34BackgroundRejectionSystem/HardCosmicRate.png}
\caption{Hard cosmic muon rate at different depths with respect to the sea level ($10^0$) \cite{HardCosmicMuonRatePlot}.\label{fig:HardCoscmicRate}}
\end{figure}

The vetos are made of a plastic scintillator block from Epic-Crystal \cite{ScintillatorVeto}. Its properties are given in Table \ref{tab:ParametersScintillatorVeto} and its energy emission spectrum is displayed in Figure \ref{fig:EmissionEnergySpectrumVeto}.

\begin{table}[htbp]
\centering{}%
\begin{tabular}{lc}
\toprule 
Property & Value \tabularnewline
\midrule
\midrule 
Base material & Polystyrene \tabularnewline
Growth method & Polymeric \tabularnewline
Density ($\gram/\cm^3$)& 1.05 \tabularnewline
Refractive index & 1.58 \tabularnewline
Soften temperature ($\degree$) & 75-80 \tabularnewline
Light output (Anthracene) & 50-60\% \tabularnewline
H/C ratio & 1.1 \tabularnewline
Emission peak (nm) & 415 (Blue) \tabularnewline
Decay Time, (ns) & 2.4 \tabularnewline
Hygroscopic & No \tabularnewline
\bottomrule
\end{tabular}
\caption{Properties of plastic scintillators from Epic-Crystals used for the cosmic vetos~\cite{ScintillatorVeto}.}
\label{tab:ParametersScintillatorVeto}
\end{table}

\begin{figure}[]
\centering
\includegraphics[scale=0.35]{3DesignPrinciples/34BackgroundRejectionSystem/EmissionEnergySpectrumVetos.png}
\caption{Emission energy spectrum of the plastic scintillator from Epic-Crystals used for the cosmic vetos\label{fig:EmissionEnergySpectrumVeto}~\cite{ScintillatorVeto}.}
\end{figure}

The energy spectrum has a peak very close to that of the scintillating fibers used, so the same photosensors are used to read them out. The dimensions of the scintillator blocks are $45 \times 17 \dot{} 1~\cm^2$ with a thickness of $1~\cm$. They are wrapped by three different layers, PTFE sheets, aluminum and black tape, as shown in Figure \ref{fig:LayersVeto}. These layers prevent external photons from reaching the plastic scintillator and prevent photons generated by the scintillator from escaping before reaching the photosensor. Two $2.5\times 2.5 ~\cm^2$ windows are made on the wrapping to allow coupling of the photosensors.


\begin{figure}[h]
\centering
    \begin{subfigure}[b]{0.23\textwidth}
    \centering
    \includegraphics[width=\textwidth]{3DesignPrinciples/34BackgroundRejectionSystem/NoCoating.jpeg}  
    \caption{\label{subfig:PlasticScintillatorNoCoating}}
    \end{subfigure}
    \hfill
    \begin{subfigure}[b]{0.23\textwidth}
    \centering
    \includegraphics[width=\textwidth]{3DesignPrinciples/34BackgroundRejectionSystem/TeflonCoating.jpeg}  
    \caption{\label{subfig:PlasticScintillatorTeflon}}
    \end{subfigure}
    \hfill
    \begin{subfigure}[b]{0.23\textwidth}
    \centering
    \includegraphics[width=\textwidth]{3DesignPrinciples/34BackgroundRejectionSystem/AluminiumCoating.jpeg}  
    \caption{\label{subfig:PlasticScintillatorAluminium}}
    \end{subfigure}
    \hfill
    \begin{subfigure}[b]{0.23\textwidth}
    \centering
    \includegraphics[width=\textwidth]{3DesignPrinciples/34BackgroundRejectionSystem/BlackTapeCoating.jpeg}  
    \caption{\label{subfig:PlasticScintillatorBlackTape}}
    \end{subfigure}
 \caption{Different layers used to wrap the cosmic veto detectors. a) Scintillator without coating. b) PTFE coating. c) Aluminium coating. c) d) Black tape coating.}
 \label{fig:LayersVeto}
\end{figure}

Considering the expected hard cosmic rate of $7 \times 10^{-3}~\cm^{-2}\second^{-1}\steradian^{-1}$ at sea level, and taking into account that the solid angle of the veto detectors is $\omega=0.5434$, calculated by integrating the solid angle of one scintillator on the other, and that the area of the veto is $765~\cm^2$, the expected hard cosmic rate on the TRITIUM cosmic vetos is $2.909~$event$/\second$. This estimation is used in section \ref{sec:TritiumActiveVeto} to determine the detection efficiency of the cosmic veto.