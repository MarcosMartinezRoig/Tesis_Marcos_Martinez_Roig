The scintillation yield provided by the manufacturer, $8000~\text{phot.}/\MeV$ only works for Minimum Ionizing Particles (MIP). As tritium electron energies are far from being MIP particles, the output light generated by the scintillating fibers was studied. For this task, the energy deposition of tritium electrons on scintillating fibers was added to the simulation.

A correction, called the Birks coefficient ($k_B$), caused by the light quenching is considered to the proporcionality between the output light per unit of apth length, $\frac{dL}{dx}$, and the energy deposited per unit of path length $\frac{dE}{dx}$\cite{BirksPaper}.

\begin{equation}
\frac{dL}{dx}= S\frac{\frac{dE}{dx}}{1+k_B\frac{dE}{dx}}
\label{eq:birkscoefficient}
\end{equation}

where the S is the scintillation yield, provided by the manufacturer. A value of $k_B=0.126~\mm/MeV$ was considered for the Birk's coefficient which is the one used for scintillators based on polystyrene \cite{BirksCoefficient}. The effect of this correction is shown and discussed in section \ref{subsec:ResultsOutputLight}.