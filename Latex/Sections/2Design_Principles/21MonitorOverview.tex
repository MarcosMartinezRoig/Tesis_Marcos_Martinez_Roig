The objective of the TRITIUM project is the design, development, construction and commissioning of an automatic station for real-time monitoring of low levels of tritium in water. To achieve this objective, the TRITIUM experimental group has developed a monitor that is based on several parts that will be explained in the sections contained in this chapter. The different parts which is contained in our monitor are:

\begin{itemize}

\item{} The tritium detector, chapter \ref{chap:Prototypes}, that is based on several modules that are read in parallel. Each module consists of hundreds of scintillating fibers, section \ref{subsec:PlasticScintillators}, read by two coincident photosensors, section \ref{subsec:Photosensors}. These scintillation fibers are directly in contact with the water sample whose tritium level we intend to measure and the photosensors included in this study are photomultiplier tubes (PMT), section \ref{subsubsec:PMTs}, or silicon photomultipliers (SiPM), section \ref{subsubsec:SiPM}.

\item{} The ultrapure water system, section \ref{sec:UltraPureWaterSystem}, that is used to condition the water sample before the measurement. This system removes all the organic particles that are dissolved in this water and all the particles whose diameter is greater than $1~\mu\meter$ without affecting the level of tritium in the sample. It is important for two reasons, on the one hand because, as we have seen in the section \ref{sec:TritiumProperties}, the mean free path of tritium in water is very short, $5$ or $6~\mu\meter$  so it is important to avoid the deposition of this particles in fibers because this would prevent us from tritium detecting. On the other hand some of this particles disolved in the water sample are natural radioactive particles that would increase the background of our detector if we didn't remove it and, due to the fact that we have few tritium events in our samples, it is very important to reduce the background of our detector as much as posible.

\item{} The background rejection system, chapter \ref{chap:DesignPrinciples} that is based on two different parts. On the one hand we use a passive shield, section \ref{subsec:SetUpPassiveShield}, that consists of a lead castle inside which we use our detector. It is used to eliminate the natural radioactive background that is found in the place where we use the Tritium detector, generally the events with relatively low energy ($<200~\MeV/$nucleon). On the other hand, we use an active veto, section \ref{subsec:SetUpActiveShield}, consisting of two (or more) plastic scintillation blocks, each read by two of the chosen photosensors. This active veto is inside the passive shield and the tritium detector is placed between both plastic scintillation blocks. This active veto is important because there are high energy events ($>200~\mega\eV$), such as cosmic events, that can travel through the passive shield and affect to the measurement of our detector. Contrary to what happens with low energy events, it is difficult to avoid that these high energy events arrive to our detector. What we will do with this active veto is to detect these high energy events and, for each of them, open narrow time windows in which we will not read the Tritium detector to prevent these events from affecting the tritium measurement.

\item{} A general electronic system, section \ref{}, that will be used to monitor all the different parts of this monitor and send an alarm if the tritium level limit, which has been set for us ($100~\becquerel/\second$), is exceeded.

\end{itemize}

When this monitor is working and several tests have been passed with which we can verify that it works correctly, it will be included in the existing early warning system in Extremadura, which consists of various types of radioactive detectors whose objective is to monitor the status of the environment around various locations including the Arrocampo nuclear power plant.