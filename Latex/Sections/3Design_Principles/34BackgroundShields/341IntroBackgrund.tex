The aim of the background rejection system is to reduce the radioactive and cosmic background that affects to the TRITIUM monitor. The TRITIUM project follows the ALARA principle for the tritium activity measurement, that is, to measure tritium activity "as low as reasonably achievable". The detection limit of tritium activity is set by the uncertainty in the activity of the background of the natural radioactivity measured by the TRITIUM detector, since tritium activities below this uncertainty cannot be distinguished from the background. Therefore, the background uncertainty must be reduced as much as possible. The total uncertainty is the quadratic sum of all the different uncertainties related to the measurement, i.e., the statistical uncertainty\footnote{Uncertainty due to the statistical nature of the radioactivity process}, $\sigma_{st}$, the systematic uncertainty\footnote{uncertainty due to the manufacturing process of the detectors}, $\sigma_{si}$, etc. Because of the Poissonian nature of the process, the statistical uncertainty is given by the square root of the measured activity, $A_{m}$, which can be reduced by minimizing detected background events.

\begin{equation}
\sigma_{T}^2 = \sigma_{st}^2 +\sigma_{si}^2; \qquad \qquad \sigma_{st;bak} = \sqrt{A_{m;bak}}
\label{eq:SquareSumUncerainty}
\end{equation} 

The background rejection system of the TRITIUM monitor reduces the background activity measured by the TRITIUM detector, minimizing the statistical component of the background uncertainty.

The background of TRITIUM has two different sources. On the one hand, radioactive elements that are present in the crust of the Earth, mainly $\ce{^{40}K}$ and elements from the four different natural radioactive series, shown in Table \ref{tab:NaturalRadioactiveSeries}. On the other hand, the cosmic ray radiation. The primary cosmic radiation, of extra-terrestrial origin, is composed of high-energy particles, mainly protons and $\alpha$ particles, which interact with the Earth's atmosphere and generate a shower mainly composed by muons, electrons, photons and neutrons.

\begin{table}[htbp]
\centering{}%
\begin{tabular}{lcccc}
\toprule 
Mass Num. & Series & Primary & Half life (y) & Final \tabularnewline
\midrule
\midrule 
4n & Thorium & $\ce{^{232}Th}$ & $1.41 \cdot{} 10^{10}$ & $\ce{^{208}Pb}$ \tabularnewline
4n+1 & Neptunium & $\ce{^{237}Np}$ & $2.14 \cdot{} 10^{6}$ & $\ce{^{209}Pb}$ \tabularnewline
4n+2 & Uranium-Radium & $\ce{^{238}U}$ & $4.51 \cdot{} 10^{9}$ & $\ce{^{206}Pb}$ \tabularnewline
4n+3 & Uranium-Actinium & $\ce{^{235}U}$ & $7.18 \cdot{} 10^{8}$ & $\ce{^{204}Pb}$ \tabularnewline
\bottomrule
\end{tabular}
\caption{Classification of natural radioactive series \cite{NaturalRadioactiveSeries1, NaturalRadioactiveSeries2}. The information displayed for each radioactive series is the multiplicity of the mass number, the name of the series, the primary and final element, and the half-life of the primary element.}
\label{tab:NaturalRadioactiveSeries}
\end{table}
Cosmic radiation depends on several parameters like the longitude, latitude, and the solar activity cycle. The spatial distribution of cosmic rays, mainly muons, follows a $cos^2(\theta)$ distribution with the zenith angle. 

Two different techniques are employed for background suppression:

\begin{enumerate}

\item{}  The soft background component, with energy below $200~\MeV/$nucleon, is stopped by a lead castle, described in section \ref{subsec:SetUpPassiveShield},

\item{} The hard background component, with energy greater than $200~\MeV/$nucleon, is much more difficult to stop and the technique employed is the use of a cosmic veto in anti-coincidence with the TRITIUM detector, reported in section \ref{subsec:SetUpActiveShield}. %that is, we will save the measured tritium event just when we don't measure any hard cosmic event in time coincidence.

\end{enumerate}