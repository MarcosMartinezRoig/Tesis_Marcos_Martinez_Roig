En el marco del proyecto \textit{Tritium}, se ha realizado un primer paso construyendo un prototipo en forma de U cuyo interior, con una capacidad de $39~\milli\liter$, se ha rellenado con un una solución de agua tritiada con una actividad de $108.11~\mega\becquerel/\liter$ (sec. $\ref{sec:Resultados}$).

Se ha conseguido obtener una señal del sistema a partir de la cual, mediante un tratamiento de datos \textit{off-line}, que incluye la substracción del fondo, se ha obtenido una señal de actividad debida únicamente tritio (fig. $\ref{senaltritio}$). Esta señal nos permite comprobar que el prototipo funciona correctamente ya que es capaz de discernir la presencia de tritio a partir de su contribución a la actividad total (background).
Sin embargo se ha visto que el prototipo presenta una eficiencia muy lejos de la prevista, lo cual implica una pérdida sucesos que una medida de la actividad del agua tritiada muy inferior al valor real. 

Para comprobar hasta que punto es ineficiente nuestro prototipo se ha realizado una serie de estimaciones que nos han permitido determinar aproximadamente la actividad efectiva máxima que esperaríamos detectar con nuestro prototipo debido únicamente al agua tritiada, cuyo valor es $215.03~\becquerel$. En esta estimación se han tenido en cuenta eficiencias tanto de las fibras como de los fotomultiplicadores. 

Sin embargo, la actividad medida con nuestro prototipo es únicamente de $0.013\becquerel$, cuatro órdenes de magnitud por debajo del valor máximo. Se trata de una pérdida de cuentas realmente importante que produce un reordenamiento de prioridades en el proyecto ya que, de lo contrario, no podremos reducir la actividad del tritio en la muestra y seguir obteniendo señal. 

La mayor prioridad , por tanto, es obtener una mayor eficiencia. Por tanto, el primer paso es determinar la causa de  esta pérdida de la señal. 

Se ha visto que un punto muy importante es el clad de la fibra, el cual permite una recolección eficiente de la luz. Debido a que las fibras utilizadas en el prototipo no presentan clad tenemos una gran pérdida de la señal en cada reflexión ($94\%$). Esto nos permite considerar que, aproximadamente, solo contribuirán a la señal los fotonesde centelleo producidos en la fibra en la dirección del ángulo sólido cubierto por las caras finales de cada fibra, es decir, el ángulo sólido cubierto por las caras finales del haz de fibras. 
A partir de una sencilla estimación  numérica, se ha visto que se produce una considerable pérdida de cuentas, llegando a factores del orden de $10^{-4}$ o $10^{-5}$ en distancias de apenas $2$ o $4~\cm$ respectivamente, valores que explicarían nuestra medida tan baja. Por tanto un primer paso en el siguiente prototipo será la utilización de fibras centelleadoras con un proceso de aluminizado que disminuyan la pérdida de luz. 

Incluir clad en las fibras centelleadoras es un proceso dificil y bastante laborioso, ya que Saint-Gobain únicamente ofrece clads comerciales de aproximadamente $40~\micro\meter $, que no puede nser empleado este experimento, ya que ningún electrón procedente de la desintegración del tritio conseguiría atravesar el clad y producir señal. Por tanto, sólo nos queda buscar algún método alternativo para obtener un clad de menor grosor o mejorar el guiado de la luz. 

Estamos considerando la posibilidad de utilizar las instalaciones del ICMOL para realizar  una deposición de materiales por evaporación en vacío, del orden de cientos de nanometros o menos. Sin embargo habrá que realizar un estudio sobre la adherencia, ya que  las fibras se encontrarán sumergidas en agua  y necesitamos que el clad  o el aluminizado no se desprenda de  la fibra. También habrá que ver cual es el material que utilizaremos para realizar el clad. Lo ideal sería utilizar un materíal plástico. Sin embargo, la evaporación no es  una técnica estudiada para materiales no metálicos, por lo que no se conoce el resultado..

También es ha visto la necesidad de cambiar la forma del prototipo. En la forma actual, forma de U, inicialmente elegida por ser más segura,  la parte del haz de fibras (aproximadamente el $33\%$) no contribuye a la señal ya que necesita un número excesivo de reflexiones para llegar al PMTs y, en cada una de estas, existirá siempre una pérdida de la señal.

Por tanto,  podemos ver que un punto decisivo será avanzar en las simulaciones para determinar el grosor y el material del clad, el diseño del prototipo, etc, que optimicen la señal obtenida. También nos permitirá estar preparados para superar, de la mejor forma futuros problemas que  encontraremos.

En último lugar, también se han comentado futuros cambios que se están estudiando pero cuya importancia es menor y podemos aplazar para un futuro tales como la automatización del sistema, utilización de electrónica de bajo ruido, etc.