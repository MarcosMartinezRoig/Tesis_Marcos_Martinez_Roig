Tritium is one of the most abundantly emitted radioisotopes by nuclear facilities and, specifically, by nuclear power plants. Large amounts of tritium are normally produced in the water of their cooling system, which are finally emitted to the environment. Due to the fact that large releases of tritium could be dangerous for human health and for the environment, there exist several regulations around the world which try to control this radioactive emissions in each country, like the Directive Europeen 2013/51/Euratom, which establishes the tritium limit for drinking water in Europe to $100~\becquerel/\liter$, or the Environmental Protection Agency, in United States, that limits tritium in drinking water to $20~\nano\curie/\liter$.

Due to the low energy of electrons emitted in the tritium decay, very sensitive detectors are needed for measuring them like LSC. The issue with LSC is that it is an off-line method and the measurement process can take 2 days or more, a time too long to detect a problem in the NPP.

Detectors based on solid scintillators are a promissing idea for building a tritium detector that works in quasi-real time. This type of detectors is developed so far succesfully but without achieving the required sensitivity of the legal limits.

The results of the TRITIUM project are presented in this thesis. In the framework of this project a quasi-real time monitor for low tritium activities in water have been developed. This monitor is based on a tritium detector that contains several detection cells which are read in parallel, several active vetos and a pasive shielding for reducing the natural background of the natural radioactivity and a water purification system to prepare the sample before being measured. Each detection cell is made up of hundreds of scintillating fibers read out by PMTs or SiPM arrays.

The final objective of this monitor will be the radiological protection around the nuclear power plant. This monitor will provide an alarm in case of an unexpected tritium release that exceeds the legal limits stablished in Europe. The final idea will be to included this monitor in the early alarm system of Extremadura consisting of several detectors the objective of which is to control the impact of Nuclear Power Plants to the environment.

\vspace{1cm}

\textbf{Keywords:} Very low-energy charged particle detectors, radiation monitoring, tritium detection, scintillators, scintillating fibers and light guides, detector design and construction technologies and materials, instruments for environmental monitoring, detector modelling and simulations.

%Tritium, Tritiated water, Real-time monitor, Nuclear Power Plant, Environmental Safety, Radiosecurity.