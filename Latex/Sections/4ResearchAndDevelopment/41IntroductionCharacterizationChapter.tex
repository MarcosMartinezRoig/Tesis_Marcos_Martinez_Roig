This chapter shows the characterization of each individual part of the TRITIUM monitor, including scintillating fibers, SiPMs (at the individual SiPM level and at the matrix level), the ultrapure water system and the background rejection system, consisting of the lead shielding and the active veto. 

This characterization is one of the most important things to do because it will help us to understand their behaviour and the results obtained with the full monitor. Furthermore, several developments have been made  to improve interesting parameters of the TRITIUM monitor components to enhance the monitor's capabilities of tritium detection.

All these studies have been carried out inside a special light-tight box, called black box, to ensure that the photons we are detecting come from the photon sources used, whether they are emitted by LEDs or by scintillators. In addition, because accurate energy calibration cannot be performed when plastic scintillators are used, most of their energy spectrums are shown in units of ADC\footnote{ADC units are the internal units, called channels, in which an analog signal is digitized after an Analog-to-Digital Converter. The number of available channels depends on the bits used in its digitization.} (channels), which are linearly proportional to the units of energy.