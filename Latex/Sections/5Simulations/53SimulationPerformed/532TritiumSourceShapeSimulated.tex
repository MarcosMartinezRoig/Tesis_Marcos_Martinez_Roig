First of all the shape of the tritium source simulated was tested. The objective of this study was to find the tritium source shape that optimaze the simulation. 

As we have said, all tritium events will be simulated one by one by Geant4. Due to the reason that the mean free path of tritium decay in water is only around $5~\mu\meter$, there will be many tritium events from the tritium solution that won't reach the scintillating fibers. These will be tritium events that don't provide us with useful information and only contribute to being time consuming and reducing available computing resources.

To optimize the simulation, we found the shape of the tritium source that minimizes the tritium events that do not reach the scintillating fibers without losing the tritium events that reach them.

The simulation designed for this test consists of a scintillating fiber with a length of $20~\cm$ and a diameter of $2~\mm$ and a surrounding tritium water source with the same length and a thickness $100$ times greater that the mean free path of tritium electrons, $0.5~\mm$, to ensure that this study take into account all possible tritium electrons that can reach to the scintillating fiber. 

We have to keep in mind that the dimensions of the fiber are not important in this study since we have only simulated energy deposition of tritium events in the fiber. That is, we have not simulated the following steps such as photon generation, propagation of these photons, etc. where the shape of the scintillating fiber becomes important.

The simulated scintillating fiber consists of a polystyrene core to which sensitive attributes has been given and the simulated tritium water source consists of water to which electron emissions with the same energy spectrum of tritium electron decay has been added.

The results of this simulation are shown in the section \ref{sec:ResultsSimulations}, where it will be discussed.
