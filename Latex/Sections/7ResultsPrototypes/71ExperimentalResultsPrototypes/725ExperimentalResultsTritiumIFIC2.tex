Comparar las cuentas que espero teóricamente con las obtenidas experimentalmente para tritium IFIC 2. Sale bastante bien.

Ajustar una curva landau a los resultados de cada espectro energético.


Tritium 2 -> Medida y comparación con 1 y 0 (y si se puede con Aveiro). Resultados de la tabla del punto 1.3 de la tesis donde comparo con otros experimentos similares, estabilización, extrapolación teórica.

Mostrar los resultados por unidad de centelleador (unidad de fibra o unidad de superficie) para poder comparar bien los prototipos.

Demostraciones teóricas de las cuentas esperadas en los detectores (en cada apartado)

Una sección con la monitorización de la señal y el fondo.

Rellenar con uan actividad grande y volver a medir. 

Vaciar el prototipo y rellenar con actividades más pequeñas. 1000 y 100 Bq/L.

Espectro de tritio + fondo medido y sacar la significancia y la relación señal ruido y todo lo que se pueda.

Comparar fondos en el laboratorio, caceres, almaraz, etc.

COmparar cada propuesta (Aveiro valencia) -> Publicaciones de cada uno y una de las sintesis.

Medir en el prototipo con SIPM.