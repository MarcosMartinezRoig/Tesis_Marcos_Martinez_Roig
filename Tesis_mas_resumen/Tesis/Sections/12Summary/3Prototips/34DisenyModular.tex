El monitor TRITIUM presenta un disseny modular, és a dir, consta de diversos prototips, anomenats mòduls quan formen part del monitor, que són llegits en paral·lel. Un disseny esquemàtic del monitor format per $10$ mòduls TRITIUM-IFIC-2 es mostra a la Figura \ref{fig:10TritiumMonitorIFIC2}.
\begin{figure}[h]
\centering
\includegraphics[scale=0.6]{12Summary/5Prototypes/55ModularTritiumDetector/Tritium_Detector_Based_On_Tritium_IFIC_2.PNG}
\caption{Disseny esquemàtic del monitor TRITIUM basat en el prototip TRITIUM-IFIC-2.\label{fig:10TritiumMonitorIFIC2}}
\end{figure}
Aquesta modularitat és una de les propietats més rellevants del monitor, ja que li confereix la possibilitat de millorar algunes de les seues característiques de forma proporcional al nombre de mòduls. Una de les propietats més importants per al projecte TRITIUM és la MDA, la millora del qual depén proporcionalment a l'arrel quadrada del nombre de mòduls. La Figura \ref{fig:MDATritiumMonitorIFIC2} mostra el comportament esperat del MDA per al monitor TRITIUM basat en els resultats experimentals obtinguts amb TRITIUM-IFIC-2 en funció del nombre de mòduls. La línia de color roig mostra l'objectiu del projecte TRITIUM de $100~\becquerel/\liter$. Com es pot veure, s'espera aconseguir aquest objectiu amb $5$ mòduls TRITIUM-IFIC-2 llegits en paral·lel.

\begin{figure}[h]
\centering
\includegraphics[scale=0.6]{12Summary/5Prototypes/55ModularTritiumDetector/MDA_vs_NP.pdf}
\caption{Mínima activitat de triti detectable pel monitor TRITIUM en funció del nombre de mòduls TRITIUM-IFIC-2 emprats.\label{fig:MDATritiumMonitorIFIC2}}
\end{figure}

Amb l'ajuda de simulacions es va estudiar com varia la resolució del monitor en funció del nombre de mòduls, paràmetre que indica la mínima variació de l'activitat de triti que és possible identificar. Com es pot observar a la Figura \ref{fig:ResolucioTritiumMonitorIFIC2}, resolucions més baixes són obtingudes amb més mòduls, permetent diferenciar variacions més xicotetes de l'activitat de la mostra. Amb $5$ mòduls TRITIUM-IFIC-2 llegits en paral·lel s'aconsegueix identificar correctament variacions de l'activitat del triti de $100~\becquerel/\liter$ en mesures d'$1~\hour$.

\begin{figure}[h]
\centering
\includegraphics[scale=0.6]{12Summary/6Simulations/62TRITIUMMonitor/621TRITIUMIFIC2/Results_Several_Detectors.pdf}
\caption{Resolució del monitor TRITIUM basat en mòduls TRITIUM-IFIC-2 en funció del nombre de mòduls.\label{fig:ResolucioTritiumMonitorIFIC2}}
\end{figure}