So far we have created the scintillating photons in the core of the fiber, which have been guided to its ends. Now, what we need is the so-called photosensor, which is an element that is able to detect these scintillating photons. Photosensors have a sensitive part that is optimized to detect photons in a range of energy (normally inside of a visible range\footnote{Photons whose wavelength is between $380~\nm$ and $750~\nm$}) with enough efficiency. After that, the photosensors create an electronic signal that carries information about these photons detected, such as their number or their detection time.

One of the most important things in the scintillation detector is that the emission spectrum of the scintillation (figure \ref{fig:EmissionSpectrumFibers} in our case) overlaps as much as possible with the detection efficiency spectrum of the photosensor used, specifically their higher peaks. In this case, the efficiency of this detector,  which is poroporcional to the multiplication of both factors at the same photon energy, will be optimized (the largest).

There are a lot of different photosensors that can be used for this purpose, whose photon detection relies on totally different physical processes, such as photoelectron multiplier tubs (PMTs), silicon photoelectron multiplier (SiPM) or charge-coupled device (CCD).  Each one of these will have different properties and we have to choose the one which fit better for our objective.

Our main proposal for our scintillation detector will be to use SiPM arrays because they are very fast (of the order of $~\nano\meter$) and have high photodetection efficiency (a maximum of around $50\%$) and high gains (multiplication faction of $10^{6}$) with a low voltage supply. On top of that, one of the most important reason of this choice is that SiPM arrays are able to detect a single photon with high efficiency, which is very important since, as we have seen in the section \ref{subsec:PlasticScintillators}, just a few photons will arrive to the sensible part of the photosensor. We will test also the PMTs, which are the conventional choice, because they are still interesting since they have lower dark count rate than an equivalent SiPM and some similar properties like its gain.



%A certain portion (in an optimal case nearly 100\%) of the scintillation photons reach the light detector, which has to be sensitive enough to detect a small number of photons. The detector then produces a signal pulse, which has a height proportional to the number of photons hitting the detector. The signal pulse of the detector is processed by the electronics, and as a result a pulse height spectrum is produced (see Section 3.5).

%This spectrum corresponds to the energy spectrum of the detected particles.
