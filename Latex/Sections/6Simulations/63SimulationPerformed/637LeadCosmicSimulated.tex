Finally the lead shielding and active vetos, detailed in section \ref{subsec:TritiumIFIC2Simulation}, were included in the simulation of the Tritium-IFIC 2 prototype. The objective of these simulations was to demonstrate its need, quantifying its effect in reducing the cosmic events detected by the prototype.

For this task, similar to that done in section \ref{subsec:FiberDiameterSimulation}, the tritium source was replaced by the cosmic events source, which was simulated through the CRY library.

As can be seen in Figures \ref{subfig:RealHardCosmicEvent} and \ref{subfig:FakeHardCosmicEvent}, two plastic scintillators were simulated with the dimensions mentioned in section \ref{subsec:SetUpActiveShield} and located above and below of the Tritium-IFIC 2 prototype simulated. 

The optical properties included to this plastic scintillators are the refractive index, the light attenuation spectrum and energy emission spectrum, the values of which were obtained from their data sheet provided by the manufacturer \cite{ScintillatorVeto}.

As shown in this figure, two PMTs, model R8520-460 from Hamamatsu company, were simulated to read each plastic scintillator, similar to that presented in section \ref{subsec:SetUpActiveShield}.

Finally, a lead shielding was simulated, whose properties were taken from the Geant4 NIST database. The dimensions of the simulated lead shielding were $60 \cdot{} 60 \cdot{} 70~\cm^3$, which is the minimum needed to accomodate the active vetos and Tritium detector module inside.  The length of the simulated lead castle, $60~\cm$, is smaller than real dimension, $148~\cm$. The reason for this is that only one tritium detector module was simulated, so the dimension of the lead shielding can be reduced to optimize simulation time and computing resources.

The results of these simulation are shown in section \ref{subsec:ResultsSimulatedBackgroundRejectionSystem}, where they are discussed.