Comparar las cuentas que espero teóricamente con las obtenidas experimentalmente para tritium IFIC 2. Sale bastante bien.

Ajustar una curva landau a los resultados de cada espectro energético.

Tritium 0 -> Tritium yfuentes radiactivas.

Tritium 1-> Tritium y comparación con Tritium 0

Aveiro -> Calibración inicial, Medida y comparación con la simulación, Reducción de cuentas debida al plomo, figuras 8a y 8b del paper experimental de Carlos, estudio teórico del límite de curíe, medida en aire y agua, medida con 10 kBq/L, extensión a mayores tiempos,

Tritium 2 -> Medida y comparación con 1 y 0 (y si se puede con Aveiro). Resultados de la tabla del punto 1.3 de la tesis donde comparo con otros experimentos similares, estabilización, extrapolación teórica.




Finally,
the simulated and the experimental spectra are compared, yielding some conclusions
regarding the PMT signal and the behavior of the scintillator in tritiated water.