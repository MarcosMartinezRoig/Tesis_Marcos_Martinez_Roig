In this section, the characterization of uncladded BCF-12 fibres from Saint-Gobain, selected for the TRITIUM monitor, is described. These fibres are compared to single clad and multiclad BCF-12 fibres with the same external diameter to quantify the influence of the clad on their photon collection efficiency. The difference between these three types of fibres is that uncladded fibres consist of a polystyrene core with a refractive index of $1.60$, whereas single clad and multiclad fibres have a PMMA clad of $30~\mu\meter$ thickness and a refractive index of $1.49$. Multiclad fibres have additionally a second fluor-acrylic clad of $10~\mu\meter$ thickness and a refractive index of 1.42. Commercial clads are too thick for tritium measurements but a sufficiently thin clad could be obtained by vapor deposition if needed. The characterization was carried out for individual scintillating fibres and consisted in a comparative study of the fibre photon collection. The setup employed, shown in Figure \ref{fig:SetUpFiberCharacterization}, consists of an optical board on which a LED and a PMT were placed in front of each other. A LED (LED435-03 from Roithner LaserTechnik Gmbh \cite{LEDRLT}), with an emission spectrum similar to that of the scintillating fibres, was used. The emission spectrum of the LED, given in Figure \ref{fig:LEDSpectrumTritium}, was measured using a spectrometer and fitted to a Gaussian function. The LED emission peak is at $434~\nano\meter$ with a $\sigma$ of $8~\nm$. The LED was fed in current mode with a sourcemeter. A calibrated Hamamatsu R8520-06SEL PMT with quantum efficiency $QE=29.76\%$ at $\lambda=430~\nano\meter$ was employed. 
\begin{figure}[h]
\centering
\includegraphics[scale=0.6]{4ResearchAndDevelopments/41Fibers/SetUp_Fiber_Characterization.png}
\caption{Setup used for fibre characterization.\label{fig:SetUpFiberCharacterization}}
\end{figure}
A $20~\cm$ long fibre was placed between the LED and the PMT, optically coupled to their end-surfaces by optical grease \cite{OpticalGrease}. Two collimators were used to ensure that only photons emitted from the LED were detected by the PMT. Two FH-ST connectors from RoHS \cite{} were used to fasten the fibre to the system. 
\begin{figure}[h]
\centering
\includegraphics[scale=0.7]{4ResearchAndDevelopments/41Fibers/LED_TRITIUM_1_std.pdf}
\caption{Emission spectrum measured for the 435-03 LED from Roithner LaserTechnik Gmbh.\label{fig:LEDSpectrumTritium}}
\end{figure}
To determine the photon collection efficiency, the rate of photons reaching the active area of the PMT was measured for the different type of fibres. The photon rate $R_{\gamma}$ reaching the photocathode was calculated from,
\begin{equation}
R_{\gamma} = \frac{\left( I_{PMT} - I_{DC} \right)}{q_e \cdot{} QE \cdot{} CE}
\label{eq:NumPhotonsFromIntensityPMT}
\end{equation}
where $I_{PMT}$ is the output current of the PMT, $I_{DC}$ is the dark current, $CE$ is the photoelectron collection efficiency and $q_e$ is the electron charge.