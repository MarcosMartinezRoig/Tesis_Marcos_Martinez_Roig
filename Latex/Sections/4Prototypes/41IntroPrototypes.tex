This chapter describes the different prototypes that have been developed in the framework of the TRITIUM experiment, which are Tritium-IFIC 0, Tritium-IFIC 1, Tritium Aveiro 0 and Tritium-IFIC 2, listed in chronological order of their construction.

On the one hand, the first two prototypes built were preliminary prototypes used to learn about tritium detection and to detect and solve problems in their designs.

On the other hand, the other two prototypes built were prototypes with a well-defined design in which no problems were found. They were built to check more subtle effects.

Each prototype was designed and built in our own workshops (at IFIC, Valencia or Aveiro, Portugal) and it was filled with tritium water following a protocol specially developed for this task.

In each prototype used, several tightness and filling tests were carried out to guarantee their radiosecurity.

Finally, in the last section, the final monitor of TRITIUM detector will be explained. It is based on modular detection units for easy scalability, where each module will be the chosen prototype (the one with the best results) previously shown throughout this chapter.