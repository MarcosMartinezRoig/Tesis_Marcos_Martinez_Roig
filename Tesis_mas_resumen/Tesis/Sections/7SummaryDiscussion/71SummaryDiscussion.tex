In this chapter, the most important results and prospects of the TRITIUM project are summarized and discussed. 

Tritium, a radioactive isotope of hydrogen, is one of the most abundantly produced radioisotopes in nuclear facilities such as nuclear power plants and research facilities. Due to its radiotoxicity, an excessive amount of tritium released into the environment could directly (drinking tritiated water) or indirectly (irrigation with tritiated water) affect human health and the environment. The legal limit of tritium activity in drinking water in Europe is $100~\becquerel/\liter$, established by the EURATOM Council Directive \cite{EURATOM_GL}. This is the most restrictive limit in the world. %Nowadays, tritium in water is mainly measured using the liquid scintillation counting technique. This technique has a very good detection sensitivity, being able to measure tritium activities as low as $1~\becquerel/\liter$. The drawback of this technique is that it takes more than 2 days to perform the measurement. In addition, the liquid scintillator is not reusable and contains toluene, a toxic chemical element. Significant progress has been done in the last decades in measurement in quasi-real-time (times less than $1~\hour$) with plastic scintillators but without achieving the low sensitivity required to measure the low-levels of tritium in water of the order of $100~\becquerel/\liter$.

The TRITIUM project was funded to investigate the feasibility of a monitor 
for measuring low tritium activities in water in quasi-real-time. The goal of this project is to design, build, install and commission a tritium in water monitor that measures tritium activities as low as $100~\becquerel/\liter$ in 1 hour or less. The TRITIUM monitor consists of three different parts:

\begin{enumerate}

\item{} The TRITIUM detector, which is composed of modules of hundreds of uncladded scintillating fibres read out in parallel. Various configurations of the TRITIUM module were tested such as different diameters of the scintillating fibres ($1~\mm$ or $2~\mm$) and types of photosensors (PMTs or SiPM arrays).

\item{} The background rejection system, which suppresses the radioactive background that affects the tritium activity measurements. This is based on a lead passive shield that reduces the soft component of cosmic rays (energies below $200~\MeV$) and the environmental radioactive background, and an active veto that reduces the hard component of cosmic rays (energies above $200~\MeV$).

\item{} The water purification system, which removes particles and minerals present in the water samples measured by the TRITIUM detector.

\end{enumerate} 

The results obtained by the TRITIUM collaboration are the following:

\begin{enumerate}
\item{} For the detector development, the following R$\&$D tasks were carried out:

\begin{enumerate}
\item{} Simulations were performed by the TRITIUM Aveiro team using the Geant4 package to optimize tritium detection efficiency. It was found that in $25~\cm$ long fibres the light signal is a factor $23\%$ larger than in $1~\meter$ long fibres. %It was also found that the TRITIUM detector background caused by cosmic ray events is smaller when $1~\mm$ diameter fibres are used instead of $2~\mm$.

\item{} A surface-conditioning method for scintillating fibres, which consists of cleaving, polishing and cleaning the fibres, was developed in the frame of this PhD thesis. A polishing machine based on Arduino technology was developed. The objective of this machine was to automate the polishing of hundreds of fibres simultaneously, a task that if done manually requires an unaffordable time for the number of fibres needed for the TRITIUM monitor. The surface-conditioning method increases by a factor of $2$ the light collected by the fibres due to polishing and an additional $30\%$ due to cleaning. 

\item{} Characterization of the scintillating fibres was performed in this PhD thesis. A photon collection efficiency of $76\pm 8\%$ was obtained, which is smaller than stated by the manufacturer ($96\%$). The typical deviation of the photon collection efficiency after applying the surface-conditioning method was also measured, obtaining about $3\%$ which is acceptable for tritium measurements.

\item{} Characterization of the S13360-1375 Hamamatsu SiPM was carried out in this PhD thesis. The most relevant parameters of the SiPM such as gain, breakdown voltage, temperature coefficient and others such as quenching resistance and terminal capacitance, were measured and compared to the values provided by the manufacturer. The charge of the single photoelectron spectrum was measured with a resolution of $1\%$ for each photoelectron peak and up to $10$ photoelectron peaks were resolved. %small uncertainties in the single photon spectrum (about $1\%$ for each photopeak) and a very good agreement with the values expected by the manufacturer.

%An additional calibration was carried out at the level of a SiPM matrix in which the probability of crosstalk between different SiPMs was measured, obtaining an insignificant probability of happening. The linearity of the SiPM signal as a function of the the number of scintillating fibres was also verified.

\item{} Due to the strong dependence of the SiPM gain on temperature, a gain stabilization method was implemented in the temperature range of interest. This method consists in compensating for the variation of the gain due to changes in the temperature by a variation of the bias voltage. Indeed, the gain depends linearly on both temperature and voltage, increasing with voltage and decreasing with temperature. This enables to stabilize the gain to its nominal value at $25\celsius$ by a variation of $59.9 \pm 1.3 ~\milli\volt/\celsius$ of the bias voltage. This stabilization method was tested, obtaining variations in the SiPM gain smaller than $0.1\%$ in the $[20-30]\celsius$ temperature range, which is the expected temperature range of operation. These results indicate that a stable operation of the SiPM readout can be obtained by an automatic implementation of a temperature-dependent SiPM bias.

\end{enumerate}

\item{} The background rejection system consists of an active veto and a passive shield. The latter consists of a lead castle of $5~\cm$ thick walls, designed by the TRITIUM CENBG team and presently installed in the Arrocampo site. A reduction factor of $5.5$ for cosmic ray events due to the lead shield was obtained by simulations performed in this PhD thesis. Most of the background of the environmental radioactivity in the Arrocampo site (which was not included in these simulations) would also be suppressed by this shield. 

The active veto, built and characterized in this PhD thesis, consists of two parallel plastic scintillator plates of $1~\cm$ thickness, separated $34.2~\cm$ distance and enclosing the TRITIUM modules. The plastic scintillator plates were threefold wrapped in PTFE, aluminium and black tape layers and each plate was read out by two photosensors. This wrapping improved the light collection by a factor of $2$ and produced a better response uniformity on the plate surface. In addition, the electronics settings that optimize the detection of hard cosmic events, such as discrimination thresholds and photosensor high voltage bias, were determined. A hard cosmic rate of $2.5~\text{events}/\text{s}$ was measured, from which an efficiency of the cosmic veto of $85\%$ is drawn by comparison to the cosmic ray rate at sea level. Finally, the dependence of the hard cosmic rate on the distance between the plastic scintillator plates was fitted to a second order polynomial, which enables to change this distance without needing to perform a new calibration of the veto. A $60\%$ reduction of the hard cosmic ray events due to the active veto was obtained through simulations. Therefore, a $92.7\%$ suppression of the cosmic ray rate by the whole background rejection system was obtained from simulations. The simulated background rejection of hard cosmic events is smaller than that found experimentally so the total cosmic ray suppression is expected to be larger than the simulated one. The actual cosmic ray suppression will be measured at the Arrocampo site.

\item{} A detailed analysis of the Arrocampo water revealed the presence of high concentrations of organic components at the site. The TRITIUM LARUEX team designed and installed a water purification system, consisting of several filtering stages that eliminate all organic matter and mineral particles of more than $1~\mu\meter$ in size. A conductivity close to $10~\mu\text{S}/\cm$ (two orders of magnitude less than the raw water) was achieved. Furthermore, the water tritium activity did not change after the purification process.

%Important results were obtained through simulations, which were implemented in the built prototypes. It was seen that the efficiency of tritium detection in the prototypes is larger when short fibres (about $20~\cm$) are used insted of long fibres ($1~\meter$). An improvement of a factor $5$ is acheived in the tritium count rate measured. Also, it was found that the tritium measurement is optimized when $1~\mm$ fibres are used, compared with  $2~\mm$, since a smaller background in the energy region of interest for the tritium measurement due to cosmic ray events is obtained.

%Three different detector prototypes, called TRITIUM-IFIC-0, TRITIUM-IFIC-1 and TRITIUM-IFIC-2, were developed in this PhD theses and this results were compared to other prototype, TRITIUM-Aveiro, developed by the TRITIUM Aveiro team 

\item{} Four different detector prototypes, called TRITIUM-IFIC-0, TRITIUM-IFIC-1, TRITIUM-Aveiro and TRITIUM-IFIC-2, listed in chronological order, were developed by the TRITIUM collaboration. The first two prototypes, TRITIUM-IFIC-0 and TRITIUM-IFIC-1 (developed in this PhD thesis), were used as a proof of concept for the detection of tritium in water with scintillating fibres and to identify the different issues that affect the detection efficiency. %The latest prototypes, TRITIUM-Aveiro (developed by the TRITIUM Aveiro team) and TRITIUM-IFIC-2 (developed in this PhD thesis), have slightly different designs. Small tritium activities were used to measure their tritium detection efficiency and MDA. %Each design has its own advantages and disadvantages and the characteristics of each one with the best results will be implemented in the final design of the TRITIUM module.

The results obtained from the first prototype demonstrated that a straight arrangement of the scintillating fibres was crucial for tritium detection. A surface-conditioning method of scintillating fibres was implemented in the TRITIUM-IFIC-1 prototype which improved the tritium detection efficiency. The use of a PTFE vessel was also found to improve the light collection due to its optical properties (reflectivity close to $93\%$ for visible light). These modifications resulted in a factor of $10$ increase in the measured count rate of tritium over the first prototype. %Finally, the use of two photosensors in time coincidence  improved the prototype MDA, since this reduces the photosensor noise. %with almost no affecting to the tritium signal.

In the latest prototypes, two photosensors in coincidence were employed to reduce the photosensor noise. These two prototypes have a similar design but with subtle differences. One of the most important differences is the scintillating fibre diameter ($2~\mm$ for the TRITIUM-Aveiro prototype and $1~\mm$ for the TRITIUM-IFIC-2 prototype). Fibres of $1~\mm$ enable to fit more of them in the same volume. This increases the total active area of the prototype (and therefore, its tritium detection efficiency) and the signal-to-background ratio (improving the MDA). Fibres of $2~\mm$ are stiffer which may facilitate the water flow through the fibre bundle and increase the effective detection area. It was obtained from simulations that the cosmic ray rate in the energy range of interest is a factor $2$ higher for scintillating fibres of $2~\mm$. Additional measurements need to be done to decide the fibre diameter of the final detector. The second important difference between the TRITIUM-Aveiro and TRITIUM-IFIC-2 prototypes is the type of photosensor proposed. TRITIUM-Aveiro uses PMTs and for TRITIUM-IFIC-2 SiPM arrays are proposed. SiPM arrays have some advantages over PMTs such as a higher photodetection efficiency which would increase the detection efficiency of the TRITIUM detector. Furthermore, the SiPMs do not need high voltage which implies a reduction of the TRITIUM monitor cost. However, SiPM arrays have some disadvantages as the need to read more channels out and to implement a gain stabilization method due to the strong dependence of SiPM gain on temperature.

The specific efficiency obtained with the TRITIUM-IFIC-2 prototype, $(141 \pm 6) \times 10^{-6}~ \second^{-1}  \liter ~ \kilo\becquerel^{-1} \cm^{-2}$, is an order of magnitude better than that obtained with the TRITIUM-Aveiro prototype, $(16 \pm 5)\times 10^{-6}~ \second^{-1}  \liter ~ \kilo\becquerel^{-1} \cm^{-2}$, most probably due to the surface-conditioning of the fibres of the IFIC prototype. In addition, $677~\becquerel/\liter$ and $218~\becquerel/\liter$ MDAs were obtained for the TRITIUM-IFIC-2 prototype for integration times of $10~\min$ and $1~\hour$, respectively. This is to be compared to $29.8~\kilo\becquerel/\liter$ and $3.6~\kilo\becquerel/\liter$ MDAs for the TRITIUM-Aveiro prototype for integration times of $1~\min$ and $1~\hour$, respectively. An integration time of $1~\hour$ can still be considered quasi-real-time. The lower the MDA the smaller the tritium activity that can be discriminated from the background.

A summary of the state-of-the-art of tritium detection in water is shown in Table \ref{tab:ComparisonResultsTritium} which includes the results obtained with the four different prototypes developed by the TRITIUM collaboration. As it can be seen in this table, TRITIUM-IFIC-2 ameliorates significantly the current state-of-the-art. The specific efficiency and the MDA are almost an order of magnitude better than the results reported in the literature.

\begin{table}[htbp]
\centering{}%
\begin{tabular}{lcrcc}
\toprule 
Reference & \parbox{5em}{$\varepsilon_{det}\times10^{-3}\\\liter~\kilo\becquerel^{-1}\second^{-1}$}  & \parbox{3.5em}{\raggedleft $F_{sci}$\\ $\cm^2$}  & \parbox{6.5em}{$~\eta_{det}\times 10^{-6}\\\liter~\kilo\becquerel^{-1}\second^{-1}~\cm^{-2}$} &  \parbox{3.5em}{MDA\\$\kilo\becquerel~\liter^{-1}$} \tabularnewline
\midrule
\midrule 
\cite{Muramatsu} & $0.39$ & $123$ & $3.13$ & $370$ \tabularnewline
\cite{Moghissi} & $4.50$ & $>424$ & $<10.6$ & $37$ \tabularnewline
\cite{Osborne} & $12$ & $3000$ & $4$ & $37$ \tabularnewline
\cite{Ratnakaran} & $41$ & $3000$ & $13.7$ & $<37$ \tabularnewline
\cite{Hofstetter1} & $2.22$ & $\sim~100$ & $<22.2$ & $25$ \tabularnewline
T-IFIC-0$\dagger$ & $2.1 \pm 0.8$ & $219$ & $10 \pm 4$ & $100$* \tabularnewline
T-IFIC-1$\dagger$ & $38.4 \pm 1.6$ & $402$ & $96 \pm 4$ & $100$* \tabularnewline
T-Aveiro$\dagger$ & $64 \pm 19$ & $4072$ & $16 \pm 5$ & $3.6$** \tabularnewline
T-IFIC-2$\dagger$ & $711 \pm 27$ & $5027$ & $141 \pm 6$ & $0.22$** \tabularnewline
\bottomrule
\end{tabular}
\caption{Results of scintillator detectors developed by different experiments (including the TRITIUM project) for tritiated water detection. This table shows the detector efficiency ($\varepsilon_{det}$), its active surface ($F_{sci}$), its specific efficiency ($\eta_{det}=\varepsilon_{det}/F_{sci}$, defined as efficiency normalized to the active surface) and its MDA.\\
* Specific activity measured, not MDA.\\ 
** MDA measured for $1~\hour$ integration time.\\
$\dagger$ This Thesis.}
\label{tab:ComparisonResultsTritium}
\end{table}

The modular structure of the TRITIUM monitor allows for scalability, which means that a lower MDA can be achieved by using a larger number of modules. The MDA of the TRITIUM monitor is expected to decrease with the square root of the number of modules. Therefore, as shown in Figure \ref{fig:MDATRITIUMmonitor}, an MDA of  $100~\becquerel/\liter$ (goal of the TRITIUM project) could be achieved by using 5 TRITIUM-IFIC-2 modules and an integration time of $1~\hour$. It has to be taken into account that the MDA reported in this PhD work was measured without the background rejection system. The tritium MDA is expected to improve when this system is included.

\item{} The stability of the tritium detection efficiency of TRITIUM-IFIC-2 was monitored for six months, obtaining a stable detector response during this time with a relative standard deviation of $2.5\%$ for the measured tritium rate. 

\item{} Finally, simulations of a TRITIUM monitor based on the TRITIUM-IFIC-2 design were carried out to determine the dependence of the tritium detection efficiency and activity resolution on integration time and the number of modules. These simulations allow us to determine the number of modules needed in the TRITIUM monitor and the integration time to be used. With $5$ modules and an integration time of $1~\hour$, a tritium activity resolution of $100~\becquerel/\liter$ is expected. This configuration is also one that has an MDA of $100~\becquerel/\liter$ as proven by the measurements (point 4).

\end{enumerate}

At present, the lead shielding, the water purification system and a TRITIUM-Aveiro module are installed at the Arrocampo site. Two additional TRITIUM-Aveiro modules and an active veto are planned to be installed as soon as possible. Moreover, three TRITIUM-IFIC-2 modules and an active veto are ready to be installed too. Their installation was delayed due to the coronavirus pandemic.