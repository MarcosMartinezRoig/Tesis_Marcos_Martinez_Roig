This chapter describes the different prototypes developed in the framework of the TRITIUM experiment, which are TRITIUM-IFIC 0, TRITIUM-IFIC 1, TRITIUM Aveiro and TRITIUM-IFIC 2, listed in chronological order of their construction. The first two prototypes built, TRITIUM-IFIC 0 and TRITIUM-IFIC 1, are preliminary prototypes used to learn about tritium detection and to improve the monitor design. The other two prototypes built, TRITIUM-Aveiro and TRITIUM-IFIC 2, are prototypes with a design in which no significant problems were found. %They were built to check more subtle effects. 

Each prototype was designed and built in the laboratories of IFIC or Aveiro and was filled with tritiated water following a method specially developed for this task. Several water tightness and filling tests were carried out for each prototype to guarantee its radiosecurity. The measurements obtained by the different prototypes during their installaiton in the laboratory are discussed in this chapter. The laboratories involved in the characterization of the prototypes are the IFIC in Valencia, the DRIM \footnote{DRIM, Deteç$\tilde{\text{a}}$o da Radiaç$\tilde{\text{a}}$o e Laboratorio Imagem Médica laboratoire (Laboratory for Radiation Detection and Medical Imaging)}, in the University of Aveiro, and the LARUEX\footnote{LARUEX, Laboratorio de Radiactividad Ambiental de la Universidad de Extremadura (Environmental Radioactivity Laboratory of the University of Extremadura)} laboratory in Extremadura. An additional section shows the measurements obtained at the Arrocampo dam, the TRITIUM monitor installation site, where the control of external atmospheric conditions is less accurate. At the end of the chapter, the final monitor of TRITIUM detector will be described. Its design is a modular structure for easy scalability, composed of the number of the final prototype needed to reach the required sensitivity.