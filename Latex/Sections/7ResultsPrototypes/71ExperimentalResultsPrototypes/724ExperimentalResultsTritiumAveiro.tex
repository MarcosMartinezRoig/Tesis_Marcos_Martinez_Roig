- Espectro energético.

- Monitorización del fondo durante meses. Desde el mes 3 hasta el mes 8

- Monitorización del fondo y de una disolución de Tritio durante unos dias. Estudio de las gaussianas de su distribución 

- Effecto del apantallamiento del plomo.

- Volume used.


Ajustar una curva landau a los resultados de cada espectro energético.

Aveiro -> Calibración inicial, Medida y comparación con la simulación, Reducción de cuentas debida al plomo, figuras 8a y 8b del paper experimental de Carlos, estudio teórico del límite de curíe, medida en aire y agua, medida con 10 kBq/L, extensión a mayores tiempos,

Mostrar los resultados por unidad de centelleador (unidad de fibra o unidad de superficie) para poder comparar bien los prototipos.

Demostraciones teóricas de las cuentas esperadas en los detectores (en cada apartado)





We have obtained a MinimumDetectable Activity lower than 5 kBq/L for a single module being limited by the cosmic background. Far from the primary goal this value is already compatible with a real-time tritium environmental surveillance monitor and a power plant pre-alert system.


Medida con una fuente de 55Fe, ya que tiene gammas muy cercanas a las del tritio, 5.9keV. Esta se situo dentro del tubo de teflón ya que si no se apantallaría y no llegaría  ala fibra. Debido a esto no podíamos usar agua en su interior.


