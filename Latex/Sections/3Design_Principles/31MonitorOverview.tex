The objective of the TRITIUM project is the design, development, construction and commissioning of an automatic station for real-time monitoring of low levels of tritium in water. To achieve this objective, the TRITIUM experimental group has developed a monitor that is based on several parts, listed below: 

\begin{itemize}

\item{} The TRITIUM detector, which will be explained in chapter \ref{chap:Prototypes}, is based on several modules that are read in parallel. Each module consists of hundreds of scintillating fibers, section \ref{subsec:PlasticScintillators}, read by two coincident photosensors, section \ref{subsec:Photosensors}. These scintillation fibers are directly in contact with the water sample whose tritium level will be measured. The photosensors included in this study are photomultiplier tubes (PMT), section \ref{subsubsec:PMTs}, and silicon photomultipliers (SiPM), section \ref{subsubsec:SiPM}.

\item{} The ultrapure water system, section \ref{sec:UltraPureWaterSystem}, is used to condition the water sample before the measurement. This system removes all the organic particles that are dissolved in this water and all the particles whose diameter is greater than $1~\mu\meter$ without affecting the level of tritium in the sample. It is important for two reasons, on the one hand because, as it has been seen in section \ref{sec:TritiumProperties}, the mean free path of tritium in water is very short, $5$ or $6~\mu\meter$,  so it is important to avoid the deposition of this particles in fibers because this would prevent the tritium electrons from reaching the fibers. On the other hand some of this particles disolved in the water sample are natural radioactive particles such as $\ce{^{40}K}$, which increase the background of our detector so, due to the fact that the water sample has few tritium events, it is very important to reduce the background of our detector as much as posible.

\item{} The background rejection system, section \ref{sec:IntroductionBackground}, is based on two different parts. On the one hand, a passive shield, section \ref{subsec:SetUpPassiveShield}, which consists of a lead castle inside of which the TRITIUM detector will be located. It is used to eliminate the natural radioactive background that is found in the place where the TRITIUM detector will be located, generally the events with relatively low energy ($<200~\MeV/$nucleon). On the other hand, an active veto, section \ref{subsec:SetUpActiveShield}, which consists of two plastic scintillation blocks located inside of a passive shielding, above and below of the TRITIUM detector, which are read by several photosensors. The objective of this active veto is to remove the remaining high energy events ($>200~\mega\eV$) from the natural background, such as cosmic events, that can travel through the passive shield and affect to the tritium measurement. Contrary to what happens with low energy events, this events are difficult to stop so, the technique used to eliminate their contribution to the TRITIUM measurement consists of reading the TRITIUM detector in anti-coincidence with the active veto.
%to detect these high energy events and, for each of them, open narrow time windows in which we will not read the Tritium detector to prevent these events from affecting the tritium measurement.

\item{} A general electronic system that will be used to monitor all the different parts of this monitor and send an alarm if the legal limit of the tritium level, $100~\becquerel/\second$, is exceeded.

\end{itemize}

Each part of this monitor were subjected to several tests to ensure its correct operation and, after that, they were installed in the Arrocampo dam. The final objective will be to include this monitor in the network of automatic stations, REA, shown in section \ref{sec:Introduction}.