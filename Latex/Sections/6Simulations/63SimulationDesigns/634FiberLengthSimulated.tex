A study to find the fiber length that optimizes the tritium detection efficiency was carried out. Two different lengths of scintillating fibers were considered in this study, $1~\meter$ and $20~\cm$, and two different tritium source activities were used, $0.5~\kilo\becquerel/\liter$ and $2.5~\kilo\becquerel/\liter$. As detected tritium decays are proportional to the active area, 5 detectors were simulated for the case of a $20~\cm$ fiber length to have the same active area. As the active area of the detector is related to its tritium detection efficiency, the advantage for using long fibers is their large active areas with a small number of cells, reducing the number of photosensors and, consequently, the price of the TRITIUM monitor. However, a smaller length of scintillating fibers reduces the photon absorption produced in the fibers, which increases the tritium detection efficiency per active area.

To find the scintillating fiber length that optimizes the tritium detection efficiency, the Tritium-Aveiro prototype, consisting of $360$ scintillating fibers of $2~\mm$ diameter, was simulated. All optical properties for the photon propagation were included in this study. 

The propagation of photons in scintillating fibers was checked. The number of photons produced in a scintillating fiber per tritium electron was compared for the electrons that reach the scintillating fiber and for only those photons detected in time coincidence by the photosensors, shown in Figure \ref{fig:PhotonsFibersYesNoPhotosensors}. Tritium events that produce a high number of photons are almost always detected but events that produce few photons are seldom detected, resulting in a peak centred at around $25$ photons.  

\begin{figure}[h]
\centering
\includegraphics[scale=0.3]{6Simulations/61TRITIUMDesign/613Length/CollectionPhotonsInFibers.png}
\caption{Number of photons produced in the fiber per tritium event for all tritium events that reach the fiber (blue histogram) and for only tritium events producing photons detected in coincidence by photosensors (red histogram) \cite{SimulationPaperCarlos}.\label{fig:PhotonsFibersYesNoPhotosensors}}
\end{figure}

%Regarding the fiber length study, 

The counts, integrated over $60~\min$ and taken over a week, are shown in Figure \ref{fig:CountsOver60minDifferentLength} as a function of time for both tritium activities and fiber lengths studied. 5 times greater signal is seen for the shorter fiber length in both cases, due mainly to the lower absorption of photons in the shorter scintillating fibers and the leakage of some photons due to partial photon collection in the fiber. In addition, non simulated effects like the dirty or mechanical imperfections of scintillating fibers increase this photon loss effect.

\begin{figure}[h]
\centering
\includegraphics[scale=0.3]{6Simulations/61TRITIUMDesign/613Length/2DifferentLength.png}
\caption{Simulations of counts integrated over $60~\min$, normalized to the same active area and taken over a week for a fiber length of $1~\meter$ (dashed lines) and $20~\cm$ (solid lines) and two different activities, $0.5~\kilo\becquerel/\liter$ (blue lines) and $2.5~\kilo\becquerel/\liter$ (red lines) \cite{SimulationPaperCarlos}. \label{fig:CountsOver60minDifferentLength}}
\end{figure}

