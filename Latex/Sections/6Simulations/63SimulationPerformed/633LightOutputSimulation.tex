The scintillation yield provided by the manufacturer, $8000~\text{phot.}/\MeV$, is only valid for minimum ionizing particles (MIP). As tritium electron energies do not correspond to MIP particles, the output light generated by the scintillating fibers was studied. For this task, the energy deposition of tritium electrons in scintillating fibers and their subsequent emission of scintillation photons was included in to the simulation.

When particles that are not MIP are detected in plastic scintillators, a light quenching effect affects the proportionality between the output light per unit of path length, $\frac{dL}{dx}$, and the energy deposited per unit of path length, $\frac{dE}{dx}$, through the so-called Birks coefficient\cite{BirksPaper}.
\begin{equation}
\frac{dL}{dx}= S\frac{\displaystyle{\frac{dE}{dx}}}{1+k_B\displaystyle{\frac{dE}{dx}}}
\label{eq:birkscoefficient}
\end{equation}
where S is the scintillation yield, provided by the manufacturer. The value of $k_B=0.126~\mm/\MeV$, typically used for scintillators based on polystyrene \cite{BirksCoefficient}, was taken for the Birk's coefficient. The effect of this correction is discussed in section \ref{subsec:ResultsOutputLight}.