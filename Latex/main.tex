\documentclass[12pt,a4paper]{book}

\input{./Secciones/1PaquetesTrabajo}

\begin{document}
\captionsetup[figure]{labelfont={bf},labelformat={default},labelsep= endash,name={Figura}}
\begin{titlepage}

\begin{center}
\vspace*{-1in}
\vspace*{1 cm}
\begin{figure}[htb]
\begin{center}
\includegraphics[scale=0.7]{Logo1.png}
\end{center}
\end{figure}
\vspace*{2 cm}


\vspace*{0.2in}
{\huge Departamento de Fisica Atómica, Molecular y Nuclear}\\
\vspace*{0.2in}
\vspace*{0.6in}
\end{center}
\vspace*{-1in}
\begin{center}
\vspace*{0.25 cm}


\begin{figure}[htb]
\begin{center}
\includegraphics[scale=0.07]{Logo2.jpg} 
\end{center}
\end{figure}
\vspace*{1 cm}

\begin{large}
\textbf{{\large Thesis}}\\
\rule{80mm}{0.1mm}\\
%\vspace*{2 cm}

\end{large}
\vspace*{0.2in}
\begin{Large}
\textbf{\LARGE Design, construction and commissioning of a tritium water detector based on scintillating fibers read out by silicon photomultiplier} \\
\end{Large}
%\vspace*{0.3in}
\vspace*{1 cm}

\begin{large}
Marcos Martínez Roig\\
\today
\end{large}
\end{center}

%\vspace*{0.3in}
%\rule{80mm}{0.1mm}\\
%\vspace*{0.1in}
\begin{large}
\begin{flushright}
\item[\bf Supervisors:\hspace{4cm} ]\quad  \\ José Díaz Medina\\
Nadia Yahlali Haddou\\
\end{flushright}
\end{large}

\end{titlepage}


$\ $
\thispagestyle{empty} % para que no se numere esta pagina
\chapter*{}

\pagenumbering{Roman} %for using romain numbers (page numering)


\begin{flushright}
\textit{Dedicated to \\
my family}
\end{flushright}

\newpage

\chapter*{Acknowledgements} \label{chap:Acknowledgements}  %pongo el asterisco para que no se numere ni aparezca en el índice
\cleardoublepage
\addcontentsline{toc}{chapter}{Acknowledgements} % para que aparezca en el indice de contenidos
%%\input{./Secciones/10Agradecimientos}

\newpage

\chapter*{Abstract} \label{chap:Abstract}
\cleardoublepage
\addcontentsline{toc}{chapter}{Abstract} % para que aparezca en el indice de contenidos



%\begin{abstract}
%Texto           del           abstract
%\end{abstract}



\newpage


\chapter*{Nomenclature and acronyms} \label{chap:NomenclatureAcronyms}  %pongo el asterisco para que no se numere ni aparezca en el índice
\begin{longtable}{p{5mm} c p{120mm} }
\multicolumn{3}{l}{Mayúsculas}\\
\\
$T$ & --- & Temperatura (ºC).\\
$V$ & --- & Volumen (m$^3$).\\
\\
\\
\multicolumn{3}{l}{Minúsculas}\\
\\
$c$ & --- & Velocidad de la luz en el vacío (m/s). La velocidad de la radiación electromagnética es independiente de la velocidad del emisor.\\
$i$ & --- & Raíz de menos uno (-).\\
\\
\\
\multicolumn{3}{l}{Letras griegas}\\
\\
$\alpha$ & --- & El principio de todo (-).\\
$\pi$ & --- & Pastel en inglés (-).\\
\end{longtable}

\let\OLDthebibliography=\thebibliography
\def\thebibliography#1{\OLDthebibliography{#1}%
\addcontentsline{toc}{chapter}{\bibname}}

\tableofcontents

%\newpage
\cleardoublepage
\addcontentsline{toc}{chapter}{Lista de figuras} % para que aparezca en el indice de contenidos
\listoffigures

%\newpage
\cleardoublepage
\addcontentsline{toc}{chapter}{Lista de tablas} % para que aparezca en el indice de contenidos
\listoftables

%%%%%%%%%%%%%%%%%%%%%%%%%%%%%%% MAIN BODY %%%%%%%%%%%%%%%



\chapter{Introduction}  \label{chap:GeneralIntroduction} %(%(I have to use latin numbers inside of this chapter))
\pagenumbering{arabic} %for using romain numbers (page numering)
%%\setcounter{page}{-1} %%first number of the counter
%%\input{./Secciones/3Introduccion4}

%\newpage
\chapter{Scintillator fibers} \label{chap:GeneralFibers}
\input{./Secciones/4Fibras}
	\section{Introduction}\label{sec:IntroductionFibers}
	%%\input{./Secciones/5SiPM/52Equipo}
	
	\section{Organic and inorganic scintillators}\label{sec:OrganicInorganicFibers}
	%\input{./Secciones/5SiPM/53CalibracionTarjeta}	

	\section{Scintillator fibers}\label{sec:ScintillatorFibers}
	%\input{./Secciones/5SiPM/53CalibracionTarjeta}	
	
	\section{Choice of the comercial scintillator fibers}\label{sec:ChoiceFiber}
	%\input{./Secciones/5SiPM/53CalibracionTarjeta}
	
	\section{Cutting device for scintillator fiber}\label{sec:CuttingFibers}
	%%\input{./Secciones/5SiPM/54Analisis}
	
	\section{Polishing task for scintillator fiber}\label{sec:PolishingTask}
	%%\input{./Secciones/5SiPM/55Temperatura}

	\section{Automatic polishing machine for scintillator fiber}\label{sec:PolishingMachine}
	%%\input{./Secciones/5SiPM/56Voltaje}
	
	\section{Splicing machine for scintillator fiber}\label{sec:SplicingMachine}
	%%\input{./Secciones/5SiPM/56Voltaje}
	
	%DE aquí para abajo todos los estudios realizados con fibras...
	%\section{Estabilización de la ganancia}\label{sec:Compensacion}
	%%\input{./Secciones/5SiPM/57Compensacion}

\chapter{Photomultiplier tubes (PMTs)} \label{sec:PMT}
%\input{./Secciones/4Fibras}
	\section{Introduction}\label{sec:IntroductionPMTs}
	%%\input{./Secciones/5SiPM/56Voltaje}
	
	\section{Calibration of the PMTs}\label{sec:CalibrationPMTs}
	%%\input{./Secciones/5SiPM/56Voltaje}
	
	\subsection{Gain calibration of the PMTs}\label{sec:GainCalibrationPMTs}
	%%\input{./Secciones/5SiPM/56Voltaje}
	
	\subsection{Aquí para abajo las demas calibraciones que haré con los PMTs}\label{sec:demáscalibraciones}
	%%\input{./Secciones/5SiPM/56Voltaje}

%\newpage
\chapter{Calibracion de los fotomultiplicadores de silicio (SiPM)} \label{chap:SiPM}
%%\input{./Secciones/5SiPM/51IntroSiPM}
	\section{Equipo y montaje experimental}\label{sec:Equipo}
	%%\input{./Secciones/5SiPM/52Equipo}
	
	%\section{Características de la tarjeta}\label{sec:Tarjeta}
	%\input{./Secciones/5SiPM/53CalibracionTarjeta}
	
	\section{Análisis de datos}\label{sec:Analisis}
	%%\input{./Secciones/5SiPM/54Analisis}
	
	\section{Calibración en temperatura}\label{sec:Temperatura}
	%%\input{./Secciones/5SiPM/55Temperatura}

	\section{Calibración en voltaje de operación}\label{sec:Voltaje}
	%%\input{./Secciones/5SiPM/56Voltaje}
	
	\section{Estabilización de la ganancia}\label{sec:Compensacion}
	%%\input{./Secciones/5SiPM/57Compensacion}

%\newpage
\chapter{Prototipo} \label{chap:Prototipo}  
%%\input{./Secciones/6Prototipos/61IntroPrototipos}
	\section{Configuración del prototipo}\label{sec:Configuracion}
	%%\input{./Secciones/6Prototipos/62Configuracionprototipo}
	
	\section{Procedimiento de llenado}\label{sec:Llenado}
	%%\input{./Secciones/6Prototipos/63Llenado}
	
	\section{Configuración de la electrónica}\label{sec:Electronica}
	%%\input{./Secciones/6Prototipos/64Configuracionelectronica}
	
	\section{Resultados}\label{sec:Resultados}
	%%\input{./Secciones/6Prototipos/65Resultados}

%\newpage
\chapter{Simulaciones} \label{chap:Simulaciones}
%%\input{./Secciones/7Simulaciones}

%\newpage
\chapter{Previsiones de futuro} \label{chap:Futuro}
%%\input{./Secciones/8Previsiones}

%\newpage
\chapter{Resultados y conclusiones} \label{chap:Conclusiones}
%%\input{./Secciones/9Conclusiones2}

\appendix
\appendixpage
\noappendicestocpagenum
\addappheadtotoc

\chapter{Más cosas}\label{App:A}
Aún faltan cosas por decir.

\chapter{Y más cosas aún}\label{App:B}
Y más cosas aún.

%\chapter{Bibliografía} \label{chap:bibliographia}
%\section {Bibliografía}
\begin{thebibliography}{100}
%Reference 1
\bibitem{IAEA} \textsc{IAEA}, 
\textit{The International Atomic Energy Agency} \href{https://www.iaea.org/}{\textbf{Webpage}}. 

%Reference 2
\bibitem{UNSCEAR} \textsc{UNSCEAR}, 
\textit{The United Nations Scientific Committee on the Effects of Atomic Radiation} \href{https://www.unscear.org/}{\textbf{Webpage}}. 

%Reference 3
\bibitem{CSN} \textsc{CSN}, 
\textit{Consejo de Seguridad Nuclear, Spain} \href{https://www.csn.es/home}{\textbf{Webpage}}.

%Reference 4
\bibitem{ICRU} \textsc{ICRU}, 
\textit{Internation Commission of Radiological Units and Measurements} \href{https://www.icru.org/}{\textbf{Webpage}}.

%Reference 5
\bibitem{ICRP} \textsc{ICRP}, 
\textit{International Commission on Radiololgical Proteccion} \href{https://www.icrp.org/}{\textbf{Webpage}}.

%Reference 6
\bibitem{ISR} \textsc{ISR}, 
\textit{International Society of Radiology} \href{https://www.isradiology.org/}{\textbf{Webpage}}.

%Reference 7
\bibitem{UN} \textsc{UN}, 
\textit{United Nations} \href{https://www.un.org/en/}{\textbf{Webpage}}. 

%Reference 8
\bibitem{REA} \textsc{CSN}, 
\textit{Red de Estaciones Automáticas, REA} \href{https://www.csn.es/mapa-de-valores-ambientales}{\textbf{Webpage}}. 

%Reference 9
\bibitem{REM} \textsc{CSN}, 
\textit{Red de Estaciones de Muestreo, REM} \href{https://www.csn.es/kprgisweb2/index.html?lang=es}{\textbf{Webpage}}. 

%Reference 10
\bibitem{100BqL}  
\href{https://eur-lex.europa.eu/eli/dir/2013/59/oj}{\textit{Council directive 2013/15/euratom}}.

%Reference 11
\bibitem{FiberDetector1a} \textsc{J. W. Berthold}, \textsc{L. A. Jeffers},
\href{https://www.osti.gov/biblio/2225-phase-final-report-situ-tritium-beta-detector}{\textit{Phase 1 Final Report for In-Situ Tritium Beta Detector}}, 
U. S. Department of Energy, McDermott Technology, Inc.,Research and Development Division, 	\textbf{DE-AC21-96MC33128}, April, 1998.

%Reference 12
\bibitem{FiberDetector1b} \textsc{J. W. Berthold}, \textsc{L. A. Jeffers}, 
\href{https://www.osti.gov/biblio/836625-MxOOUa/native/}{\textit{In Situ Tritium Beta Detector}}, U. S. Department of Energy, McDermott Technology, Inc. (MTI), Technology development data sheet, \textbf{DE-AC21-96MC33128}, May, 1999.

%Reference 13
\bibitem{CommonEmissionTritium} \textsc{X- Hou},  
\textit{Tritium and \ce{^{14}C} in the environmental and nuclear facilities: Sources and analytical methods}, Journal of the Nuclear Fuel Cycle and Waste Technology (JNFCWT), 16 (2018), 11-39 \href{https://doi.org/10.7733/jnfcwt.2018.16.1.11}{\textbf{DOI: 10.7733/jnfcwt.2018.16.1.11}}.

%Reference 14
\bibitem{CrossSeccionNeutrons}  
\textit{REFERENCIAAAAAAAA}.

%Reference 15
\bibitem{PercentageEnergySpain}
\href{https://www.ree.es/es/datos/publicaciones/informe-anual-sistema/informe-del-sistema-electrico-espanol-2019}{\textit{Avance del informe del sistema eléctrico español, 2019}}, 
\textbf{Red eléctrica española}.

%Reference 16
\bibitem{60ReactorsChina}
\href{https://www.europapress.es/internacional/noticia-china-construira-menos-60-centrales-nucleares-proxima-decada-20160916210159.html}{\textit{China construirá 60 centrales nucleares en la próxima década}}, 
\textbf{Europa press}.

%Reference 17
\bibitem{35MillionsUSA}
\href{https://www.energynews.es/estados-unidos-centrales-nucleares/}{\textit{Inversión de EE. UU. de 35 millones para centrales nucelares}}, \textbf{Energy News}

%Reference 18
\bibitem{ThreeMileIsland}
\href{www.world-nuclear.org/information-library/safety-and-security/safety-of-plants/three-mile-island-accident.aspx}{\textit{Three mile island accident}}, \textbf{World Nuclear Association}.

%Reference 19
\bibitem{EIAOutlook}
\textit{International Energy Outlook 2013}. \href{https://www.eia.gov/outlooks/ieo/}{\textbf{U. E. Energy Information Administration}}.

%Reference 20
\bibitem{FERMILAB}
\href{https://www.fnal.gov/pub/tritium/}{Tritium at Fermilab}.

%Reference 21
\bibitem{BrookHavenNationalLaboratory}
\href{https://www.bnl.gov/hfbr/decommission.php}{\textbf{Brookhaven National Laboratory (BNL)}}.

%Reference 22
\bibitem{TrackingTritium} \textsc{Aleksandra Sawodni}, \textsc{Anna Pazdur}, \textsc{Jacek Pawlyta}, 
\href{http://yadda.icm.edu.pl/baztech/element/bwmeta1.element.baztech-article-BAT3-0035-0005}{\textit{Measurements of Tritium Radioactivity in Surface Water on the Upper Silesia Region}}, Journal on Methods and Applications of Absolute Chronology, Geochronometria, Vol. 18, pp 23-28 \textbf{2000}.

%Reference 23
\bibitem{TritiumDiscovery} \textsc{M. L. Oliphant}, \textsc{P. Harteck} and \textsc{E. Rutherford},  
\href{https://royalsocietypublishing.org/doi/10.1098/rspa.1934.0077}{\textit{Transmutation Effects observed with Heavy Hydrogen}}, Nature, 133, 413 (1934)\href{https://doi.org/10.1038/133413a0}{\textbf{DOI: 10.1038/133413a0}}.

%Reference 24
\bibitem{TritiumIsolate} \textsc{Luis W. Alvarez} and \textsc{R. Cornog},  
\textit{Helium and Hydrogen of Mass 3}, Physical Review Journals Archive, 56, 613 (1939)\href{https://doi.org/10.1103/PhysRev.56.613}{\textbf{DOI: 10.1103/PhysRev.56.613}}.

%Reference 25
\bibitem{TritiumHandling} 
\href{https://www.twirpx.com/file/1977676/}{\textit{DOE Handbook: Primer on Tritium Safe Handling Practices}}, U. S. Departament Of Energy Washington, D.C. 20585.

%Reference 26
\bibitem{OxigenTritium} \textsc{Robert Haight}, \textsc{Joseph Wermer} and \textsc{Michael Fikani},
\textit{Tritium Production by Fast Neutrons on Oxygen: An Integral Experiment}, Journal of Nuclear Science and Technology, 39:sup2, 1232-1235, \href{https://doi.org/10.1080/00223131.2002.10875326}{\textbf{DOI: 10.1080/00223131.2002.10875326}}. 

%Reference 27
\bibitem{FranceTritiumEnvironment} \textsc{Institut de Radioprotection et de Sureté Nucléaire}
\textit{Tritium and the environment}, \href{https://www.irsn.fr/EN/Research/publications-documentation/radionuclides-sheets/environment/Pages/Tritium-environment.aspx}{\textit{Tritium and the environment}}, IRSN, Enhancing nuclear safety. 

%Referencia 28
\bibitem{CrossSeccionNeutrino} \textsc{},
\textit{REFERENCIAAAA}, \textbf{}

%Referencia 29
\bibitem{TritiumDecayEnergyLevels} 
\href{https://www-nds.iaea.org}{\textit{International Atomic Energy Agency}}.

%Referencia 30
\bibitem{TritiumDecayImage} 
\href{https://conexioncausal.wordpress.com}{\textit{Tritium decay image}}.

%Referencia 31
\bibitem{TritiumEspectrum} \textsc{Zhang Lin},
\href{https://www.mdpi.com/2073-4352/10/2/105/htm}{\textit{Simulation and Optimization Design of SiC-Basaed PN Betavoltaic Microbattery Using Tritium Source}}, MDPI Open Access Journal \textit{12/02/2020}, \textbf{DOI:10.3390/cryst10020105}

%Reference 32
\bibitem{MeanFreePathDocument} \textsc{Blauvelt, R.K.}, \textsc{Deaton, M.R.} and \textsc{Gill, J.T.},
\textit{Health Physics Manual of Good Practices for Tritium Facilities}, EG and G Mound Applied Technologies, Miamisburg, OH (United States), Technical Report,  01 December 1991, \href{https://doi.org/10.2172/266889}{\textbf{DOI: 10.2172/266889}}. 

%Reference 33
\bibitem{EstimationTritiumDosi} \textsc{Tsuyoshi Masuda} and \textsc{Toshitada Yoshioka},
\textit{Estimation of radiation dose from ingested tritium in humans by administration of deuterium-labelled compounds and food}, Scientific reports, 02 Febrary 2021, \href{https://doi.org/10.1038/s41598-021-82460-5}{\textbf{DOI: 10.1038/s41598-021-82460-5}}. 

%Reference 34
\bibitem{EstimationTritiumDosiRats} \textsc{Z. Pietrzak-Flis}, \textsc{I. Radwan}, \textsc{Z. Major} and \textsc{M. Kowalska},
\textit{Tritium Incorporation in Rats Chronically Exposed to Tritiated Food or Tritiated Water for Three Successive Generations}, Journal of Radiation Research, Vol 22, Issue 4, December 1981, page 434-442 \href{https://doi.org/10.1269/jrr.22.434}{\textbf{DOI: 10.1269/jrr.22.434}}. 

%Reference 35
\bibitem{EstimationTritiumDosiKangarooRats} \textsc{J.R. Martin} and \textsc{J.J. Koranda},
\textit{Biological Half-Life Studies of Tritium in Chronically Exposed Knagaroo Rats}, Journal of Radiation Research, Vol 50, Issue 2, May 1972, page 426-440 \href{https://www.jstor.org/stable/3573500?seq=1#metadata_info_tab_contents}{\textbf{PMID: 5025235}}. 

%Reference 36
\bibitem{StraumeTritiumHazard} \textsc{T Straume} and \textsc{A. L. Carsten},
\textit{Tritium radiobiology and relative biological effectiveness}, Health Physics, Vol. 65, Number 6, December 1993, \href{https://pubmed.ncbi.nlm.nih.gov/8244712/}{\textbf{DOI: 10.1097/00004032-199312000-00005 }}. 

%Reference 37
\bibitem{RytoemaaTritiumHazard} \textsc{Rytoemaa, T.}, \textsc{Saltevo, J.} and \textsc{Toivonen, H.},
\href{http://inis.iaea.org/search/search.aspx?orig_q=RN:11535484}{\textit{Radiotoxicity of Tritium-Labelled Molecules}}, International Atomic Energy Agency symposium, IAEA, Vienna: Biological Implications of Radionuclides Released from Nuclear Industries, INIS Vol. 11, INIS Issue. 13, Reference Number, 11535484, 1979. 

%Reference 38
\bibitem{ICRP_GL} \textsc{International Commission on Radiological Protection, ICRP},
\href{https://www.icrp.org/publication.asp?id=icrp\%20publication\%2060}{\textit{Recommendations of the ICRP. Annals of the ICRP, 21(1.3), 1991a. 1990. Oxford, Pergamon Press (Publication 60).}}. 

%Reference 39
\bibitem{WHO_GL} \textsc{World Health Organization, WHO}, 
\href{http://www.who.int/water_sanitation_health/dwq/
GDWQ2004web.pdf}{\textit{Guidelines for Drinking-Water Quality. Vol 1. Third Edition. Geneve, Switzerland, 2004}}. 

%Reference 40
\bibitem{ICRP_factor} \textsc{International Commission on Radiological Protection, ICRP}, 
\href{https://www.icrp.org/publication.asp?id=ICRP\%20Publication\%2072}{\textit{Age-dependent doses to members of the public from intake of radionuclides: Part 5. Compilation of ingestion and inhalation dose coefficients. Oxford, Pergamon Press (International Commission on Radiological Protection Publication 72), 1996}}. 

%Reference 41
\bibitem{Switzerland_GL} \textsc{Département fédéral de l'intérieur, DFI (Federal Department of the Interior)},
\href{www.admin.ch/ch/f/rs/8/817.021.23.fr.pdf}{\textit{Ordonnance du DFI sur les substances etrangères et les composants dans les denrées alimentaires (817.021.23)}}, 2006, Switzerland (in French).

%Reference 42
\bibitem{Ontario_GL} \textsc{Ontario Ministry of the Environment},
\href{https://atrium.lib.uoguelph.ca/xmlui/handle/10214/15832}{\textit{Ontario Drinking Water Objectives. Toronto, Ontario, 1994}}. 

%Reference 43
\bibitem{Quebec_GL} \textsc{Québec},
\href{https://numerique.banq.qc.ca/patrimoine/details/52327/3582272?docref=fxoJ-qgA5cus5Upw-L_NHg}{\textit{Résultats du programme de surveillance de l’environnement du site de Gentilly. Rapport annuel 2006. Québec, Canada.}}. 

%Reference 44
\bibitem{Russia_GL} \textsc{Russia},
\href{http://www.wdcb.ru/mining/zakon/NRB99.htm}{\textit{NRB-99 Radiation Safety Norms}}, 2007. 

%Reference 45
\bibitem{Australia_GL} \textsc{Australian Government}, \textsc{National Health and Medical Reserch Council} and \textsc{Natural Resource Management Ministerial Council},
\href{https://www.nhmrc.gov.au/about-us/publications/australian-drinking-water-guidelines}{\textit{AustralianDrinking Water Guideilnes 6}}, National Water Quality Managment Strategy,Version 3.6, Updated March 2011. 

%Reference 46
\bibitem{Finland_GL} \textsc{Nuclear Energy Agency, NEA},
\href{https://www.oecd-nea.org/jcms/pl_23551/finland}{\textit{Radiation and Nuclear Safety Authority}}, 1993. Radioactivity of Household Water. ST 12.3. Erweko Paintuote, Helsinki, Finland, 1994. 

%Reference 47
\bibitem{California_GL} \textsc{Office of Environmental HEalth Hazard Assessment, OEHHA},
\href{https://oehha.ca.gov/water/public-health-goal/public-health-goals-six-chemicals-drinking-water}{\textit{Public Health Goals for Chemicals in Drinking Water-Tritium. OEHHA, California ENfironmental Protection Agency, California USA, September, 2007}}. 

%Reference 48
\bibitem{USEPA_GL} \textsc{United States Environmental Protection Agency, US EPA},
\href{https://www.epa.gov/dwreginfo/radionuclides-rule}{\textit{Drinking Water Requirements for States and Public Water Systems}}, Radionuclides Rule, 1976. 

%Reference 49
\bibitem{France_GL} \textsc{Institut de radioprotection et de sûreté nucléaire, IRSN (Radioprotection and Nucelar Safety Institute)},
\href{https://www.google.com/url?sa=t&rct=j&q=&esrc=s&source=web&cd=&ved=2ahUKEwiskum8mYLwAhXLB2MBHWLgAkoQFjAAegQIBBAD&url=https\%3A\%2F\%2Fwww.actu-environnement.com\%2Fmedia\%2Fpdf\%2Fnews-32705-bilan.pdf&usg=AOvVaw0oCSJP78IgV1Tek0T4_6z1}{\textit{Bilan de l’état radiologique  de l’environnement français  de 2015 à 2017}}. France. 

%Reference 50
\bibitem{Germany_GL} \textsc{Bundesamt für Strahlenschutz, BMU (Federal Office for Radiation Protection)},
\href{http://doris.bfs.de/jspui/handle/urn:nbn:de:0221-20100331990}{\textit{ Environmental Radioactivity and Radiation Exposure}}, Annual Report, 2005, (Jahresbericht 2005). BMU, Bonn, Germany (in German). 

%Reference 51
\bibitem{Spain_GL} \textsc{Consejo de Seguridad Nuclear, CSN, Nuclear Safety Council},
\href{https://www.csn.es/en/normativa-del-csn/normativa-espanola}{\textit{National Regulation of Radionuclides}}. 

%Reference 52
\bibitem{EURATOM_GL} \textsc{European Atomic Energy Community, EURATOM},
\href{https://eur-lex.europa.eu/eli/dir/2013/59/oj}{\textit{Council directive 2013/15/euratom}}, October, 2013. Laying down requirements for the protection of the health of the general public with regard to radioactive substances in water intended for human consumption. 

%Referencia 53
\bibitem{LSCothers} \textsc{M. N. Al-Haddad}, \textsc{A. H. Fayoumi} and \textsc{F. A. Abu-Jarad},
\textit{Calibration of a liquid scintillation counter to assess tritium levels in various samples}, Nuclear Instruments and Methods in PHysics Research A, Volume 438, Issues 2-3, December 1999, Pages 356-361, \href{https://doi.org/10.1016/S0168-9002(99)00272-7}{\textbf{DOI: 10.1016/S0168-9002(99)00272-7}}.

%Referencia 54
\bibitem{HofstetterSeveral} \textsc{K. J. Hofstetter} and \textsc{H. T. Wilson},
\textit{Aqueous Effluent Tritium Monitor Development}, Fusion Technology, Volume 21, 2P2, Pages 446-451, March 1992, \href{https://doi.org/10.13182/FST92-A29786}{\textbf{DOI: 10.13182/FST92-A29786}}.

%Referencia 55
\bibitem{0.6Bq_L} \textsc{M. Palomo}. \textsc{A. Peñalver}, \textsc{C. Aguilar} and \textsc{F. Borrull},
\textit{Tritium activity levels in environmental water samples from different origins}, Applied Radiation and Isotopes, Volume 65, Issue 9, September 2007, Pages 1048-1056, \href{https://doi.org/10.1016/j.apradiso.2007.03.013}{\textbf{DOI: 10.1016/j.apradiso.2007.03.013}}.

%Referencia 56
\bibitem{OnlineLSC} \textsc{R. A. Sigg}, \textsc{J. E. McCarty}, \textsc{R. R. Livingston} and \textsc{M. A. Sanders},
\textit{Real-time aqueous tritium monitor using liquid scintillation counting}, FNuclear Instrument and Methods in Physics Research A, Volume 353, Issues 1-3, 30 Decembre 1994, Pages 494-498 \href{https://doi.org/10.1016/0168-9002(94)91707-8}{\textbf{DOI: 10.1016/0168-9002(94)91707-8}}.

%Referencia 57
\bibitem{IonizationChamber1} \textsc{N. P. Kherani},
\textit{An alternative approach to tritium-in-water monitoring}, Nuclear and Methods in PHysics Research A, Volume 484, Issues 1-3, 21 May 2002, Pages 650-659 \href{https://doi.org/10.1016/S0168-9002(01)02008-3}{\textbf{DOI: 10.1016/S0168-9002(01)02008-3}}

%Referencia 58
\bibitem{IonizationChamber2} \textsc{Z. Chen}, \textsc{S. Peng}, \textsc{D. Meng} \textsc{Y. He} and \textsc{H. Wang},
\textit{Theoretical study of energy deposition in ionization chambers for tritium measurements}, Review of Scientific Instruments, 84, 103302, 2013, \href{https://dx.doi.org/10.1063/1.4825032}{\textbf{DOI: 10.1063/1.4825032}}.

%Referencia 59
\bibitem{Calorimeter1} \textsc{C. G. Alecu}, \textsc{U. Besserer}, \textsc{B. Bornschein}, \textsc{B. Kloppe}, \textsc{Z. Köllö} and \textsc{J. Wendel},
\textit{Reachable Accuracy and Precision for Tritium Measurements by Calorimetry at TLK}, Fusion Science and Technology, 60:3, 937-940, \href{https://doi.org/10.13182/FST11-A12569}{\textbf{DOI: 10.13182/FST11-A12569}}.

%Referencia 60
\bibitem{Calorimeter2} \textsc{A. Bükki-Deme}, \textsc{C. G. Alecu}, \textsc{B. Kloppe} and \textsc{B. Bornschein},
\textit{First results with the upgraded TLK tritium calorimeter IGC-V0.5}, Fusion Engineering and Design, Volume 88, Issue 11, November 2013, Pages 2865-2869 \href{https://doi.org/10.1016/j.fusengdes.2013.05.066}{\textbf{DOI: 10.1016/j.fusengdes.2013.05.066}}.

%Referencia 61
\bibitem{XRays1} \textsc{M. Matsuyama}, \textsc{Y. Torikai}, \textsc{M. Hara} and \textsc{K. Watanabe},
\textit{New Technique for non-destructive measurements of tritium in future fusion reactors}, IAEA Nuclear Fusion, Volume 47, Number 7, S464, June 2007, \href{https://doi.org/10.1088/0029-5515/47/7/S09}{\textbf{DOI: 10.1088/0029-5515/47/7/S09}}.

%Referencia 62
\bibitem{XRays2} \textsc{M. Matsuyama},
\textit{Development of a new detection system for monitoring high-level tritiated water}, Fusion Engineering and Design, Volume 83, Issue 10-12, December 2008, Pages 1438-1441 \href{https://doi.org/10.1016/j.fusengdes.2008.05.023}{\textbf{DOI: 10.1016/j.fusengdes.2008.05.023}}.

%Referencia 63
\bibitem{Bremstrahlung} \textsc{S. Niemes}, \textsc{M. Sturm}, \textsc{R. Michling} and \textsc{B. Bornschein},
\textit{High Level Tritiated Water Monitoring by Bremsstrahlung Counting Using a Silicon Dift Detector}, Fusion Science and Technology, 67:3, 507-510, 2015, \href{https://doi.org/10.13182/FST14-T66}{\textbf{DOI: 10.13182/FST14-T66}}.

%Referencia 64
\bibitem{APD} \textsc{K. S. Shah}, \textsc{P. Gothoskar}, \textsc{R. Farrell} and \textsc{J. Gordon},
\textit{High Efficiency Detection of Tritium Using Silicon Avalanche Photodiodes}, IEEE Transactions on Nuclear Science, Volume 44, Issue 3, June 1997, \href{https://doi.org/10.1109/23.603750}{\textbf{DOI: 10.1109/23.603750}}

%Referencia 65
\bibitem{Spectrometry} \textsc{P. Jean-Baptiste}, \textsc{E. Fourré}, \textsc{A. Dapoigny}, \textsc{D. Baumier}, \textsc{N. Baglan} and \textsc{G. Alanic},
\textit{\ce{^{3}He} mass spectrometry for very low-level measurement of organic tritium in environmental samples}, Journal of Environmental Radioactivity, Volume 101, Issue 2, Febrary 2010, Pages 185-190, \href{https://doi.org/10.1016/j.jenvrad.2009.10.005}{\textbf{DOI: https://doi.org/10.1016/j.jenvrad.2009.10.005}}. 

%Referencia 66
\bibitem{Ring} \textsc{C. Bray}, \textsc{A. Pailoux} and \textsc{S. Plumeri},
\textit{Tritiated water detection in the 2.17 $\mu$M spectral region by cavity ring down spectroscopy},  Nuclear Instruments and Methods in Physics Research A, Volume 789, 21 July 2015, Pages 43-49, \href{https://doi.org/10.1016/j.nima.2015.03.064}{\textbf{DOI: 10.1016/j.nima.2015.03.064}}. 

%Referencia 67
\bibitem{Muramatsu} \textsc{M. Muramatsu}, \textsc{A. Koyano} and \textsc{N. Tokanuga},
\textit{A Scintillation Probe for Continuous Monitoring of Tritiated Water}, Nuclear Instruments and Methods, Volume 54, Issue 2, October 1967, Page 325-326, \href{https://doi.org/10.1016/0029-554X(67)90645-3}{\textbf{DOI: 10.1016/0029-554X(67)90645-3}}.

%Referencia 68
\bibitem{Moghissi} \textsc{A. A. Moghissi}, \textsc{H. L. Kelley}, \textsc{C. R. Phillips} and \textsc{J. E. Regnier},
\textit{A Tritium Monitor Based on Scintillation}, Nuclear Instruments and Methods, Volume 68, Issue 1, 1 Febrary 1969, Page 159, \href{https://doi.org/10.1016/0029-554X(69)90705-8}{\textbf{DOI: 10.1016/0029-554X(69)90705-8}}.

%Referencia 69
\bibitem{Osborne} \textsc{R. V. Osborne},
\textit{Detector for Tritium in Water}, Nuclear Instruments and Methods, Volume 77, Issue 1, 1 January 1970, Page 170-172, \href{https://doi.org/10.1016/0029-554X(70)90596-3}{\textbf{DOI: 10.1016/0029-554X(70)90596-3}}.

%Referencia 70
\bibitem{Ratnakaran} \textsc{A. N. Singh}, \textsc{M. Ratnakaran} and \textsc{K. G. Vohra},
\textit{An Online Tritium-in-Water Monitor}, Nuclear Instruments and Methods, Volume 236, Issue 1, 1 May 1985, Page 159-164, \href{https://doi.org/10.1016/0168-9002(85)90141-X}{\textbf{DOI: 10.1016/0168-9002(85)90141-X}}.

%Referencia 71
\bibitem{Ratnakaran2000} \textsc{M. Ratnakaran}, \textsc{R. M. Revetkar}, \textsc{R. K. Samant} and \textsc{M. C. Abani},
\href{https://inis.iaea.org/search/search.aspx?orig_q=RN:32015986}{\textit{A Real-time Tritium-In-Water Monitor for Measurement Of Heavy Water Leak To The Secondary Coolant}}, International congress of the International Radiation Protection Association, Volume 32, Issue 15, 14-19 May 2000, P-3a-197, Reference number: \textbf{32015986}

%Referencia 72
\bibitem{Hofstetter1} \textsc{K. J. Hofstetter} and \textsc{H. T. Wilson},
\textit{Aqueous Effluent Tritium Monitor Development}, Fusion Technology, Volume 21, 2P2, 1992, Pages 446-451, \href{https://doi.org/10.13182/FST92-A29786}{\textbf{DOI: 10.13182/FST92-A29786}}.

%Referencia 73
\bibitem{Hofstetter2} \textsc{K. J. Hofstetter} and \textsc{H. T. Wilson},
\href{https://www.osti.gov/biblio/6865647-continuous-tritium-effluent-water-monitor-savannah-river-site}{\textit{Continuous Tritium Effluent Water Monitor at the Savannah River Site}}, International conference on advances in liquid scintillation, Vienna (Austria), 14-18 September 1992.

%Referencia 74
\bibitem{TRITIUM} \textit{Tritium, Interreg Sudoe Program}. 
\href{https://tritium-sudoe.eu/es-es/homepage}{\textbf{Tritium website}}.

%Referencia 75
\bibitem{Geant4WebPage} \textsc{CERN Collaboration},
\textit{Geant4: A toolkit for the simulation of the passage of particles through matter.}. \href{https://geant4.web.cern.ch/node/1}{\textbf{Website}}.

%Referencia 76
\bibitem{Knoll} \textsc{Glenn F. Knoll}, 
\textit{Radiation Detection and Measurement}, Third Edition, John Wiley and Sons, Inc. 1999.

%Referencia 77
\bibitem{Leo} \textsc{William R. Leo},
\textit{Techniques for Nuclear and Particle Physics Experiments: a how-to approach}, Second Revised Edition, Springer-Verlag Berlin Heidelberg GmbH, 1994. \href{https://doi.org/10.1007/978-3-642-57920-2}{\textbf{DOI: 10.1007/978-3-642-57920-2}}. 

%Referencia 78
\bibitem{DataSheetBCF12Fiber} \textsc{Saint-Gobain Ceramics and Plastics, Inc.},
\textit{Scintillating Optical Fibers}, It's What's Inside that Counts, 2005-14. \href{https://www.crystals.saint-gobain.com/products/scintillating-fiber}{\textbf{Data sheet}}. 

%Referencia 79
\bibitem{TFGAlberto} \textsc{},
\textit{}, . \href{}{\textbf{}}. 

%Referencia 80
%\bibitem{DataSheetKuraray}
%\textit{Plastic Scintillating Fibers}, Scintillating Fibers, Wavelength Shifting Fibers and Clear Fibers. \href{https://www.kuraray.com/products/psf}{\textbf{Data sheet}}. 

%Referencia 80
\bibitem{Snell} \textsc{},
\textit{}, \href{}{\textbf{}}. 

%Referencia 81
\bibitem{DataSheetPMTs} \textsc{HAMAMATSU PHOTONICS K.K.},
\textit{Photonmultiplier tube R8520-406/R8520-506}. \href{https://www.hamamatsu.com/eu/en/product/type/R8520-406/index.html}{\textbf{Data sheet}}.

%Referencia 82
\bibitem{CalibrationPMTsNEXT} \textsc{Javier Pérez Pérez},
 \href{https://next.ific.uv.es/cgi-bin/DocDB/public/ShowDocument?docid=48}{\textit{Caracterización de los Fotomultiplicadores R8520-06SEL para NEXT}, 25-06-2010}.

%Referencia 83
\bibitem{TesisNEXTSiPMs} \textsc{David Lorca Galindo},
\href{https://dialnet.unirioja.es/servlet/tesis?codigo=101465}{\textit{Tesis: SiPM based tracking for detector calibration in NEXT}}, Departamento de física atómica, molecular y nuclear, Universidad de Valencia (UV), Valencia, Spain, \textit{03/2015}.

%Referencia 84
\bibitem{OSI} \textsc{OSI Optoelectronics}, 
\href{https://osioptoelectronics.com/standard-products/default.aspx?gclid=EAIaIQobChMIkYrLif_37QIVDNTtCh3NuwpkEAAYASAAEgKMJ_D_BwE}{\textit{Characteristics and Applications}}.

%Referencia 85
\bibitem{DataSheetHammamatsu_1_SiPM_50} \textsc{HAMAMATSU PHOTONICS K.K. Solid State Division},
\textit{MPPC Multi-Pixel Photon Counter S13360-6050}. \href{https://www.hamamatsu.com/eu/en/product/type/S13360-6050CS/index.html}{\textbf{Data sheet}}.

%Referencia 86
\bibitem{DataSheetHammamatsu_1_SiPM_75} \textsc{HAMAMATSU PHOTONICS K.K. Solid State Division},
\textit{MPPC Multi-Pixel Photon Counter S13360-6075}. \href{https://www.hamamatsu.com/eu/en/product/type/S13360-6075CS/index.html}{\textbf{Data sheet}}.

%Referencia 87
\bibitem{DataSheetHammamatsu_array_SiPM_6050} \textsc{HAMAMATSU PHOTONICS K.K. Solid State Division},
\textit{MPPC Multi-Pixel Photon Counter S13361-6050}. \href{https://www.hamamatsu.com/us/en/product/type/S13361-6050AE-04/index.html}{\textbf{Data sheet}}.

%Referencia 88
\bibitem{DataSheetHammamatsu_array_SiPM_3050} \textsc{HAMAMATSU PHOTONICS K.K. Solid State Division},
\textit{MPPC Multi-Pixel Photon Counter S13361-3050}. \href{https://www.hamamatsu.com/jp/en/product/type/S13361-3050AE-08/index.html}{\textbf{Data sheet}}.

%Referencia 89
\bibitem{DataSheetSensL} \textsc{SensL sense light},
\textit{Introduction to the SPM TECHNICAL NOTE}. February 2017 \href{https://sensl.com/}{\textbf{Document}}.

%Referencia 90
\bibitem{DataSheetKeithley6487} \textsc{KEITHLEY, a greater measure of confidence},
\textit{Model 6487 Picoammeter/voltage source, Manual reference}. \href{https://pdf.directindustry.com/pdf/keithley-instruments/6487-picoammeter-voltage-source/1438-619876.html}{\textbf{Data sheet}}.

%Referencia 91
\bibitem{DataSheetHVSupplyTennelec} \textsc{Tennelec},
\textit{Model TC 952 High Voltage Supply, Manual reference}. \href{https://groups.nscl.msu.edu/nscl_library/manuals/tennelec/tennelec.htm}{\textbf{Data sheet}}.

%Referencia 92
\bibitem{DataSheetHVSupplyWenzel} \textsc{Wenzel Electronik},
\textit{Model N 1330-4 High Voltage Power Supply}. \href{https://wenzel-elektronik.de}{\textbf{Website}}.

%Referencia 93
\bibitem{DataSheetFANINOUT} \textsc{Philips Scientific},
\textit{Model 740 Quad Linear Fan-In/Out, Manual reference}. \href{https://prep.fnal.gov/catalog/hardware_info/phillips_scientific/740.html}{\textbf{Data sheet}}.

%Referencia 94
\bibitem{DataSheetPreAmp} \textsc{ORTEC},
\textit{Model 9326 FastPreamplifier, Manual reference}. \href{https://www.ortec-online.com/products/electronics/preamplifiers/9326}{\textbf{Data sheet}}.

%Referencia 95
\bibitem{DataSheet575Amp} \textsc{ORTEC},
\textit{Model 575A Amplifier, Manual reference}. \href{https://www.ortec-online.com/products/electronics/amplifiers/575a}{\textbf{Data sheet}}.

%Referencia 96
\bibitem{DataSheet671Amp} \textsc{ORTEC},
\textit{Model 671 Spectroscopy Amplifier, Manual reference}. \href{https://www.ortec-online.com/products/electronics/amplifiers/671}{\textbf{Data sheet}}.

%Referencia 97
\bibitem{DataSheetDiscriminator} \textsc{ORTEC},
\textit{Model CF8000 Octal Constant-Fraction Discriminator, Manual reference}. \href{https://www.ortec-online.com/products/electronics/fast-timing-discriminators/cf8000}{\textbf{Data sheet}}.

%Referencia 98
\bibitem{DataSheetDiscriminatorCAEN} \textsc{CAEN},
\textit{Model 84, 4 channels discriminator}. \href{https://www.caen.it/}{\textbf{Website}}.

%Referencia 99
\bibitem{DataSheetCoincidenceLeCroy} \textsc{LeCroy},
\textit{Model 465 Coincidence Unit, Manual reference}. \href{https://prep.fnal.gov/catalog/hardware_info/lecroy/nim/465.html}{\textbf{Data sheet}}.

%Referencia 100
\bibitem{DataSheetCoincidenceCERN} \textsc{CERN},
\textit{Coincidence Unit Type N6234, Manual reference}. \href{}{\textbf{Data sheet}}.

%Referencia 101
\bibitem{DataSheetGateAndDelay} \textsc{ORTEC},
\textit{Model 416A Gate and Delay Generator, Manual reference}. \href{https://www.ortec-online.com/products/electronics/delays-gates-and-logic-modules/416a}{\textbf{Data sheet}}.

%Referencia 102
\bibitem{DataSheetMCA} \textsc{AmpTek},
\textit{MCA8000D, Pocket MCA, Digital Multichannel Analyzer, Manual reference}. \href{https://www.amptek.com/products/multichannel-analyzers/mca-8000d-digital-multichannel-analyzer}{\textbf{Data sheet}}.

%Referencia 103
\bibitem{PETSYS} \textit{PETsys Electronics}. \href{https://www.petsyselectronics.com/web/private/login}{\textbf{Website}}.

%Referencia 104
\bibitem{OpticalFibers} \textsc{Saint-Gobain Ceramics and Plastics, Inc.},
\textit{Optical fiber BCF-98, Manual reference}. \href{https://www.crystals.saint-gobain.com/products/scintillating-fiber}{\textbf{Manual reference}}.

%Referencia 105
\bibitem{LEDThorlabs} \textsc{Thorlabs},
\textit{LED430L - 430 nm LED with a Glass Lens, 8 mW, TO-18}. \href{https://www.thorlabs.com/thorproduct.cfm?partnumber=LED430L}{\textbf{Datasheet}}.

%Referencia 106
\bibitem{NaturalRadioactiveSeries1} \textsc{Pall Theodórsson},
\textit{Measurement of weak radioactivity}, World Scientific, 1996.

%Referencia 107
\bibitem{NaturalRadioactiveSeries2} \textsc{R D Evans},
\textit{The Atomic Nucleus}, McGraw-Hill, Inc., 1996.

%Referencia 108
\bibitem{PDG} \textsc{P.A. Zyla et al.},
\textit{(Particle Data Grup), PDG, Prog. Theor. Exp. Phys. \textbf{2020} no. 8, 083C01 (2020)}. \href{https://pdg.lbl.gov/}{\textbf{Website}}~\href{https://academic.oup.com/ptep/article/2020/8/083C01/5891211}{\textbf{DOI: 10.1093/ptep/ptaa104}}.

%Referencia 109
\bibitem{HardCosmicMuonRate} \textsc{Hiroyuki SAGAWA \& Itsumasa URABE (2001)},
\textit{Estimation of Absorbed Dose Rates in Air Based on Flux Densities of Cosmic Ray Muons and Electrons on the Ground Level in Japan}, Journal of Nuclear Science and Technology, 38:12, 1103-1108, \href{https://doi.org/10.1080/18811248.2001.9715142}{\textbf{DOI: 10.1080/18811248.2001.9715142}}.

%Referencia 110
\bibitem{HardCosmicMuonRatePlot} \textsc{T. Szücs, D. Bemmerer, T. P. Reinhardt, K. Schmidt, M. P Takács, A. Wagner, L. Wagner, D. Weinberger and K. Zuber},
\textit{Cosmic-ray induced background intercomparison with actively shielded HPGe detectors at underground locations}. \href{https://arxiv.org/abs/1503.00457v2}{\textbf{DOI:10.1140/epja/i2015-15033-0}}.

%Referencia 111
\bibitem{ScintillatorVeto} \textsc{Epic Crystal},
\textit{Plastic scintillator of Epic Crystal, Manual reference}. \href{http://www.epic-crystal.com/others/plastic-scintillator.html}{\textbf{Data sheet}}.

%Referencia 112
\bibitem{DiamondThorlabs} \textsc{Thorlabs},
\textit{Guide to connectorization and polishing optical fibers}, 2006. \href{https://www.thorlabs.de/thorproduct.cfm?partnumber=FN96A}{\textbf{Manual Reference}}.

%Referencia 113
\bibitem{GuillotineIFO} \textsc{Indistroañ fiber optical},
\textit{POF Cutter block}. \href{https://i-fiberoptics.com/tool-detail.php?id=105&cat=cutters}{\textbf{Website}}.

%Referencia 114
\bibitem{AngleBlade} \textsc{David Sáez-Rodríguez, Kristian Nielsen, Ole Bang and David John Webb},
\textit{Simple Room Temperature Method for Polymer Optical Fibre CLeaving}, Journal of lightwave technology, vol 33, No. 23, December 1, 2015. \href{https://ieeexplore.ieee.org/document/7274313}{\textbf{DOI:10.1109/JLT.2015.2479365}}.

%Referencia 115
\bibitem{TemperatureBlade} \textsc{S.H. Law, J.D. Harvey, R.J. Kruhlak, M. Song, E. Wu, G.W. Barton, M.A. van Eijkelenborg and M.C.J. Large},
\textit{Cleaving of microstructured polymer optical fibres}. \href{https://www.researchgate.net/publication/228880071_Cleaving_of_microstructured_polymer_optical_fibres}{\textbf{DOI:10.1016/j.optcom.2005.08.011}}.

%Referencia 116
\bibitem{StepperMotors} \textsc{Nanotec},
\textit{ST4209S1404-A - STEPPER MOTOR NEMA 17}. \href{https://en.nanotec.com/products/463-st4209s1404-a}{\textbf{Data sheet}}.

%Referencia 117
\bibitem{ArduinoUNO}
\textit{ARDUINO}, \href{https://www.arduino.cc/}{\textbf{Website}}.

%Referencia 118
\bibitem{CNCShield}
\textit{CNC shield V3.0}, \href{https://osoyoo.com/2017/04/07/arduino-uno-cnc-shield-v3-0-a4988/}{\textbf{Reference manual}}.

%Referencia 119
\bibitem{A4988Driver} \textsc{Allegro}
\textit{Driver Pololu A4988, DMOS Microstepping Driver with Translator And Overcurrent Protection}, \href{https://www.alldatasheet.es/datasheet-pdf/pdf/455036/ALLEGRO/A4988.html}{\textbf{Data sheet}}.

%Referencia 120
\bibitem{DRV8825Driver} \textsc{Texas Instruments}
\textit{Driver DRV8825 Stepper Motor Controller IC}, \href{https://www.ti.com/product/DRV8825?utm_source=google&utm_medium=cpc&utm_campaign=asc-null-null-GPN_EN-cpc-pf-google-wwe&utm_content=DRV8825&ds_k=DRV8825+Datasheet&DCM=yes&gclid=EAIaIQobChMIworWtYba7gIVqoFQBh10_QfhEAAYASAAEgLPn_D_BwE&gclsrc=aw.ds}{\textbf{Data sheet}}.

%Referencia 121
\bibitem{TMC2208Driver}
\textit{Driver TMC2208, Step/Dir Drivers for Two-Phase Bipolar Stepper Motors up to 2A peak- StealthChop for Quiet Movement- UART Interface Option}, \href{https://datasheetspdf.com/pdf/1142008/TRINAMIC/TMC2225/1}{\textbf{Data sheet}}.

%Referencia 122
\bibitem{LEDRLT} \textsc{Roithner LaserTechnik Gmbh}
\textit{LED435-03, 20 mW, 20 mA}, \href{http://www.roithner-laser.com/led_diverse.html}{\textbf{Reference}}.

%Referencia 123
\bibitem{FiberConnector} \textsc{},
\textit{}, \href{}{\textbf{Reference}}.

%Referencia 124
\bibitem{OpticalGrease} \textsc{Saint-Gobain Ceramics and Plastics, Inc.},
\textit{BC-630, Silicone Optical Grease}, \href{https://www.crystals.saint-gobain.com/}{\textbf{Website}}.

%Referencia 125
\bibitem{BlackBlancket} \textsc{Thorlabs},
\textit{BK5 - Black Nylon, Polyurethane-Coated Fabric, 5'x9' (1.5m x 2.7m) x 0.005" (0.12 mm) Thick}, \href{https://www.thorlabs.com/thorproduct.cfm?partnumber=BK5}{\textbf{Datasheet}}.

%Referencia 126
\bibitem{WettingProperty} \textsc{San Nopco company},
\textit{Wetting property}, \href{https://www.sannopco.co.jp/eng/products/function/function4.php}{\textbf{Website}}.

%Referencia 127
\bibitem{TurbiditySystem} \textsc{Hanna Instruments},
\textit{Multiparamétrico con opciones GPS, sonda autoregistradora, turbidez e ISE}, \href{https://www.hannainst.es/parametros/4654-multiparametrico-portatil-con-portasondas-multisensor-ph-orp-ce-od-temperatura.html#/507-cable_m-4_m/512-portasondas-si/513-portasondas_registrador-no/514-gps-no/515-turbidez-no}{\textbf{Website}}.

%Referencia 107
%\bibitem{ElectronicMicroscopeSCSIE} \textsc{SCSIE}
%\textit{Microscopio electrónico de barrido por emisión de campo, Marca: Hitachi S4800}, \href{https://www.uv.es/uvweb/servicio-central-soporte-investigacion-experimental/es/equipamientos-instalaciones/todos-equipamientos-instalaciones/microscopio-electronico-barrido-emision-campo-1285887466621/OCTRecurs.html?id=1286169920510}{\textbf{Website}}.

%Referencia 128
\bibitem{ScalerDataSheet} \textsc{CAEN company},
\textit{Quad Scaler And Preset Counter-Timer, N1145}, \href{https://www.caen.it/products/n1145/}{\textbf{Datasheet}}.

%Referencia 129
\bibitem{TritiumSourceTechnicalFile} \textsc{Physikalisch-Technische Bundesanstalt, PTB, Braunschweig and Berlin, Germany}
\textit{Calibration Certificate of tritium source, PTB-6.11-2005-1442}.

%Referencia 130
\bibitem{DataSheetBCF10Fiber} \textsc{Saint-Gobain Ceramics and Plastics, Inc.},
\textit{Scintillating Optical Fibers}, It's What's Inside that Counts, 2005-14. \href{https://www.crystals.saint-gobain.com/products/scintillating-fiber}{\textbf{Data sheet}}. 

%Referencia 131
\bibitem{DataSheetPMTsAveiro} \textsc{HAMAMATSU PHOTONICS K.K.},
\textit{Photonmultiplier tube R2154-02 2"}. \href{https://www.hamamatsu.com/eu/en/product/type/R2154-02/index.html}{\textbf{Data sheet}}.

%Referencia 132
\bibitem{PowerSupplyAveiroDataSheet} \textsc{HAMAMATSU PHOTONICS K.K.},
\textit{High Voltage Power Supply C11152-01}. \href{https://www.hamamatsu.com/jp/en/product/type/C11152-01/index.html}{\textbf{Data sheet}}.

%Referencia 133
\bibitem{MAX5500DataSheet} \textsc{Maxim Integrated},
\textit{Low-Power, Quad, 12-Bit, Votlage-Output DACs with Serial Interface}. \href{https://www.maximintegrated.com/en/products/analog/data-converters/digital-to-analog-converters/MAX5500.html}{\textbf{Data sheet}}.

%Referencia 134
\bibitem{VetoAveiro} \textsc{Saint-Gobain},
\textit{Scintillating plastic grown with polymeric method}. \href{https://www.epic-crystal.com/others/plastic-scintillator.html}{\textbf{Data sheet}}.

%Referencia 135
\bibitem{CREMATPreAmplifierDataSheet} \textsc{CREMAT Inc.},
\textit{CR 111-R2.1 Charge sensitive preamplifier}. \href{https://www.cremat.com/home/charge-sensitive-preamplifiers/}{\textbf{Data sheet}}.

%Referencia 136
\bibitem{OPA656} \textsc{Texas Instruments},
\textit{OPA656 Wideband, Unity-Gain Stable, FET-Input Operational Amplifier}. \href{https://www.ti.com/product/OPA656}{\textbf{Data sheet}}.

%Referencia 137
\bibitem{LT111} \textsc{Linear Technology},
\textit{LT111A}. \href{https://datasheetspdf.com/pdf/57354/LinearTechnology/LT111/1}{\textbf{Data sheet}}.

%Referencia 138
\bibitem{Stretcher} \textsc{Texas Instruments},
\textit{SN74AHC1G32 Single 2-Input Positive-OR Gate}. \href{https://www.ti.com/product/SN74AHC1G32}{\textbf{Data sheet}}.

%Referencia 139
\bibitem{ANDGate} \textsc{Texas Instruments},
\textit{SN74LVC1G11DBVR Single 3-Input Positive-AND Gate}. \href{https://www.ti.com/store/ti/en/p/product/?p=SN74LVC1G11DBVR}{\textbf{Data sheet}}.

%Referencia 140
%\bibitem{Geant4WP} \textsc{CERN collaboration}, 
%\textit{Geant4, A simulation toolkit}. \href{https://geant4.web.cern.ch/node/1}{\textbf{Website}}.

%Referencia 141
\bibitem{Geant4P} \textsc{J. Allison}, 
\textit{Geant4 - A simulation toolkit}. \href{https://doi.org/10.1016/S0168-9002(03)01368-8}{\textbf{DOI:10.1016/S0168-9002(03)01368-8}}

%Referencia 142
\bibitem{CRYwebsite} \textsc{Plot Nuclear Data (NADS)}, 
\textit{Physics simulation packages, CRY (cosmic-ray particle showers)}. \href{https://nuclear.llnl.gov/simulation/}{\textbf{Website}}

%Referencia 143
\bibitem{CRYpaper} \textsc{Chris Hagmann}, \textsc{David Lange} and \textsc{Douglas Wright}, 
\textit{Cosmic-Ray particle Showers Generator (CRY) for Monte Carlo Transport Codes}, IEEE Nuclear Scinece Symposium conference record. Nuclear Scinece Symposium 2:1143-1146, January 2007. \href{https://www.researchgate.net/publication/4313740_Cosmic-ray_shower_generator_CRY_for_Monte_Carlo_transport_codes}{\textbf{DOI:10.1109/NSSMIC.2007.437209}}

%Referencia 144
\bibitem{WaterPropertiesSimulation} \textsc{H. Buiteveld}, \textsc{J.H.M. Hakvoort}, \textsc{M. Donze}, 
\textit{Optical properties of pure water, in:Proc. 2258 Ocean Optics XII, Bergen, Norway, 1994}. \href{https://www.spiedigitallibrary.org/conference-proceedings-of-spie/2258/1/Optical-properties-of-pure-water/10.1117/12.190060.short?SSO=1}{\textbf{DOI:10.1117/12.190060}}

%Referencia 145
\bibitem{TritiumEmissionSpectrum} \textsc{S. Maertens}, \textsc{et al.},  \textit{Sensitivity of next-generation tritium beta-decay experiments for keV-scale sterile neutrinos, J. Cosmol. Astropart. Phys. 2015 (2015) 020}. \href{https://iopscience.iop.org/article/10.1088/1475-7516/2015/02/020}\textbf{DOI:10.1088/1475-7516/2015/02/020}

%Referencia 146
\bibitem{NEMODataSimulation} \textsc{J. Argyriades}, \textsc{et al.},
\textit{Spectral modeling of scintillator for the NEMO-3 and SuperNEMO detectors, Nucl.Instrum. Methods A 625 (2011) 20-28}. \href{}{\textbf{}}

%Referencia 147
\bibitem{CAENV1724} \textsc{CAEN, Tools for Discovery},
\textit{CAEN V1724, 8 Channels, 14 bit, 100MS/s Digitalizer}. \href{https://www.caen.it/products/v1724/}{\textbf{Data sheet}}

%Referencia 147
\bibitem{CurieLimit} \textsc{Lloyd, A. Currie},
\textit{Limits for Qualitative Detection and QUantitative Determination. Application to Radiochemistry, Anal. Chem. 1968, 40, 3, 586-593, March 1, 1968}. \href{https://doi.org/10.1021/ac60259a007}{\textbf{DOI: 10.1021/ac60259a007}}

%~\cite{Ivo}
\end{thebibliography}

\end{document}