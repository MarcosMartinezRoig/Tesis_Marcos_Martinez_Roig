The lead shielding and active vetos, described above, were included in the simulation of the Tritium-IFIC 2 prototype. The objective of these simulations was to quantify the reduction of cosmic background detected by the prototype. For this task, the tritium source was replaced by the cosmic events source, which was simulated through the CRY library.

%As can be seen in Figures \ref{subfig:RealHardCosmicEvent} and \ref{subfig:FakeHardCosmicEvent}, two plastic scintillators were simulated with the dimensions mentioned in section \ref{subsec:SetUpActiveShield} and located above and below of the Tritium-IFIC 2 prototype simulated. 

The optical properties included to the plastic scintillators of the active veto are the refractive index, the light attenuation spectrum and energy emission spectrum, the values of which were obtained from their data sheet provided by the manufacturer \cite{ScintillatorVeto}. Two PMTs, model R8520-460 from Hamamatsu, were simulated to read each plastic scintillator, similar to that presented in section \ref{subsec:SetUpActiveShield}.

The lead shielding was simulated with properties taken from the Geant4 NIST database. The dimensions of the simulated lead shielding were $60 \times 60 \times 70~\cm^3$, which is the minimum needed to accomodate an active veto and Tritium detector module inside. The length of the simulated lead castle, $60~\cm$, is smaller than real dimension, $148~\cm$. The reason for this is that only one tritium detector module was simulated, so the dimension of the lead shielding can be reduced to optimize simulation time and computing resources.

Similar to the simulations used for the study of the TRITIUM-IFIC 2 prototype, analogous variables were used as tests, which were systematically verified, to ensure that all the steps of the simulations were carried out correctly.

Three different simulations were carried out to independently quantify how the tritium detection is affected due to both, the lead shield and the cosmic veto. The first simulation consists of a TRITIUM-IFIC 2 prototype and the cosmic ray source, in the second simulation a lead shield was added and for the third simulation, the cosmic veto was also included.

The cosmic events detected by the TRITIUM-IFIC 2 prototype are reduced around 5.5 times when a lead shielding with walls of $5~\cm$ is included. This reduction is mainly caused due to the stop of the weak cosmic radiation (energy lower than $200~\MeV$). It has to be taken into account that the natural backgrounds of the place are not included in this simulation. This radioactive background will be also stopped by the lead shielding, so the expected reduction of the radioactive background due to the passive veto is even better.

Regarding to the cosmic events that pass through the lead shield and reach the TRITIUM IFIC 2 prototype, which are the hard cosmic radiation, a percentage of around $10\%$ is detected with the cosmic veto and, therefore, removed to the tritium measurement.

Therefore, the usefulness of both parties of the background rejection system, lead shield and active veto, has been demonstrated by quantifying their effect.