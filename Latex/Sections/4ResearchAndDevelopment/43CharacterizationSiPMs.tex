This section describe the work that has been carried out to characterize the SiPMs used in the TRITIUM experiment, model S13360-6075 from Hammamtsu Photonics company. It will consist of measuring some of the most important parameters of this SiPM that will affect the tritium measurement such us their break down voltage or their gain.

Furthermore, a temperature compensation method has been developed and experimentally tested to compensate for temperature variations of the SiPM since it has been seen to greatly affect to its correct operation and, therefore, the measurement of tritium.

The setup used in this characterization is shown in section \ref{subsubsec:SiPMsElectronicalSystem}, Figure \ref{fig:PCBs_LEDSpectrum}. Furthermore, these measurements were performed inside of a climatic chamber whose temperature and humidity were controled with a precision of $0.1~\degree$ and $0.1\%$ respectively. We have to take into account, when the system was stabilized, variations with the same size of this precision were observed.

First of all we measured to breakdown voltage of the SiPM and their quenching resistance which was calculed from the measurement of their current-voltage curves. To measure it we need to work without amplification of the electronic system so, insatead to use the second and third PCB, we connected the output of the black box directly to the picoammeter previously mentioned \cite{DataSheetKeithley6487}. A LabView program was used to automate the taking of measurements.

On the one hand, we feed the SiPM using a forward bias voltage from $0~\volt$ to $1.58~\volt$ in steps of $0.005~\volt$, whose result is shown in Figure \ref{}.

FIGURAAA

EXPLICACIOON

On the other hand, we feed the SiPM using the reverse bias voltage...



the black box output was connected directly to the keithley to measure the current and using

This calibration will consist of two different parts. On the one hand, we measure the IV curves of the SiPM, from which we will calculate the breakdown voltage and the quenching resistance of the SiPM and, on the other hand, we will measure energy spectrums from which we will calculate the gain of the SiPM, the equivalent capacity of the SiPM, etc.

To measure the energy spectrums we will use the setup shown in section \ref{} and to measure the IV curves a small modification of this setup will be made which consists of... 




Pequeñas zonas de deplexión crean grandes capacidades que producen alto ruido de los SiPMs

Trabajo de Fernando Hueso.

Explicar electrónica que se utiliza como la tarjeta y demas. Poner el esquema electrónico de la tarjeta y referencia a Marc de NEXT por haberla construido. Explicar la forma del pulso como en la tesis de Karina Asnar

Paper de Nadia para la PDE.

Las medidas se han hecho en la camara del IFIMED (ver lo que tengo apuntado en el TFM y dar gracias al IFIMED)

la banda prohibida es pequeña por lo que algunos electrones pueden excitarse termicamente y pasar a la banda de conducción -> RUIDO
Cuando hable del ruido, afterpulses, crosstalk... intro en MPPC hammatsu data sheet

cuando hable de la capacidad del SiPM utilizar el punto 10.3.2 del Leo -> pag. 226

resumen de como varía cada magnitud con la temperatura y el voltaje. Tesis SiPMs.

Inocherent light source!

fired cells.


paper NADIA

B. Photon Detection Efficiency -> entero.

Aunque el aumentar el voltaje inverso mejora la eficiencia de detección de fotones, también aumenta la corriente oscura. Reducir la temperatura sin embargo disminuye la corriente oscura.

Para contar el número de veces que dos o más fotones son detectados simultáneamente, el umbral les establecido en N – 0.5 p.e. (donde N es un número arbitrario de fotones). Al contar el número de pulsos que exceden este umbral se puede saber el número de veces que se han detectado simultáneamente N o más fotones (Manual Hamamatsu, 2008, 2007)

En este trabajo se estudió tanto la variación en la resolución en energía de un SiPM como la variación del centroide de un pico (explicado a detalle en los próximos capítulos) con la temperatura y el voltaje. Para ello se estableció una electrónica de adquisición adecuada para la mayor eliminación de ruido posible y óptima resolución (explicada en el Capítulo 3).


Superponer 2 plots con el LED a distintos voltajes (2 o mas... probar varios voltajes a Vov recomendado y quedarnos con los mejores).

Figura 11 de la tesis de cristales monoliticos

La ganancia M, depende exponencialmente de la tensión de polarización inversa del dispositivo (Fig. 11, derecha). Sin embargo, en la región de operación de los APDs con M ~ 100, un cambio relativo de tensión de polarización corresponde a un cambio lineal en la ganancia, con pendientes típicas de 10 \%/V. Además, la temperatura debe estar debidamente estabilizada en un sistema con APDs ya que estos detectores sufren variaciones importantes con la temperatura, típicamente ~ 2-3 \%/ oC

Todo el apartado de SiPMs de esta tesis.

Cuando hablemos del PDE:
El número de celdas de un SiPM dependerá de la aplicación específica. Será lo suficientemente elevado para detectar la cantidad de fotones esperada pero sin exceder innecesariamente este valor, ya que cada celda necesita de espacio para las resistencias de quenching de cada APD y para la separación y aislamiento entre las diferentes celdas. Cuanto mayor sea el número de celdas, mayor será el espacio muerto y menor su eficiencia. Por el contrario, un número de celdas inferior con celdas de mayor tamaño, implica una alta eficiencia de detección de fotones (Photon Detection efficiency, PDE) pero un rango dinámico bajo. La PDE en un SiPM se define como su eficiencia cuántica por el ratio entre el área sensitiva y el área total del dispositivo, lo que se conoce como factor de llenado (fill factor) y se representa con “epsilon”, por la probabilidad de que un fotoelectrón comience un proceso de avalancha (Ec. 19). Aparte de la longitud de onda, 

Este parrafo justirca porque me quedo con el SiPM array de area de SiPM mayor... debido a la sensibilidad... punto 4.2 de la tesis de SiPM (LARGA)



the noise level scales with the area of the device. 




If each ionization process could be considered independent of the others, the fluctuations would then be described by a Poisson distribution where the variance (sigma 2 ) would be equal to the mean number of ionization electrons, N I . However, the fluctuations in the mean number of ionization electrons present a lower value, as predicted by Fano’s theory [66], being proportional to a factor F, known as the Fano Factor, which multiplies the mean primary ionization yield. 


Tiene un factor de amplificación interno que depende exclusivamente de las características de la union p-n y de la resistencia quenching