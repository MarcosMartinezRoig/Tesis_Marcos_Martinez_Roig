Legislaciones de todos los paises -> Internet, pestañas guardadas

Hablar de las distintas organizaciones que controlan el panorama. Hablar de la red e vigilancia radiologica ambiental, REA y REM y poner el mapa.

La legislación se basa en los principios del sistema de proteccion radiologica propuestos por la ICRP, que son justificacion, ALARA y dosis limites. Explicar bien estos principios y que es el ICRP.

Hablar de las distintas legislaciones y sus limites. Remarcar que la de EEUU es 7 veces menos restrictiva que la de España. Y la de Canada es todavía menos restrictiva debido al hecho que las Centrales nucleares funcionan con agua pesada e, irremediablemente, sus niveles de tritio son más altos. Centrarnos en la de españa. PONER LA TABLA DE LA DIRECTIVA DEL CONSEJO EURATOM.


PONER LOS GRÁFICOS MEIDDOS POR EL REM DE SU NIVEL NORMAL. ¿PORQUE ES IMPORTANTE MEDIRLO? (en concreto donde vamos a medirlo nosotros, es una central nuclear) poner graficos de antes i despues de la central nuclear de cofrentes y ver que efectivamente el tritio aumenta (en las aguas superficiales, que son las usadas por la central nuclear, no en las subterraneas). Ver como luego se diluye.



Due to that, it is important to develop a legislation which control the bla, bla y granticen la prevención de la incidencia de efectos biologicos la cual se presentará en la sección \ref{}. 

Por ello, para evitar la dispersión de material radiactivo de forma innecesaria en el medio ambiente se debe realizar un análisis de costes-beneficios que justifiquen, con beneficio neto positivo, el uso de elementos radiactivos en cualquier actividad tecnológica desarrollada por hombre.

Ver temas 6, 7 y 8 donde habla de NPP y el accidente de fukushima.


We have to take into account that the limit of the emission of each radioactive element depends on the government agency who manages it, and the regulation directives that is implemented in that place so, as a consequence, it is different in each country. For example, in Europe, these limits are fixed by the EURATOM Council Directive and the limit for tritium in drinking water, established in $2013$, is $A = 100~\becquerel/\liter$ \cite{100BqL}. In USA, these limits are set by the United States Environmental Protection Agency (U. S. EPA) and this limit for tritium in drinking water is $A = 20~\nano\curie/\liter = 740~\becquerel/\liter$ \cite{740BqL}, which was stablished in $2014$.