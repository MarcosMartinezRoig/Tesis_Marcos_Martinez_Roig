This section shows the last prototypes, TRITIUM-Aveiro 0 and TRITIUM-IFIC 2, in which the problems previously found are solved and a well-defined design is developed for them.

A different design was employed in both prototypes so that they can allow the reading of a large number of fibers arranged in a straight position with two photosensors in time coincidence, in a safe way.

Particular attention to tritium detection efficiency was paid in these prototypes, which is the reason why they use many more fibers than the preliminary prototypes and a time coincidence reading of these fibers is done using two photosensors.

Furthermore, the activity of the radioactive liquid source of tritium, which is used to fill these prototypes, is much lower than previous prototypes since the interesting parameter to be measured is their low detection level, LDL.

The design of both prototypes is very similar and their objective is to test the subtle effects caused by the little difference, such us the diameter of the fibers used, and to choose the one with the best results which will be included in the final design of the TRITIUM monitor.