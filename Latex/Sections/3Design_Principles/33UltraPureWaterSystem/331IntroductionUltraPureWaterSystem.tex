The aim of using an ultrapure water system is to condition the sample before the measurement. It is important for two reasons:

\begin{itemize}

\item{} On the one hand, it is important because, as we saw in section \ref{sec:TritiumProperties}, the mean free path of tritium electrons in water (our case) is around $5~\mu\meter$ and even less for solid materials like organic material.

For detecting this tritium decay, we need that the electron of its decay reaches the fiber, so we must keep our detector clean. If the analyzed water sample contains particles that can be deposited on the fibers of our detector, it can form a layer of matter, which prevents the tritium electrons from reaching the fibers, reducing the tritium detection efficiency until it becomes impossible to measure tritium.

\item{} On the other hand, we have to keep in mind that, as we will see in chapter \ref{chap:Prototypes}, the tritium monitor does not have any spectrometric capabilities that can be used to distinguish other radioactive elements from tritium. That means that, all the radioactive element included in the analyzed water sample will be computed as a tritium event.

With this system we can remove all particles up to a diameter of $1~\mu\meter$ and organic matter, which means that we remove all particles and molecules other than water. Since tritium is the only radioactive element that can be practically equal to water (when it is in the $\ce{HTO}$ form, the majority form in wihch tritium are present in the water sample), with this process we remove all particles radioactive elements other than tritium and the amount of tritium present in the sample is not affected

\end{itemize} 

In summary, with the ultrapure water system we get to keep our detector clean, ensuring the stability of its detection efficiency and we eliminate all radioactive particles other than tritium, maintaining the activity of the tritium in the sample, so we do not need any spectroscopic capabilities in our detector to distinguish radioactive elements. Both reasons has been tested with experimental measurements that will be shown in the chapter \ref{sec:CharacterizationUltraPureWaterSystem}.