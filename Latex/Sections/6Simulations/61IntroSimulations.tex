This chapter describes the Monte Carlo simulations performed in the TRITIUM experiment to optimize the design of the TRITIUM detector, understand its behaviour and investigate its limitations. These simulations are divided into two different sections. The first section contains the results of several simulations used to improve the design of the TRITIUM detector while the second exhibits the results obtained for the simulation of a full TRITIUM monitor based on an active veto read in anticoincidence with several TRITIUM-IFIC-2 prototypes. Furthermore, several tests were carried out to verify the correct simulation of the different steps such as the simulated tritium source, the energy deposition in the fibers and the production of photons in them. The simulation environment employed was Geant4 \cite{Geant4WebPage, Geant4P}.