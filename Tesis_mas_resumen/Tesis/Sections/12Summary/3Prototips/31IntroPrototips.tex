En la col·laboració TRITIUM, un total de quatre prototips del detector de triti van ser desenvolupats. Els dos primers, anomenats TRITIUM-IFIC-0 i TRITIUM-IFIC-1, són prototips a una escala menor. Aquests van ser utilitzats per aprendre sobre la detecció de triti en aigua i estudiar la viabilitat de possibles millores per optimitzar-la. Els dos darrers prototips, anomenats TRITIUM-Aveiro i TRITIUM-IFIC-2, són prototips funcionals amb un disseny similar. Aquests són utilitzats per avaluar la importància de certes diferències subtils en el disseny i elaborar un prototip final que incloga aquelles opcions que optimitzen l'eficiència en la detecció del triti. Durant el plenat d'aquests prototips amb aigua tritiada es van realitzar diverses proves d'estanqueïtat per garantir la radioseguretat de l'entorn. De forma complementària es van realitzar diverses simulacions amb la llibreria Geant 4 \cite{Geant4WebPage} que van ser utilitzades per obtenir més informació sobre quines possibles millores en els dissenys dels prototips incrementarien les característiques rellevants del monitor.