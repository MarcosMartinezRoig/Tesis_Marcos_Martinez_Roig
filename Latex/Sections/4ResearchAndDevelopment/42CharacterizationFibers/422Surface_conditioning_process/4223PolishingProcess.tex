As can be seen in Figure \ref{subfig:CleaveFiberEnd}, a slightly darkened zone at the bottom of the fiber is created, which is an inavoidable effect of the cleaving process. To remove that, a polishing process implemented by Thorlabs was applied \cite{DiamondThorlabs}. 

\textbf{Hand Polishing Method.}

The Thorlabs polishing method, shown in Figure \ref{fig:HandPolishingMethod}, consists on a kit based on a special fiber connector from Thorlabs which is used for rubbing the fibers during two minuts with five different polishing papers made out of aluminum  oxyde grain, with a decreasing grain size, $30~\mu\meter$, $20~\mu\meter$, $12~\mu\meter$, $5~\mu\meter$ and $0.3~\mu\meter$, describing on the the paper a shape of an 8 for 2 minuts (approximately 120 movements). 

\begin{figure}[h]
\centering
\includegraphics[scale=0.75]{4ResearchAndDevelopments/41Fibers/Hand_Polishing_Kit.png}
\caption{Hand polishing method implemented by Thorlabs.\label{fig:HandPolishingMethod}}
\end{figure}

The result obtained after polishing is shown in Figure \ref{subfig:PolishFiberEnd}, where it can be noted that the darkened zone has completely disappeared and the fiber end is completely clear without any damage or imperfection. Therefore, with this conditioning method, all the requirements imposed were fulfilled, obtaining fibers with optimal light transmission. 

\begin{figure}
\centering
    \begin{subfigure}[b]{0.5\textwidth}
    \centering
    \includegraphics[width=\textwidth]{4ResearchAndDevelopments/41Fibers/CutEndFiberGood.png}  
    \caption{\label{subfig:CleaveFiberEnd}}
    \end{subfigure}
    \hfill
    \begin{subfigure}[b]{0.45\textwidth}
    \centering
    \includegraphics[width=\textwidth]{4ResearchAndDevelopments/41Fibers/CutAndPolishedFiberEnd.png}  
    \caption{\label{subfig:PolishFiberEnd}}
    \end{subfigure}
 \caption{Result of the polishing process. a) Fiber end after cleaving b) Fiber end after cleaving and hand polishing with Thorlabs technique. Pictures taken with the microscope PB 4161 from EUROMEX.}
 \label{fig:ResultofPolishingProcess}
\end{figure}

\textbf{Automatic Polishing Machine.}

The issue with the polishing method implemented by Thorlabs is that it takes more than 10 minutes to polish each fiber, too long to polish the thousands of fibers needed for the TRITIUM monitor\footnote{Tritium monitor is composed of several modules based on hundred of scintillating fibers.} (see section \ref{sec:TritiumMonitor}). This is why an automated polishing process has been developed within this thesis work. The goal of this effort was to ensure a better light coupling and transmission of light of the scintillating fibers to the photosensors. 


%As mentioned above, tens of thousands of fibers had to be prepared and conditioned for the TRITIUM monitor\footnote{Tritium monitor is composed of several modules based on hundred of scintillating fibers.} (see section \ref{sec:TritiumMonitor}). Cleaving this large number of fibers with out home-made cleaving device was a relatively fast process. Polishing them however is quite time consuming, as it takes ten minutes to hand polish one fiber. Hand polishing thoudands of fibers,  would result in an unaffordable amount of time. This is why an automated polishing process has been developed within this thesis work. The goal of this effort was to ensure a better light coupling and transmission of light of the scintillating fibers to the photosensors. 

A machine was designed, built and tested for automatically polishing several plastic scintillating fibers at the same time. This polishing machine is easily scalable to larger number of fibers.

\begin{figure}[h]
\centering
\includegraphics[scale=0.75]{4ResearchAndDevelopments/41Fibers/GeneralViewPolishingMchine.png}
\caption{Polishing machine developed for TRITIUM.\label{fig:GeneralViewPolishingMachine}}
\end{figure}
This automatic polishing machine, displayed in Figure \ref{fig:GeneralViewPolishingMachine}, consists of two parts: 1) A polishing table, where the fibers are polished 2) The electronics, based on Arduino technology, that operates the movement of the polishing paper:

\begin{enumerate}
\item{} The polishing table, shown in Figure \ref{subfig:PolishingTable}, is divided in two parts: the static part, where the fibers are fixed, and the movable part on bottom of the previous one, where the polishing papers are fixed. It was decided to establish the polishing movement on the plate with the polishing sheets, because of its lighter weight and in order to avoid possible damaging movements to the fibers.

The static part (the fiber holder plate), shown in Figure \ref{subfig:PolishingTable}, consists of a plastic piece built with a 3D printer and locked to the system by four vertical screws. There are two nuts on each screw used to set the relative height and the inclination of fibers relative to the polishing papers. This piece contains one hundred holes in which the fibers are lodged. 

As the fibers are too light ($0.16~\gram$) to make by gravity the necessary pressure on the polishing paper, a plastic belt and a piece of metal with a weight of around $1.5~\gram$ were attached to the individual fibers, as shown in Figure \ref{subfig:FiberMetailcPiece}, to increase their contact pressure, in similar way as with the hand polishing connectors. 

The movable part consists of a flat PMMA plate of $18 \times 18~\cm^2$ to which the polishing paper is attached. This part is locked to two horizontal screws, perpendicular to each other that are used to set its position in the XY plane (horizontal plane), as shown in Figure \ref{subfig:HorizontalAxis}.

The polishing system contains several switches, model DB1 6A250, mounted on a piece made with a 3D printer, shown in Figures \ref{subfig:PolishingTable}, \ref{subfig:HorizontalAxis} and \ref{subfig:3DSwitchPiece}, which are used to find the origin of coordinates when the system is reinitiated and to stop the movable part when the end of the path is reached. 

\begin{figure}
\centering
    \begin{subfigure}[b]{0.55\textwidth}
    \centering
    \includegraphics[width=\textwidth]{4ResearchAndDevelopments/41Fibers/PolishingTable.png}  
    \caption{\label{subfig:PolishingTable}}
    \end{subfigure}
    \hfill
    \begin{subfigure}[b]{0.3\textwidth}
    \centering
    \includegraphics[width=\textwidth]{4ResearchAndDevelopments/41Fibers/PieceOfFiber.png}  
    \caption{\label{subfig:FiberMetailcPiece}}
    \end{subfigure}
    \hfill
    \begin{subfigure}[b]{0.55\textwidth}
    \centering
    \includegraphics[width=\textwidth]{4ResearchAndDevelopments/41Fibers/HorizontalAxis2.png}  
    \caption{\label{subfig:HorizontalAxis}}
    \end{subfigure}
    \hfill
    \begin{subfigure}[b]{0.4\textwidth}
    \centering
    \includegraphics[width=\textwidth]{4ResearchAndDevelopments/41Fibers/Switch.png}  
    \caption{\label{subfig:3DSwitchPiece}}
    \end{subfigure}
 \caption{Components of the fiber polishing machine. a) Polishing table. b) Fiber with metal piece. c) Horizontal screws and PMMA plate. d) A movement switch with its cables inserted inside its holding piece.}
 \label{fig:PolishingTable}
\end{figure}

\item{} The electronics which controls the automatic movement of the polishing paper, shown in Figure \ref{fig:ElectronicSystemPolishingMachine}, is based on Arduino technology.

This electronics consists of two stepper motors, model NEMA ST4209S1404-A \cite{StepperMotors}, which move the horizontal screws on which the PMMA plate holding the polishing paper is attached. These motors are controlled by an Arduino UNO \cite{ArduinoUNO} that uses a CNC shield \cite{CNCShield} in which two different drivers are connected to control the stepper motors, one driver for each stepper motor.

Drivers are controllers that allow to manage stepper motors in a simple way. It is very important to choose the correct controller for the system because the controller limits the supply power to the motors, avoiding damaging them. Instead of using the Pololu A4988 drivers \cite{A4988Driver}, which is one of the most widely used drivers, the first choice was the DRV8825 driver \cite{DRV8825Driver}. DRV8825 allows to power the motor with higher voltage and intensities ($45~\volt$ and $2.5~\ampere$) than A4988 ($35~\volt$ and $2~\ampere$). Also, the DRV8825 controller includes a new microstepping mode ($1/32$) compared to the A4988 ($1/16$) with which we get more accurate and smooth movements. Finally the drivers were replaced by the TMC2208 \cite{TMC2208Driver}, much less noisy since it includes the \textit{StealthChop} function with which the noise is practically eliminated. Furthermore, this controller is much more accurate owing to its a microstepping mode of $1/256$. The voltage and current used to power the motors are $35~\volt$ and $2~\ampere$ which are sufficient for the whole system since the current of the motors is limited to $1.33~\ampere$. The excess current will be transformed into heat that has to be dissipated from the system. Overheating of the drivers may cause loss of steps, producing wrong movements or even destroying the driver. Therefore, a cooling system is needed to ensure the correct operation of the polishing system. The cooling system, shown in Figure \ref{fig:ElectronicSystemPolishingMachine}, consists of a copper piece\footnote{The copper is one of the best thermal conductor at STP} in contact with both controllers and a fan, used to prevent heat accumulation inside the electronics box. The cooling power can be inproved by using a PELTIER cell.

\begin{figure}[h]
\centering
\includegraphics[scale=0.75]{4ResearchAndDevelopments/41Fibers/ElectronicPolishingMachine.png}
\caption{Electronic system of Polishing machine.\label{fig:ElectronicSystemPolishingMachine}}
\end{figure}

\end{enumerate}

This polishing machine is controlled by a Raspberry Pi computer board \cite{RaspberryPi} using the Universal G-code Sender software (a grafical interface based on the GRBL package). There are several useful pre-programmed functions such as "HOME" with which the system, using the switches, finds its origin of coordinate every time the system is turned on. The software also has the possibility of loading a file containing the G-code to be executed. In the TRITIUM project, the 120 movements required for each polishing paper are loaded in this way. 

\textbf{Experimental Test.}

Finally, this machine was tested with twenty fibers of $15~\cm$ length. This fibers, after polishing, were arranged in a bunch and fixed to the structure shown in Figure \ref{fig:BunchWith2PMTsCoincidence} and two PMTs, located at the bundle ends and read in coincidence, was used to measure the light transmision of the fibers.

\begin{figure}[]
\centering
\includegraphics[scale=0.6]{4ResearchAndDevelopments/41Fibers/FiberBunch2PMTsCoincidence.png}
\caption{Setup used to test the effect of the fiber polishing on light transmission to the PMTs.\label{fig:BunchWith2PMTsCoincidence}}
\end{figure}

The light transmission was measured before and after polishing. These measurements were carried out using two radioactive sources, a $\ce{^{60}Co}$ gamma source of $715~\becquerel$ activity, and a $\ce{^{90}Sr}$ beta source of $17.8~\kilo\becquerel$ activity. The radioactive sources were placed next to the fiber bundle, in the middle of it (at $7.5~\cm$ from each PMT) and the energy spectra was recorded for both radioactive sources, which are exhibited in Figure \ref{fig:ResultsOfPolishingMachine}.

\begin{figure}
\centering
    \begin{subfigure}[b]{1\textwidth}
    \centering
    \includegraphics[width=\textwidth]{4ResearchAndDevelopments/41Fibers/Co_60_PolishingMachine_ZOOM.pdf}  
    \caption{\label{subfig:EnergySpectrumCo60PolishingTest}}
    \end{subfigure}
    \hfill
    \begin{subfigure}[b]{1\textwidth}
    \centering
    \includegraphics[width=\textwidth]{4ResearchAndDevelopments/41Fibers/Sr_90_PolishingMchine_ZOOM.pdf}  
    \caption{\label{subfig:EnergySpectrumSr90PolishingTest}}
    \end{subfigure}
 \caption{Energy spectra recorded with polished and unpolished fibers. a) for the $\ce{^{60}Co}$ source b) for the $\ce{^{90}Sr}$ source}
 \label{fig:ResultsOfPolishingMachine}
\end{figure}

As it can be seen in Figure \ref{fig:ResultsOfPolishingMachine}, both energy spectra are shifted to the right after polishing, which means that the detected events have more energy (more photons per event reach the PMTs). This energy increase was more than 40\% ($(42 \pm 4.6)\%$ for gamma source and $(49 \pm 8.4)\%$ for beta source) with respect to the unpolished fibers. In summary, with the polishing machine, the photon collection efficiency of the fibers was improved  (mainly due to the improvement of the interface between fibers and PMTs). It is very important to achieve a high detection efficiency as the expected number of photons per tritium event is quite low, less than 20 as it has been demostrated with simulations and experimental measurements.