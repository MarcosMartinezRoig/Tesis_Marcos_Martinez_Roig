So far the scintillating photons has been created in the core of the fiber, which have been guided to its ends. Now, the so-called photosensor is needed, which is an element that is able to detect these scintillating photons. Photosensors have a sensitive part that is optimized to detect photons in a range of energy (normally in the visible range) with certain probability (efficiency). After that, the photosensors create an electronic signal that carries information about these photons detected such as their number or their detection time.

There are a lot of different photosensors that can be used for this purpose, the photon detection of which relies on totally different physical processes, such as photoelectron multiplier tubs (PMTs), silicon photoelectron multiplier (SiPM) or charge-coupled device (CCD).  Each one of these will have different properties and it has to be chosen the one which fit better for the objective of the experiment.

One of the most important things to optimize the efficiency of a scintillation detector is that the emission spectrum of the scintillator (Figure \ref{fig:EmissionSpectrumFibers} for the fibers used) overlaps as much as possible with the detection efficiency spectrum of the photosensor chosen, specifically their higher peaks. The detection efficiency spectrum is a way of expressing the probability of detecting photons at several wavelength. In this case, the efficiency of this detector,  which is poroporcional to the multiplication of both factors at the same photon energy, will be the largest.

The main proposal of TRITIUM will be to use SiPM arrays because they are very fast (of the order of $~\nano\second$) and have high photodetection efficiency (a maximum of around $50\%$) and high gains (multiplication faction of $10^{6}$) with a low voltage supply. On top of that, one of the most important reason of this choice is that SiPM arrays are able to detect a single photon with high efficiency, which is very important due to the low amount of photons generated by tritium event, shown in section \ref{subsec:PlasticScintillators}. The PMTs, which are the conventional choice, will be also tested because they are still interesting since they have lower dark count rate than an equivalent SiPM and some similar properties like its gain or its fast signals.



%A certain portion (in an optimal case nearly 100\%) of the scintillation photons reach the light detector, which has to be sensitive enough to detect a small number of photons. The detector then produces a signal pulse, which has a height proportional to the number of photons hitting the detector. The signal pulse of the detector is processed by the electronics, and as a result a pulse height spectrum is produced (see Section 3.5).

%This spectrum corresponds to the energy spectrum of the detected particles.
