This section shows the characterization of the ultrapure water system, whose objective is to ensure that the quality of the output water sample of the ultrapure water system, purified water from now on, is good enough to overcome the requirements of the TRITIUM detector. There are three different requirements that this ultrapure water system must meet:

\begin{itemize}
\item{} First of all, we need to achieve a low enough conductivity\footnote{Conductivity is the ability of the material to conduct electrical current. In liquids, conductivity is related to the presence of salts (presence of positive and negative ions)} of the water, around $10~\mu\sievert/\cm$, so that external particles disolved in the water don't be deposited on the fibers, drastically reducing the detector efficiency due to such a low mean free path of the tritium electron.

\item{} Second, we must note that this system does not have any spectral capabilities that can be used to distinguish between several radioactive isotopes, so to measure only tritium, this system must remove all external radioactive particles (other that tritium isotope) from the sample.

\item{} Lastly, the tritium activity should not be affected by this process. 

\end{itemize}

To verify that these requirements have been exceeded, a characterization of the water sample was carried out before and after the ultrapure water system, called raw water and purified water respectively, which consists of measuring the water sample conductivity and the activity of each radioactive element that are present in the sample. The turbidity and the chemical components of the water sample will also be measured.

It must be taken into account that, so that the sample was representative of the raw water sample, it was taken from a tap located very close to the input of the ultrapure water system, which is located at 40 meters from the TRITIUM monitor and two meters deep. It has been seen that it is very important to take the sample where we introduce the sample to the ultrapure water system since variations of up to $25\%$ in the tritium activity has been measured between both points (due to its diffusion).

First, the chemical components of the water were measured before the ultra-purification process by a physico-chemical analysis, which was carried out a few years ago. It is shown in Table \ref{tab:ChemicalComponentsRawWater}.

\begin{table}[htbp]
%%\centering
\begin{center}
\begin{tabular}{|c|c|}
\hline
Chemical components & Concentration ($\milli\gram/\liter$)\\
\hline \hline \hline
$\ce{CO_{3}H^-}$ & $154$ \\ \hline
$\ce{Mg}$ & $46$ \\ \hline
$\ce{Ca}$ & $105$ \\ \hline
$\ce{NO_{3}^-}$ & $16$ \\ \hline
$\ce{Cl^-}$ & $196$ \\ \hline
$\ce{NO_{2}^-}$ & $0.03$ \\ \hline
$\ce{K}$ & $11$ \\ \hline
$\ce{Na}$ & $173$ \\ \hline
$\ce{SO_{4}^-}$ & $217$ \\ \hline
Dry Residue & $1029$ \\ \hline
\end{tabular}
\caption{Chemical components and turbidity measured in the raw water sample.}
\label{tab:ChemicalComponentsRawWater}
\end{center}
\end{table}

EXPLICACIÓN DE ESTA. SE SUPONE QUE LA AGUA PURIFICADA NO POSEE NADA DE ESTO???

Its turbidity\footnote{The turbidity of water is the loss of its transparency due to dissolved particles, normally measured in NTU, Nephelometric Units of Turbidity, which measure the intensity of the scattered light at 90 degrees.} was also measured using the Hanna Hi 9829 portable multiparameter system from Hanna Instruments \cite{TurbiditySystem}, obtaining a value of $29$ NTU.

Second, the conductivity was measured for both, raw and purified water. To do so, the same multiparameter system was used, the Hanna Hi 9829. These measurements, together with the measurement of the conductivity of reject water(explained in section \ref{subsec:SetUpWaterSystem}), are presented in Table \ref{tab:ConductivityValues}.

\begin{table}[htbp]
%%\centering
\begin{center}
\begin{tabular}{|c|c|c|c|}
\hline
Date & Raw ($\mu\sievert/\cm$) & Pure ($\mu\sievert/\cm$) & Reject ($\mu\sievert/\cm$) \\
\hline \hline \hline
$1/8/18$ & $970$ & $11.85$ & $1442$ \\ \hline
$7/8/18$ & $958$ & $11.8$ & $1632$ \\ \hline
$14/8/18$ & $966$ & $12.04$ & $1725$ \\ \hline
$22/8/18$ & $980$ & $12.54$ & $1702$ \\ \hline
$28/8/18$ & $987$ & $9.9$ & $1692$ \\ \hline
$5/9/18$ & $1009$ & $12.02$ & $1645$ \\ \hline
\end{tabular}
\caption{Measurements of the conductivity for several samples of each water type (raw water, pure water and reject water).}
\label{tab:ConductivityValues}
\end{center}
\end{table}	

As can be seen in the first column, we have too high values of conductivity of raw water to be introduced directly into the tritium detector. The reason of that is because this type of water contains many different ions(saw in Table \ref{tab:ChemicalComponentsRawWater}) that we have to eliminate to prevent them from depositing on the fibers. 

We can see in the second column that the conductivity values of pure water have been reduced by almost two orders of magnitude, reaching values close to $10~\mu\sievert/\cm$, our initial objective, exceeding this requirement.

Finally, as can be checked in the third column, the reject water has high values of conductivity. The reason of that is because it contains the ions that has been removed to the pure water.

Third, each different radioactive element present in the water samples of both types, raw water and purified water, was identified and their activities was measured. For this task, a high purity germanium detector, HPGe, was used to measure this samples and a gamma analysis was carried out to find the natural gamma emitters (those that come from the natural radioactive series, Table \ref{tab:NaturalRadioactiveSeries}) and artificial gamma emitters with long enough lifetime to be measured (those that come from the activation of nuclear fission of neutrons).

The energy spectrum of these measurements is presented in Figures ... where the different elements can be identified and their activities are shown in Table...

FIGURAAAAS y tablaaa

As can be seen in Figures ... we have several radioactive elements in the raw water sample whose activities are not negligible compared to the activity of the tritium we intend to measure $100~\becquerel/\liter$. Therefore, as we have said, it is very important to remove these elements to the water sample as it will affect the tritium measurement.

As we can see in Figure \ref{}, these radioactive elements have been completely eliminated after the ultrapurification process leaving only tritium in the water sample, achieving to exceed other requirements for the ultrapure water system.

Lastly, the tritium activity was measured to see how it is affected by the ultra-purification process. This measurement was carried out using the Quantulus system, which consists of a liquid scintillator, which has been mixed with tritium water, readout by PMTs. 

Before this measurement, each water sample was filtered at 0.45 microns to remove any particles that would affect the extinction of the scintillation signal.

The Table \ref{tab:ActivityTritiumValues} show several measurements of the activity for different tritium samples of each water types (raw water, reject water and purified water).

\begin{table}[htbp]
%%\centering
\begin{center}
\begin{tabular}{|c|c|c|}
\hline
Date & Raw ($\becquerel/\liter$) & Pure ($\becquerel/\liter$) \\
\hline \hline \hline
$7/8/18$ & $24 \pm 3$ & $26 \pm 4$ \\ \hline
$11/12/19$ & $13.2 \pm 2.1$ & $13.85 \pm 2.2$ \\ \hline
$15/01/20$ & $30.6 \pm 4.2$ & $30 \pm 4$ \\ \hline
\end{tabular}
\caption{Measurements of the activity for several samples of both water types (raw water and pure water).}
\label{tab:ActivityTritiumValues}
\end{center}
\end{table}	

As can be see, tritium activity is not affected by this system, exceeding the last requirement of the ultrapure system. We have checked that the tritium activity of reject water is also the same.