The scintillating photons created in the core of the fiber and guided to its end are detected by photosensors. Photosensors have a sensitive part that is optimized to detect photons in a range of energy (usually in the visible range) with a certain probability, called quantum efficiency. The photosensors produce an electronic signal that carries information about the detected photons such as their number, detection time, etc. There are many available photosensors that rely on various physical processes, such as photoelectron multiplier tubes (PMTs), silicon photoelectron multipliers (SiPM) or charge-coupled devices (CCD).  %Each one of these will have different properties and it has to be chosen the one which fit better for the objective of the experiment.

The optimization of the efficiency of a scintillation detector is essential. To do so, the emission spectrum of the scintillator (Figure \ref{fig:EmissionSpectrumFibers} for the fibers used) must overlap as much as possible with the detection efficiency spectrum of the photosensor chosen. The detection efficiency spectrum gives the probability of detecting photons as a function of wavelength. The efficiency of a detector is proportional to the product of both, the emission and the detection efficiency spectra, and this is largest when both spectra match.

The proposal of TRITIUM is to use SiPM arrays because they are very fast (of the order of $\nano\second$) and have a high photodetection efficiency of about $50\%$, a high gain (multiplication factor of $10^{6}$) and need a low voltage supply. The most important reason of this choice is that SiPM arrays are able to detect a single photon with high efficiency, which is a fundamental aspect due to the low amount of photons generated by tritium decay. The PMTs, which are the conventional choice, were also tested because they have lower dark count rate than SiPM and similar properties like gain and timing.



%A certain portion (in an optimal case nearly 100\%) of the scintillation photons reach the light detector, which has to be sensitive enough to detect a small number of photons. The detector then produces a signal pulse, which has a height proportional to the number of photons hitting the detector. The signal pulse of the detector is processed by the electronics, and as a result a pulse height spectrum is produced (see Section 3.5).

%This spectrum corresponds to the energy spectrum of the detected particles.
