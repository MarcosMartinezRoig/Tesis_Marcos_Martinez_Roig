The aim of the background rejection system is to reduce the TRITIUM radioactive background. The TRITIUM project follows the ALARA principle for the tritium activity measurement, that is, to measure tritium activity "as low as reasonably achievable". The detection limit of tritium activity is set by the uncertainty in the activity of the radioactive background, since tritium activities below this uncertainty cannot be distinguished from the background. Therefore, the background uncertainty must be reduced as much as possible. The total uncertainty is the quadratic sum of all the different uncertainties related to the measurement, i. e., the statistical uncertainty\footnote{Uncertainty due to the statistical nature of the radioactivity process}, $\sigma_{st}$, the systematic uncertainty\footnote{uncertainty due to the manufacture of the detectors}, $\sigma_{si}$, etc. The background rejection system of TRITIUM monitor minimizes the statistical component. Because of the Poissonian nature of the process, the statistical uncertainty is given by the square root of the measured activity, $A_{m}$, which can be reduced by minimizing detected background events.

\begin{equation}
\sigma_{T}^2 = \sigma_{st}^2 +\sigma_{si}^2; \qquad \qquad \sigma_{st;bak} = \sqrt{A_{m;bak}}
\label{eq:SquareSumUncerainty}
\end{equation} 

The background of TRITIUM is due to natural radioactivity and has two different sources. On the one hand, radioactive elements that are present in the crust of the Earth, mainly $\ce{^{40}K}$ and elements from the four different natural radioactive series, shown in Table \ref{tab:NaturalRadioactiveSeries}. On the other hand, the cosmic ray radiation. The primary cosmic radiation is composed of high-energy particles, mainly protons and $\alpha$ particles, but, after interacting with the Earth's atmosphere, they generate a shower mainly composed by muons, electrons, photons and neutrons.

\begin{table}[htbp]
\centering{}%
\begin{tabular}{lcccc}
\toprule 
Mass Num. & Series & Prim. el. & Half life (y) & Final isotope \tabularnewline
\midrule
\midrule 
4n & Thorium & $\ce{^{232}Th}$ & $1.41 \cdot{} 10^{10}$ & $\ce{^{208}Pb}$ \tabularnewline
4n+1 & Neptunium & $\ce{^{237}Np}$ & $2.14 \cdot{} 10^{6}$ & $\ce{^{209}Pb}$ \tabularnewline
4n+2 & Uranium-Radium & $\ce{^{238}U}$ & $4.51 \cdot{} 10^{9}$ & $\ce{^{206}Pb}$ \tabularnewline
4n+3 & Uranium-Actinium & $\ce{^{235}U}$ & $7.18 \cdot{} 10^{8}$ & $\ce{^{204}Pb}$ \tabularnewline
\bottomrule
\end{tabular}
\caption{Classification of natural radioactive series \cite{NaturalRadioactiveSeries1, NaturalRadioactiveSeries2}.}
\label{tab:NaturalRadioactiveSeries}
\end{table}
Cosmic radiation depends on several parameter like the longitude, latitude, latitude and the solar activity cycle. The spatial distribution of cosmic rays, mainly muons, follows a $cos^2(\theta)$ distribution with the zenith angle. 

To remove the background two different techniques are employed for the weak and the hard radiation,

\begin{enumerate}

\item{}  The weak radiation, which is any radiation with energy below $200~\MeV/$nucleon, is stopped by a lead castle, described in section \ref{subsec:SetUpPassiveShield},

\item{} The hard radiation, that is any radiation of energy greater than $200~\MeV/$nucleon, is much more difficult to stop and the technique employed is a cosmic veto in anti-coincidence with the TRITIUM detector, reported in section \ref{subsec:SetUpActiveShield}. %that is, we will save the measured tritium event just when we don't measure any hard cosmic event in time coincidence.

\end{enumerate}