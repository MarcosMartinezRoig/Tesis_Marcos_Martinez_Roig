The different prototypes developed in the framework of the TRITIUM project are described in this chapter. They are named TRITIUM-IFIC-0, TRITIUM-IFIC-1, TRITIUM Aveiro and TRITIUM-IFIC-2, listed in chronological order of their construction. The first two prototypes, TRITIUM-IFIC-0 and TRITIUM-IFIC-1, are preliminary prototypes used to learn about tritium detection and to improve the monitor design. The other two prototypes, TRITIUM-Aveiro and TRITIUM-IFIC-2, are prototypes with an optimized design, based on the lessons learned from the former prototypes. Each prototype was designed and built in the laboratories of IFIC or Aveiro and was filled with tritiated water following a method specially developed for this task. Several water tightness and filling tests were carried out for each prototype. The measurements obtained by the different prototypes are discussed in this chapter. The laboratories involved in the characterization of the prototypes are the Valencia IFIC, the Aveiro DRIM and the Extremadura LARUEX\footnote{LARUEX, Laboratorio de Radiactividad Ambiental de la Universidad de Extremadura (Environmental Radioactivity Laboratory of the University of Extremadura)}. An additional section shows the measurements obtained at the Arrocampo dam, the TRITIUM monitor installation site, where the control of external atmospheric conditions is less accurate. At the end of the chapter, the final monitor of the TRITIUM detector is described.