0Basado en la tesis de NEXT y completar con cosas del TFG de Andrea Chana y la tesis larga.

Me quedo con el SiPM array de area de SiPM mayor debido a la sensibilidad... punto 4.2 de la tesis de SiPM (LARGA).

resumen de como varía cada magnitud con la temperatura y el voltaje. Tesis SiPMs.

tabla con las propiedades del SiPM.

Hablar de la electrónica como en el tema 5 de la tesis larga.

Inocherent light source!

fired cells.

Trabajo de Fernando Hueso.

Con la serie S13360 de SiPMs (la nuestra) -> The MPPCs inherits the superb low afterpulse characteristics of previous products and further provide lower crosstalk and lower dark count.

Describir un APD -> Tesis NEXT.
Describir un SiPM -> Tesis NEXT, completar con lo de Andrea y la tesis larga.
Descrpción de SiPMs array. Data sheet 

SiPMs are small solid state photosensors with signal levels similar to PMTs and excellent photon counting capabilities. Their use as imaging cells in NEXT-100 has moreover the advantage of providing an optimal spatial resolution (active area down to 1 mm 2 ) at a moderate cost and a reduced radioactivity budget (micro Bq/kg level for the 238 U and 232 Th radioactive chains [21]). SiPMs have an optimal sensitivity in the blue spectral region (around 440nm).

B. Photon Detection Efficiency -> entero.


Algunas de estas ventajas son: inmunidad a campos magnéticos, facilidad de uso (simpleza de resultados generados), tamaño compacto, buena linealidad, electrónica simple, alta ganancia a bajo voltaje de polarización, alta eficiencia de detección de fotones (Photon detection efficiency), tiempo de respuesta corto, alto ritmo de conteo (count rate), buena resolución temporal, y un amplio rango de respuesta espectral (MANUAL MPPC HAMAMATSU, España et al., 2009, Ramili et al, 2012). Se ha comprobado que estos aparatos tienen buen desempeño bajo campos magnéticos de 0 a 7 Tesla, al igual que su conveniencia para adquisiciones tipo PET bajo la presencia de gradientes cambiantes y secuencias de radiofrecuencia de MRI (España et al., 2009, p. 1)

Aunque el aumentar el voltaje inverso mejora la eficiencia de detección de fotones, también aumenta la corriente oscura. Reducir la temperatura sin embargo disminuye la corriente oscura.

Para contar el número de veces que dos o más fotones son detectados simultáneamente, el umbral les establecido en N – 0.5 p.e. (donde N es un número arbitrario de fotones). Al contar el número de pulsos que exceden este umbral se puede saber el número de veces que se han detectado simultáneamente N o más fotones (Manual Hamamatsu, 2008, 2007)

En este trabajo se estudió tanto la variación en la resolución en energía de un SiPM como la variación del centroide de un pico (explicado a detalle en los próximos capítulos) con la temperatura y el voltaje. Para ello se estableció una electrónica de adquisición adecuada para la mayor eliminación de ruido posible y óptima resolución (explicada en el Capítulo 3).


Superponer 2 plots con el LED a distintos voltajes (2 o mas... probar varios voltajes a Vov recomendado y quedarnos con los mejores).

Intro 1.1 de la tesis "Uso de fotomultiplicadores de silicio para medidas der alta velocida y baja intensidad luminosa".

"Solid state technologies have evolved a lot in recent decades and the development of new high sensitivity semiconductor devices is a reality nowadays. In fact, current Silicon Photomultipliers (SiPMs) are replacing PMTs in many fields, and probably, they will not have rival in a next future."

Basarse en la memoria de Andrea Chana y completar

Un fotomultiplicador es un aparato que convierte señales luminosas en señales eléctricas.

poner PDE en la lista de nomenclatura

poner APD en nomenclatura

Durante los últimos años se han introducido fotodetectores de estado sólido compatibles con altos campos magnéticos: APDs y más recientemente SiPMs. Esta tendencia de uso de detectores compactos, unido al aumento del número de canales en los detectores, ha hecho imprescindible el desarrollo de electrónica integrada en PET. La arquitectura electrónica típica para PET se basa en la digitalización y lectura de la carga que proporcionan los fotosensores, para lo cual se realiza un proceso de amplificación de bajo ruido y adaptación de señal al digitalizador. Sin embargo, existen hoy en día otras formas de detección alternativas, como esquemas time over threshold (ToT) o digitalizadores directos de formas de onda.

Los detectores basados en semiconductores, que comprenden a los APDs y a los SiPMs, son insensibles a los campos magnéticos y actualmente ofrecen una alternativa a la opción clásica basada en PMTs. Ambos dispositivos generan portadores libres debido al efecto fotoeléctrico, produciendo una corriente eléctrica proporcional a la intensidad de la luz incidente. Cuando los fotodetectores son capaces de multiplicar fotoelectrones primarios, se pueden producir corrientes eléctricas grandes incluso en el caso de que incida un único fotón.

Figura 11 de la tesis de cristales monoliticos

Este proceso multiplicativo en un APD también produce un ruido estocástico inherente que no está presente en un fotodiodo convencional y que se conoce con el nombre de factor de ruido en exceso (F). Este ruido aumenta según lo hace la ganancia M del APD (Ec.15), lo que obliga a operar con una ganancia interna moderada, ya que este ruido en exceso es uno de los principales factores limitantes con estos dispositivos.

La ganancia M, depende exponencialmente de la tensión de polarización inversa del dispositivo (Fig. 11, derecha). Sin embargo, en la región de operación de los APDs con M ~ 100, un cambio relativo de tensión de polarización corresponde a un cambio lineal en la ganancia, con pendientes típicas de 10 \%/V. Además, la temperatura debe estar debidamente estabilizada en un sistema con APDs ya que estos detectores sufren variaciones importantes con la temperatura, típicamente ~ 2-3 \%/ oC

Todo el apartado de SiPMs de esta tesis. Lo de antes era de APDs.

IMPORTANTE PARA JUSTIFICAR EL SIPM ARRAY ESCOGIDO -> 4X4:

El número de celdas de un SiPM dependerá de la aplicación específica. Será lo suficientemente elevado para detectar la cantidad de fotones esperada pero sin exceder innecesariamente este valor, ya que cada celda necesita de espacio para las resistencias de quenching de cada APD y para la separación y aislamiento entre las diferentes celdas. Cuanto mayor sea el número de celdas, mayor será el espacio muerto y menor su eficiencia. Por el contrario, un número de celdas inferior con celdas de mayor tamaño, implica una alta eficiencia de detección de fotones (Photon Detection efficiency, PDE) pero un rango dinámico bajo. La PDE en un SiPM se define como su eficiencia cuántica por el ratio entre el área sensitiva y el área total del dispositivo, lo que se conoce como factor de llenado (fill factor) y se representa con “epsilon”, por la probabilidad de que un fotoelectrón comience un proceso de avalancha (Ec. 19). Aparte de la longitud de onda, 


señal lineal con la luz entrante.

MPPC Multi-Pixel Photon Counter

Linear response with the incoming photons.

Ver apuntes generales comentados de photosensores en la intro de photosensores

Los tubos fotodetectores (PMT) son los dispositivos adecuados para esto, aunque, sin embargo, los avances tecnologicos de las ultimas decadas en tecnologia de semiconductores ha permitido el desarrollo de los fotodiodos de avalancha operados en modo Geiger (APD, Avalanche PhotoDiode)

Actualmente, la tecnologia de semiconductores ha permitido el desarrollo de fotodiodos que, por sus prestaciones, pueden reemplazar adecuadamente a los PMT. 

4.5. Recent trends on SiPMs
The size of commercially available SiPMs is currently limited to approximately 10 mm 2 [164] Roncali-2011 . The restricted are is dictated both by manufacturing and noise considerations. 

Most of the devices produced up today have an area of 1 x 1 mm 2 . SiPMs of 3 x 3 mm 2 and 4 x 4 mm 2 have been also produced, but their use is limited since the noise level scales with the area of the device. 2D matrixes of 1 x 1 mm 2 SiPMs devices on the same substrate are suitable. Such devices increase the detection area maintaining on each read-out channel a noise level corresponding to a SiPM of 1 mm 2 and providing at the same time 2D position information with a spatial resolution of 1 mm 2 [165] Dinu-2009 .


produced in the process. If each ionization process could be considered independent of the others, the fluctuations would then be described by a Poisson distribution where the variance (sigma 2 ) would be equal to the mean number of ionization electrons, N I . However, the fluctuations in the mean number of ionization electrons present a lower value, as predicted by Fano’s theory [66], being proportional to a factor F, known as the Fano Factor, which multiplies the mean primary ionization yield. 

NEXT:
To shield NEXT-100 from the remanent external flux of high-energy gamma rays, a lead castle structure has been designed and built (see Figure 3.9). The lead wall has a thickness of 20 cm and is made of layers of staggered lead bricks held with a steel structure. The lead bricks have standard dimensions (200 x 100 x 50 mm 3 ), and by requirement, an activity in uranium and thorium lower than 0.4 mBq/kg. The lead castle, with a total weight of 60 tons,




Por su parte, el fotomultiplicador de Silicio (SiPM en adelante), desarrollado a finales del pasado milenio, acomete finalmente con éxito el reto de detectar, medir temporalmente y cuantificar señales poco luminosas hasta el nivel de un solo fotón. Los SiPM ofrecen una alternativa altamente atractiva que replica minuciosamente las propiedades de detección en condiciones de baja luminosidad de los PMT, proporcionando al mismo tiempo todos y cada uno de los beneficios de los detectores de estado sólido. Un SiPM ofrece la opción de trabajar con bajos voltajes de operación (menos de
100 V normalmente), posee insensibilidad a los campos magnéticos, destaca por su robustez mecánica y excelente uniformidad de la respuesta y es además muy compacto en dimensiones.