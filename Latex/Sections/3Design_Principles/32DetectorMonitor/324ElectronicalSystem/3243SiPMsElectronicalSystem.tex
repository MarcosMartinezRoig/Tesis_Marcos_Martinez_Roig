The SiPMs in the TRITIUM experiment are arranged in matrices of $4\times 4$. The electronic system chosen to process and analyze the output signals of the SiPM arrays is PETsys \cite{PETSYS}, displayed in Figure \ref{fig:PETSYS}, which is a commercial system prepared to work with SiPM matrices from Hamamatsu. PETsys provides time and energy digitalization, including the charge integrations QDCs\footnote{charge-to-digital converter} and TDCs\footnote{time-to-digital converter}, resulting in a complete acquisition and digitization system capable of working with up to 1024 SiPM. This system consists of a basic board to which 16 different SiPM matrices can be connected with up to 64 SiPM per matrix. This number of channels is needed in the TRITIUM project because, as shown in section \ref{sec:TritiumMonitor}, the TRITIUM monitor consists of a large number of SiPM matrices with 16 channels per matrix.

\begin{figure}[h]
\centering
\includegraphics[scale=0.8]{3DesignPrinciples/32Tritium_detector/PETSYS_System.png}
\caption{Different parts of PETsys system\label{fig:PETSYS}~\cite{PETSYS}.}
\end{figure}
Although the capacity provided by PETsys should be enough for the requirements of the TRITIUM project, TRITIUM is a modular detector with scalable sensitivity. This means that, if an inprovement of TRITIUM limits is needed to improve its sensitivity or to further reduce the background, more photosensors would be needed. Therefore, the electronics should be able to increase its capacity in a scalable way. This requeriment is fulfilled by PETsys since it has an additional module, called Clock and Trigger, to which up to sixteen different PETsys basic boards can be connected. Theses sixteen PETsys basic boards are read in parallel, giving a total system capacity of reading 256 SiPM matrices (16384 SiPMs\footnote{$1024\cdot{}16 = 16384$}). 

PETsys software is based on C++ and Python scripts to drive the main tasks required, such as time coincidence options between SiPM (or even SiPM matrices) or energy discrimination. This software is open source, giving the possibility to modify the current scripts or to develop others with additional functions. PETsys has a time resolution better than $30~\pico\second$ which is one of the best time resolutions of commercial systems available and its price is around $10$\euro$/$ channel, which is cheaper than similar electronic systems.

As reported in section \ref{sec:CharacterizationSiPM}, the SiPM matrix temperature is an important parameter. The PETsys system has the ability to monitor the temperature of the SiPM matrices and ASICS employed to control them. Temperature monitoring is important to ensure the correct functioning of both photosensors and system. PETsys has the possibility of developing new scripts to implement the stabilization method of the SiPM gain reported in section \ref{sec:CharacterizationSiPM}.

Some characterization measurements were carried out using the PETsys system to ensure that the system works properly but the SiPM characterization was carried out at the level of a single channel (individual SiPM). The reason is that the output information of PETsys is already integrated and digitized, so it does not allow the SiPM to be calibrated. Therefore, to characterize a SiPM, a different electronic system was used to read up to eight different SiPMs. This system consists of a PCB\footnote{PCB, Printed Circuit Board} that provides the SiPM bias voltage and reads the SiPM output signal. An example of the electronic scheme (provided by Hamamatsu) in which this PCB is based is shown in Figure \ref{fig:PCBSiPM}.

\begin{figure}
\centering
    \begin{subfigure}[b]{0.5\textwidth}
    \centering
    \includegraphics[width=\textwidth]{3DesignPrinciples/32Tritium_detector/SiPMPCB.png}  
    \caption{\label{subfig:ElectronicBoardSiPM}}
    \end{subfigure}
    \hfill
    \begin{subfigure}[b]{0.45\textwidth}
    \centering
    \includegraphics[width=\textwidth]{3DesignPrinciples/32Tritium_detector/ElectronicSchemePCBSiPM.png}  
    \caption{\label{subfig:ElectronicSchemePCBSiPM}}
    \end{subfigure}
    \hfill
 \caption{a) Electronic board used to provide the SiPM bias voltage and to read the SiPM output signal. b) Electronical scheme in which this PCB is based.}
 \label{fig:PCBSiPM}
\end{figure}
The PCB was feed at $\pm6~\volt$ using the voltage source ISOTECH, model IPS-4303 \cite{VoltageSourceISOTECH} and the SiPM was feed using the electrometer KETHLEY, model 6517B \cite{VoltageSourceKethley}, that achieves a resolution of $1~\milli\volt$, low enough to ensure that this voltage variations does not affect the SiPM gain. The output signal of this PCB is connected to an oscilloscope, model WwaveRunner 625Zi from TELEDYNE LECROY \cite{OscilloscopeIFIMED} that records the data which were subsequently analized by ROOT\footnote{ROOT is a framework for data processing, based on C ++ and object-oriented technology, developed at CERN and widely used in nuclear and particle physics.} scripts.