This thesis is build up as follow:...



The objetive of this three-phase project was design, develop and instalation and commisioning a automatical system capable of detection tritium which we find in the water that is used by the nuclear power plants for their cooling system. The initial idea was quantifying its activity in quasi-real time before discharging it into public rivers or seas. 

We are speaking all the time about getting the measurement in quasi-real time, that is, in less of 10 minuts but what about the real time? It is obviously imposible because we are mesuring the activity, I mean, we are counting tritium decays in samples with low activities (supposedly less than $100~\becquerel/\liter$) so we need some time in order to get enough stadistic with which we can distinguish the tritium signal to the background in our system.

In order to get the measurement in quasi-real time we need to work \textit{in situ}, that's, we need that our detector is able to work in the same place that we take the sample. The reason of this fact is because if we need to move the sample we lose a lot of time (1 or 2 days in some cases). Furthermore if we avoid to move the samples we get a:

\begin{itemize}
\item{} faster detector because we eliminates the process of taking the sample, the chain-of-custody until this sample arrive to this laboratory and the complexity which involve these tasks. 

\item{} better detector since if we can work \textit{in site}, our measurements can be more frequent hence we will can identify cahnges in the activity earlier.

\item{} cheaper detector because we have removed all the cost associated with the sample collection, chain-of-custody of this sample, shipping of this sample to the laboratory and analysis thereof there. Not only we have eliminated the material costs attached to this tasks (material, transport, etc) but we have also eliminated the costs attached to the specialized staff who are involving in these tasks. With our detector we will only need frequent calibrations each time, which we consider suitable, in order to ensure the correct operation of this detector

\item{} safer detector since the personal exposure dose is reduced and the changes in activity are detected fastly. On top of that we remove the possibles mistakes which can be done by specialized staff who follow each protocol of this tasks.

\end{itemize} 

Dividiremos este trabajo en seis partes:
\begin{enumerate}
\item{} En primer lugar, se realizará un estudio sobre las fibras centelleadoras para  determinar  el protocolo de manipulación para obtener un  procedimiento de preparación de  un haz de fibras centelleadoras con un buen rendimiento óptico. 

\item{} En segundo, lugar estudiaremos el procedimiento de calibración de los SiPM,  fundamental para el experimento.  No se necesita realizar una calibración de los PMT, paso igualmente importante al anterior, ya que este trabajo fue realizado recientemente por otro componente del grupo.

\item{} En tercer lugar, se describirá  el primer prototipo diseñado, formado por un haz de 35 fibras centelleadoras sin clad leídas por PMT,  incluyendo el protocolo del proceso de llenado con agua tritiada que tuvo que ser desarrollado para cumplir los requisitos de protección radiológica y evitar contaminación accidental,  y los  resultados obtenidos con el mismo.

\item{} En cuarto lugar, se presentarán las simulaciones realizadas con el programa de Geant4 en una configuración sencilla.

\item{} En quinto lugar, se expondrán aspectos a estudiar en el futuro inmediato y en etapas posteriores, durante la Tesis Doctoral. 


\item{} Se presentarán, finalmente, los resultados, logros y conclusiones alcanzadas en el desarrollo del  trabajo.

\end{enumerate}