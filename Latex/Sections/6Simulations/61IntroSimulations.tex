The Monte Carlo simulations performed in the TRITIUM experiment are described in this chapter. They were carried out to optimize the design of the TRITIUM detector, understand its behaviour and investigate its limitations. This chapter is divided into two different sections. The first section contains the results of several simulations used to improve the design of the TRITIUM detector, while the second section describes the simulation results of a full TRITIUM monitor composed of several TRITIUM-IFIC-2 prototypes read out in anticoincidence mode with an active cosmic veto. Furthermore, several tests were carried out to verify the correct simulation of the different steps such as the simulated tritium source, the energy deposition in the fibers and the production of photons in them. The simulation environment employed is Geant4 \cite{Geant4WebPage, Geant4P}.