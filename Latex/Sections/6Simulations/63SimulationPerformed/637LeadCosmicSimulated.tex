The lead shielding and active vetos, described above, were included in the simulation of the Tritium-IFIC 2 prototype. The objective of these simulations was to quantify the reduction of cosmic background detected by the prototype. For this task, the tritium source was replaced by the cosmic events source, which was simulated through the CRY library.

%As can be seen in Figures \ref{subfig:RealHardCosmicEvent} and \ref{subfig:FakeHardCosmicEvent}, two plastic scintillators were simulated with the dimensions mentioned in section \ref{subsec:SetUpActiveShield} and located above and below of the Tritium-IFIC 2 prototype simulated. 

The optical properties included to the plastic scintillators of the active veto are the refractive index, the light attenuation spectrum and energy emission spectrum, the values of which were obtained from their data sheet provided by the manufacturer \cite{ScintillatorVeto}. Two PMTs, model R8520-460 from Hamamatsu, were simulated to read each plastic scintillator, similar to that presented in section \ref{subsec:SetUpActiveShield}.

The lead shielding was simulated with properties taken from the Geant4 NIST database. The dimensions of the simulated lead shielding were $60 \times 60 \times 70~\cm^3$, which is the minimum needed to accomodate an active veto and Tritium detector module inside. The length of the simulated lead castle, $60~\cm$, is smaller than real dimension, $148~\cm$. The reason for this is that only one tritium detector module was simulated, so the dimension of the lead shielding can be reduced to optimize simulation time and computing resources.

The results of these simulation are shown in section \ref{subsec:ResultsSimulatedBackgroundRejectionSystem}, where they are discussed.