The prototypes so-called TRITIUM-Aveiro and TRITIUM-IFIC-2 are reported in this section. In these prototypes, a different design was used compared to the previous ones (TRITIUM-IFIC-0 and TRITIUM-IFIC-1) so that they can safely allow the reading of a large number of fibers arranged in a straight mode with two photosensors operating in time coincidence. A better tritium detection efficiency was obtained in these prototypes through the use of a much larger number of scintillating fibers than in the preliminary prototypes, and the time-coincidences read out mode of the photosensors. Furthermore, the activity of the radioactive liquid source of tritium used to fill these prototypes was much lower than that used for the first prototypes, in order to measure their minimum detactable activity, MDA. 

A similar design was used for the latest prototypes and subtle differences were included to check which ones optimize the tritium detection. The main differences between both prototypes are:

\begin{itemize}

\item{} The diameter of the scintillating fiber, $2~\mm$ for TRITIUM-Aveiro and $1~\mm$ for TRITIUM-IFIC 2. The use of a larger diameter facilitates the flow of water around the fibers, reducing problems related to surface tension and ensuring that the entire active volume of the fibers participates in tritium detection. In addition, a large radius increases the rigidity of the fiber, improving its robustness. However, this large radius decreases the signal-to-background ratio. The detector active volume for $2~\mm$ fibers is smaller for the same volume than for $1~\mm$ fibers and the internal volume of the fibers, unreachable by tritium decay electrons, is larger for $2~\mm$ fibers, producing a higher background. As a result, a lower signal-to-background ratio is obtained.


%On the one hand $1~\mm$ are better for the tritium detection since the mean free path of the tritium decay electrons inside the fiber is about $5~\mu\meter$. The part of the scintillating fiber deeper than that, larger for $2~\mm$ scintillating fibers will only contribute to the background of the prototype, masking the tritium signal. In addition, more $1~\mm$ fibers can be used in the same space, achieving a larger active area of the detector. Therefore, fewer photosensors are needed for the TRITIUM monitor, lowering its price. On the other hand $2~\mm$ improves the flow of the water through the scintillating fiber bunch, a crucial point since it is directly proportional to the active area of the prototype.

\item{} The scintillating fibers methods used to prepare the fibers before its use. The entire surface-conditioning method (section \ref{subsec:SurfaceConditioningProcess}), consisting in the cutting, polishing and cleaning methods, are applied in the scintillating fibers used in the TRITIUM-IFIC-2 prototype. However, only the cutting method is applied to the scintillating fibers used in the TRITIUM-Aveiro prototype.

\item{} Different photosensors, PMTs for TRITIUM-Aveiro and SiPM arrays for TRITIUM-IFIC-2. Although most of the current development with TRITIUM-IFIC-2 was made with PMTs, the final purpose is to use SiPM arrays with which larger photodetection efficiencies than PMTs with a similar price can be achieved. In addition, no high voltage is needed, lowering the price of its supply voltage. However, it is necessary to read many more channels, which raises its price.

\item{} A different electronic system is used to process and analyze the signals of the photosensors. The TRITIUM-Aveiro prototype uses a home-made PCB-based electronic system, which is cheaper than the commercial system used by TRITIUM-IFIC 2, PETsys. Nevertheless, the PETsys system is more stable and it is prepared for the scalability property of the detector, allowing read out much more SiPM arrays without any development. However, the PETsys system is more stable and meets the TRITIUM monitor scalability requirement, allowing more SiPM arrays to be read without any additional development.

\end{itemize}

The development and operation of these two prototypes aimed at defining the final design and construction options for the module used in the TRITIUM monitor.
