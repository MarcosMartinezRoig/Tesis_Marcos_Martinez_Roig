TRITIUM-IFIC-1 was designed to correct the issues found in TRITIUM-IFIC-0. The main improvements were:

\begin{enumerate}

\item{} The fibre bundle was arranged straight to optimize the photon collection efficiency of the fibres. In addition, a PTFE matrix was used to maintain a distance of $1~\mm$ between fibres.

\item{} A special fibre cleaning method, described in section \ref{sec:CharacterizationScintillatingFibers}, was applied to the fibres to improve the quality of the interface between fibres and tritiated water. This method produces a better wetting property of the fibres, which improves their photon collection efficiency.

\item{} A PTFE vessel was used to improve the collection of photons inside the prototype. Indeed, PTFE has a reflectivity close to $100\%$ at the fibre scintillating wavelengths. Thus, the photons that escape from fibres and hit the vessel walls are reflected back into the scintillating fibres.

\end{enumerate}

The TRITIUM-IFIC-1 prototype consists of 64 straight scintillating fibres of $20~\cm$ length, arranged in an $8\times 8$ PTFE squared matrix, as shown in Figure \ref{fig:TeflonStructureFibersTritiumIFIC1}.
\begin{figure}[h]
\centering
\includegraphics[scale=0.4]{5Prototypes/52PreliminarPrototypes/522TritiumIFIC1/FiberMatrixTeflonStructure.png}
\caption{PTFE structure used to arrange the fibres of TRITIUM-IFIC-1 prototype in a matrix of $8 \times 8$.\label{fig:TeflonStructureFibersTritiumIFIC1}}
\end{figure}
This structure is placed within a cylindrical PTFE vessel of $48~\mm$ diameter and $200~\mm$ length, shown in Figure \ref{fig:TeflonVesselTritumIFIC1}. 
\begin{figure}
\centering
    \begin{subfigure}[b]{0.30\textwidth}
    \centering
    \includegraphics[width=\textwidth]{5Prototypes/52PreliminarPrototypes/522TritiumIFIC1/TeflonVesselTritiumIFIC1a.png}  
    \caption{\label{subfig:TeflonVesselTritumIFIC1a}}
    \end{subfigure}
    \hfill
    \begin{subfigure}[b]{0.45\textwidth}
    \centering
    \includegraphics[width=\textwidth]{5Prototypes/52PreliminarPrototypes/522TritiumIFIC1/TeflonVesselTritiumIFIC1b.png}  
    \caption{\label{subfig:TeflonVesselTritumIFIC1b}}
    \end{subfigure}
 \caption{Pictures of the TRITIUM-IFIC-1 PTFE vessel.}
 \label{fig:TeflonVesselTritumIFIC1}
\end{figure}
The cleaning process described in section \ref{sec:CharacterizationScintillatingFibers} was applied to the fibres to achieve a better tritiated water-fibre interface. A PVC piece was used to couple a photosensor to the prototype and to prevent external light, as shown in Figure \ref{fig:TritumIFIC1}. 
\begin{figure}[h]
\centering
\includegraphics[scale=0.4]{5Prototypes/52PreliminarPrototypes/522TritiumIFIC1/TritiumIFIC1a.png}
\caption{A picture of the TRITIUM-IFIC-1 prototype. The photosensor lodging is shown.\label{fig:TritumIFIC1}}
\end{figure}
The prototype was instrumented with a PMT, model R8520-06SEL from Hamamatsu Photonics \cite{DataSheetPMTs}, coupled directly to the fibre bundle by optical grease \cite{OpticalGrease}. The quantum efficiency of this PMT is $28.66\%$ at $\lambda=430~\nano\meter$.  The PMT high voltage was $-800~\volt$. The DAQ was the same as for TRITIUM-IFIC-0. In a first measurement, this prototype was filled with pure water ($118~\milli\liter$, uncertainty of $0.05\%$) and several background measurements were taken during a week. Subsequently, it was emptied and refilled with $118~\milli\liter$ (uncertainty of $0.05\%$) tritiated water of $99.696~\kilo\becquerel/\liter$ activity. The measured signal and background energy spectra are shown in Figure \ref{subfig:SignalBackgroundEnergySpectraTritiumIFIC1}. The tritium spectrum is shown in Figure \ref{subfig:TritiumEnergySpectraTritiumIFIC1}. The rates obtained from the integration of the spectra are given in Table \ref{tab:CountsPerSecondTRITIUMIFIC1}. 

\begin{figure}
\centering
    \begin{subfigure}[b]{1\textwidth}
    \centering
    \includegraphics[width=\textwidth]{5Prototypes/52PreliminarPrototypes/522TritiumIFIC1/TritiumIFIC1Signals.pdf}  
    \caption{\label{subfig:SignalBackgroundEnergySpectraTritiumIFIC1}}
    \end{subfigure}
    \hfill
    \begin{subfigure}[b]{1\textwidth}
    \centering
    \includegraphics[width=\textwidth]{5Prototypes/52PreliminarPrototypes/522TritiumIFIC1/TritiumIFIC1Clear.pdf}  
    \caption{\label{subfig:TritiumEnergySpectraTritiumIFIC1}}
    \end{subfigure}
 \caption{Energy spectra measured by TRITIUM-IFIC-1. a) Signal and background energy spectra. b) Tritium energy spectrum.}
 \label{fig:EnergySpectraTRITIUMIFIC1}
\end{figure}

\begin{table}[htbp]
\centering{}%
\begin{tabular}{cc}
\toprule 
Spectrum & Rate (Hz) \tabularnewline
\midrule
\midrule 
Signal & $7.82 \pm 0.11$ \tabularnewline
Background & $3.99 \pm 0.08$ \tabularnewline  
Tritium & $3.83 \pm 0.13$ \tabularnewline
\bottomrule
\end{tabular}
\caption{Counting rates obtained by TRITIUM-IFIC-1.}
\label{tab:CountsPerSecondTRITIUMIFIC1}
\end{table}
The tritium detection efficiency obtained for TRITIUM-IFIC-1 is 
$$\eta=(3.84 \pm 0.16)\cdot{} 10^{-2}~\liter\:\kilo\becquerel^{-1}\second^{-1}$$
The specific efficiency obtained is
$$S=(9.6 \pm 0.4)\cdot{} 10^{-5}~\liter\:\kilo\becquerel^{-1}\second^{-1}\cm^{-2}$$
which is a factor ten better than that of TRITIUM-IFIC-0. Furthermore, compared to the scintillating detectors of tritium in water given in table \ref{tab:PlasticScinTritium}, the efficiency of this prototype is very close to the best result obtained by Singh \cite{Ratnakaran, Ratnakaran2000}, $\eta=4,1 \cdot{} 10^{-2}~\liter\ \kilo\becquerel^{-1}\second^{-1}$, and the specific efficiency, which is the most relevant parameter for comparison, is almost 5 times larger than that obtained by Hofstetter \cite{Hofstetter1, Hofstetter2}, $S<22.2 \cdot{} 10^{-6}~\liter\ \kilo\becquerel^{-1}\second^{-1}\cm^{-2}$.
