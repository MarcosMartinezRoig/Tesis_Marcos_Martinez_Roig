First of all the shape of the simulated tritated water source was optimized. 

The mean free path of tritium electrons in water are only around $5~\mu\meter$, so there are many electrons that don't reach the scintillating fibers, electrons that don't provide useful information and only contribute to being time consuming and reducing available computing resources.

To optimize the simulation, the dimensions of the simulated tritium source was studied. The goal of this study was to minimizes the tritium events that do not reach the scintillating fibers avoiding losing the tritium events that reach them.

This simulation test consists of a scintillating fiber with a length of $20~\cm$ and a diameter of $2~\mm$ and a surrounding tritiated water source with the same length and a thickness of $0.5~\mm$ ($100$ times greater that the mean free path of tritium electrons) to ensure that this study take into account all possible tritium electrons that can reach to the scintillating fiber. 

The dimensions of the fiber are not important in this study since only the energy deposition of tritium electrons in the fiber were simulated. That is, this simulation doesn't include the following steps such as photon generation, propagation of these photons, etc. in which the shape of the scintillating fiber becomes important.

%The simulated scintillating fiber consists of a polystyrene core to which sensitive attributes has been given and the simulated tritiated water source consists of water to which electron emissions with the same energy spectrum of tritium electron decay has been added.

The results of this simulation are shown in section \ref{subsec:ResultsShapeSource}, where they are discussed.