The last prototype developed for TRITIUM was TRITIUM-IFIC-2, marked as A in Figure \ref{fig:TritiumIFIC2}. This prototype, built in the IFIC workshop, consists of a cylindrical PTFE vessel, shown in Figure \ref{fig:Tritium-IFIC2_vessels}, with a similar shape to that of Aveiro. The internal length and diameter of the PTFE vessel were $210~\mm$ and $36~\mm$ respectively. This prototype contains $800$ uncladded BCF-12 scintillating fibers of $200~\mm$ length and $1~\mm$ diameter. This number is larger than that in Aveiro's prototype and is contained in a smaller volume. The fibers used were cleaved, polished and cleaned with the conditioning processes described in section \ref{sec:CharacterizationScintillatingFibers}. These scintillating fibers were freely and tightly arranged while standing straight, allowing water to flow among them. Two PMMA windows, located at the ends of the fiber bundle, allowed to read out the scintillation light in a similar way as in the Aveiro's prototype. 

\begin{figure}[h]
\centering
\includegraphics[scale=0.4]{5Prototypes/53FinalPrototypes/532TritiumIFIC2/Tritium_IFIC_2_full_module.jpg}
\caption{TRITIUM-IFIC-2 prototype (A) and active veto (E) within the metallic structure (D).\label{fig:TritiumIFIC2}}
\end{figure}

\begin{figure}
\centering
    \begin{subfigure}[b]{0.35\textwidth}
    \centering
    \includegraphics[width=\textwidth]{5Prototypes/53FinalPrototypes/532TritiumIFIC2/Tritium_IFIC_2_vessel1.png}  
    \caption{\label{subfig:Tritium_IFIC_2_vessel}}
    \end{subfigure}
    \hfill
    \begin{subfigure}[b]{0.3\textwidth}
    \centering
    \includegraphics[width=\textwidth]{5Prototypes/53FinalPrototypes/532TritiumIFIC2/Tritium_IFIC_2_vessel2.png}  
    \caption{\label{subfig:TritiumIFIC2_vessel_with_PVC_caps}}
    \end{subfigure}
 \caption{a) TRITIUM-IFIC-2 PTFE vessel. b) TRITIUM-IFIC-2 PTFE vessel with PVC caps in which a groove is made for the cable connection to the SiPM.}
 \label{fig:Tritium-IFIC2_vessels}
\end{figure}

A $5~\mm$ width PMMA optical windows is sufficient to guarantee tightness, since the detector works at very low water pressure. Two clamps allow to keep the water tightness of the prototype, similar to the TRITIUM-Aveiro prototype. PMMA was chosen for its optical properties, especially its transmission coefficient, shown in Figure \ref{fig:PMMATransmissionSpectrum}, which was measured for visible light at ICMOL laboratories. This transmission coefficient is approximately $95\%$ for the working wavelength ($435~\nm$). Slightly better transmission coefficients can be achieved with other materials such as quartz or sapphire but they are much more expensive.

\begin{figure}[h]
\centering
\includegraphics[scale=0.6]{5Prototypes/53FinalPrototypes/532TritiumIFIC2/TransmissionSpectrumPMMA_cut_at_low_energy.pdf}
\caption{Light transmission spectrum of a $5~\mm$ thick PMMA plate, measured at ICMOL laboratory. \label{fig:PMMATransmissionSpectrum}}
\end{figure}	

A water inlet/outlet was implemented in the PTFE vessel, B in Figure \ref{fig:Tritium-IFIC2_vessels}, to allow a constant water flux, as in the TRITIUM-Aveiro prototype.

In the first laboratory measurements, two PMTs model Hamamatsu R8520-460 \cite{DataSheetPMTs} were used to compare the results to those of the previous prototypes. However, measurements of the TRITIUM-IFIC-2 prototype with SiPM arrays controlled by PETsys were also performed, as this is the final readout option for this prototype. PETsys has a graphical user interface, shown in Figure \ref{fig:GUI_PETSYS}. It allows the remote control of all the different input and ouput options such as the supply voltage for the SiPM arrays, thresholds, etc., via computer terminal. 

\begin{figure}[h]
\centering
\includegraphics[scale=0.38]{5Prototypes/53FinalPrototypes/532TritiumIFIC2/GUI_PETSYS.png}
\caption{Graphical User Interface (GUI) of PETsys.\label{fig:GUI_PETSYS}}
\end{figure}

Two PVC caps, located at both ends of the prototype were used to provide a light-tight environment to the SiPM arrays. An aluminum structure was designed and built to house up to 10 TRITIUM-IFIC-2 modules and two cosmic vetos, marked as D in Figure \ref{fig:TritiumIFIC2}.

In Arrocampo site, the available space inside the lead shield box may accommodate up to 5 structures. This means that the final TRITIUM moitor may accommodate up to 50 TRITIUM-IFIC-2 modules and 5 different cosmic vetos.

Two identical TRITIUM-IFIC-2 prototypes were built, as for the TRITIUM-IFIC-0 prototype. One of them was filled with pure water and used to measure the background and the other was filled with a radioactive liquid source of tritium and employed to measure the signal. The water volume in both cases was $82~\milli\liter$ (uncertainty of $0.05\%$). The activity of the tritium source used for this prototype was $10~\kilo\becquerel/\liter$ (uncertainty of $2.24\%$), which was prepared by diluting a sample of tritiated water in pure water. The signal and background energy spectra are shown in Figure \ref{subfig:SignalBackgroundEnergySpectraTritiumIFIC2}. The energy spectrum of tritium, Figure \ref{subfig:TritiumEnergySpectraTritiumIFIC2}, was obtained by substracting the background to the signal. The rates obtained from these three spectra are given in Table \ref{tab:CountsPerSecondTRITIUMIFIC2}. 

\begin{figure}
\centering
    \begin{subfigure}[b]{0.73\textwidth}
    \centering
    \includegraphics[width=\textwidth]{5Prototypes/53FinalPrototypes/532TritiumIFIC2/TritiumIFIC2SignalsHigherZOOM_NP.pdf}  
    \caption{\label{subfig:SignalBackgroundEnergySpectraTritiumIFIC2}}
    \end{subfigure}
    \hfill
    \begin{subfigure}[b]{0.73\textwidth}
    \centering
    \includegraphics[width=\textwidth]{5Prototypes/53FinalPrototypes/532TritiumIFIC2/TritiumIFIC2ClearHigherZOOM_NP.pdf}  
    \caption{\label{subfig:TritiumEnergySpectraTritiumIFIC2}}
    \end{subfigure}
 \caption{Energy spectra measured with TRITIUM-IFIC-2 prototype. a) Signal and background energy spectra. b) Tritium energy spectrum.}
 \label{fig:EnergySpectraTRITIUMIFIC2}
\end{figure}

\begin{table}[htbp]
\centering{}%
\begin{tabular}{cc}
\toprule 
Spectrum & Counts/second \tabularnewline
\midrule
\midrule 
Signal prototype & $19.05 \pm 0.18$ \tabularnewline
Background prototype & $11.54 \pm 0.14$ \tabularnewline  
Tritium counts & $7.11 \pm 0.23$ \tabularnewline
\bottomrule
\end{tabular}
\caption{Counting rates measured by TRITIUM-IFIC-2 prototype.}
\label{tab:CountsPerSecondTRITIUMIFIC2}
\end{table}
The tritium detection efficiency obtained for this prototype is $(7.11 \pm 0.28)\cdot{} 10^{-1}~\frac{cps}{\kilo\becquerel/\liter}$. This efficiency is larger than those reported in the literature, Table \ref{tab:PlasticScinTritium}. This is an expected result since the active area of this prototype is the largest. To remove the active area effect, the specific efficiency was measured, obtaining a value of 
$$S=(14.1 \pm 0.6)\cdot{} 10^{-5}~\frac{cps}{\kilo\becquerel/\liter \cdot{} \cm^{2}}$$
Again, it can be observed that this prototype has the largert specific efficiency reported for tritium detection, demostrating that its design the best design currently developed for detection of low activies of tritium in water.

The energy spectrum is given in ADC channels, since an energy calibration for a plastic scintillator is not accurate due to the large uncertainty in the number of photons produced per energy event. Nevertheless, a detector calibration in units of photons detected per event can be obtained from the single-photon distribution of the PMTs. The PMTs used to read this prototype was decoupled from the prototype and covered with a special black blanket to screen the PMT from external photons. The distribution measured and fitted to a Gaussian function is shown in Figure \ref{subfig:SinglePhotonDistributionIFIC2}. As can be seen, the mean and uncertainty of the single photon signal are around $172$ and $66$ ADC channels, respectively, for one of the PMT and $173$ and $57$ ADC channels, respectively, for the other. The tritium signal given in number of photons detected per event, shown in Figure \ref{subfig:TritiumSignalTRITIUMIFIC2}, is obtained as the ratio of the energy spectrum to the single-photon distribution mean. A maximum of $15$ photons are measured per tritium event, which is in agreement with the results of the simulations shown in Chapter \ref{chap:Simulations}. 

\begin{figure}
\centering
    \begin{subfigure}[b]{0.73\textwidth}
    \centering
    \includegraphics[width=\textwidth]{5Prototypes/53FinalPrototypes/532TritiumIFIC2/SinglePhotonDistribution2.pdf}  
    \caption{\label{subfig:SinglePhotonDistributionIFIC2}}
    \end{subfigure}
    \hfill
    \begin{subfigure}[b]{0.73\textwidth}
    \centering
    \includegraphics[width=\textwidth]{5Prototypes/53FinalPrototypes/532TritiumIFIC2/PhotonsPerTritiumEvent.pdf}  
    \caption{\label{subfig:TritiumSignalTRITIUMIFIC2}}
    \end{subfigure}
 \caption{a) Single photon distribution measured with TRITIUM-IFIC-2 prototype. b) Tritium energy spectrum measured with TRITIUM-IFIC-2 prototype in photons detected per event.}
 \label{fig:PhotonsPerTritiumEventIFIC2}
\end{figure}

%\begin{figure}[h]
%\centering
%\includegraphics[scale=0.6]{5Prototypes/53FinalPrototypes/532TritiumIFIC2/SinglePhotonDistribution.pdf}
%\caption{Single photon energy distribution measured with the PMT used in TRITIUM-IFIC-2 prototype.\label{fig:SinglePhotonDistributionIFIC2}}
%\end{figure}

%\begin{figure}[h]
%\centering
%\includegraphics[scale=0.6]{5Prototypes/53FinalPrototypes/532TritiumIFIC2/PhotonsPerTritiumEvent.pdf}
%\caption{Tritium signal measured with the TRITIUM-IFIC-2 prototype and expressed in number of photones per tritium event detected.\label{fig:TritiumSignalTRITIUMIFIC2}}
%\end{figure}


%As can be seen, a maximum of $15$ photons are generated per tritium event, which corresponds to the best situation. To compare the value obtained with the expected one, the different energies and efficiencies involved are taken into account. Considering a maximum energy for the tritium electron detected, $18.6~\keV$, a scintillation yield of $8000~\text{ph}/\MeV$ for the fibers, a maximum collection efficiency for the fibers, $7\%/\meter$, the fiber length, $20~\cm$ (which increases the collection efficiency by a factor of 5), and the PMT efficiency, $29\%$, the maximum number of photons produced for a tritium event detected with TRITIUM-IFIC-2 prototype is $15$. As can be seen, this is perfectly in accordance with the measurement.

A monitoring of both prototypes, signal and background, were carried out during several months. The rates measured are shown in Figure \ref{fig:MonitorizationTRITIUMIFIC2}. No quenching of the signal was observed, which indicates the detector efficiency remained stable, within statistical and systematic uncertainties, over up to 6 months. 

\begin{figure}[h]
\centering
\includegraphics[scale=0.6]{5Prototypes/53FinalPrototypes/532TritiumIFIC2/Signal_Background_stability_ZOOM.pdf}
\caption{Signal and background rates for a long time measurement.\label{fig:MonitorizationTRITIUMIFIC2}}
\end{figure}

Finally, the Minimum Detectable Activity (MDA) was calculated. To do so, fourteen different measurements of the background were done using two different integration times, $10~\min$ and $60~\min$. The mean value and standard deviation of these measurements are shown in the Table \ref{tab:CurrieLawTRITIUMIFIC2}. The minimum net counts with a probability of a  false-negative less than $5\%$ , $N_D$, and with a probability of a false-positive less than $5\%$, $L_C$, were calculated by applying the Currie's law, equation \ref{eq:EquationNetCounts} and they are included in Table \ref{tab:CurrieLawTRITIUMIFIC2}.

\begin{table}[htbp]
\centering{}%
\begin{tabular}{ccccc}
\toprule 
Int. time (min.) & Mean V. & Std. Dev. & $L_C$ & $N_D$ \tabularnewline
\midrule
\midrule 
$10$ & $5635$ & $82$ & $191$ & $384$ \tabularnewline
$60$ & $33969$ & $158$ & $368$ & $737$ \tabularnewline
\bottomrule
\end{tabular}
\caption{Mean value and standard deviation of the counts of fourteen background measurements. Minimum net counts obtained by applying the Currie's Law, $L_C$ and $N_D$.}
\label{tab:CurrieLawTRITIUMIFIC2}
\end{table}

Therefore, $N_D'$, which is the counts referred to the detector signal (before background subtraction), are $6019$ and $34706$ counts for an integration time of $10~\min$ and $60~\min$ respectively. Then, the MDA of tritium can be obtained from the $N_D'$ values by associating the mean value of the background counts to a zero tritium activity and the mean value of the signal counts to a tritium activity of $10~\becquerel/\liter$, assuming counts scale linearly with the activity. This results in a MDA of $677~\becquerel/\liter$ and $218~\becquerel/\liter$ for the integration time of $10~\min$ and $60~\min$ respectively. 

In addition, it has to be taken into account that one of the most important properties of the TRITIUM detector is its scalability, which means that better results can be achieved by using a large number of modules. The MDA of the TRITIUM monitor is expected to be reduced by a factor of $\sqrt{\text{number of prototypes}}$, according to the equation \ref{eq:EquationNetCounts}. This relationship is shown in Figure \ref{fig:MDATRITIUMmonitor}, where it can be seen that the goal of the TRITIUM project (to be able to measure $100~\becquerel/\liter$ (red line) in quasi-real time) is achieved by using $45$ TRITIUM-IFIC-2 modules read out in parellel with an integration time of $10~\min$ and, the cheaper and more realistic option of using $5$ TRITIUM-IFIC-2 modules read out in parallel with an integration time of $1~\hour$. %Therefore, the cheaper and realistic option is to use $5$ TRITIUM-IFIC-2 read out in parallel with an integration time of $1~\hour$ 

\begin{figure}[h]
\centering
\includegraphics[scale=0.7]{5Prototypes/53FinalPrototypes/532TritiumIFIC2/MDA_1_hour_and_10_min_vs_N_Prototypes_logY.pdf}
\caption{Minimum detectable activity, MDA, as a function of the number of TRITIUM-IFIC-2 prototypes read out in parallel for an integration time of $10~\min$ (blue line) and $1~\hour$ (black line). The dotted red line indicates the goal of the TRITIUM project, $100~\becquerel/\liter$. \label{fig:MDATRITIUMmonitor}}
\end{figure}

%Medir en el prototipo con SIPM.

%As the sensitivity of the TRITIUM monitor scales with the number of TRITIUM modules used, the results obtained with the TRITIUM monitor should improve those results by a factor of $\sqrt{N}$, where N is the number of modules used.