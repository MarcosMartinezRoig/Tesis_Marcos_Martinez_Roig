Tritium is one of the most frequently emitted radioisotopes in a nuclear power plant. It could be dangerous for health and for the environment so there's exist several legislations which try to control this radiactive emissions like Directive Europeen 2013/51/Euratom, which establish the tritium limit in $100~\becquerel/\liter$ in drinking water, or U. S. Environmental Protection Agency, whose tritium limit is established in $740~\becquerel/\liter$ in drinking water.

Nowadays, due to the low energy emitted in the tritium decay, we need high sensitive detectors for measuring it like LSC. But this is a off-line method whose measurement process can take up to 3 or 4 days, whose response time could be a problem if there are any problem with the NPP.

Detectors based on solid scintillators is a promissing idea for building a tritium detector that works in quasi-real time. This type of detectors has been developed so far succesfully but without achieving enough sensibility for measuring the legal limits.

In this study the results of TRITIUM project is presented. In the framework of this project we have developed a quasi-real time monitor for low tritium activities in water. This monitor is based on a tritium detector, which we read in parallel, that contains on several detection cells with hundreds of scintillating fibers read out by PMTs or SiPM arrays, several pasive shielding and active vetos for reduce the natural background to our system and a ultra pure water system to keep our detector clean.

The final objective of this monitor will be the radiological protection around the nuclear power plant. This monitor will provide an alarm in case of an unexpected tritium release. It will be included in the early alarm system consisting of several detectors whose objective is to reduce the impact of Nuclear Power Plants to the environment

%LEER LOS 2 ABSTRACS DE ANA MAS LOS ABSTRACS QUE TENGO YO DE LAS CONFERENCIAS Y ARTICULOS (ARTICULOS ALEATORIOS MAS EL DE CARLOS, EL MIO, EL DE NADIA, ETC)