Due to the radiological risk of tritium, described above, it is important that the current legislation limits the release of tritium to the environment ensuring that the levels are below a safe value for health.

The guidelines to limit the radioactive elements in drinking water for many countries. They are based on the radiation protection methodology developed by the ICRP \cite{ICRP_GL} and the recommendations of the world health organization (WHO) \cite{WHO_GL}. The objective of the international radiation methodology is to  protect people and the environment from the negative effects of ionizing radiations but allowing beneficial activities that involve a reasonable radiation exposure. It is based on three main points, which are:
\begin{enumerate}
\item{} The justification: The benefit from radiological exposure must outweigh the detriment to health that it causes.
\item{} The ALARA principle ("As Low As Reasonably Achievable"): The radiological exposure must be kept as low as possible considering social and economic factors.
\item{} The dose limitation: Limit that must never be exceeded.
\end{enumerate}

While the ICRP recommends a maximum dose of $1~\milli\sievert/$yr, excluding the natural background and medical interventions, the WHO is more conservative, recommending a maximum dose of $0.1~\milli\sievert/$yr, which correspond to less than $5\%$ of the annual dose due to background radiation, $2.42~\milli\sievert/$year.

The guideline reference level of each radionuclide in drinking water, GL, is usually calculated from these recomendations using the equation,
\begin{equation}
GL (\becquerel/\liter) = \frac{RDL}{DCF \cdot{} q}
\label{eq:Guideline}
\end{equation}
where RDL is the reference dose level, DCF is the dose conversion factor (the normal used value for tritium is $1.8 \dot{} 10^{-11}~\sievert/ \becquerel$, provided by ICRP \cite{ICRP_factor}) and $q$ is an estimation of the annual volume of drinking water consumed (normally assumed two liters per day, $730~\liter/$yr).

The GL calculated for tritium in drinking water according to the ICRP and WHO recommendations is $76,103~\becquerel/\liter$ and $7,610~\becquerel/\liter$  respectively. It means that tritiated water with activities below these values is considered not harmful for health.

Based on these recommendations, each country has created organizations in charge of developing its own legislation on radionuclide limits. In Spain, the responsible organization of this task is the CSN. Most of the countries in the world implement the RDL of $0.1~\milli\sievert/$yr recommended by the WHO. The legal limit for tritium in drinking water in this case is $7,610~\becquerel/\liter$  but it is often approximated in different ways. Some countries like Switzerland \cite{Switzerland_GL} or some organizations like the WHO \cite{WHO_GL} take this value as $10,000~\becquerel/\liter$. Others like some countries of Canada, such as Ontario and Québec, truncate this value to the first number $7,000~\becquerel/\liter$ \cite{Ontario_GL, Quebec_GL}. There are other countries like Russia which use the much more accurate approximation value of $7,700~\becquerel/\liter$ \cite{Russia_GL}. There are other countries like Australia that prefer to implement the RDL of $1~\milli\sievert/$yr, recommended by the ICRP, the legal limit of which is $76,103~\becquerel/\liter$ \cite{Australia_GL}. Other countries like Finland are based in the ICRP recommendations and use only half of this value, $0.5~\milli\sievert/$yr, rounded to a legal limit of $30,000~\becquerel/\liter$ for tritium in drinking water \cite{Finland_GL}.

There are two different exceptions to these recommendations:
\begin{enumerate}
\item{} Most of the USA states like California use a RDL of $4~\milli\rem~(0.04~\milli\sievert)$, which corresponds to a legal limit of $20~\nano\curie/\liter~(740~\becquerel/\liter)$ \cite{California_GL}. This value was proposed by the United States Environmental Protection Agency (US EPA) as a result of an analysis performed \cite{USEPA_GL}.

\item{} Most of the EU countries, such as France, Germany or Spain, impose an GL of $100~\becquerel/\liter$, which is one of the most restrictive limit in the world \cite{France_GL, Germany_GL, Spain_GL}. This value arise from the consideration that it is an indicator of the presence of other radionuclides more dangerous than tritium. These limits are fixed by the EURATOM Council Directive \cite{EURATOM_GL}. 
\end{enumerate}

All limits mentioned in this section are summarized in table \ref{tab:LegalLimitTritium}.

\begin{table}[htbp]
\begin{center}
\begin{tabular}{|c|c|}
\hline
Country/Agency & Legal limit of tritium in water ($\becquerel/\liter$)\\
\hline \hline \hline
ICRP & $76,103$ \\ \hline
WHO & $10,000$ \\ \hline
Switzerland & $10,000$ \\ \hline
Canada & $7,000$ \\ \hline
Russia & $7,700$ \\ \hline
Australia & $76,103$ \\ \hline
Finland & $30,000$ \\ \hline
United States & $740$ \\ \hline
European Union & $100$ \\ \hline
\end{tabular}
\caption{Legal limit of tritium in drinking water stablished in each country.}
\label{tab:LegalLimitTritium}
\end{center}
\end{table}