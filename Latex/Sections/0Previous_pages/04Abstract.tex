Tritium is one of the most abundantly emitted radioisotopes in a nuclear power plant. Large quantities of tritium are normally produced in the water of their cooling system, which are finally emitted to the environment. Due to the fact that high quantities of tritium could be dangerous for human health and for the environment, there exist several normative around the world which try to control this radioactive emissions in each country, like the Directive Europeen 2013/51/Euratom, which establishes the tritium limit in drinking water in Europe to $100~\becquerel/\liter$, or the U. S. Environmental Protection Agency, in United States, whose tritium limit in drinking water is established at $740~\becquerel/\liter$.

Nowadays, due to such a low energy emitted in the tritium decay, we need high sensitive detectors for measuring it like LSC. The problem with LSC is that it is an off-line method the measurement process of which can take up to 3 or 4 days, too much time if there are any problem with the NPP.

Detectors based on solid scintillators is a promissing idea for building a tritium detector that works in quasi-real time. This type of detectors has been developed so far succesfully but without achieving enough sensibility for measuring the legal limits.

In this study the results of TRITIUM project is presented. In the framework of this project we have developed a quasi-real time monitor for low tritium activities in water. This monitor is based on a tritium detector that contains several detection cells which we read in parallel, several active vetos and a pasive shielding for reducing the natural background of our system and an ultrapure water system to prepare the sample before we measure. Each detection cell is made up of hundreds of scintillating fibers read out by PMTs or SiPM arrays.

The final objective of this monitor will be the radiological protection around the nuclear power plant. This monitor will provide an alarm in case of an unexpected tritium release. It will be included in the early alarm system of Extremadura consisting of several detectors the objective of which is to reduce the impact of Nuclear Power Plants to the environment.

\vspace{1cm}

\textbf{Keywords:} Tritium in water, Real-time monitor, Nuclear Power Plant, ENvironmental Safety, ...