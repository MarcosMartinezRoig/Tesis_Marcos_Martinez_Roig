There exist two different proposals for the photosensors employed in the TRITIUM monitor, PMTs and SiPM arrays. Each type of photosensor has advantages and disadvantages and must be experimentally tested to ensure the most suitable option. PMTs are used in the TRITIUM prototypes developed by Aveiro experimental group while SiPM arrays are used in the TRITIUM prototypes developed at IFIC, in Valencia.

PMTs with and without gain have also been used in Valencia to perform various laboratory measurements such as R\&D studies with fibers, characterization of the active veto and test measurements of the TRITIUM prototypes developed.

The IFIC experimental group of TRITIUM has chosen the SiPM matrix option for the TRITIUM photosensor option for its advantages over PMTs, which are compactness and robustness, necessary to work for several years without supervision, its larger efficiency for the detection of photons in the visible range, critical parameter for the TRITIUM monitor, and its economic price, which are cheaper than PMTs not only the photosensor itself, but also the electronic sytem necessary to feed them and process and analyze their output signal. However, the SiPM arrays have a higher dark count rate compared to the PMTs, which is a relevant disadvantage for the purpose of TRITIUM monitor since low activities of tritium are intended to be measured.