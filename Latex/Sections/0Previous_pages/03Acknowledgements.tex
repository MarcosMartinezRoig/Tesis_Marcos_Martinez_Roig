Muchas personas han colaborado a lo largo de estos cuatro años para hacer posible este trabajo. En primer lugar me gustaría agradecer enormemente a mis tutores, Jose Díaz Medina y Nadia Yahlaly Haddou, por intentar transmitirme en todo momento el máximo de conocimientos, sin los cuales nada de esto hubiera sido posible. También agradecer su tiempo ya que, a pesar de sus obligaciones, siempre han encontrado un hueco en sus agendas para ayudarme con cualquier problema. Estoy muy agradecido por su cercanía y sus consejos, tanto en el mundo laboral como en la vida personal, los cuales me han ayudado a llegar a donde estoy ahora.

En segundo lugar me gustaría agradecer a mis compañeros de laboratorio Andrea Esparcia Córcoles, Ana Bueno Fernández, Marcos Llanos Exposito y sobre todo a Mireia Simeo Vinaixa, con quienes los largas e interminables trabajos de laboratorio han sido mucho más fáciles además de ayudarme a mejorar y comprender mis medidas experimentales.  También me gustaría agradecer mis compañeros de trabajo, Clodoaldo Roldán García, Teresa Cámara García, Vanesa Delgado Belmar y Rosa..., por su ayuda, su tiempo y sus esfuerzos en hacer todo esto haya sido posible.

En tercer lugar me gustaría agradecer a toda la gente del proyecto TRITIUM, gente que me ha acogido con los brazos abiertos, que me ha ayudado en todo lo que les ha sido posible y de la que he tenido el placer de trabajar a su lado, aprendiendo una gran cantidad de conocimientos.

También agradecer a David Calvo Diaz-Aldagalán, ingeniero del IFIC, por su ayuda y su tiempo con los diseños de las PCBs utilizadas en el proyecto TRITIUM.

También me gustaría dar las gracias a mucha gente con la que, aun sin haber aportado directamente en mi trabajo, no habría sido posible conseguir estos resultados sin su ayuda:

\begin{itemize}

\item{} A los investigadores del proyecto DUNE, Anselmo Cervera, Miguel Angel García, Justo Martin-Albo y LA XICA , los cuales me han ayudado con su tiempo, sus conocimientos e su instrumentación experimental a completar mi estudio de los SiPMs. 

\item{} A los ingenieros del proyecto NEXT, Vicente, Marc, Javier, Sara, con quienes he compartido laboratorio y, aun sin trabajar directamente con ellos, siempre estaban dispuestos a echar una mano. 

\item{} A los investigadores del IFIMED (Gabriela Llosá, Ana Ros, Marina Borja Lloret, Jhon Barrio, Rita Viegas y Jorge Roser) e investigadores del proyecto ATLAS (Urmila Soldevila y Carlos Mariña) por prestarnos sus instalaciones y, en concreto, sus sistemas de control de temperatura el cual fué necesario para la calibración de los SiPMs. 

\item{} A los investigadores del laboratorio de espectrometría gamma, , Berta Rubio, Cesar Domingo, Ion Ladarescu y Luis Caballero, quienes me ayudaron prestandome el material necesario para realizar cierto experimentos.

\item{} A los investigadores del ICMOL, HOMBRE y Lidón Gil, por ayudarnos con el sistema de limpieza de las fibras centelleadoras.

\item{} A toda la gente de los distintos departamentos del IFIC (mecánica, electrónica y electrónica) que hay ayudado con su tiempo y sus consejos.

\end{itemize} 

Finalmente me gustaría agradecer a mis compañeros del IFIC, Mireia, Pepe, Kevin, Jose, Pablo, Victor, Ana y David los cuales, aun sin haber aportado directamente a trabajo, compartir mis días con ellos ha sido uno de los mayores apoyos de esta aventura.

Finalmente me gustaría agradecer al programa INTERREG-SUDOE su ayuda y confianza depositada en el proyecto TRITIUM.