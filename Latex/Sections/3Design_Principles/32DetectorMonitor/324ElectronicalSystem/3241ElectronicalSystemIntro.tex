The electronic system is in charge of reading, processing and analyzing the output signal of photosensors and providing the desired output information. This electronic system depends on the type of output information and on the detector configuration used.

This section details the electronic readout used for the TRITIUM detector when SiPM arrays are used (PETsys). This section also shows the electron systems used for the laboratory characterization tests and I\&D, in which a single SiPMs and one or several PMTs was used.

It is important to note that the electronic systems used for laboratory testing will not be used to read the TRITIUM detector in the final location, Arrocampo dam. For this task, as there are two different porposals, two electronic systems are used; the PETsys system are used when the TRITIUM detector use SiPM arrays and an especific electronic system has been developed, designed and tested by Aveiro experimental group is used to read the TRITIUM detector when PMTs are employed. This electronic system is explained in Appendix \ref{App:ElectronicSystemAveiro}.

%In each type of detector configuration, this electronic system will also be different depending on the type of information we want to obtain. For example, it will be different if we want to obtain an energy spectrum as output information or simply a number such as the number of counts per second of our detector or the output electrical current. Each electron system that has been used in our experiment will be explained during this section.

 