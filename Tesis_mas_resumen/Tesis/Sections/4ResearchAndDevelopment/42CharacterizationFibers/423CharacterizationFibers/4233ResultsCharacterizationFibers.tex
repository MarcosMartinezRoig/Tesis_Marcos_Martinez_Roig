Cleaving and polishing add a systematic dispersion $\sigma_{sys-SF}$ to the instrinsic fibre response. In addition, the positioning of the connectors that lock the fibre to the experimental setup produces an additional systematic uncertainty $\sigma_{sys-pos}$. Since both uncertainties are independent, the total systematic uncertainty is given by,

\begin{equation}
\sigma_{sys} = \sqrt{\sigma^2_{sys-SF} + \sigma^2_{sys-pos} }
\label{eq:TotalUncertaintyFiberCharacterization}
\end{equation}
Two different experiments were designed, the first giving only the systematic uncertainty $\sigma_{sys-pos}$, and the second the total uncertainty. The uncertainty due to fibre positioning $\sigma_{sys-pos}$ was measured to extract $\sigma_{sys-SF}$ from the total systematic uncertainty. Thus, $\sigma_{sys-SF}$ is given by,

\begin{equation}
\sigma_{sys-SF} = \sqrt{\sigma^2_{sys} - \sigma^2_{sys-pos} }
\label{eq:TMUncertaintyFiberCharacterization}
\end{equation}
The test designed to measure $\sigma_{sys-pos}$ consisted in preparing one conditioned fibre of each type (uncladded, single clad and multiclad), all of $1~\mm$ diameter and $20~\cm$ length. Each fibre was locked to the setup, and the PMT photocurrent was measured for the LED fed at $1~\milli\ampere$. These measurements were repeated ten times for each fibre, removing and inserting the fibre each time. The mean $\bar{x}$, the standard deviation  and the relative standard deviation $\sigma^{rel}_{sys-pos}$
of the PMT photocurrent for each fibre type are shown in Table \ref{tab:PositionStandardDeviation}. The relative standard deviation is defined as
\begin{equation}
\sigma^{rel}_{sys-pos} = \frac{\sigma_{sys-pos}}{\bar{x}}
\label{eq:RelativeStandardDesviation}
\end{equation}
As it can be noticed, larger photon rates are obtained for single clad and multiclad fibres than for uncladded fibres. The reason could be that the interface between the core and the clad of the fibre is more uniform for single clad and multiclad fibres than for uncladded fibres for which the interface is air. External conditions as dirt may produce significant interface fluctuations. The statistical error of the intensity is three orders of magnitude smaller than the systematic uncertainties $\sigma_{sys-pos}$ and is neglected.

\begin{table}[htbp]
\centering{}%
\begin{tabular}{lccc}
\toprule 
Fibre type & Photon rate ($\nano\second^{-1}$) & $\sigma_{sys-pos}$ ($\nano\second^{-1}$) & $\sigma^{rel}_{sys-pos}$ (\%) \tabularnewline
\midrule
\midrule 
Uncladded & $524.09 \pm 0.01$ & $17.7$ & $3.4$ \tabularnewline
Single Clad & $1071.70 \pm 0.01$ & $9.1$ & $0.9$ \tabularnewline
Multiclad & $949.93 \pm 0.03$ & $9.9$ & $1$ \tabularnewline
\bottomrule
\end{tabular}
\caption{Mean, standard deviation and relative standard deviation due to fibre positioning in the setup of the photon rate that reach the PMT for $0.1~\milli\ampere$ LED intensity.}
\label{tab:PositionStandardDeviation}
\end{table}

%It is also seen in the table that a second clad slightly reduces the collection efficiency. The reason could be that a second clad layer reduces the radius of the fibre core proportionally, keeping the external diameter at $1~\mm$.

%In addition to the $\sigma_{pos}$ measurement, we have measured the number of photons collected by each type of fibre in the same situation, which is higher for single clad and even higher for multiclad. It means that the clad has an appreciable effect on the fibre collection efficiency and it could be a possible point to futur studies.

To determine the total systematic uncertainty $\sigma_{sys}$, ten different samples of each fibre type were prepared and the response of each fibre was measured as described above. These measurements were done for $0.05$, $0.1$, $0.15$ and $0.2~\milli\ampere$ LED current. The results for uncladded fibres are plotted in Figure \ref{fig:10samplesNC}, where it can be seen that, although each fibre shows a linear trend with increasing LED intensity, a dispersion of the fibre response is clearly observed. Similar results, displayed in figure \ref{fig:10samplesThreeTypes}, were obtained for single clad and multiclad fibres.

\begin{figure}[h]
\centering
\includegraphics[scale=0.7]{4ResearchAndDevelopments/41Fibers/10_Different_samples_NoClad.pdf}
\caption{Rate of photons reaching the PMT for $10$ uncladded fibres. Error bars are smaller than the dot size.\label{fig:10samplesNC}}
\end{figure}

\begin{figure}
\centering
    %\begin{subfigure}[b]{0.6\textwidth}
    %\centering
    %\includegraphics[width=\textwidth]{4ResearchAndDevelopments/41Fibers/10_Different_samples_NoClad.pdf}  
    %\caption{Number of photons/ns reaching the PMT for uncladded fibres.\label{subfig:10samplesNC}}
    %\end{subfigure}
    %\hfill
    \begin{subfigure}[b]{1\textwidth}
    \centering
    \includegraphics[width=\textwidth]{4ResearchAndDevelopments/41Fibers/10_Different_samples_SingleClad.pdf}  
    \caption{\label{subfig:10samplesSC}}
    \end{subfigure}
    \hfill
    \begin{subfigure}[b]{1\textwidth}
    \centering
    \includegraphics[width=\textwidth]{4ResearchAndDevelopments/41Fibers/10_Different_samples_MultiClad.pdf}  
    \caption{\label{subfig:10samplesMC}}
    \end{subfigure}
 \caption{Photon rate reaching the PMT for ten different fibres. a) Single clad fibres. b) Multiclad fibres. Error bars are smaller than the dot size.}
 \label{fig:10samplesThreeTypes}
\end{figure}
The average photon rate and the relative standard deviation versus LED intensity for each type of fibre are given in Tables \ref{tab:10DifferentSamples} and \ref{tab:RelativeStandardDeviation3FiberTypes} respectively, and are plotted in Figure \ref{fig:AveregeThreeFiberTypes} where it can be noticed that the average fibre response is linear with current. Single clad and multiclad fibres give larger signals than uncladded fibres (a factor two in the case of single clad) which indicates that the clad has a significant effect on the fibre collection efficiency. It can also be observed in Table \ref{tab:RelativeStandardDeviation3FiberTypes} that the relative standard deviation $\sigma^{rel}_{sys}$ does not vary with the LED intensity. The largest uncertainty was found for single clad fibres, despite of their higher light collection. This is most probably due to the cleaving process that produces cracks in the clad as observed in Figure \ref{fig:CleavingFiberEnd}. This damage seems to be reduced for multi-clad fibres, probably due to their larger mechanical resistance.

\begin{table}[h]
\centering{}%
\begin{tabular}{lccc}
\toprule 
 & \multicolumn{3}{c}{Photon rate ($10~\nano\second^{-1}$)} \tabularnewline
\midrule
Intensity (mA) & Uncladded & Single clad & MultiClad \tabularnewline
\midrule
\midrule 
$0.05$ & $24.5 \pm 1.1$ & $38 \pm 3$ & $37.7 \pm 1.5$ \tabularnewline
$0.1$ & $57 \pm 3$ & $92 \pm 7$ & $87 \pm 4$ \tabularnewline
$0.15$ & $92 \pm 4$ & $149 \pm 12$ & $140 \pm 6$ \tabularnewline
$0.2$ & $127 \pm 6$ & $205 \pm 17$ & $193 \pm 8$ \tabularnewline
\bottomrule
\end{tabular}
\caption{Photons rate versus LED intensity for the different type of fibres.}
\label{tab:10DifferentSamples}
\end{table}

\begin{table}[h]
\centering{}%
\begin{tabular}{lccc}
\toprule 
 & \multicolumn{3}{c}{$\sigma^{rel}_{sys}(\%)$} \tabularnewline
\midrule
Intensity (mA) & Uncladded & Single clad & MultiClad \tabularnewline
\midrule
\midrule
$0.05$ & $4.4$ & $8.7$ & $4$ \tabularnewline
$0.1$ & $4.6$ & $8$ & $4$ \tabularnewline
$0.15$ & $4.3$ & $8.1$ & $4$ \tabularnewline
$0.2$ & $4.4$ & $8.1$ & $3.9$ \tabularnewline
\midrule 
Mean & $4.4$ & $8.2$ & $4$ \tabularnewline
\bottomrule
\end{tabular}
\caption{Relative standard deviation of the photon rate $\sigma^{rel}_{sys}(\%)$ versus LED intensity for the different fibre types.}
\label{tab:RelativeStandardDeviation3FiberTypes}
\end{table}

\begin{figure}[h]
\centering
\includegraphics[scale=0.6]{4ResearchAndDevelopments/41Fibers/10_Different_Samples_Average_3_Fiber_Types.pdf}
\caption{Average photon rate versus LED current for 10 samples of different fibre types (uncladded, single clad and multiclad fibres). Error bars are smaller than the dot size.\label{fig:AveregeThreeFiberTypes}}
\end{figure}

%The relative standard deviation are also presented in these tables, where we it can be seen that the dispersion of each fibre type for different LED intensities is practically negligible, which again verifies the correct behavior of the system. 

%There is only one point (uncladded fibre with $0.1~\milli\ampere $) that is higher than we expect. We can see in Table \ref{tab:10DifferentSamplesNoClad} that the reason for this is that its standard deviation is too high (as high as the measurement for uncladded fibres with $0.15~\milli\ampere$). The reason was found in the sample 9, whose measurement was very different from the average, incresing the standard deviation, probably due to a problem in the measurement process. We discard this sample because this result is not representative.

The average of $\sigma^{rel}_{sys}$, $\sigma^{rel}_{sys-pos}$ and $\sigma^{rel}_{sys-SF}$ are given in Table \ref{tab:RelativeStandardDeviations}. The smallest relative standard deviation was found for uncladded fibres, which means that the damage occurs mainly in the fibre clad, as illustrated in Figure \ref{fig:ResultofPolishingProcess} where cracks in the clad due to the cleaving process can be seen. It was checked under microscope that this damage only occurs at the end of the fibre. Also, the largest relative standard deviation is obtained for single clad fibres, which indicates that a second clad increases the tolerance of the fibre to conditioning.

\begin{table}[htbp]
\centering{}%
\begin{tabular}{lccc}
\toprule 
Fibre type & $\sigma^{rel}_{sys}$ (\%) & $\sigma^{rel}_{sys-pos}$ (\%) & $\sigma^{rel}_{sys-SF}$ (\%) \tabularnewline
\midrule
\midrule 
Uncladded & $4.4$ & $3.4$ & $2.9$ \tabularnewline
Single clad & $8.2$ & $0.9$ & $8.1$ \tabularnewline
Multiclad & $4$ & $1$ & $3.8$ \tabularnewline
\bottomrule
\end{tabular}
\caption{Measured relative standard deviations $\sigma^{rel}_{sys}$, $\sigma^{rel}_{sys-pos}$ and $\sigma^{rel}_{sys-SF}$.}
\label{tab:RelativeStandardDeviations}
\end{table}

In summary, the relative statistical deviation due to fibre conditioning was quantified for the different fibre types. It was found that a fibre clad improves the photon collection efficiency but at the cost of worsening its standard deviation. Larger uncertainties (a factor two) in the light collection were observed in single clad fibres compared to multiclad and uncladded ones. This may be due to the damage in the clad produced by cleaving of fibres. %Therefore, it was decided to use uncladded fibres for the TRITIUM detector. 

The absolute photon collection efficiency of $10~\cm$ scintillating fibres $CE_{10}$ was measured for each type of fibre. Ten different fibres of $10~\cm$ length for each fibre type were prepared and the photon rates were measured. The results are summarized in Table \ref{tab:10DifferentSamplesAlltypes}. $CE_{10}$ was calculated as the ratio of the collected photon rate to that of a fibre of $20~\cm$ length, 
\begin{equation}
CE_{10}=\frac{R_{ph}(20~\cm)}{R_{ph}(10~\cm)}=e^{-10/L}=96\%
\label{eq:CollectionEfficiency}
\end{equation}
where $L=270~\cm$ is the attenuation length provided by the manufacturer and an exponential attenuation of the photon rate $R_{ph}$ with length is assumed \cite{Leo},
\begin{equation}
R_{ph}(x) = R_{ph}(x_0) \times e^{-(x-x_0)/L}
\label{eq:ExponentialAttenuation}
\end{equation}
The measured $CE_{10}$, shown in Table \ref{tab:CollectionEfficiencyOfFibers}, were somewhat smaller than the calculated from the attenuation length for all the scintillating fibre types.

\begin{table}[htbp]
\centering{}%
\begin{tabular}{lccc}
\toprule 
 & \multicolumn{3}{c}{Photon rate ($10^{2}~\nano\second^{-1}$)} \tabularnewline
\midrule
Intensity (mA) & Uncladded & Single clad & MultiClad \tabularnewline
\midrule
\midrule
$0.05$ & $3.2 \pm 0.6$ & $5.5 \pm 0.7$ & $4.8 \pm 0.8$ \tabularnewline
$0.1$ & $7.4 \pm 1.4$ & $12.7 \pm 1.6$ & $11.1 \pm 1.9$ \tabularnewline
$0.15$ & $11.8 \pm 2.3$ & $19.8 \pm 2.3$ & $18\pm 3$ \tabularnewline
$0.2$ & $16 \pm 3$ & $25.1 \pm 2.1$ & $23 \pm 4$ \tabularnewline
\bottomrule
\end{tabular}
\caption{Average photon rate versus LED intensity for 10 different samples of $10~\cm$ length for uncladded, single clad and multiclad fibres.}
\label{tab:10DifferentSamplesAlltypes}
\end{table}

%The collection efficiency  to $CE_{100}$ was calculated from $CE_{10}$ by assuming an exponential attenuation of the signal in length as follow \cite{}.

\begin{table}[htbp]
\centering{}%
\begin{tabular}{lcc}
\toprule 
Fibre type & $CE_{10}$ (\%) \tabularnewline
\midrule
\midrule 
Uncladded & $76 \pm 8$ \tabularnewline
Single clad & $78 \pm 6$ \tabularnewline
Multiclad & $83 \pm 7$ \tabularnewline
\bottomrule
\end{tabular}
\caption{Measured average collection efficiency $CE_{10}$ for different types of scintillating fibres.}
\label{tab:CollectionEfficiencyOfFibers}
\end{table}


%The collection efficiency, $CE_{100}$, given by the manufacturer Saint-Gobain is in the range $3.44\%-7\%$ \cite{DataSheetBCF12Fiber}. Our measurements, given in Table \ref{tab:CollectionEfficiencyOfFibers}, are close but slightly higher than the manufacturer values which could be attributed to our use of collimated photons.

%As collimated photons were used in this study, the fact that our results are in the best side is justified. As it can be seen in Table \ref{tab:CollectionEfficiencyOfFibers}, our measured values are very close to those provided by the manufacturer. %The difference between this value for the three types of fibre studied is not as large as it was expected. A possible reason is that the difference in fibre length is only $10~\cm$ and it may not be enough to see this effect. It could be interesting to repeat these tests with a larger difference in fibre length.