Plastic scintillators are easy to machine to any desired shape. The chosen shape for TRITIUM detector is the fiber, specifically, commercial fibers BCF-12 from Saint-Gobain Crystals Inc \cite{DataSheetBCF12Fiber}. This type of fiber was chosen as the result of a comparative study \cite{TFGAlberto} among some of the best-known commercial manufacturers. The BCF-12 fibers consist of a scintillating polystyrene core with the possibility of being covered by polymethylmethacrylate (PMMA) claddings. % (smaller refractive index than core in order to archieve a critical angle) or a multicladding (second cladding) with even smaller refractive index.

When a particle deposits all or part of its kinetic energy in a scintillating fiber, photons are produced in the fiber core as a result of the fluorescence process. The number of photons produced depends on the scintillating efficiency and its value is around $2.4\%$ for the BCF-12 fibers, which means that a scintillation yield of about $8000$ photons per $\MeV$ is produced for a mip. For instance, for tritium electrons of $18.6~\keV$, these fibers emit at least 149 photons, as electrons of these energies are not mips. The emission spectrum of the fibers employed in this work is shown in Figure \ref{fig:EmissionSpectrumFibers}.

\begin{figure}[htbp]
\centering
\includegraphics[scale=0.5]{3DesignPrinciples/32Tritium_detector/EmisionBCF12.png}
\caption{Emission spectrum of BCF-12 scintillating fibers of Saint-Gobain\label{fig:EmissionSpectrumFibers}~\cite{DataSheetBCF12Fiber}.}
\end{figure}

The scintillation light is guided to the sensitive part of the photosensor. A single photon produces a signal with a probability called quantum efficiency. Photons are guided in fibers according to the Snell's law \cite{Snell}. The guiding mechanism is determined by the interface between the core and the surrounding material. When a photon hits this interface, it is refracted (and therefore lost) following the Snell equation \cite{Snell}, 
\begin{equation}
n_0~sen\theta_0 = n_1~sen\theta_1
\label{eq:Snell}
\end{equation}
where $\theta_0$ is the incident angle formed by the photon and the normal to the surface of the first medium with refractive index $n_0$, and $\theta_1$  is the refraction angle formed by the photon and the normal to the second medium with refractive index $n_1$. If the surrounding material has a lower refractive index than the core of the fiber, as it is the case with scintillating fibers, there exist a critical angle, $\theta_c$, beyond which photons will be totally reflected ($\theta_1 = 90\degree$) and therefore kept within the fiber as illustrated in Figure \ref{fig:Fiber_physic},
\begin{equation}
\theta_c = \arcsin\left(\frac{n_1}{n_0} \right)
\label{eq:CriticAngle}
\end{equation}

The trapping efficiency or photon collection efficiency is defined as the efficiency of the scintillator to guide photons. For BCF-12 fibers with optical clad this efficiency is between $3.4\%$ and $7\%$ per meter of fiber (depending on the emission point, it is minimum on the fiber axis and maximum near the core-clad interface). Therefore, from the $148$ photons initially created by a tritium decay electron of $18.6~\keV$, and assuming a light yield of $8000$ photons$/\MeV$, around $52$ photons are guided along the $20~\cm$ fiber length of the TRITIUM detector, assuming a $7\%$ trapping efficiency and an exponencial attenuation with the fiber length. Thus, the output signal is small and is in the energy range of the spectrum where electronic noise is already significant. In Figure \ref{fig:Fiber_physic}, the light collection in a fiber is illustrated.

\begin{figure}[htbp]
\centering
\includegraphics[scale=0.5]{3DesignPrinciples/32Tritium_detector/Fiber_data_sheet.png}
\caption{Photon collection in a single clad fiber\label{fig:Fiber_physic}~\cite{DataSheetBCF12Fiber}.}
\end{figure}
The cladding material has a higher refractive index than air and water. Therefore, it increases the critical angle and reduces the light collection. However, it is useful for protecting the core surface from dirt and aggressive external agents that would reduce the light collection. Three different cases are shown in Table \ref{tab:CriticalAngles}, where the cladding effect is illustrated.
\begin{table}[htbp]
\centering{}%
\begin{tabular}{lcc}
\toprule 
Material & Refractive index & critical angle ($\degree$) \tabularnewline
\midrule
\midrule 
Air & 1 & $38.68$ \tabularnewline
Water & 1.33 & $56.23$ \tabularnewline
Cladding of PMMA & 1.49 & $68.63$ \tabularnewline
\bottomrule
\end{tabular}
\caption{Critical angles associated to different interfaces between polystyrene ($n_0=1.6$) and other materials.}
\label{tab:CriticalAngles}
\end{table}
As can be seen, the critical angle for uncladded fibers surrounded by water or air is smaller than for cladded fibers, which implies a larger trapping efficiency. However, in practice, it is difficult to achieve a perfect air-core or water-core interface, and this affects light collection. As commercial claddings are thicker ($30~\micro \meter$) than the mean free path of tritium decay electrons in water (around $5~\micro\meter$), cladded fibers are not an option for the TRITIUM detector. Hence, special attention is needed for achieving a good enough water-core interface. To achieve this goal a special method was developed in the ICMOL laboratory\footnote{ICMOL, \textit{Instituto de Ciencia Molecular}, is a research institute located in the \textit{Parc Científic} of the University of Valencia.} for preparing fibers for tritium detection, described in section \ref{subsec:SurfaceConditioningProcess}. The relevant parameters of the scintillating fibers used for the TRITIUM detector are given in Table \ref{tab:ParametersFibersBCF12}.

\begin{table}[htbp]
\centering{}%
\begin{tabular}{lc}
\toprule 
Property & Value \tabularnewline
\midrule
\midrule 
Core material & Polystyrene \tabularnewline
Core refractive index & 1.60 \tabularnewline
Density ($\gram/\cm^3$) & 1.05 \tabularnewline
Cladding material & Acrylic (PMMA) \tabularnewline
Cladding refractive index & 1.49 \tabularnewline
Cladding thickness & $3\%$  $\varnothing$ \tabularnewline
Numerical aperture & 0.58 \tabularnewline
Trapping efficiency & $3.4\%$ to $7\%$ \tabularnewline
$\#$ of H atoms per cc (core) & $4.82 \cdot{} 10^{22}$ \tabularnewline
$\#$ of C atoms per cc (core) & $4.85 \cdot{} 10^{22}$ \tabularnewline
$\#$ of electrons per cc (core) & $3.4 \cdot{} 10^{23}$ \tabularnewline
Radiation length (cm) & 42 \tabularnewline
Emission peak (nm) & 435 (blue) \tabularnewline
Decay time (ns) & 3.2 \tabularnewline
1/e Attenuation length (m) & 2.7 \tabularnewline
Scintillator yield (\#$\gamma$/MeV) & $\sim 8000$ \tabularnewline
Operating Temperature & $-20\celsius$ to $50\celsius$ \tabularnewline
\bottomrule
\end{tabular}
\caption{Properties of BCF-12 scintillating fibers from Saint-Gobain Inc. \cite{DataSheetBCF12Fiber}.}
\label{tab:ParametersFibersBCF12}
\end{table}