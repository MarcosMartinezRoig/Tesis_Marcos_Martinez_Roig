In this section the last two prototypes are shown, Tritium-Aveiro 0 and Tritium-IFIC 2, in which the problems previously found are solved and a well-defined design is developed for them.

A different design was developed for these prototypes so that they can allow the reading of a large number of fibers arranged in a straight position with two PMTs in time coincidence, in a safe way.

In these prototypes we pay particular attention to tritium detection efficiency, which is the reason why they use many more fibers than the preliminary prototypes and a time coincidence reading of these fibers is done using two photosensors.

Furthermore, the activity of the radioactive liquid source of tritium, which is used to fill these prototypes, will be lower since we are interested in measure their sensitivity.

The design of both prototypes is very similar and their objective is to test the subtle effects caused by the little difference, such us the diameter of the fibers used, and to choose the one with the best results which will be the final design of the Tritium detector module.