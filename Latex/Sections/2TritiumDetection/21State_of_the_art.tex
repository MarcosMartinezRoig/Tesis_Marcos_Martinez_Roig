Measurement of tritium activity is one of the routine environmental controls that are carried out in the vecinity of nuclear research facilities and nuclear power plants during their energy production lifetime. Consequently, this measurement has been carried out with different available technologies under development to improve the state of the art of tritium detection. The most employed techniques are summarized in Table \ref{tab:DifferentThecnics}.

\begin{table}[htbp]
\begin{center}
\begin{tabular}{|c|c|c|c|c|}
\hline
 & LSC & IC & Calorimetry & BIXS\\
\hline \hline \hline
\parbox{5em}{\centering Measured\\ quantity} & \parbox{5em}{\centering Scintillation\\ photons} &  \parbox{5em}{\centering Ionization\\ current} & heat & X-rays\\ \hline
LDL & $\sim\becquerel$ & $10-100~\kilo\becquerel$ & $\sim~\giga\becquerel$ & $\sim~\mega\becquerel$ \\ \hline
Sample form & Liquid & Gas, vapor & All & All \\ \hline
%Disadvantages & & Gas, vapor & All & All \\ \hline
\end{tabular}
\caption{State-of-the-art in the tritium detection for different techniques. This table show the measured quantity, low detection level (LDL) and the sample form for four different techniques, liquid scintillator counting (LSC), ionization chamber (IC), calorimetry and beta induced X-rays spectrometry (BIXS)}
\label{tab:DifferentThecnics}
\end{center}
\end{table}

Nowadays, the most used technique for measuring tritium in water is liquid scintillator counting (LSC). This technique consists of mixing a liquid sample (some ml for environmental measurements or less for higher activities) with liquid scintillator. This mixture is usually made in a ratio of 50:50 but it depends on the detection system and the activity of the samples \cite{LSCothers, HofstetterSeveral}. In this technique, the $\beta$ energy emitted from the sample excites the molecular energy levels of the liquid scintillator which promptly decays emitting several photons with a well-know energy (fluorescence), usually in the visible spectrum. Finally these photons are detected with photosensors, which convert the optical signal into measurable electrical charge. The liquid scintillator technique has a very good detection sensitivity for low activity levels of tritiated water ($<1~\becquerel/\liter$) \cite{0.6Bq_L} but it has the problems of long measurement time (up to 2 days) and of producing chemical waste, since liquid scintillator contains toluene which is toxic. In addition, this technique requires special staff for sampling, chain-of-custody and lab analysis which require economical and time resources. In order to avoid this problem some unsuccessful efforts have been made in order to build a monitor of tritium with LSC \cite{OnlineLSC}. 

The ionization chamber (IC) consists of a gas chamber (sample) which contains electrodes connected to different voltages. These electrodes collect the ionization current that is produced due to the $\beta$ radiation. It is a simple and fast system, but it has the problem of high Low Detection Limit ($> 10~\kilo\becquerel$) and of requiring the samples to be in a state of gas or steam \cite{IonizationChamber1, IonizationChamber2}.

The calorimetry method is based on the measurement of the heat generated in the detection medium (normally platinum) \cite{Calorimeter1, Calorimeter2}. The problem with this technique is that it has a high LDL, of the order of $\giga\becquerel$, and requires long measurement time, 2 days or more.

The Beta Induced X-ray Spectrometry (BIXS) is based on the measurement of the bremsstrahlung radiation produced by the tritium decay electrons, using a \ce{NaI(Tl)} crystal couplet to a PMT  \cite{XRays1, XRays2} or Silicon Drift Detector (SDD) \cite{Bremstrahlung}. The problem with this technique is a high LDL, of the order of $\mega\becquerel$.

There are additional methods for tritium detection, although they are less employed or less experimentally developed, each one with its own advantages and limitations. For example, the Avalanche PhotoDiode (APD) cannot be used in contact with water \cite{APD},  the mass spectrometry which needs to store the sample several months before taking the measurement \cite{Spectrometry} and the Cavity ring spectroscopy requires a special optical configuration that is not possible outside a laboratory \cite{Ring}.

All the above techniques are offline methods that need long time for sample collection, shipment to the laboratory and activities measurements. Therefore, they cannot be used for in-situ monitoring of tritium in water. The liquid scintillation technique is the only one with sufficiently small Low-Detection-Limit to fulfill the compliance of $100~\becquerel/\liter$ in tritium of the water samples, stablished by the EURATOM directive. 

The purpose of the TRITIUM project is to develop an alternative method, based on solid scintillators, that allows to accomplish the requirements of in-situ monitoring of levels as low as the legal limit in Europe, $100~\becquerel/\liter$, in quasi-real time. There are several studies that was developed with solid scintillators so far:

\begin{enumerate}

\item{} The study done by M. Muramatsu, A. Koyano and N. Tokunaga in 1967 who used a scintillator plate read out by two PMTs in coincidence \cite{Muramatsu}.

\item{} The study carried out by the A. A. Moghissi, H. L. Kelley, C. R. Phillips and J. E. Regnier in 1969 that used one hundred plastic fibers coated with anthracene powder and read out by two PMTs in coincidence \cite{Moghissi}.

\item{} The study performed by R. V. Osborne in 1969 that used sixty stacked scintillator plates read out by two PMTs in coincidences \cite{Osborne}.

\item{} The study done by A. N. Singh, M. Ratnakaran and K. G. Vohra in 1985, that used a scintillator sponge read out by PMTs in electronic coincidence \cite{Ratnakaran, Ratnakaran2000}.

\item{} The study carried out by K. J. Hofstetter and H. T. Wilson in 1991, that tested different shapes of scintillator plastics like several sizes of beads, fibers, etc. The better result obtained for solid plastic scintillator was a tritium detection efficiency of the order of $10^{-3}$ \cite{Hofstetter1, Hofstetter2}.

\end{enumerate}
%%Para poner varias lineas en \parbox separar la ecuación en 2 ecuaciones y poner \\ entre ellas.
\begin{table}[htbp]
\begin{center}
\begin{tabular}{|c|c|c|c|c|}
\hline
Study & \parbox{5.5em}{\centering $\varepsilon_{det}(\frac{cps \cdot{} 10^{-3}}{\kilo\becquerel/\liter})$}  & \parbox{4.5em}{\centering $F_{sci}$ ($\cm^2$)}  & \parbox{6.5em}{\centering $\eta_{det}(\frac{cps \cdot{} 10^{-6}}{\kilo\becquerel/ \liter \cdot{} \cm^2})$} & LDL ($\kilo\becquerel / \liter$)\\
\hline \hline \hline
Muramatsu & $0.39$ & $123$ & $3.13$ & $370$ \\ \hline
Moghissi & $4.50$ & $>424.1$ & $<10.6$ & $37$ \\\hline
Osborne & $12$ & $3000$ & $4$ & $37$ \\ \hline
Singh & $41$ & $3000$ & $13.7$ & $<37$ \\ \hline
Hofstetter & $2.22$ & $\sim~100$ & $<22.2$ & $25$ \\ \hline
\end{tabular}
\caption{Results of different scintillator detector for tritiated water detection. This table shows the efficiency of the detector ($\varepsilon_{det}$), its active surface ($F_{sci}$), its specific efficiency ($\eta_{det}=\varepsilon_{det}/F_{sci}$), defined as its efficiency normalized to its active surface, and its low detection-level (LDL) for each study listed above.}
\label{tab:PlasticScinTritium}
\end{center}
\end{table}

%COMPROBAR QUE ESTAN BIEN TODOS LOS DATOS (sobretodo areas, lo otro esta comprobado. A lo mejor puedo calcular el area del ultimo caso)

The results of these experiments are sumarized in Table \ref{tab:PlasticScinTritium}. As can be seen in the first column, the intrinsic detector efficiency, $\varepsilon_{det}$, is very different in these experiments. As one of the most important factor that affect the efficiency is the active surface of the plastic scintillator, $F_{sci}$, which varies largely with the detector type, the specific detector efficiency (third column) is used in order to compare these detectors, which is the intrinsic detector efficiency normalized to this active surface. It can be checked that, effectively, these specific efficiencies are quite similar. The specific efficiency obtained by Moghissi for scintillating fibers is sufficiently high to justify our choice of scintillating fibers as a detection medium. Finally, as can be seen in the last column, the LDL in all these experiments are of the order of a few tens of $\kilo\becquerel/\liter$. Thus, to develop a detector with much lower LDL is essential to comply with the EURATOM directive of 100 Bq/L of tritium in water for human consumption.