In this chapter, the most important results obtained in this PhD work and in the TRITIUM project are summarized and discussed. These results were previously presented in the chapter \ref{chap:ResearchandDevelopment} (the R\&D tasks of the TRITIUM monitor), the chapter \ref{chap:Prototypes} (results obtained with each TRITIUM prototype developed) and the chapter \ref{chap:Simulations} (results of the TRITIUM simulations).


\begin{enumerate}
\item{} During the Research \& Development tasks, the design of the TRITIUM module and its components (scintillating fibers, photosensors, etc) were characterized and optimized, as well as the water purification system and the background rejection shield. As concerning the detector development, the following results were obtained withing this PhD work:

\begin{enumerate}
\item{} A surface-conditioning method for the scintillating fibers, which consists in cleaving, polishing and cleaning the fibers following specific rules, was developed. A polishing machine, based on arduino technology, was developed, which is capable of polishing hundreds of fibers automatically. The objective of this machine was to automate the manual polishing of hundreds of fibers simultaneously, a task unattainable for the amount of fibers needed in the TRITIUM monitor. The surface-conditioning method was tested, achieving an increase of the light collected by the fibers of a factor 2 due to polishing and $25\%$ due to cleaning. Furthermore, a characterization of the uncladded scintillating fibers was carried out, in which the photon collection efficiency and the increase of its uncertainty due to the surface-conditioning method was quantified. The photon collection efficiency obtained for the uncladded scintillating fibers is $76\pm 8\%$ which is slightly smaller than expected by the manufacturer ($96\%$). The systematic uncertainty for the photon collection efficiency after applying the surface-conditioning method on uncladded scintillating fibers was measured, obtaining $2.86\%$, which is a small enough uncertainty to be acceptable for the tritium measurements.

\item{} A characterization of the SiPM model S13360-6075 from Hamamatsu was carried out. In this characterization, some of the most relevant parameters of the SiPM for the detection of tritium, such as gain of SiPM, the breakdown voltage, the temperature coefficient and others like the quenching resistance, the terminal capacitance, were measured and compared with their valuers provided by the manufacturer. The charge of the single photoelectron spectrum was measured with a resolution of $1\%$ for each photoelectron peak and up to $10$ photoelectron peaks were resolved. %small uncertainties in the single photon spectrum (about $1\%$ for each photopeak) and a very good agreement with the values expected by the manufacturer.

%An additional calibration was carried out at the level of a SiPM matrix in which the probability of crosstalk between different SiPMs was measured, obtaining an insignificant probability of happening. The linearity of the SiPM signal as a function of the the number of scintillating fibers was also verified.

Due to the strong dependence of the SiPM gain on temperature, a gain stabilization method was implemented in the temperature range of interest, $[20-30]~\celsius$. This consists in compensating the variation of the gain, due to changes in the temperature, by a variation in the bias voltage. Indeed, the gain depends linearly on both temperature and voltage bias with opposite slopes. This allows to stabilize the gain to its nominal value at $25~\celsius$ by a variation of $59.9 \pm 1.3 ~\milli\volt/\celsius$ in the bias voltage. This stabilization method was tested experimentally, obtaining variations of $0.1\%$ in the SiPM gain in the $[20-30]~\celsius$ temperature range, which is the expected range for the operation of the TRITIUM monitor. These results indicate that a stable operation of the SiPM readout can be obtained by an automatic implementaiton of this temperature dependent bias of the SiPMs in the TRITIUM monitor.

\item{} The background rejection system is divided into active and passive veto. The passive shield, which consists of a lead castle with walls of $5~\cm$ width, was designed by the TRITIUM CNBG team and installed in Arrocampo. This passive shield is aimed at reducing the soft component of the radioactive background that affects the tritium measurement (energies below $200~\MeV$). This background component is caused mainly by the background of the natural radioactivity of the detector location and the soft cosmic ray events. A factor of $5.5$ reduction for cosmic ray events due to the lead shield was obtained through simulations. In addition, it is expected that the background of the natural radioactivity (which was not included in these simulations) would be also suppressed by this shield. An active veto was built and tested, which is aimed to mitigate the hard component of the background that affects the tritium measurement (energies above $200~\MeV$), mainly caused by hard cosmic ray events. The active veto consists of two parallel plastic scintillating plates of $1~\cm$ thickness, located at $34.2~\cm$ distance on top and bottom of the TRITIUM modules respectively, and read out each one by two photosensors. The plastic scintillator was wrapped with a PTFE layer, an aluminium layer and a black tape layer. With this coverage an improvement of a factor $2$ was measured in the light collection and a spatial uniformity on the detector surface. In addition, the experimental conditions that optimize the detection of the hard cosmic events, such as discrimination thresholds and photosensors bias high voltage, were found. A hard cosmic rate of $2.5~\text{events}/\text{s}$ was experimentally obtained, which gives an efficiency of the cosmic veto of $85\%$ comparing with the expected measured count rate with that. Furthermore, the energy spectrum measured fits very well with the expected Laudau function. Finally the dependence of the hard cosmic rate on the distance between both scintillating plastics was studied and quantified, allowing this distance to be changed without the need to perform a new calibration of the active veto. A $60\%$ reduction of the hard cosmic ray events due to the active veto was obtained through simulations. Therefore a $92.5\%$ reduction of the cosmic ray events due to the full background rejection system is obtained through simulations and will be further measured in the detector location site at Arrocampo.

\item{} Regarding the water purification system, a detailed analysis of the water of Arrocampo was carried out, where the samples will be taken. Due to the presence of the high concentrations of organic components in this sample, a water purification system is needed. A water purification system was designed and installed in Arrocampo dam by the TRITIUM LARUEX team and its level of the purification was tested. Conductivities close to $10~\mu\text{S}/\cm$ (two orders of magnitud less than the initial sample) were achieved. Furthermore, the tritium activity in the sample was found to remain unchanged after the process of purification.

\end{enumerate}

%Important results were obtained through simulations, which were implemented in the built prototypes. It was seen that the efficiency of tritium detection in the prototypes is larger when short fibers (about $20~\cm$) are used insted of long fibers ($1~\meter$). An improvement of a factor $5$ is acheived in the tritium count rate measured. Also, it was found that the tritium measurement is optimized when $1~\mm$ fibers are used, compared with  $2~\mm$, since a smaller background in the energy region of interest for the tritium measurement due to cosmic ray events is obtained.

\item{} Four different prototypes were developed by the TRITIUM collaboration, called TRITIUM-IFIC-0, TRITIUM-IFIC-1, TRITIUM-Aveiro and TRITIUM-IFIC-2, listed in chronological order of their construction. The first two prototypes, TRITIUM-IFIC-0 and TRITIUM-IFIC-1, were used as proof of principle for the detection of tritium in water with scintillating fibers and identify the different issues that affect the detection efficiency. The lastest prototypes, TRITIUM-Aveiro and TRITIUM-IFIC-2, have slightly different designs. Small tritium activities were used to measure their tritium detection efficiency and MDA. %Each design has its own advantages and disadvantages and the characteristics of each one with the best results will be implemented in the final design of the TRITIUM module.

\begin{itemize}

\item{} In the first prototypes, the use of a straight arrangement of the scintillating fibers was found to be a critical point for the tritium detection. In addition, a surface-conditioning method was succesfully implemented in the scintillating fiber, obtaining an improvement in the tritium detection. The use of a PTFE vessel was also found to improve the light collection in the TRITIUM prototypes due to its optical properties (a reflectivity close to $95\%$ for visible light). All these improvements were applied on the TRITIUM-IFIC-1 prototype, obtaining an improvement in the measured count rate of tritium of more than a factor $10$. Finally, the use of two photosensors in time coincidence mode was needed to improve the prototype MDA, of the prototype since this reduces the background measured by the prototype, mainly the photosensors noise. %with almost no affecting to the tritium signal.

\item{} The lastest prototypes are based on a similar design but with subtle differences. One of the most important difference is the use a different diameter for the scintillating fibers ($2~\mm$ for the TRITIUM-Aveiro prototype and $1~\mm$ for the TRITIUM-IFIC-2 prototype). The use of $1~\mm$ fibers allows to arrange more scintillating fibers in the same space, increasing the total active area of the prototype (and, therefore, its efficiency to the tritium detection) and the signal-to-background ratio (and, therefore, improving its MDA). The use of $2~\mm$ fibers may facilitate the water flow through the fiber bunch, which may optimize the effective detection area. In addition, $2~\mm$ fibers are more resistant which could be important if higher water fluxes are used, which is not the case of the TRITIUM monitor. It was observed through simulations that the cosmic rate in the energy region of interest of tritium is a factor $2$ higher for scintillating fibers of $2~\mm$. Additional experimental tests need to be done to decide the final fiber diameter. A possible way to take this decision could be to build two identical TRITIUM prototypes (either TRITIUM-Aveiro or TRITIUM-IFIC-2), one of them with $1~\mm$ fibers and the other with $2~\mm$ fibers to compare the results. 

The second important difference between the TRITIUM-Aveiro and TRITIUM-IFIC-2 prototypes is the type of photosensor proposed. TRITIUM-Aveiro use PMTs and the TIRITIUM-IFIC-2 purpose SiPM arrays. SiPM arrays have some advantages such as a higher photodetection efficiency, which would increase the detection efficiency of the TRITIUM detector. Furthermore, the SiPMs do not need high voltage, which implies a reduction of the TRITIUM monitor cost. However, SiPM arrays have some disadvantages as the need to read out more channels, which complicates the electronic. In addition, a gain stabilization method is needed due to the strong dependence of the SiPMs gain on temperature.

The specific efficiency obtained with the TRITIUM-IFIC-2 prototype, $141.45 \pm 5.52~\frac{cps \cdot{} 10^{-6}}{\kilo\becquerel/ \liter \cdot{} \cm^2}$, is better than that obtained with the TRITIUM-Aveiro prototype, $15.93 \pm 4.77~\frac{cps \cdot{} 10^{-6}}{\kilo\becquerel/ \liter \cdot{} \cm^2}$, most probably due to the surface-conditioning method apply to the fibers of the last two IFIC prototypes. In addition, a lower MDA was obtained for the TRITIUM-IFIC-2 prototype, $677~\becquerel/\liter$ for an integration time of $10~\min$ or $218~\becquerel/\liter$ for an integration time of $1~\hour$, compared to the the TRITIUM-Aveiro prototype MDA, $29.8~\kilo\becquerel/\liter$ for an integration time of $1~\min$ or $5~\kilo\becquerel/\liter$ for an integration time of $1~\hour$. A lower MDA allows to discriminate smaller tritium activities from the background.

A summary of the state-of-the-art of the tritium detection is shown in Table \ref{tab:ComparisonResultsTritium}, which includes the results obtained with the four different prototypes developed in the TRITIUM project. As it can be seen, the TRITIUM-IFIC-2 prototype overcome the current state-of-the-art, obtaining a specific efficiency and MDA better that the best result obtained in other experiments (Singh or Hofstetter), almost an order of magnitud better in both parameters. As it can be seen, the specific efficiency obtained for the TRITIUM-Aveiro prototype is smaller than expected. 

\begin{table}[htbp]
\centering{}%
\begin{tabular}{lcccc}
\toprule 
Study & \parbox{5.5em}{$\varepsilon_{det}(\frac{cps \cdot{} 10^{-3}}{\kilo\becquerel/\liter})$}  & \parbox{4.5em}{$F_{sci}$ ($\cm^2$)}  & \parbox{6.5em}{$\eta_{det}(\frac{cps \cdot{} 10^{-6}}{\kilo\becquerel/ \liter \cdot{} \cm^2})$} & MDA ($\kilo\becquerel / \liter$) \tabularnewline
\midrule
\midrule 
Muramatsu & $0.39$ & $123$ & $3.13$ & $370$ \tabularnewline
Moghissi & $4.50$ & $>424.1$ & $<10.6$ & $37$ \tabularnewline
Osborne & $12$ & $3000$ & $4$ & $37$ \tabularnewline
Singh & $41$ & $3000$ & $13.7$ & $<37$ \tabularnewline
Hofstetter & $2.22$ & $\sim~100$ & $<22.2$ & $25$ \tabularnewline
T-IFIC-0 & $2.11 \pm 0.85$ & $219.91$ & $9.59 \pm 3.87$ & $100$* \tabularnewline
T-IFIC-1 & $38.42 \pm 1.61$ & $402.12$ & $95.55 \pm 4.01$ & $100$* \tabularnewline
T-Aveiro & $64.87 \pm 19.41$ & $4071.50$ & $15.93 \pm 4.77$ & $5$** \tabularnewline
T-IFIC-2 & $711.03 \pm 27.77$ & $5026.55$ & $141.45 \pm 5.52$ & $0.22$** \tabularnewline
\bottomrule
\end{tabular}
\caption{Results of scintillator detectors developed for several experiments (including the TRITIUM project) for tritiated water detection. This table shows the efficiency of the detector ($\varepsilon_{det}$), its active surface ($F_{sci}$), its specific efficiency ($\eta_{det}=\varepsilon_{det}/F_{sci}$), defined as its efficiency normalized to its active surface, and its MDA for each study listed above. "*" specific activity measured, not MDA. "**" MDA measured for $1~\hour$ integration time.}
\label{tab:ComparisonResultsTritium}
\end{table}

\end{itemize}

The MDA achieved with the TRITIUM-IFIC-2 prototype is $218~\becquerel/\liter$ for an integration time of $1~\hour$, which can be still considered a quasi-real time. One of the most relevant properties of the TRITIUM monitor is that it is scalable, which means that better results can be achieved by using a larger number of modules. The MDA of the TRITIUM monitor is expected to be reduced by a factor $\sqrt{\text{Number of modules}}$ with respect to the MDA obtained with one module. Therefore, as it is shown in Figure \ref{fig:MDATRITIUMmonitor}, an MDA of  $100~\becquerel/\liter$ (goal of the TRITIUM project) could be achieved using 5 TRITIUM-IFIC-2 modules and an integration time of $1~\hour$, which is the option chosen by the TRITIUM collaboration. The idea is to first install three TRITIUM-IFIC-2 modules with which all possible problems of working with several TRITIUM modules read out in parallel will be detected and solved, and then, to intall the two TRITIUM-IFIC-2 modules remained to reach the MDA of $100~\becquerel/\liter$.

It has to be taken into account that the MDA reported in this PhD work was measured without the installation of the background rejection system. The MDA of these TRITIUM prototypes are expected to improve when the background rejection system is included.

The stability of the tritium detection efficiency of the latest TRITIUM prototypes was verified during six months, obtaining a stable behavior of the detector during this time.

\item{} Finally, simulations were carried out to determine the dependence of the tritium detection efficiency and activity resolution of the TRITIUM-IFIC-2 prototype on integration times and number of prototypes. These simulations, which agree with the experimental measurements obtained with this prototype, allow us to make the decision of how many modules are used in TRITIUM monitor and which integration time use. It can be used three different modules read in parallel with an integration time of $30~\min$, with which a difference of $250~\becquerel/\liter$ in the tritium concentration is expected to be measured. It could also be chosen to use five different modules with an integration time of $10~\min$ with which a difference of $200~\becquerel/\liter$ in the tritium concentration is expected to be measured.

\end{enumerate}

Currently, the lead shielding, the water purification system and a TRITIUM-Aveiro prototype are installed in Arrocampo dam. Two additional TRITIUM-Aveiro prototypes and an active veto are planned to be installed as soon as possible. Furthermore, three TRITIUM-IFIC-2 prototypes and an active veto are ready to be installed too, the installation of which was delayed due to the coronavirus pandemic.